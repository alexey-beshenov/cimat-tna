\documentclass{article}

\usepackage[utf8]{inputenc}
\usepackage[spanish]{babel}

\usepackage{amsmath,amssymb}

\title{Teoría de números algebraicos. Tarea 1}
\author{Alexey Beshenov (alexey.beshenov@cimat.mx)}
\date{19 de agosto de 2020}

\begin{document}

\maketitle

\thispagestyle{empty}

\noindent Elija y resuelva \emph{cuatro} problemas de la siguiente lista.

\begin{enumerate}
\item Para $d \ge 3$ libre de cuadrados demuestre que $2$ es irreducible,
  pero no es primo en los anillos
  \begin{enumerate}
  \item[a)] $\mathbb{Z} [\sqrt{-d}]$,
  \item[b)] $\mathbb{Z} [\sqrt{d}]$ para $d \equiv 1 \pmod{4}$.
  \end{enumerate}

  Concluya que estos no son dominios de factorización única.

\item Sea $p \equiv 1 \pmod{3}$ un primo racional. Usando la factorización única
  en $\mathbb{Z} [\zeta_3]$, demuestre que los números $u,v \in \mathbb{Z}$ en
  la expresión $4p = u^2 + 27 v^2$ están bien definidos salvo el signo.

\item Verifique sin computadora si la congruencia
  $$x^3 \equiv 2 - 3\zeta_3 \pmod{23}$$
  tiene solución en $\mathbb{Z} [\zeta_3]$.

  \emph{Sugerencia: en total en $(\mathbb{Z} [\zeta_3]/(23))^\times$ habrá
    $\frac{23^2 - 1}{3} = 177$ cubos y no es una buena idea enumerarlos uno por
    uno\dots}

\item Encuentre las soluciones enteras de $y^2 = x^3 - 4$.

  \emph{Sugerencia: $y^2 + 4 = (y + 2i)\,(y - 2i)$.}

\item Consideremos la ecuación $x^2 - 7y^2 = n$, donde
  $$n = 2,3,4,5,6,7,8,9,10.$$
  ¿Para cuáles de estos $n$ existen soluciones enteras?
  Demuestre que en este caso hay un número infinito de ellas.

\item Calcule el índice de subgrupo
  \[ \Bigl[ \mathbb{Z} \Bigl[\frac{1+\sqrt{5}}{2}\Bigr]^\times :
            \mathbb{Z} [\sqrt{5}]^\times \Bigr]. \]
\end{enumerate}

\noindent\rule{50pt}{1pt}

\noindent Fecha límite: viernes, 28 de agosto.

\end{document}

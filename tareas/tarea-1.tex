\documentclass{article}

\usepackage[utf8]{inputenc}
\usepackage[spanish]{babel}

\usepackage{amsmath,amssymb,amsthm}

\newcounter{tarea}
\setcounter{tarea}{1}
\theoremstyle{definition}
\newtheorem{ejercicio}{Ejercicio}[tarea]

\newenvironment{solucion}{\begin{proof}[Solución]}{\end{proof}}

\usepackage{mathspec}

\setmainfont{PT Serif}
\setsansfont{Montserrat}
\setmonofont{PT Mono}

\title{Teoría de números algebraicos\\Tarea 1}
\author{Alexey Beshenov (alexey.beshenov@cimat.mx)}
\date{19 de agosto de 2020}

% \def\solutions{true}

\begin{document}

{\sffamily\bfseries\maketitle}

\noindent Fecha límite: viernes, 28 de agosto.

\ifdefined\solutions
\else
\thispagestyle{empty}
\fi

\begin{ejercicio}
  Para $d \ge 3$ libre de cuadrados demuestre que $2$ es irreducible,
  pero no es primo en los anillos
  \begin{enumerate}
  \item[a)] $\mathbb{Z} [\sqrt{-d}]$,
  \item[b)] $\mathbb{Z} [\sqrt{d}]$ para $d \equiv 1 \pmod{4}$.
  \end{enumerate}

  Concluya que estos no son dominios de factorización única.

  \ifdefined\solutions
  \begin{solucion}
    Primero, para la irreducibilidad de $2$, la norma sobre
    $\mathbb{Z} [\sqrt{-d}]$ viene dada por $a^2 + db^2$.    
    Tenemos $N (2) = 4$, y se ve que para $d \ge 2$ no hay elementos de norma
    $2$. Esto implica que $2$ es irreducible.

    En el anillo $\mathbb{Z} [\sqrt{d}]$ con $d \equiv 1 \pmod{4}$ la norma
    viene dada por
    $$a^2 - db^2 \equiv a^2 - b^2 \pmod{4}.$$
    Los cuadrados módulo $4$ son $0$ y $1$, y de allí se ve que
    $a^2 - b^2 \not\equiv 2 \pmod{4}$. Esto demuestra la irreducibilidad de $2$
    en el caso b).

    Si $d$ es par, notamos que $2 \mid \sqrt{-d}\,\sqrt{-d}$, pero
    $2 \nmid \sqrt{-d}$ (aquí usamos que $4\nmid d$). Si $d$ es impar, entonces
    $2 \mid (1 + \sqrt{-d})\,(1 - \sqrt{-d})$, pero $2 \nmid 1 \pm \sqrt{-d}$.
    Esto demuestra que $2$ no es primo en los anillos $\mathbb{Z} [\sqrt{-d}]$.

    De la misma manera, para $d$ impar se tiene
    $2 \mid (1 + \sqrt{d})\,(1 - \sqrt{d})$, pero $2 \nmid (1 \pm \sqrt{d})$.
    Esto demuestra que $2$ no es primo en $\mathbb{Z} [\sqrt{d}]$ con $d$
    impar.

    Recordemos que en un dominio de factorización única todo irreducible debe
    ser primo, así que acabamos de probar que los anillos en cuestión no tienen
    factorización única.
  \end{solucion}
  \fi
\end{ejercicio}

\begin{ejercicio}
  Sea $p \equiv 1 \pmod{3}$ un primo racional. Usando la factorización única en
  $\mathbb{Z} [\zeta_3]$, demuestre que los números $u,v \in \mathbb{Z}$ en la
  expresión $4p = u^2 + 27 v^2$ están bien definidos salvo el signo.

  \ifdefined\solutions
  \begin{solucion}
    Supongamos que se tiene
    $$4p = u^2 + 3 v^2 = u'^2 + 3 v'^2,$$
    donde
    $$v \equiv v' \equiv 0 \pmod{3}.$$
    Notamos que necesariamente
    $$u \equiv v\pmod{2}, \quad u' \equiv v'\pmod{2},$$
    y además, $3 \nmid u$, $3 \nmid u'$. Un pequeño cálculo demuestra que
    $$p = \pi\,\overline{\pi} = \pi'\,\overline{\pi'},$$
    donde
    $$\pi = \frac{u+v}{2} + v\zeta_3, \quad \pi' = \frac{u'+v'}{2} + v'\zeta_3$$
    son primos, ya que tienen norma $p$. La factorización única implica que
    $$\pi \sim \pi' \quad\text{o}\quad \pi \sim \overline{\pi'}.$$
    Calculamos que
    $$\overline{\pi} = \frac{u-v}{2} - v\zeta_3.$$
    Entonces, cambiando el signo de $v$, podemos asegurarnos de que
    $\pi \sim \pi'$; es decir, $\pi = \epsilon \pi'$ con
    $\epsilon \in \{ \pm 1, \pm \zeta_3, \pm \zeta_3^2 \}$.
    Consideremos una por una las seis posibilidades.

    \begin{itemize}
      \item Si $\epsilon = +1$, entonces
        $$\frac{u+v}{2} + v\zeta_3 = \frac{u+v}{2} + v'\zeta_3,$$
        así que $v' = v$ y $u' = u$.

      \item Si $\epsilon = -1$, entonces
        $$\frac{u+v}{2} + v\zeta_3 = -\frac{u+v}{2} - v'\zeta_3,$$
        de donde $v' = -v$ y $u = -u$.

      \item Si $\epsilon = \zeta_3$, se obtiene
        $$\frac{u+v}{2} + v\zeta_3 = -v' + \frac{u'-v'}{2}\zeta_3.$$
        Esto implicaría que $u = -2v' - v$, pero luego $3 \mid u$,
        y no es el caso.

      \item El caso de $\epsilon = -\zeta_3$ se descarta de manera similar.

      \item En fin, si $\epsilon = \pm \zeta_3^2$, entonces
        $\pi = \pm \zeta_3^2 \pi'$ implica que $\pi' = \pm\zeta_3\pi$, y este
        caso ya fue considerado.
    \end{itemize}

    Podemos concluir que $\epsilon = \pm 1$, lo que implica que $u = \pm u'$
    y $v = \pm v'$.
  \end{solucion}
  \fi
\end{ejercicio}

\begin{ejercicio}
  Verifique sin computadora si la congruencia
  $$x^3 \equiv 2 + 3\zeta_3 \pmod{23}$$
  tiene solución en $\mathbb{Z} [\zeta_3]$.

  \emph{Sugerencia: en total en $(\mathbb{Z} [\zeta_3]/(23))^\times$ habrá
    $\frac{23^2 - 1}{3} = 176$ cubos y no es una buena idea enumerarlos uno por
    uno\dots}

  En general, dado un primo racional $p \equiv 2 \pmod{3}$, ¿cuándo
  $2 + 3\zeta_3$ es un cubo módulo $p$?

  \ifdefined\solutions
  \begin{solucion}
    Notamos que $N (2 + 3\zeta_3) = 7$ es un primo racional, así que
    $2 + 3\zeta_3$ es un primo en $\mathbb{Z} [\zeta_3]$. Además,
    $2 + 3\zeta_3 \equiv 23 \equiv 2 \pmod{3}$; se trata de primos
    primarios y se aplica la reciprocidad cúbica
    $$\left(\frac{2 + 3\zeta_3}{23}\right)_3 = \left(\frac{23}{2 + 3\zeta_3}\right)_3.$$
    Recordemos que $\mathbb{Z} [\zeta_3]/(2 + 3\zeta_3) \cong \mathbb{F}_7$,
    y por lo tanto la pregunta se reduce a ver si $23$ es un cubo módulo $7$.
    Hay solo $(7-1)/3 = 2$ cubos módulo $7$, y estos son claramente $\pm 1$.
    Tenemos $23 \equiv 2 \pmod{7}$. Entonces, la respuesta es negativa.

    En general, para un primo racional $p \equiv 2 \pmod{3}$ el número
    $2 + 3\zeta_3$ es un cubo módulo $p$ si y solamente si
    $p \equiv \pm 1 \pmod{7}$.
  \end{solucion}
  \fi
\end{ejercicio}

\begin{ejercicio}
  Encuentre las soluciones enteras de $y^2 = x^3 - 4$.

  \emph{Sugerencia: $y^2 + 4 = (y + 2i)\,(y - 2i)$.}

  \ifdefined\solutions
  \begin{solucion}
    Primero un \emph{spoiler}: una búsqueda indica que hay cuatro soluciones
    $$(x,y) = (2, \pm 2), ~ (5, \pm 11),$$
    pero hay que verificar que no hay otras. Para esto factorizamos en
    $\mathbb{Z} [i]$
    $$x^3 = (y + 2i)\,(y - 2i).$$

    Primero, supongamos que $y$ es impar. Si un primo de Gauss $\pi$ divide a
    $y + 2i$ e $y - 2i$, entonces $\pi \mid 4i$, y por lo tanto $\pi \sim 1+i$.
    Pero $y$ es impar, así que $y \pm 2i \equiv 1 \pmod{1+i}$. Esto implica que
    $$\operatorname{mcd} (y + 2i, y-2i) = 1.$$
    La factorización única en $\mathbb{Z} [i]$ nos permite concluir que
    $$y + 2i = u\,(a + bi)^3,$$
    donde $u \in \mathbb{Z} [i]^\times$. Todas las unidades en $\mathbb{Z} [i]$
    son cubos, así que podemos asumir que $u = +1$. Escribamos
    \[ y + 2i = (a+bi)^3 = a^3 - 3ab^2 + (3a^2b - b^3)\,i
              = a\,(a^2 - 3b^2) + b\,(3a^2 - b^2)\,i. \]
    De la ecuación $2 = b\,(3a^2 - b^2)$ se ve que las soluciones enteras son
    $(a,b) = (\pm 1, 1)$ y $(\pm 1,-2)$. Estas corresponden a
    $y = \pm 2$ (pero $y$ es impar, así que este caso se puede descartar por
    el momento) y $y = \pm 11$.

    Ahora bien, supongamos que $y$ es par. Analizando la ecuación
    $y^2 = x^3 - 4$, se ve que la máxima potencia de $2$ que puede dividir
    a $y^2$ es $4$, así que $y = 2y'$, donde $2 \nmid y'$. Se obtiene
    $$x^3 = 4\,(y' + i)\,(y' - i).$$
    Aquí
    $$\operatorname{mcd} (y' + i, y' - i) = 1+i.$$
    Entonces, podemos escribir
    $$x^3 = 2^3 \cdot \frac{y' + i}{1 + i} \cdot \frac{y' - i}{1 - i},$$
    donde los factores son coprimos. Esto implica que
    $$\frac{y' + i}{1 + i}$$
    es un cubo. Escribamos
    \[ y' + i = (1 + i)\,(a + bi)^3
              = (a-b)\,(a^2 + 4ab + b^2) + (a+b)\,(a^2 - 4ab + b^2)\,i. \]
    Se sigue que
    $$a + b = \pm 1, \quad a^2 - 4ab + b^2 = \mp 1.$$
    Las soluciones enteras son $(a,b) = (0,-1), (-1,0)$, y estas nos dan
    $y' = \pm 1$, así que $y = \pm 2$.
  \end{solucion}
  \fi
\end{ejercicio}

\begin{ejercicio}
  Consideremos la ecuación $x^2 - 7y^2 = n$, donde
  $$n = 2,3,4,5,6,7,8,9,10.$$
  ¿Para cuáles de estos $n$ existen soluciones enteras?
  Demuestre que en este caso hay un número infinito de ellas.

  \ifdefined\solutions
  \begin{solucion}
    Primero consideremos la ecuación
    $$x^2 - 7y^2 = 1.$$
    Una solución no trivial es $(8, 3)$. Esta corresponde a la unidad
    $$u = 8 + 3\sqrt{7} \in \mathbb{Z} [\sqrt{7}]^\times.$$
    Luego,
    $\pm u^k$ para todo $k \in \mathbb{Z}$ son también unidades, y son
    diferentes: $u^k = u^\ell$ para $k\ne \ell$ sucede solamente
    cuando $u = \pm 1$.

    En realidad, se puede verificar que $u$ es la unidad fundamental y
    $$\mathbb{Z} [\sqrt{7}]^\times = \{ \pm 1 \} \times \langle u\rangle,$$
    pero esto no es necesario para el ejercicio.

    De la misma manera, una solución de $x^2 - 7y^2 = n$ corresponde a un
    elemento $\alpha = x + y\sqrt{7}$ con $N (\alpha) = n$, y luego
    $N (u^k \alpha) = N (\alpha) = n$, así que los números
    $$x' + y'\sqrt{7} = u^k \alpha$$
    para diferentes $k$ nos dan diferentes soluciones. Esto demuestra que
    si hay una solución de $x^2 - 7y^2 = n$, entonces habrá un número infinito
    de ellas.

    \vspace{1em}

    Reduciendo módulo $7$ se obtiene $x^2 \equiv n \pmod{7}$. Los cuadrados
    módulo $7$ son $1$, $2 \equiv 3^2$ y $4$. Esto demuestra que para
    $n = 3,5,6,10$ no hay soluciones. Por otra parte, reduciendo módulo $4$,
    notamos que $x^2 - 7y^2 \equiv x^2 + y^2 \pmod{4}$, así que
    $n \not\equiv 3\pmod{4}$.  De esta manera descartamos $n = 3$ y $7$.

    Nos quedan $n = 2,4,8,9$, y para estos valores es fácil encontrar una
    solución. Notamos que si $n$ es un cuadrado, entonces existe una solución
    obvia $(x,y) = (\sqrt{n}, 0)$, así que para $n = 4, 9$ habrá soluciones.

    Para $n = 2$ se encuentra la solución $(3,1)$ que corresponde a
    $N (3 + \sqrt{7}) = 2$. Luego
    $$N ((3 + \sqrt{7})^3) = N (3 + \sqrt{7})^3 = 8.$$
    Calculamos que
    $$(3 + \sqrt{7})^3 = 90 + 34\sqrt{7},$$
    así que $(90, 34)$ es una solución para $n = 8$. También se puede notar que
    $(6,2)$ es una solución.
  \end{solucion}
  \fi
\end{ejercicio}

\end{document}

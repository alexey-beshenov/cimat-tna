\documentclass{article}

\usepackage[utf8]{inputenc}
\usepackage[spanish]{babel}

\usepackage{amsmath,amssymb,amsthm}

\usepackage{tikz-cd}
\usetikzlibrary{babel}

\newcounter{tarea}
\setcounter{tarea}{4}
\theoremstyle{definition}
\newtheorem{ejercicio}{Ejercicio}[tarea]

\newenvironment{solucion}{\begin{proof}[Solución]}{\end{proof}}

\usepackage{array}
\newcolumntype{x}[1]{>{\centering\hspace{0pt}}p{#1}}

\usepackage{mathspec}

\setmainfont{PT Serif}
\setsansfont{Montserrat}
\setmonofont{PT Mono}

\newcommand{\ZZ}{\mathbb{Z}}
\newcommand{\FF}{\mathbb{F}}
\newcommand{\QQ}{\mathbb{Q}}
\renewcommand{\O}{\mathcal{O}}

\DeclareMathOperator{\Gal}{Gal}

\title{Teoría de números algebraicos\\Tarea 4}
\author{Alexey Beshenov (alexey.beshenov@cimat.mx)}
\date{15 de septiembre de 2020}

% \def\solutions{true}

\begin{document}

{\sffamily\bfseries\maketitle}

\noindent Fecha límite: viernes, 25 de septiembre.

\ifdefined\solutions
\else
\thispagestyle{empty}
\fi

\vspace{1em}

\begin{ejercicio}
  Encuentre la fórmula para el discriminante del polinomio
  $$x^n + ax + b.$$
\end{ejercicio}

\begin{ejercicio}
  Sea $K/\QQ$ un campo de números y $\alpha \in \O_K$ un elemento entero tal que
  $\alpha \notin m \O_K$ para $m > 1$. Demuestre que en este caso
  existe una base de $\O_K$ sobre $\ZZ$ que contiene $\alpha$. En particular,
  demuestre que $\O_K$ siempre admite una base que contiene $1$.
\end{ejercicio}

\begin{ejercicio}
  Sea $d$ un entero libre de cuadrados. Consideremos el campo cúbico
  $K = \QQ (\sqrt[3]{d})$. Denotemos $\alpha = \sqrt[3]{d}$ y consideremos
  un elemento
  $$\beta = a + b\alpha + c\alpha^2, \quad a,b,c \in \QQ.$$

  \begin{enumerate}
  \item[a)] Calcule las trazas 
    $T_{K/\QQ} (\beta)$, $T_{K/\QQ} (\alpha\beta)$, $T_{K/\QQ} (\alpha^2\beta)$
    y la norma $N_{K/\QQ} (\beta)$.

  \item[b)] Si $\beta \in \O_K$, entonces las trazas y normas de arriba son
    números enteros. Use esto para concluir que
    $\O_K  \subseteq \frac{1}{3} \ZZ [\alpha]$.

  \item[c)] Use estas consideraciones para calcular el anillo de enteros $\O_K$
    y discriminante $\Delta_K$ (¡la respuesta depende de $d$!).
  \end{enumerate}
\end{ejercicio}

\begin{ejercicio}
  Encuentre el anillo de enteros $\O_K$ y discriminante $\Delta_K$ para
  los campos cúbicos $\QQ (\sqrt[3]{6})$ y $\QQ (\sqrt[3]{12})$.
\end{ejercicio}

\begin{ejercicio}
  Consideremos el campo cúbico $K = \QQ (\sqrt[3]{17})$.

  \begin{enumerate}
  \item[a)] Calcule el anillo de enteros $\O_K$ y discriminante $\Delta_K$.
  \item[b)] Describa las factorizaciones de primos racionales $p \in \ZZ$ en
    $\O_K$.
  \item[c)] Describa los ideales primos $\mathfrak{p} \subset \O_K$
    tales que $N_{K/\QQ} (\mathfrak{p}) \le 30$.
  \item[d)] Describa todos los ideales $I \subset \O_K$
    tales que $N_{K/\QQ} (I) \le 30$.
  \end{enumerate}
\end{ejercicio}

\end{document}

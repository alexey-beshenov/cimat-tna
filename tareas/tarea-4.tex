\documentclass{article}

\usepackage[utf8]{inputenc}
\usepackage[spanish]{babel}

\usepackage{amsmath,amssymb,amsthm}

\usepackage{multicol}

\usepackage{tikz-cd}
\usetikzlibrary{babel}

\newcounter{tarea}
\setcounter{tarea}{4}
\theoremstyle{definition}
\newtheorem{ejercicio}{Ejercicio}[tarea]

\newenvironment{solucion}{\begin{proof}[Solución]}{\end{proof}}

\usepackage{framed}

\usepackage{mathspec}

\setmainfont{PT Serif}
\setsansfont{Montserrat}
\setmonofont{PT Mono}

\newcommand{\ZZ}{\mathbb{Z}}
\newcommand{\FF}{\mathbb{F}}
\newcommand{\QQ}{\mathbb{Q}}
\renewcommand{\O}{\mathcal{O}}

\DeclareMathOperator{\GL}{GL}
\DeclareMathOperator{\Res}{Res}
\renewcommand\gcd{\operatorname{mcd}}

\title{Teoría de números algebraicos\\Tarea 4}
\author{Alexey Beshenov (alexey.beshenov@cimat.mx)}
\date{15 de septiembre de 2020}

% \def\solutions{true}

\begin{document}

{\sffamily\bfseries\maketitle}

\noindent Fecha límite: viernes, 25 de septiembre.

\ifdefined\solutions
\else
\thispagestyle{empty}
\fi

\vspace{1em}

\begin{ejercicio}
  Encuentre la fórmula para el discriminante del polinomio
  $$x^n + ax + b.$$

  \ifdefined\solutions
  \begin{solucion}
    Este cálculo es bien conocido, pero espero que no hayan \emph{googleado}
    la respuesta inmediatamente :--) Daré una prueba que tiene sentido en
    el contexto de nuestra clase. Hemos visto la fórmula
    \[ \Delta (f) = (-1)^{n\choose 2}\,\Res (f,f')
                  = (-1)^{n \choose 2} \prod_{1 \le i \le n} f' (\alpha_i), \]
    donde $\alpha_1,\ldots,\alpha_n$ son las raíces de $f$.
    En nuestro caso $f = x^n + ax + b$ y su derivada es
    $f' = n x^{n-1} + a$. Para una raíz $\alpha$ de $f$ calculamos
    \[ \alpha f' (\alpha) = n\,\alpha^n + a\,\alpha =
       n\,(-a\alpha - b) + a\,\alpha = -nb - (n-1)\,a\,\alpha. \]
    Entonces,
    $$\frac{\alpha}{(n-1)\,a}\,f' (\alpha) = -\frac{nb}{(n-1)\,a} - \alpha.$$
    Ahora
    \[ \prod_i f' (\alpha_i) =
       \frac{(n-1)^n\,a^n}{\prod_i \alpha} \, \prod_i \left(-\frac{nb}{(n-1)\,a} - \alpha_i\right)
       = \frac{(n-1)^n\,a^n}{(-1)^n b}\,f\left(-\frac{nb}{(n-1)\,a}\right). \]
    Calculamos que
    \[ f\left(-\frac{nb}{(n-1)\,a}\right) =
       (-1)^n\,\frac{n^n\,b^n}{(n-1)^n\,a^n} - \frac{nb}{(n-1)} + b. \]
    De aquí
    $$\prod_i f' (\alpha_i) = n^n\,b^{n-1} - (-1)^n\,(n-1)^{n-1}\,a^n.$$
    Entonces, podemos escribir la fórmula del discriminante como
    $$\Delta (f) = (-1)^{n\choose 2}\,((-1)^{n+1}\,(n-1)^{n-1}\,a^n + n^n\,b^{n-1}).$$

    Técnicamente hablando, en algún momento hemos ocupado la división por
    $\alpha_i$ y por $a$, asumiendo de manera implícita que $a,b\ne 0$.
    Sin embargo, de todos modos, el resultante puede ser escrito como
    el determinante de alguna matriz formada por los coeficientes de $f$ y $f'$,
    y entonces este es un polinomio en los coeficientes de $f$. Como
    consecuencia, si nuestra fórmula es válida para $a,b \ne 0$, esta debe ser
    válida para $a = 0$ o $b = 0$. (Les doy esta justificación tramposa para no
    considerar diferentes casos por separado :--)
  \end{solucion}
  \fi
\end{ejercicio}

\begin{ejercicio}
  Sea $K/\QQ$ un campo de números y $\alpha \in \O_K$ un elemento entero tal que
  $\alpha \notin m \O_K$ para $m > 1$. Demuestre que en este caso
  existe una base de $\O_K$ sobre $\ZZ$ que contiene $\alpha$. En particular,
  demuestre que $\O_K$ siempre admite una base que contiene $1$.

  \ifdefined\solutions
  \begin{solucion}
    Esta pregunta es sobre álgebra lineal. Dado que $\O_K \cong \ZZ^n$, estamos
    simplemente afirmando que si $\vec{a} = (a_1,\ldots,a_n)$ es un vector
    que cumple la condición $\gcd (a_1,\ldots,a_n) = 1$, entonces $\vec{a}$ está
    contenido en alguna base de $\ZZ^n$. Esto equivale a decir que existe una
    matriz entera invertible de $n\times n$ que contiene el vector $\vec{a}$
    como una de sus columnas (o filas). Está claro que esto es imposible si
    $a_1,\ldots,a_n$ no son coprimos: el determinante de la matriz será
    divisible por $d = \gcd (a_1,\ldots,a_n)$. Para $n = 2$ la afirmación está
    clara:
    \[ \gcd (a_1, a_2) = 1 \iff
       \det\begin{pmatrix} a_1 & b_1 \\ a_2 & b_2 \end{pmatrix} = \pm 1
       \text{ para algunos }b_1,b_2 \in \ZZ. \]
    Se puede dar una prueba por inducción que a partir de $\vec{a}$   
    construye una matriz en $\GL_n (\ZZ)$ que contiene $\vec{a}$. A saber, si
    $\gcd (a_1,\ldots,a_n) = 1$, consideremos los números
    $\frac{a_1}{d},\ldots,\frac{a_{n-1}}{d}$, donde
    $d = \gcd (a_1,\ldots,a_{n-1})$. En este caso por la hipótesis de inducción
    habrá una matriz de determinante $\pm 1$
    \[ \begin{pmatrix}
      a_1/d & b_{11} & \cdots & b_{1,n-2} \\
      \vdots & \vdots & \ddots & \vdots \\
      a_{n-1}/d & b_{n-1,1} & \cdots & b_{n-1,n-2}
    \end{pmatrix}. \]
    Ahora $\gcd (a_1,\ldots,a_n) = \gcd (d, a_n) = 1$, así que se tiene
    $x d + y a_n = 1$ para algunos $x,y\in \ZZ$. Se puede verificar que
    \[ \det \begin{pmatrix}
      a_1 & b_{11} & \cdots & b_{1,n-2} & y\,a_1/d \\
      \vdots & \vdots & \ddots & \vdots & \vdots \\
      a_{n-1} & b_{n-1,1} & \cdots & b_{n-1,n-2} & y\,a_{n-1}/d \\
      a_n & 0 & \cdots & 0 & x
    \end{pmatrix} = \pm 1. \]
    (Despeje el determinante respecto a la última fila.)

    \pagebreak

    Propongo ver un argumento un poco distinto y más natural.

    Consideremos el cociente de $\ZZ$-módulos $M = \ZZ^n/\ZZ\vec{a}$. Afirmamos
    que este es un $\ZZ$-módulo libre (de rango $n-1$). Dado que $M$ es un grupo
    abeliano finitamente generado (siendo un cociente de $\ZZ^n$), es suficiente
    probar que $M$ es libre de torsión. La torsión significa en este caso que
    existe un vector $\vec{b} \notin \ZZ \vec{a}$, tal que para algún $c \ne 0$
    se cumple $c \vec{b} \in \ZZ \vec{a}$. Pero no es difícil ver que esto es
    imposible bajo nuestra hipótesis sobre $\vec{a}$.

    Ahora tenemos $\ZZ^n/\ZZ\vec{a} \cong \ZZ^{n-1}$. Esto significa que existen
    algunos vectores $\vec{e}_1, \ldots, \vec{e}_{n-1} \in \ZZ^n$ tales que sus
    imágenes en el cociente $\ZZ^n/\ZZ\vec{a}$ forman una base. Pero luego
    $\vec{e}_1, \ldots, \vec{e}_{n-1}, \vec{a}$ es una base de $\ZZ^n$.
  \end{solucion}
  \fi
\end{ejercicio}

\begin{ejercicio}
  Sea $d$ un entero libre de cuadrados. Consideremos el campo cúbico
  $K = \QQ (\sqrt[3]{d})$. Denotemos $\alpha = \sqrt[3]{d}$ y consideremos
  un elemento
  $$\beta = a + b\alpha + c\alpha^2, \quad a,b,c \in \QQ.$$

  \begin{enumerate}
  \item[a)] Calcule las trazas 
    $T_{K/\QQ} (\beta)$, $T_{K/\QQ} (\alpha\beta)$, $T_{K/\QQ} (\alpha^2\beta)$
    y la norma $N_{K/\QQ} (\beta)$.

  \item[b)] Si $\beta \in \O_K$, entonces las trazas y normas de arriba son
    números enteros. Use esto para concluir que
    $\O_K  \subseteq \frac{1}{3} \ZZ [\alpha]$.

  \item[c)] Use estas consideraciones para calcular el anillo de enteros $\O_K$
    y discriminante $\Delta_K$ (¡la respuesta depende de $d$!).
  \end{enumerate}

  \ifdefined\solutions
  \begin{solucion}
    Las trazas son
    \[ T_{K/\QQ} (\beta) = 3a, \quad
       T_{K/\QQ} (\alpha\beta) = 3cd, \quad
       T_{K/\QQ} (\alpha^2\beta) = 3bd, \]
    y la norma es
    $$N_{K/\QQ} (\beta) = a^3 - 3 abcd + b^3 d + c^3 d^2.$$

    Ahora si $\beta \in \O_K$, entonces las trazas y normas de arriba son
    números enteros. Las condiciones para las trazas quieren decir que existen
    $a',b',c' \in \ZZ$ tales que
    \[ a = \frac{a'}{3}, \quad b = \frac{b'}{3d}, \quad c = \frac{c'}{3d}. \]
    Lo que queremos ver es que $d \mid b'$ y $d \mid c'$, y para esto vamos a revisar
    la norma de $3\beta$ que resulta ser igual a
    $$a'^3 - \frac{3 a' b' c'}{d} + \frac{b'^3}{d^2} + \frac{c'^3}{d} \in \ZZ.$$
    Entonces,
    $$-\frac{3 a' b' c' + c'^3}{d} + \frac{b'^3}{d^2} \in \ZZ.$$
    Supongamos que para algún $p \mid d$ se tiene $p \nmid b'$. En este caso
    tomando las valuaciones $p$-ádicas de la expresión de arriba llegamos a una
    contradicción. Entonces, $d \mid b'$. Ahora sustituyendo $b = \frac{b'}{3}$,
    de la misma manera se ve que $d \mid c'$. Esto demuestra que
    $\beta \in \frac{1}{3}\ZZ[\alpha]$. Entonces,
    $$\ZZ [\alpha] \subseteq \O_K \subseteq \frac{1}{3}\ZZ[\alpha].$$
    Podemos analizar el cociente $\frac{1}{3}\ZZ[\alpha]/\ZZ[\alpha]$ para ver cuáles
    elementos enteros faltan a $\ZZ [\alpha]$; es decir, cuáles elementos entre
    $$\beta = \frac{1}{3}\,(a + b\alpha + c\alpha^2)$$
    son enteros, donde $0 \le a,b,c < 3$. Escribamos el polinomio característico
    de $\beta$ (lo calculé en PARI/GP):
    \[ x^3 - ax^2 + \frac{a^2 - bcd}{3}\,x - \frac{a^3 + b^3 d + d^2 c^3 - 3 abcd}{27}. \]

    Si $b = 0$ o $c = 0$, entonces el coeficiente de $x$ no será entero, y si
    $a = 0$, el término constante no es entero. Esto nos dice que $abc \ne 0$,
    y tenemos que analizar ocho diferentes casos. Además, notamos que $\beta$
    es entero si y solamente si $-\beta$ lo es. Módulo $3$, esto corresponde
    a pasar de $(a,b,c)$ a $(2a, 2b, 2c)$. Entonces, en realidad no son ocho
    casos diferentes, sino solamente cuatro.
    
    En cada caso la condición sobre el coeficiente de $x$ nos dice cuál es el
    resto de $d$ módulo $3$, mientras que la condición sobre el término
    constante quiere decir algo sobre $d$ módulo $3^3$.

    \begin{center}\renewcommand{\arraystretch}{1.5}
      \begin{tabular}{cccccc}
        $a$ & $b$ & $c$ & $d~(3)$ & $27\times\text{term. const.}$ & $d~(3^3)$ \\
        \hline
        $1$ & $1$ & $1$ & $1$ & $d^2 - 2d + 1$ & $1,10,19$ \\
        $1$ & $1$ & $2$ & $2$ & $8d^2 - 5d + 1$ & --- \\
        $1$ & $2$ & $1$ & $2$ & $d^2 + 2d + 1$ & $8, 17, 26$ \\
        $1$ & $2$ & $2$ & $1$ & $8d^2 - 4d + 1$ & --- \\
        \hline
      \end{tabular}
    \end{center}
    Al final, la respuesta depende de $d \pmod{9}$. Nos salió lo siguiente:
    \begin{itemize}
    \item Si $d \equiv 1 \pmod{9}$, entonces el elemento
      $$\beta = \frac{1}{3} + \frac{1}{3}\alpha + \frac{1}{3}\alpha^2$$
      es entero (el otro que nos saldrá es $2\beta$). Tenemos
      $$\O_K = \ZZ [\alpha,\beta] = \ZZ \oplus \alpha\ZZ \oplus \beta\ZZ.$$
      Para verificar el resultado, calculamos que el discriminante
      correspondiente será
      \[ \Delta_K = \det \begin{pmatrix}
        T (1) & T (\alpha) & T (\beta) \\
        T (\alpha) & T (\alpha^2) & T (\alpha\beta) \\
        T (\beta) & T (\alpha\beta) & T (\beta^2) \\
      \end{pmatrix} = \det \begin{pmatrix}
        3 & 0 & 1 \\
        0 & 0 & d \\
        1 & d & (2d+1)/3 \\
      \end{pmatrix} = -3d^2. \]
      Por otra parte,
      $\Delta (\ZZ [\alpha]) = -27 d^2$, y luego
      $[\O_K : \ZZ [\alpha]] = 3$. Podemos también escribir la expresión para
      $\Delta (\ZZ[\beta])$ y concluir que
      $[\O_K : \ZZ [\beta]] = \frac{d-1}{9}$.

    \item Si $d \equiv 8 \pmod{9}$, entonces el elemento
      $$\beta = \frac{1}{3} + \frac{2}{3}\alpha + \frac{1}{3}\alpha^2$$
      es entero. En este caso también se tiene $\Delta_K = -3d^2$ y
      $[\O_K : \ZZ [\alpha]] = 3$. Además, es posible ver que
      $[\O_K : \ZZ [\beta]] = \frac{d-8}{9}$.

    \item En el resto de los casos cuando $d \not\equiv \pm 1\pmod{9}$, no habrá
      elementos enteros adicionales.
    \end{itemize}

    Cuando $d \equiv\pm 1 \pmod{9}$, entonces $\O_K = \ZZ [\alpha,\beta]$,
    y en el caso contrario, $\O_K = \ZZ [\alpha]$.
  \end{solucion}
  \fi
\end{ejercicio}

\begin{ejercicio}
  Encuentre el anillo de enteros $\O_K$ y discriminante $\Delta_K$ para
  los campos cúbicos $\QQ (\sqrt[3]{6})$ y $\QQ (\sqrt[3]{12})$.

  \ifdefined\solutions
  \begin{solucion}
    En el caso de $\QQ (\sqrt[3]{6})$, el ejercicio anterior nos dice que
    $\O_K = \ZZ [\sqrt[3]{6}]$, y el discriminante correspondiente será
    $\Delta_K = \Delta (x^3 - 6) = -27\cdot 6^2 = -2^2\cdot 3^5$.

    Ahora para $\QQ (\sqrt[3]{12})$, el número $12$ no es libre de cuadrados,
    así que no podemos aplicar el mismo argumento. Denotemos
    $\alpha = \sqrt[3]{12}$. No es difícil notar otro elemento entero:
    $\beta = \alpha^2/2 = \sqrt[3]{18}$. Calculamos que
    \[ \Delta (\ZZ [\alpha]) = -2^4\cdot 3^5, \quad
       \Delta (\ZZ [\beta]) = -2^2\cdot 3^7. \]
    Ahora consideremos
    $$\ZZ [\alpha,\beta] = \ZZ \oplus \alpha\ZZ \oplus \beta\ZZ$$
    (note que $\alpha\beta = 6$, $\alpha^2 = 2\beta$, $\beta^2 = 6\alpha$).
    Calculamos
    \[ \Delta (\ZZ [\alpha,\beta]) = \det \begin{pmatrix}
      T (1) & T (\alpha) & T (\beta) \\
      T (\alpha) & T (\alpha^2) & T (\alpha\beta) \\
      T (\beta) & T (\alpha\beta) & T (\beta^2) \\
    \end{pmatrix} = \det \begin{pmatrix}
      3 & 0 & 0 \\
      0 & 0 & 18 \\
      0 & 18 & 0 \\
    \end{pmatrix} = -2^2\cdot 3^5. \]

    Esto implica que
    $$[\O_K : \ZZ [\alpha,\beta]] = m = 1, 2, 3, 6, 9, 18.$$
    Podemos analizar la integridad de los elementos en el cociente
    $\frac{1}{m}\ZZ[\alpha]/\ZZ[\alpha]$. Para esto consideremos los elementos
    $$\gamma = \frac{a}{m} + \frac{b}{m}\alpha + \frac{c}{m}\beta,$$
    donde $0 \le a,b,c < m$. Basta considerar $m = 18$. En teoría, todo esto se
    puede hacer a mano, hasta cierto punto. Por ejemplo, calculamos
    $$T_{K/\QQ} (\gamma) = \frac{3a}{m} = \frac{a}{6},$$
    de donde $a = 0,6,12$, lo que quita una parte del trabajo. Podemos
    sustituir estos valores de $a$ y, por ejemplo, analizar $N_{K/\QQ} (\gamma)$,
    pero mejor hacerlo con la computadora.

    \begin{framed}\small
\begin{verbatim}
? m = 18;
? nrm = norm (1/m*Mod(a + b*x + c*x^2/2, x^3-12))
% = 1/5832*a^3 - 1/324*c*b*a + (1/486*b^3 + 1/324*c^3)

{ for (a=0,m-1, for (b=0,m-1, for (c=0,m-1,
        if (denominator(eval(nrm)) == 1,
          print ([a,b,c])
        )
  )))
};

[0, 0, 0]
/* ¡y nada más! */
\end{verbatim}
    \end{framed}
    Esto implica que $\O_K = \ZZ [\alpha,\beta]$. Ya calculamos
    $\Delta_K$. Notamos que para $\QQ (\sqrt[3]{6})$ y $\QQ (\sqrt[3]{12})$ nos
    salió el mismo discriminante.
  \end{solucion}
  \fi
\end{ejercicio}

\begin{ejercicio}
  Consideremos el campo cúbico $K = \QQ (\sqrt[3]{17})$.

  \begin{enumerate}
  \item[a)] Calcule el anillo de enteros $\O_K$ y discriminante $\Delta_K$.
  \item[b)] Describa las factorizaciones de primos racionales $p \in \ZZ$ en
    $\O_K$.
  \item[c)] Describa los ideales primos $\mathfrak{p} \subset \O_K$
    tales que $N_{K/\QQ} (\mathfrak{p}) \le 10$.
  \item[d)] Describa todos los ideales $I \subseteq \O_K$
    tales que $N_{K/\QQ} (I) \le 10$.
  \end{enumerate}

  \ifdefined\solutions
  \begin{solucion}
    Denotando $\sqrt[3]{17}$ por $\alpha$, el ejercicio 3 nos dice que
    \[ \O_K = \ZZ [\alpha,\beta], \quad
       \beta = \frac{1}{3} + \frac{2}{3}\,\alpha + \frac{1}{3}\,\alpha^2. \]
    Además, allí escribimos la fórmula curiosa
    $[\O_K : \ZZ [\beta]] = \frac{d-8}{9}$, y para $d = 17$ tenemos suerte
    y $\O_K = \ZZ [\beta]$. Por otra parte, $[\O_K : \ZZ [\alpha]] = 3$.
    Esto significa que podemos aplicar el Kummer--Dedekind al polinomio mínimo
    de $\beta$:
    $$x^3 - x^2 - 11x - 12.$$
    Sin embargo, será más fácil considerar el polinomio $x^3 - 17$ para todos
    los primos excepto $p = 3$. En ese caso excepcional tenemos
    $$x^3 - x^2 - 11x - 12 \equiv x\,(x + 1)^2 \pmod{3},$$
    lo que nos da la factorización
    $$3\O_K = \mathfrak{p}_3 \, \mathfrak{p}_3'^2 = (3, \beta)\,(3, 1+\beta)^2,$$
    donde $N_{K/\QQ} (\mathfrak{p}_3) = N_{K/\QQ} (\mathfrak{p}_3') = 3$.
    Para los primos distintos de $3$, las factorizaciones son las siguientes.

    \begin{itemize}
    \item $17\O_K = \mathfrak{p}^3$, donde $\mathfrak{p} = \sqrt[3]{17}\O_K$.
      Tenemos $N_{K/\QQ} (\mathfrak{p}) = 17$.

    \item Si $p \equiv 2 \pmod{3}$ y $p \ne 17$, entonces
      $p\O_K = \mathfrak{p} \, \mathfrak{p}'$, lo que viene de la
      factorización
      $$x^3 - 17 \equiv (x-a) \times \text{polinomio cuadrático} \pmod{p}.$$
      Aquí $N_{K/\QQ} (\mathfrak{p}) = p$ y $N_{K/\QQ} (\mathfrak{p}') = p^2$.

    \item Para $p \equiv 1 \pmod{3}$ hay dos opciones.
      Si $17$ no es un cubo módulo $p$, entonces $\mathfrak{p} = p\O_K$ es un
      ideal primo y $N_{K/\QQ} (\mathfrak{p}) = p^3$.
      Si $17$ es un cubo módulo $p$, entonces
      $p\O_K = \mathfrak{p} \, \mathfrak{p}' \, \mathfrak{p}''$, lo que viene
      de la factorización
      $$x^3 - 17 \equiv (x - a_1)\,(x - a_2)\,(x - a_3) \pmod{p}.$$
    \end{itemize}

    Ahora si nos interesan los ideales primos de norma
    $N_{K/\QQ} (\mathfrak{p}) \le 10$, esto en particular implica que
    $\mathfrak{p} \mid p$, donde $p < 10$. Para obtener todos los ideales de
    norma $N_{K/\QQ} (I) \le 10$, hay que multiplicar los ideales primos
    correspondientes (usando la unicidad de descomposición de ideales en ideales
    primos).

    Los primos $p \equiv 2 \pmod{3}$ que nos interesan son $p = 2, 5$, y las
    factorizaciones de $x^3 - 17$ son
    \begin{align*}
      p=2\colon & (x+1)\,(x^2 + x + 1),\\
      p=5\colon & (x+2)\,(x^2 + 3x + 4).
    \end{align*}
    El primo $p = 7$ es inerte, y el ideal primo correspondiente
    $\mathfrak{p} = 7\O_K$ tiene norma $7^3$ y no nos interesa. Ahora los
    ideales primos de norma $< 10$ que salen de la lista de arriba son
    nada más los siguientes:
    \begin{align*}
      N=2\colon & \mathfrak{p}_2 = (2, 1+\alpha),\\
      N=3\colon & \mathfrak{p}_3 = (3, \beta), ~ \mathfrak{p}_3' = (3, 1+\beta),\\
      N=4\colon & \mathfrak{p}_2' = (2, 1 + \alpha + \alpha^2),\\
      N=5\colon & \mathfrak{p}_5 = (5, 2 + \alpha).
    \end{align*}
    Si nos interesan todos los ideales de norma $\le 10$, hay que
    considerar los productos:
    \begin{align*}
      N=1\colon & \O_K, \\
      N=2\colon & \mathfrak{p}_2,\\
      N=3\colon & \mathfrak{p}_3, ~ \mathfrak{p}_3',\\
      N=4\colon & \mathfrak{p}_2^2, ~ \mathfrak{p}_2',\\
      N=5\colon & \mathfrak{p}_5,\\
      N=6\colon & \mathfrak{p}_2\,\mathfrak{p}_3, ~ \mathfrak{p}_2\,\mathfrak{p}_3',\\
      N=8\colon & \mathfrak{p}_2^3, ~ \mathfrak{p}_2\,\mathfrak{p}_2',\\
      N=9\colon & \mathfrak{p}_3^2, ~ \mathfrak{p}_3\,\mathfrak{p}_3', ~ \mathfrak{p}_3'^2,\\
      N=10\colon & \mathfrak{p}_2\,\mathfrak{p}_5,
    \end{align*}

    En total, nos salieron $15$ ideales. Podemos comprobarlo con PARI/GP.
    Allí la función \texttt{ideallist($K$,$N$)} devuelve los ideales en $\O_K$
    de norma $\le N$ como una lista separada por normas
    \begin{framed}\small
\begin{verbatim}
? L = ideallist (nfinit(x^3-17),10);
? vector (#L, i, #L[i])
% = [1, 1, 2, 2, 1, 2, 0, 2, 3, 1]
? vecsum(%)
% = 15
\end{verbatim}
    \end{framed}

    Las consideraciones similares demuestran que para cualquier campo de números
    $K/\QQ$ y $N$ fijo hay un número finito de ideales
    $I \subseteq \O_K$ con la norma $N_{K/\QQ} (I) \le N$.
  \end{solucion}
  \fi
\end{ejercicio}

\end{document}

\documentclass{article}

\usepackage[utf8]{inputenc}
\usepackage[spanish]{babel}

\usepackage{amsmath,amssymb,amsthm}

\usepackage{tikz-cd}
\usetikzlibrary{babel}
\usetikzlibrary{calc}

\newcounter{tarea}
\setcounter{tarea}{6}
\theoremstyle{definition}
\newtheorem{ejercicio}{Ejercicio}[tarea]

\newenvironment{solucion}{\begin{proof}[Solución]}{\end{proof}}

\usepackage{framed}

\usepackage{mathspec}

\setmainfont{PT Serif}
\setsansfont{Montserrat}
\setmonofont{PT Mono}

\DeclareMathOperator{\Gal}{Gal}

\newcommand{\FF}{\mathbb{F}}
\newcommand{\ZZ}{\mathbb{Z}}
\newcommand{\QQ}{\mathbb{Q}}
\renewcommand{\O}{\mathcal{O}}
\newcommand{\CC}{\mathbb{C}}
\renewcommand{\Re}{\operatorname{Re}}

\newcommand{\legendre}[2]{\left(\frac{#1}{#2}\right)}

\title{Teoría de números algebraicos\\Tarea 6}
\author{Alexey Beshenov (alexey.beshenov@cimat.mx)}
\date{8 de octubre de 2020}

% \def\solutions{true}

\begin{document}

{\sffamily\bfseries\maketitle}

\ifdefined\solutions
\else
\thispagestyle{empty}
\fi

\vspace{1em}

\begin{ejercicio}
  Para un campo cuadrático $\QQ (\sqrt{d})$ encuentre $n$ tal que
  $\QQ (\sqrt{d}) \subset \QQ (\zeta_n)$.

  \ifdefined\solutions
  \begin{solucion}
    Tenemos $\sqrt{-1} \in \QQ (\zeta_4)$ y $\sqrt{\pm 2} \in \QQ (\zeta_8)$.
    Si $p$ es un primo impar, sabemos que $\sqrt{p^*} \in \QQ (\zeta_p)$, donde
    $p^* = (-1)^{\frac{p-1}{2}}\,p$ (véase la tarea 3). Si $p \equiv 3\pmod{4}$,
    de todos modos $\sqrt{p} \in \QQ (\zeta_{4p})$. Ahora si $d$ es un entero
    libre de cuadrados, podemos factorizarlo como $\pm 2^e p_1 \cdots p_s$,
    donde $e = 0, 1$ y los $p_i$ son impares, y las consideraciones de arriba
    nos dicen que $\QQ (\sqrt{d})$ es un subcampo de
    $\QQ (\zeta_{8 p_1\cdots p_s})$.

    Este es un caso particular del teorema de Kronecker--Weber que afirma que
    cualquier campo abeliano $K/\QQ$ se encaja en un campo ciclotómico.
  \end{solucion}
  \fi
\end{ejercicio}

\begin{ejercicio}
  Para $K = \QQ (\sqrt[4]{2},i)$ calcule el grupo $\Gal (K/\QQ)$ y describa
  los subcampos de $K$.

  \ifdefined\solutions
  \begin{solucion}
    Primero hay que ver que $K/\QQ$ es una extensión de Galois. Esto se sigue
    del hecho de que $K$ es el campo de descomposición del polinomio
    $$x^4 - 2 = (x - \sqrt[4]{2})\,(x + \sqrt[4]{2})\,(x - i\sqrt[4]{2})\,(x + i\sqrt[4]{2}).$$
    Se trata del compositum de campos $\QQ (\sqrt[4]{2})$ y $\QQ (i)$,
    y como una base de $K$ sobre $\QQ$ podemos tomar
    \[ 1, \, 2^{1/4}, \, 2^{1/2}, \, 2^{3/4}, \,
       i, \, i2^{1/4}, \, i2^{1/2}, \, i2^{3/4}. \]

    Hay dos automorfismos evidentes:
    $\sigma\colon \sqrt[4]{2} \mapsto i\sqrt[4]{2}$ de orden $4$
    y $\tau\colon i \mapsto -i$ de orden $2$
    (la conjugación compleja). Calculamos que
    $\sigma\tau = \tau\sigma^3 \ne \tau\sigma$. De estas consideraciones se ve
    que $\sigma$ y $\tau$ generan un grupo de $8$ elementos que será isomorfo al
    grupo diédrico $D_4$ (también conocido como $D_8$).  Aquí está el diagrama
    de subgrupos. Hay $5$ subgrupos de índice $4$ y $3$ subgrupos de índice $2$.

    \begin{center}
      \begin{tikzpicture}
        \node (D4) {$G$};
        \node (s41) at (-3,2) {$\langle\tau,\sigma^2\rangle$};
        \node (s42) at (0,2) {$\langle\sigma\rangle$};
        \node (s43) at (+3,2) {$\langle\tau\sigma,\sigma^2\rangle$};

        \node (s21) at (-4,4) {$\langle\tau\rangle$};
        \node (s22) at (-2,4) {$\sigma\langle\tau\rangle\sigma^{-1}$};
        \node (s23) at (0,4) {$\langle\sigma^2\rangle$};
        \node (s24) at (+2,4) {$\langle\tau\sigma\rangle$};
        \node (s25) at (+4,4) {$\tau\langle\tau\sigma\rangle\tau^{-1}$};

        \draw (s41.south) -- (D4.north west);
        \draw (s42.south) -- (D4.north);
        \draw (s43.south) -- (D4.north east);

        \draw (s23.south west) -- ($(s41.north)+(0.2,0)$);
        \draw (s23.south) -- (s42.north);
        \draw (s23.south east) -- ($(s43.north)-(0.2,0)$);
        \draw (s21.south) -- ($(s41.north)-(0.2,0)$);
        \draw ($(s22.south)-(0.3,0)$) -- (s41.north);
        \draw ($(s24.south)+(0.3,0)$) -- (s43.north);
        \draw (s25.south) -- ($(s43.north)+(0.2,0)$);

        \node (s1) at (0,6) {$\{ 1 \}$};
        \draw ($(s1.south)-(0.2,0)$) -- (s21.north);
        \draw ($(s1.south)-(0.1,0)$) -- (s22.north);
        \draw (s1.south) -- (s23.north);
        \draw ($(s1.south)+(0.1,0)$) -- (s24.north);
        \draw ($(s1.south)+(0.2,0)$) -- (s25.north);
      \end{tikzpicture}
    \end{center}

    Es un poco trabajoso, pero no es muy difícil calcular uno por uno los
    subcampos fijos correspondientes.

    \begin{center}
      \begin{tikzpicture}
        \node (D4) {$G$};
        \node (s41) at (-3,2) {$\QQ (\sqrt{2})$};
        \node (s42) at (0,2) {$\QQ (i)$};
        \node (s43) at (+3,2) {$\QQ (\sqrt{-2})$};

        \node (s21) at (-4,4) {$\QQ (\sqrt[4]{2})$};
        \node (s22) at (-2,4) {$\QQ (i\sqrt[4]{2})$};
        \node (s23) at (0,4) {$\QQ (\zeta_8)$};
        \node (s24) at (+2,4) {$\QQ (\zeta_8 \sqrt[4]{2})$};
        \node (s25) at (+4,4) {$\QQ (\zeta_8^3 \sqrt[4]{2})$};

        \draw (s41.south) -- (D4.north west);
        \draw (s42.south) -- (D4.north);
        \draw (s43.south) -- (D4.north east);

        \draw (s23.south west) -- ($(s41.north)+(0.2,0)$);
        \draw (s23.south) -- (s42.north);
        \draw (s23.south east) -- ($(s43.north)-(0.2,0)$);
        \draw (s21.south) -- ($(s41.north)-(0.2,0)$);
        \draw ($(s22.south)-(0.3,0)$) -- (s41.north);
        \draw ($(s24.south)+(0.3,0)$) -- (s43.north);
        \draw (s25.south) -- ($(s43.north)+(0.2,0)$);

        \node (s1) at (0,6) {$\QQ (\sqrt[4]{2},i)$};
        \draw ($(s1.south)-(0.2,0)$) -- (s21.north);
        \draw ($(s1.south)-(0.1,0)$) -- (s22.north);
        \draw (s1.south) -- (s23.north);
        \draw ($(s1.south)+(0.1,0)$) -- (s24.north);
        \draw ($(s1.south)+(0.2,0)$) -- (s25.north);

        \node at ($(s21)!.5!(s22)$) {$\cong$};
        \node at ($(s24)!.5!(s25)$) {$\cong$};
      \end{tikzpicture}
    \end{center}
    (Para verificar los cálculos, note que
    $\zeta_8 = \frac{\sqrt{2}}{2} + i\frac{\sqrt{2}}{2}$.)
  \end{solucion}
  \fi
\end{ejercicio}

\begin{ejercicio}
  Consideremos el campo bicuadrático $K = \QQ (\sqrt{-3}, \sqrt{5})$.

  \begin{enumerate}
  \item[1)] Describa cómo los primos racionales se factorizan en $\O_K$.

  \item[2)] Calcule la densidad de primos que corresponden a cada tipo de
    descomposición.
  \end{enumerate}

  \ifdefined\solutions
  \begin{solucion}
    En $K$ tenemos tres subcampos cuadráticos:
    \[ F_1 = \QQ (\sqrt{-3}), \quad
    F_2 = \QQ (\sqrt{5}), \quad
    F_3 = \QQ (\sqrt{-15}). \]

    La descomposición en $F_i$ se determina por los símbolos de Legendre
    correspondientes. Todo depende del resto de $p$ módulo $15$:
    \begin{align*}
      \legendre{-3}{p} & = \begin{cases}
        +1, & p\equiv 1 \pmod{3},\\
        -1, & p\equiv 2 \pmod{3};
      \end{cases}\\
      \legendre{5}{p} & = \begin{cases}
        +1, & p\equiv 1,4 \pmod{5},\\
        -1, & p\equiv 2,3 \pmod{5};
      \end{cases}\\
      \legendre{-15}{p} & = \begin{cases}
        +1, & p\equiv 1,2,4,8 \pmod{15},\\
        -1, & p\equiv 7,11,13,14 \pmod{15}.
      \end{cases}
    \end{align*}

    La extensión $K/\QQ$ es de Galois, así que $e_p\,f_p\,g_p = 4$. Primero
    podemos ver qué pasa con los primos ramificados. Tenemos $K = F_1 F_2$
    y los disciminantes son $\Delta_{F_1} = -3$, $\Delta_{F_2} = 5$, así que
    $\Delta_K = 3^2\cdot 5^2$, y los primos ramificados en $K$ son $3$ y $5$.

    Si $p = 3$, entonces $p$ se ramifica en $F_1$ y es inerte en $F_2$. Esto
    quiere decir que $2 \mid e_3$ y $2 \mid f_3$, así que el tipo de
    descomposición será $\mathfrak{p}^2$.

    Si $p = 5$, entonces $p$ se ramifica en $F_2$ y es inerte en $F_1$, así
    que el tipo de descomposición es $\mathfrak{p}^2$.

    Para los primos no ramificados se cumple $f_p\,g_p = 4$.

    Si $p$ se escinde en uno de los subcampos cuadráticos, pero es inerte en
    otro, entonces $2 \mid f_p$ y $g_p \ge 2$, así que el tipo de
    descomposición será $\mathfrak{p}_1\,\mathfrak{p}_2$, donde $f = f' = 2$.
    Los primos correspondientes son los siguientes.

    \begin{itemize}
    \item Si $p \equiv 2, 8 \pmod{15}$, entonces $p$ es inerte en $F_1$ y $F_2$,
      pero se escinde en $F_3$.

    \item Si $p \equiv 7, 13 \pmod{15}$, entonces $p$ se escinde en $F_1$, pero
      es inerte en $F_2$ y $F_3$.

    \item Si $p \equiv 11, 14 \pmod{15}$, entonces $p$ es inerte en $F_1$ y
      $F_3$, pero se escinde en $F_2$.
    \end{itemize}

    Nos queda el caso de $p \equiv 1,4 \pmod{15}$ cuando $p$ se escinde en los
    tres subcampos cuadráticos. Nos gustaría probar que en este caso la
    factorización tiene forma
    $p\O_K = \mathfrak{p}_1\,\mathfrak{p}_2\,\mathfrak{p}_3\,\mathfrak{p}_4$.
    Esto es equivalente a probar que para todo ideal primo
    $\mathfrak{p} \subset \O_K$ tal que $\mathfrak{p} \mid p$ se tiene
    $[\O_K/\mathfrak{p} : \FF_p] = 1$. En otras palabres, hay que ver que para
    todo $\alpha \in \O_K$ existe un entero racional $a \in \ZZ$ tal que
    $\alpha \equiv a \pmod{\mathfrak{p}}$. Consideremos los ideales
    $\mathfrak{p}_1 = \mathfrak{p} \cap \O_{F_1}$ y
    $\mathfrak{p}_2 = \mathfrak{p} \cap \O_{F_2}$. Estos son ideales primos en
    $\O_{F_1}$ y $\O_{F_2}$ respectivamente. Por nuestra hipótesis, se tiene
    $[\O_{F_1}/\mathfrak{p}_1 : \FF_p] = 1$ y $[\O_{F_2}/\mathfrak{p}_2 : \FF_p]
    = 1$. En otras palabras, cualquier elemento $\alpha \in \O_{F_i}$ es
    congruente a algún entero racional módulo $\mathfrak{p}_i$. Ahora
    $K = F_1 F_2$, y todo elemento $\alpha \in \O_K$ tiene forma
    $\sum_i \alpha_i \beta_i$, donde $\alpha_i \in \O_{F_1}$ y
    $\beta_i \in \O_{F_2}$. Dado que
    $\mathfrak{p} \mid \mathfrak{p}_1 \O_K$ y
    $\mathfrak{p} \mid \mathfrak{p}_2 \O_K$, sabemos que cada $\alpha_i$ y
    $\beta_i$ se reduce a un entero racional módulo $\mathfrak{p}$, y por ende
    $\sum_i \alpha_i \beta_i$ cumple con la misma propiedad. Esto termina la
    prueba.

    Notamos que el argumento que acabamos de ver se generaliza al siguiente
    resultado. Si un primo racional $p \in \ZZ$ se factoriza en $[K : \QQ]$
    ideales primos en $\O_K$, entonces se dice que $p$
    \textbf{se escinde completamente} en $K$. Ahora si $K$ es el compositum de
    $F_1$ y $F_2$, y $p$ se escinde completamente en $F_1$ y $F_2$, entonces este
    también se escinde completamente en $K$.

    Según el teorema de Dirichlet sobre primos en progresiones aritméticas,
    las densidades son entonces $\frac{3}{4}$ para
    $\mathfrak{p}_1\,\mathfrak{p}_2$ y $\frac{1}{4}$ para
    $\mathfrak{p}_1 \, \mathfrak{p}_2 \, \mathfrak{p}_3 \, \mathfrak{p}_4$.
  \end{solucion}
  \fi
\end{ejercicio}

\begin{ejercicio}
  Sea $p$ un número primo y $\chi$ el carácter de Dirichlet de orden $2$ mód
  $p$, definido por el símbolo de Legendre $\chi (n) = \legendre{n}{p}$.

  \begin{enumerate}
  \item[1)] Demuestre que
    $\exp (g (\chi)\,L (1,\chi)) = \prod_n (1 - \zeta_p^n)\,\prod_r (1 - \zeta_p^r)^{-1}$,
    donde $g (\chi) = \sum_{1 \le a \le p-1} \chi (a)\,\zeta_p^a$,
    y los productos son sobre los no-residuos y residuos cuadráticos mód $p$
    respectivamente.

  \item[2)] Use la parte anterior para calcular $L (1,\chi)$, donde $\chi$ es el
    carácter de orden $2$ mód $5$.
    (Para el valor numérico en PARI/GP, basta digitar \texttt{lfun(5,1)})
  \end{enumerate}

  \ifdefined\solutions
  \begin{solucion}
    Denotemos
    $$P = \prod_n (1 - \zeta_p^n) \, \prod_r (1 - \zeta_p^r)^{-1}.$$
    Tenemos
    $$\log P = \sum_n \log (1 - \zeta_p^n) - \sum_r \log (1 - \zeta_p^r) = \sum_a -\chi (a) \, \log (1 - \zeta_p^a).$$
    La serie $-\log (1-z) = \sum_{n\ge 1} \frac{z^n}{n}$ converge para $|z| < 1$
    y también converge para $z = \zeta_p^r$ (el teorema de Abel). Podemos
    escribir
    $$\log P = \sum_{m\ge 1} \frac{1}{m}\,\sum_a \chi (a)\,\zeta_p^{am}.$$
    Ahora ocupamos la identidad para las sumas cuadráticas de Gauss
    $$\sum_a \chi (a)\,\zeta_p^{am} = g (\chi)\,\chi (m).$$
    Entonces,
    $$\log P = \sum_a \chi (a)\,\sum_{m\ge 1} \frac{\chi (m)}{m} = g (\chi)\,L (1,\chi).$$
    Esto establece la identidad deseada $\exp (g (\chi)\,L (1,\chi)) = P$.

    En particular, si $p = 5$, calculamos
    $$g (\chi) = \zeta_5 - \zeta_5^2 - \zeta_5^3 + \zeta_5^4 = \sqrt{5}$$
    y
    \[ P = \frac{(1 - \zeta_5^2)\,(1 - \zeta_5^3)}{(1 - \zeta_5)\,(1 - \zeta_5^4)}
         = 1 - \zeta_5^2 - \zeta_5^3 = \frac{3+\sqrt{5}}{2}. \]
    Ahora
    $$L (1,\chi) = \frac{1}{\sqrt{5}}\,\log \frac{3+\sqrt{5}}{2}.$$
    Lo podemos confirmar con PARI/GP:
    \begin{framed}
\begin{verbatim}
? 1/sqrt(5) * log ((3 + sqrt(5))/2)
% = 0.43040894096400403888943323295060542543
? lfun (5,1)
% = 0.43040894096400403888943323295060542542
\end{verbatim}
    \end{framed}
    De manera similar, si $p = 3$, entonces $g (\chi) = \zeta_3 - \zeta_3^2 = \sqrt{-3}$
    y $P = \frac{1 - \zeta_3^2}{1 - \zeta_3} = 1 + \zeta_3 = \zeta_6$.
    Así nos quedamos con la fórmula
    $$\exp (2\pi i/6) = \exp (\sqrt{-3} \, L (1,\chi)).$$
    Entonces (módulo $2\pi i\ZZ$) se cumple
    $\frac{2\pi i}{6} = i\sqrt{3} \, L (1,\chi)$.
    De aquí $L (1,\chi) = \frac{\pi}{3\sqrt{3}}$.
    \begin{framed}
\begin{verbatim}
? lfun (-3,1)
% = 0.60459978807807261686469275254738524409
? Pi/(3*sqrt(3))
% = 0.60459978807807261686469275254738524409
\end{verbatim}
    \end{framed}
  \end{solucion}
  \fi
\end{ejercicio}

\begin{ejercicio}
  Consideremos funciones $f,g\colon \ZZ_{>1} \to \CC$ y las series de
  Dirichlet correspondientes
  $F (s) = \sum_{n\ge 1} \frac{f(n)}{n^s}$ y
  $G (s) = \sum_{n\ge 1} \frac{g(n)}{n^s}$.

  \begin{enumerate}
  \item[1)] Demuestre que cuando las series convergen absolutamente en $s$,
    se tiene
    $F (s)\cdot G (s) = \sum_{n\ge 1} \frac{(f\ast g) (n)}{n^s}$,
    donde
    $(f\ast g) (n) = \sum_{d\mid n} f (d) \, g \left(\frac{n}{d}\right)$.

  \item[2)] Sean $\mu (n)$ la función de Möbius,
    $\tau (n)$ el número de divisores,
    $\sigma (n) = \sum_{d \mid n} d$ la suma de divisores,
    y $\phi (n)$ la función de Euler. Demuestre que
  \end{enumerate}
  \[ \sum_{n\ge 1} \frac{\mu (n)}{n^s} = \frac{1}{\zeta (s)}, ~
     \sum_{n\ge 1} \frac{\tau (n)}{n^s} = \zeta (s)^2, ~
     \sum_{n\ge 1} \frac{\sigma (n)}{n^s} = \zeta (s)\,\zeta (s-1), ~
     \sum_{n\ge 1} \frac{\phi (n)}{n^s} = \frac{\zeta (s-1)}{\zeta (s)}. \]

  \ifdefined\solutions
  \begin{solucion}
    Usando la convergencia absoluta, podemos cambiar el orden de términos
    y escribir
    \[ F (s)\cdot G (s)
    = \Bigl(\sum_{m\ge 1} \frac{f (m)}{m^s}\Bigr) \cdot
    \Bigl(\sum_{n\ge 1} \frac{g (n)}{n^s}\Bigr)
    = \sum_{m,n\ge 1} \frac{f (m) \, g (n)}{(mn)^s}
    = \sum_{n\ge 1} \sum_{d \mid n} \frac{f (d) \, g (n/d)}{n^s}. \]
    Además, la serie que acabamos de obtener también converge absolutamente
    en $s$, dado que
    \[ \sum_{n\ge 1} \frac{|(f \ast g) (n)|}{|n^s|} \le
    \sum_{n\ge 1} \sum_{d\mid n} \frac{|f (d)|\cdot |g (n/d)|}{|n^s|} =
    \Bigl(\sum_{m\ge 1} \frac{|f (s)|}{|n^s|}\Bigr)\cdot
    \Bigl(\sum_{n\ge 1} \frac{|g (s)|}{|n^s|}\Bigr) < \infty. \]

    La serie $\zeta (s) = \sum_{n\ge 1} \frac{1}{n^s}$ converge
    absolutamente para
    $\Re s > 1$, así que la serie $\sum_{n\ge 1} \frac{\mu (n)}{n^s}$
    converge absolutamente para $\Re s > 1$.

    Denotemos por $1$ la función constante $n \mapsto 1$. Tenemos
    \[ (\mu\ast 1) (n) = \sum_{d\mid n} \mu (d) = \begin{cases}
      1, & \text{si }n = 1,\\
      0, & \text{si }n > 1.
    \end{cases}. \]
    De hecho, escribiendo $n = p_1^{e_1}\cdots p_s^{e_s}$, para $n > 1$, tenemos
    $$\sum_{d\mid n} \mu (d) = \sum_{(e_1,\ldots,e_n)} \mu (p_1^{e_1}\cdots p_s^{e_s}),$$
    donde $e_i = 0$ o $1$. Luego,
    \[ \sum_{d\mid n} \mu (d) = 1 - s + {s \choose 2} - {s \choose 3} + \cdots + (-1)^s
    = (1-1)^s = 0. \]

    Para la segunda identidad, calculamos
    $$(1 \ast 1) (n) = \sum_{d\mid n} 1 = \tau (n).$$
    Entonces, $\zeta (s)^2 = \sum_{n\ge 1} \frac{\tau (n)}{n^s}$, y esta serie
    converge absolutamente para ${\Re s > 1}$.

    Para la tercera identidad, notamos que
    $$\zeta (s-1) = \sum_{n\ge 1} \frac{1}{n^{s-1}} = \sum_{n\ge 1} \frac{n}{n^s},$$
    y esta serie converge absolutamente para $\Re s > 2$. Calculamos entonces
    $$(1\ast id) (n)= \sum_{d\mid n} d = \sigma (n).$$
    Entonces, $\zeta (s)\,\zeta (s-1) = \sum_{n\ge 1} \frac{\sigma (n)}{n^s}$,
    y esta serie converge absolutamente para $\Re s > 2$.

    En fin, para la última identidad, calculamos
    \[ \frac{\zeta (s-1)}{\zeta (s)}
    = \Bigl(\sum_{m\ge 1} \frac{m}{m^s}\Bigr)\cdot
    \Bigl(\sum_{n\ge 1} \frac{\mu (n)}{n^s}\Bigr). \]
    Ahora
    $$(id\ast\mu) (n) = \sum_{d\mid n} \mu (d)\,\frac{n}{d} = \phi (n).$$
    Esto se sigue de la fórmula $\sum_{d\mid n} \phi (d) = n$ y la inversión
    de Möbius. Entonces, podemos concluir que
    $\frac{\zeta (s-1)}{\zeta (s)} = \sum_{n\ge 1} \frac{\phi (n)}{n^s}$,
    y esta serie converge absolutamente para $\Re s > 2$.
  \end{solucion}
  \fi
\end{ejercicio}

\end{document}

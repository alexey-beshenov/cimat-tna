\documentclass{article}

\usepackage[utf8]{inputenc}
\usepackage[spanish]{babel}

\usepackage{amsmath,amssymb,amsthm}

\usepackage{tikz-cd}
\usetikzlibrary{babel}
\usetikzlibrary{calc}

\newcounter{tarea}
\setcounter{tarea}{8}
\theoremstyle{definition}
\newtheorem{ejercicio}{Ejercicio}[tarea]

\newenvironment{solucion}{\begin{proof}[Solución]}{\end{proof}}

\usepackage{framed}

\usepackage{mathspec}

\setmainfont{PT Serif}
\setsansfont{Montserrat}
\setmonofont{PT Mono}

\DeclareMathOperator{\Cl}{Cl}
\DeclareMathOperator{\covol}{covol}
\DeclareMathOperator{\vol}{vol}

\newcommand{\FF}{\mathbb{F}}
\newcommand{\ZZ}{\mathbb{Z}}
\newcommand{\QQ}{\mathbb{Q}}
\newcommand{\RR}{\mathbb{R}}
\renewcommand{\O}{\mathcal{O}}
\newcommand{\CC}{\mathbb{C}}
\renewcommand{\Re}{\operatorname{Re}}

\newcommand{\legendre}[2]{\left(\frac{#1}{#2}\right)}

\title{Teoría de números algebraicos\\Tarea 8}
\author{Alexey Beshenov (alexey.beshenov@cimat.mx)}
\date{28 de octubre de 2020}

% \def\solutions{true}

\begin{document}

{\sffamily\bfseries\maketitle}

\ifdefined\solutions
\else
\thispagestyle{empty}
\fi

\vspace{1em}

\begin{ejercicio}
  Demuestre que si $X$ es un conjunto convexo simétrico compacto tal que
  $\vol X = 2^n\cdot \covol \Lambda$, entonces $X \cap \Lambda \ne \{ 0 \}$.
\end{ejercicio}

\begin{ejercicio}
  Para $t > 0$ consideremos el conjunto convexo simétrico
  $$X_t = \{ (x_\tau)_\tau \in K_\RR \mid |x_\tau| < t\text{ para todo }\tau \}.$$
  Calcule que
  $$\vol (X_t) = 2^{r_1}\,(2\pi)^{r_2}\,t^n.$$
\end{ejercicio}

\begin{ejercicio}
  Supongamos que $d = p_1\cdots p_s$, donde $s > 1$ y los $p_i$ son diferentes
  primos y consideremos el campo cuadrático imaginario $K = \QQ (\sqrt{-d})$.
  Demuestre que los ideales correspondientes
  $\mathfrak{p}_1,\ldots,\mathfrak{p}_s \subset \O_K$ generan un subgrupo en
  $\Cl (K)$ isomorfo a $(\ZZ/2\ZZ)^{s-1}$.
\end{ejercicio}

\begin{ejercicio}
  Calcule los grupos de clases de campos
  \[ \QQ (\sqrt{-110}), \quad
     \QQ (\sqrt{-127}), \quad
     \QQ (\sqrt{33}), \quad
     \QQ (\sqrt[3]{19}), \quad
     \QQ (\sqrt{-5}, \sqrt{-11}). \]
\end{ejercicio}

\begin{ejercicio}
  Sea $K/\QQ$ un campo de números. Demuestre que para cualquier ideal
  $I \subset \O_K$ existe una extensión finita $L/K$ tal que el ideal
  correspondiente $I\,\O_L$ es principal.
\end{ejercicio}

\begin{ejercicio}
  Consideremos una sucesión exacta corta de $R$-módulos
  $$0 \to M' \xrightarrow{i} M \xrightarrow{p} M'' \to 0$$

  \begin{enumerate}
  \item[1)] Demuestre que si $M''$ es un $R$-módulo libre, entonces
    el homomorfismo $p$ admite una \textbf{sección} $s\colon M'' \to M$
    tal que $p\circ s = id_{M''}$.

  \item[2)] Demuestre si existe una sección $s$ como arriba, entonces
    $M'\oplus M'' \cong M$.
  \end{enumerate}
\end{ejercicio}

\end{document}

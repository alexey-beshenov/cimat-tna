\documentclass{article}

\usepackage[utf8]{inputenc}
\usepackage[spanish]{babel}

\usepackage{amsmath,amssymb,amsthm}

\usepackage{tikz-cd}
\usetikzlibrary{babel}

\newcounter{tarea}
\setcounter{tarea}{2}
\theoremstyle{definition}
\newtheorem{ejercicio}{Ejercicio}[tarea]
\newtheorem*{ejercicio-adicional}{Ejercicio adicional}

\newenvironment{solucion}{\begin{proof}[Solución]}{\end{proof}}

\newcommand{\ZZ}{\mathbb{Z}}
\newcommand{\QQ}{\mathbb{Q}}
\newcommand{\FF}{\mathbb{F}}

\DeclareMathOperator{\Pic}{Pic}

\renewcommand{\gcd}{\operatorname{mcd}}
\renewcommand{\O}{\mathcal{O}}

\usepackage{mathspec}

\setmainfont{PT Serif}
\setsansfont{Montserrat}
\setmonofont{PT Mono}

\usepackage[perpage,symbol]{footmisc}
\renewcommand{\thefootnote}{\ifcase\value{footnote}\or(*)\or
(**)\or(***)\or(****)\or(\#)\or(\#\#)\or(\#\#\#)\or(\#\#\#\#)\or($\infty$)\fi}

\title{Teoría de números algebraicos\\Tarea 2}
\author{Alexey Beshenov (alexey.beshenov@cimat.mx)}
\date{26 de agosto de 2020}

% \def\solutions{true}

\begin{document}

{\sffamily\bfseries\maketitle}

\noindent Fecha límite: viernes, 4 de septiembre.

\ifdefined\solutions
\else
\thispagestyle{empty}
\fi

\begin{ejercicio}
  Demuestre que para $\alpha \in \mathbb{Z} [i]$ no nulo se tiene
  $$N_{\mathbb{Q} (i)/\mathbb{Q}} (\alpha) = \# (\mathbb{Z} [i]/(\alpha)).$$

  \noindent (Más adelante veremos un resultado general.)

  \ifdefined\solutions
  \begin{solucion}
    Pongamos $\alpha = a + bi$. Tenemos
    $$(\alpha) = \{(c + di)\,(a + bi) \mid c, d \in \ZZ \}.$$
    Podemos escribir
    $$(c + di)\,(a + bi) = c \cdot (a + bi) + d \cdot (-b + ai),$$
    así que como $\ZZ$-módulo se tiene
    $$(\alpha) = (a + bi) \ZZ \oplus (-b + ai) \ZZ.$$

    Nos interesa el cociente de $\ZZ[i]$ por $(\alpha)$, o de manera
    equivalente, el cociente de $\ZZ[i]$ por el subgrupo generado por $(a, b)$ y
    $(-b, a)$. El número de elementos en el cociente será igual a\footnote{Este
      es un caso particular del siguiente resultado general: para
      $v_1, \ldots, v_n \in \ZZ^n$ denotemos por $A$ la matriz formada por los
      $v_i$. Luego, si $\det A \ne 0$, entonces
      $$[\ZZ^n : \langle v_1, \ldots , v_n\rangle] = \det A$$
      (y si $\det A = 0$, el índice no es finito).}

    \[ \det \begin{pmatrix} a & -b \\ b & a\end{pmatrix}
       = a^2 + b^2 = N(\alpha). \]

    Esta es la prueba elemental que tenía en mente. Otra opción es la
    siguiente: para $\alpha \notin \ZZ[i]^\times$ podemos tomar la
    factorización en primos de Gauss
    $$\alpha \sim \pi_1^{e_1} \cdots \pi_s^{e_s}.$$
    La multiplicatividad de la norma nos da
    $$N(\alpha) = N (\pi_1)^{e_1} \cdots N(\pi_s)^{e_s}.$$
    Además, por el teorema chino del resto,
    \[ \ZZ[i]/(\alpha) \cong
       \ZZ[i]/(\pi_1^{e_1}) \times \cdots \times \ZZ[i]/(\pi_s^{e_s}). \]

    Así el problema se reduce al probar que para un primo de Gauss $\pi$ se
    cumple $\# (\ZZ[i]/(\pi^e)) = N (\pi)^e$. Ya lo observamos para $e = 1$, y
    por el lema que vimos al inicio de clase 8, se tiene
    $\#(\ZZ[i]/(\pi^e)) = \#(Z[i]/(\pi))^e$.
  \end{solucion}
  \fi
\end{ejercicio}

\begin{ejercicio}
  Para $R = \mathbb{Z} [\sqrt{-5}]$ encuentre todos los ideales maximales
  $\mathfrak{p} \subset R$ tales que $R/\mathfrak{p} \cong \mathbb{F}_{23}$.

  \ifdefined\solutions
  \begin{solucion}
    Hay varios modos de hacerlo.  Por ejemplo, podemos ocupar la idea de
    Kummer–Dedekind. Si $\ZZ[\sqrt{-5}]/\mathfrak{p} \cong \FF_{23}$, entonces
    $23 \in \mathfrak{p}$. Estos ideales corresponden a los ideales maximales en
    el anillo cociente
    \begin{align*}
      \ZZ[\sqrt{-5}]/(23) & \xrightarrow{\cong} \FF_{23} [x]/(x^2 + 5),\\
      a + b \mapsto \sqrt{-5} ~\text{mód}~ 23 & \mapsto \overline{a + bx}.
    \end{align*}
    De la factorización $x^2 + 5 = (x + 8)\,(x - 8)$ en $\FF_{23} [x]$ podemos
    concluir que los ideales en cuestión son $(23, \pm 8+\sqrt{-5})$, o mejor
    dicho, $\mathfrak{p} = (23, 8 + \sqrt{-5})$,
    $\overline{\mathfrak{p}} = (23, 8 - \sqrt{-5})$. De hecho, estos ideales no
    son principales.
  \end{solucion}
  \fi
\end{ejercicio}

\begin{ejercicio}
  Demuestre que el ideal $(23,x)$ no es invertible en el anillo
  $\mathbb{Z} [x]$.

  \ifdefined\solutions
  \begin{solucion}
    El número $23$ no tiene mucho que ver con el ejercicio, podemos poner en su
    lugar cualquier primo $p$ y considerar el ideal $\mathfrak{m} = (p, x)$.
    $$\mathfrak{m}^{-1} = \{ f \in \QQ (x) \mid pf, xf \in \ZZ[x] \}.$$
    La condición $pf \in \ZZ[x]$ significa que en el denominador de $f$ a lo
    sumo puede estar $p$, mientras que la condición $xf \in \ZZ[x]$ significa
    que en el denominador a lo sumo puede estar $x$. Todo esto implica que
    $\mathfrak{m}^{-1} = \ZZ[x]$, y entonces
    $\mathfrak{m} \mathfrak{m}^{-1} = \mathfrak{m} \ne \ZZ[x]$.

    Otra manera de verlo sería la siguiente: notamos que
    $\mathfrak{m}^2 = (p^2, px, x^2)$, y para el ideal $I = (p, x^2)$ se tiene
    $\mathfrak{m}^2 \subsetneq I \subsetneq \mathfrak{m}$. Las inclusiones son
    estrictas: por ejemplo, no es difícil comprobar que
    $p \notin \mathfrak{m}^2$ y $x \notin I$. Si $\mathfrak{m}$ fuera
    invertible, tendríamos
    $\mathfrak{m} \subsetneq I \mathfrak{m}^{-1} \subsetneq \ZZ[x]$,
    pero el ideal $\mathfrak{m}$ es maximal.
  \end{solucion}
  \fi
\end{ejercicio}

\begin{ejercicio}
  Consideremos el anillo $\mathbb{Z} [\sqrt{5}]$ y los ideales
  \[ \mathfrak{p}_2 = (2, 1 + \sqrt{5}), \quad
     \mathfrak{p}_{11} = (11, 4 + \sqrt{5}). \]
  Determine si son invertibles y encuentre $I^{-1}$ en cada caso.

  \ifdefined\solutions
  \begin{solucion}
    En cada caso se puede calcular $I^{-1}$ a mano y luego multiplicar
    $II^{-1}$, pero esto es algo aburrido\dots{} Podemos notar que
    \[ \mathfrak{p}_2^2 = (4, 2 + 2\sqrt{5}, 6 + 2\sqrt{5})
       = (2) \cdot (2, 1 + \sqrt{5}) = 2\ZZ[\sqrt{5}] \cdot \mathfrak{p}_2. \]
    Ahora si $\mathfrak{p}_2$ fuera invertible, esto implicaría que
    $\mathfrak{p}_2 = 2\ZZ[ \sqrt{5}]$, pero
    $1 + \sqrt{5} \notin 2\ZZ[\sqrt{5}]$. Entonces, el ideal no es invertible.
  
    Respecto al ideal $\mathfrak{p}_{11}$, en realidad este es principal:
    $11 = (4 + \sqrt{5})\,(4 - \sqrt{5})$.  Entonces,
    $\mathfrak{p}_{11} = (4 + \sqrt{5})$,
    $\mathfrak{p}_{11}^{-1} = \left(\frac{1}{4 + \sqrt{5}}\right)$.

    Ahora calculamos
    \[ \mathfrak{p}_2^{-1} = \{ \alpha \in \QQ(\sqrt{5}) \mid
       2\alpha, (1 + \sqrt{5}) \alpha \in \ZZ[\sqrt{5}] \}. \]

    De la primera condición tenemos
    $\alpha = \frac{a}{2} + \frac{b}{2}\sqrt{5}$, donde $a,b \in \ZZ$.
    Para la segunda condición, primero calculamos
    $(1 + \sqrt{5})\,\alpha = \frac{a + 5b}{2} + \frac{a + b}{2}\sqrt{5}$,
    así que $a \equiv b \pmod{2}$, y luego
    $\mathfrak{p}_2^{-1} = \ZZ\Bigl[\frac{1 + \sqrt{5}}{2}\Bigr]$.
    Solo para comprobar otra vez más que el ideal no es invertible, podemos
    calcular
    \[ \mathfrak{p}_2 \mathfrak{p}_2^{-1} =
       (2, 1 + \sqrt{5})\,\left(1, \frac{1 + \sqrt{5}}{2}\right) =
       (2, 1 + \sqrt{5}, 3 + \sqrt{5}) =
       \mathfrak{p}_2 \ne \ZZ[\sqrt{5}]. \qedhere \]
  \end{solucion}
  \fi
\end{ejercicio}

\begin{ejercicio}
  Asumiendo que $\operatorname{Pic} (\mathbb{Z} [\sqrt{-37}]) \cong \mathbb{Z}/2\mathbb{Z}$,
  demuestre que la curva elíptica $y^2 = x^3 - 37$ no tiene puntos enteros.

  \ifdefined\solutions
  \begin{solucion}
    Reduciendo la ecuación módulo $4$, notamos que $y$ debe ser par. Escribamos
    $x^3 = (y + \sqrt{-37})\,(y - \sqrt{-37})$.  Usando que $y$ es par, podemos
    ver que el ideal $(y + \sqrt{-37}, y - \sqrt{-37})$ contiene $2y$ e
    \[ 37 = \left(\frac{y}{2} - \sqrt{-37}\right)\,(y + \sqrt{-37})
            - \frac{y}{2}\,(y - \sqrt{-37}). \]

    De la ecuación $y^2 = x^3 - 37$ se ve que $37 \nmid y$, así que
    $\gcd(2y, 37) = 1$. Esto demuestra que
    $(y + \sqrt{-37}) + (y - \sqrt{-37}) = \ZZ[\sqrt{-37}]$. Dado que los
    ideales son coprimos, tenemos $(y + \sqrt{-37}) = I^3$ para algún ideal
    $I$. Pero $\Pic(\ZZ[\sqrt{-37}]) = \ZZ/2\ZZ$ implica que el mismo $I$ es
    principal. Escribamos $I = (a + b\sqrt{-37})$. Tenemos
    $$y + \sqrt{-37} = a\,(a^2 - 111b^2 ) + b\,(3a^2 - 37b^2)\sqrt{-37}.$$
    Se ve que la ecuación $b\,(3a^2 - 37b^2) = \pm 1$ no tiene soluciones
    $a, b \in \ZZ$, y entonces nuestra ecuación $y^2 = x^3 - 37$ tampoco tiene
    soluciones enteras.
  \end{solucion}
  \fi
\end{ejercicio}

\begin{ejercicio-adicional}
  Encuentre el anillo de enteros $\mathcal{O}_K$ para
  $K = \mathbb{Q} (\sqrt{3}, \sqrt{5})$.

  \noindent (Hay modos listos de hacerlo, pero también se pueden ocupar cálculos
  directos como veremos el lunes para los campos cuadráticos; véase
  \emph{Kenneth S. Williams, Integers of biquadratic fields,
    Canad. Math. Bull. 13 (1970), 519--526}.)

  \ifdefined\solutions
  \begin{solucion}
    Este ejercicio no es muy instructivo porque más adelante veremos otro modo
    mejor de hacerlo.\footnote{Respuesta breve: $K = \QQ (\sqrt{3})$ y
      $K' = \QQ (\sqrt{5})$ son campos linealmente disjuntos con
      $\gcd (\Delta_K,\Delta_{K'}) = 1$. Esto será explicado más adelante.}

    \[ \begin{tikzcd}
      & \QQ(\sqrt{3},\sqrt{5})\ar[-]{dl}\ar[-]{d}\ar[-]{dr} \\
      \QQ(\sqrt{3})\ar[-]{dr} &
      \QQ(\sqrt{5})\ar[-]{d} &
      \QQ(\sqrt{15})\ar[-]{dl} \\
      & \QQ
      \end{tikzcd} \]
    Tomemos $1, \sqrt{3}, \sqrt{5},\sqrt{15}$ como una base de $K/\QQ$ y
    consideremos un elemento genérico $\alpha \in \O_K \setminus \QQ$ y sus
    conjugados $\alpha',\alpha'',\alpha''' \in \O_K \setminus \QQ$:
    \begin{align*}
      \alpha & = a + b\sqrt{3} + c\sqrt{5} + d\sqrt{15},\\
      \alpha' & = a - b\sqrt{3} + c\sqrt{5} - d\sqrt{15},\\
      \alpha'' & = a + b\sqrt{3} - c\sqrt{5} - d\sqrt{15},\\
      \alpha''' & = a - b\sqrt{3} - c\sqrt{5} + d\sqrt{15}.
    \end{align*}
    La integralidad de estos elementos implica que
    \[ \alpha + \alpha'' \in \ZZ [\sqrt{3}], \quad
       \alpha + \alpha' \in \ZZ \Bigl[\frac{1+\sqrt{5}}{2}\Bigr], \quad
       \alpha + \alpha''' \in \ZZ [\sqrt{15}]. \]
    Como consecuencia,
    $$\alpha = \frac{1}{2} (a' + b'\sqrt{3} + c'\sqrt{5} + d'\sqrt{15}),$$
    donde $a',b',c',d' \in \ZZ$.

    Primero, analizando los casos cuando $\alpha$ está en uno de los tres
    subcampos cuadráticos, notamos que necesariamente se cumple
    \[ \tag{*} a' \equiv c', ~ b' \equiv d' \pmod{2}. \]

    Ahora empieza la verdadera pesadilla: si $\alpha$ no está en los subcampos
    cuadráticos, vamos a escribir su polinomio mínimo que coincide con el
    polinomio característico. Lo calculé en PARI/GP.

\begin{verbatim}
? K = nfinit(t^2-3);
? L = rnfinit(K, x^2-5);
? u = rnfeltreltoabs(L,t);
? v = rnfeltreltoabs(L,x);
? f = charpoly ((a + b*u + c*v + d*u*v)/2)
% = .....
? r = a^2 - 15*d^2;
? s = a^2 - 3*b^2 - 5*c^2 + 15*d^2;
? t = 2*(a*d - b*c);
? f == x^4 - 2*a*x^3 + (r+s/2)*x^2
+ (15*d*t - a*s)/2*x + (s^2-15*t^2)/16
% = 1
\end{verbatim}

    El resultado es
    \[ x^4  - 2a' x^3 + \left(r + \frac{s}{2}\right)\,x^2 +
       \frac{15 d' t - a' s}{2}\,x + \frac{s^2 - 15\,t^2}{16}, \]
    donde
    \[ r = a'^2 - 15 d'^2, \quad
       s = a'^2 - 3b'^2 - 5c'^2 + 15d'^2, \quad
       t = 2\,(a' d' - b' c'). \]
    Se puede ver que los coeficientes del polinomio mínimo son enteros si y
    solamente si $s \equiv t \equiv 0 \pmod{4}$, y esto sucede si y solamente si
    se cumple la condición (*). Entonces,
    \begin{align*}
      \O_K & = \Bigl\{ \frac{1}{2} (a' + b'\sqrt{3} + c'\sqrt{5} + d'\sqrt{15})
      \Bigm| a' \equiv c', ~ b' \equiv d' ~ (2) \Bigr\} \\
      & = \ZZ \Bigl[\sqrt{3}, \frac{1+\sqrt{5}}{2}\Bigr] \\
      & = \ZZ \oplus \sqrt{3}\ZZ \oplus \frac{1 + \sqrt{5}}{2}\ZZ
      \oplus \frac{\sqrt{3}+\sqrt{15}}{2}\,\ZZ.
    \end{align*}
    Los elementos de arriba son claramente enteros, y uno podría adivinar desde
    el principio que la respuesta es
    $\ZZ \Bigl[\sqrt{3}, \frac{1+\sqrt{5}}{2}\Bigr]$, pero todavía no sabríamos
    cómo probarlo de manera lista\dots{} La moraleja de estos cálculos:
    trabajando solamente con la definición de $\O_K$, sin usar ninguna idea
    especial, no se puede llegar muy lejos\dots
  \end{solucion}
  \fi
\end{ejercicio-adicional}

\end{document}

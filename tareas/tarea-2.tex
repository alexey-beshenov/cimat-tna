\documentclass{article}

\usepackage[utf8]{inputenc}
\usepackage[spanish]{babel}

\usepackage{amsmath,amssymb,amsthm}

\newcounter{tarea}
\setcounter{tarea}{2}
\theoremstyle{definition}
\newtheorem{ejercicio}{Ejercicio}[tarea]
\newtheorem*{ejercicio-adicional}{Ejercicio adicional}

\newenvironment{solucion}{\begin{proof}[Solución]}{\end{proof}}

\usepackage{mathspec}

\setmainfont{PT Serif}
\setsansfont{Montserrat}
\setmonofont{PT Mono}

\title{Teoría de números algebraicos\\Tarea 2}
\author{Alexey Beshenov (alexey.beshenov@cimat.mx)}
\date{26 de agosto de 2020}

% \def\solutions{true}

\begin{document}

{\sffamily\bfseries\maketitle}

\noindent Fecha límite: viernes, 4 de septiembre.

\ifdefined\solutions
\else
\thispagestyle{empty}
\fi

\begin{ejercicio}
  Demuestre que para $\alpha \in \mathbb{Z} [i]$ no nulo se tiene
  $$N_{\mathbb{Q} (i)/\mathbb{Q}} (\alpha) = \# (\mathbb{Z} [i]/(\alpha)).$$

\noindent (Más adelante veremos un resultado general.)
\end{ejercicio}

\begin{ejercicio}
  Para $R = \mathbb{Z} [\sqrt{-5}]$ encuentre todos los ideales maximales
  $\mathfrak{p} \subset R$ tales que $R/\mathfrak{p} \cong \mathbb{F}_{23}$.
\end{ejercicio}

\begin{ejercicio}
  Demuestre que el ideal $(23,x)$ no es invertible en el anillo
  $\mathbb{Z} [x]$.
\end{ejercicio}

\begin{ejercicio}
  Consideremos el anillo $\mathbb{Z} [\sqrt{5}]$ y los ideales
  \[ \mathfrak{p}_2 = (2, 1 + \sqrt{5}), \quad
     \mathfrak{p}_{11} = (11, 4 + \sqrt{5}). \]
  Determine si son invertibles y encuentre $I^{-1}$ en cada caso.
\end{ejercicio}

\begin{ejercicio}
  Asumiendo que $\operatorname{Pic} (\mathbb{Z} [\sqrt{-37}]) \cong \mathbb{Z}/2\mathbb{Z}$,
  demuestre que la curva elíptica $y^2 = x^3 - 37$ no tiene puntos enteros.
\end{ejercicio}

\begin{ejercicio-adicional}
  Encuentre el anillo de enteros $\mathcal{O}_K$ para
  $K = \mathbb{Q} (\sqrt{3}, \sqrt{5})$.

  \noindent (Hay modos listos de hacerlo, pero también se pueden ocupar cálculos
  directos como veremos el lunes para los campos cuadráticos; véase
  \emph{Kenneth S. Williams, Integers of biquadratic fields,
    Canad. Math. Bull. 13 (1970), 519--526}.)
\end{ejercicio-adicional}

\end{document}

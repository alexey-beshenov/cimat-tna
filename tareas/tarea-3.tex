\documentclass{article}

\usepackage[utf8]{inputenc}
\usepackage[spanish]{babel}

\usepackage{amsmath,amssymb,amsthm}

\newcounter{tarea}
\setcounter{tarea}{3}
\theoremstyle{definition}
\newtheorem{ejercicio}{Ejercicio}[tarea]
\newtheorem*{ejercicio-adicional}{Ejercicio adicional}

\newenvironment{solucion}{\begin{proof}[Solución]}{\end{proof}}

\usepackage{mathspec}

\setmainfont{PT Serif}
\setsansfont{Montserrat}
\setmonofont{PT Mono}

\title{Teoría de números algebraicos\\Tarea 3}
\author{Alexey Beshenov (alexey.beshenov@cimat.mx)}
\date{2 de septiembre de 2020}

% \def\solutions{true}

\begin{document}

{\sffamily\bfseries\maketitle}

\noindent Fecha límite: viernes, 11 de septiembre.

\ifdefined\solutions
\else
\thispagestyle{empty}
\fi

\vspace{1em}

Consideremos el campo ciclotómico $K = \mathbb{Q} (\zeta_8)$.
Más adelante veremos un modo adecuado para probar que
$\mathcal{O}_K = \mathbb{Z}[\zeta_8]$, pero por el momento se puede aceptar este
resultado.

\begin{ejercicio}
  Usando el teorema de Kummer--Dedekind, describa las factorizaciones de
  $p \mathcal{O}_K$ en ideales primos para diferentes primos racionales $p$.

  \noindent (La respuesta depende de $p$ mód $8$.)
\end{ejercicio}

\begin{ejercicio}
  Encuentre las subextensiones
  $\mathbb{Q} \subset F \subset \mathbb{Q} (\zeta_8)$ y las factorizaciones de
  $p \mathcal{O}_F$ para cada una de estas.

  \noindent (Para encontrar las subextensiones, use la teoría de Galois.)
\end{ejercicio}

\begin{ejercicio}
  Considerando la descomposición de primos racionales $p$ en $\mathcal{O}_K$,
  demuestre que $\zeta_p \notin \mathbb{Q} (\zeta_q)$ para diferentes primos
  impares $p \ne q$.
\end{ejercicio}

\begin{ejercicio}
  Para el campo ciclotómico $K = \mathbb{Q} (\zeta_p)$ el grupo de Galois
  $\operatorname{Gal} (K/\mathbb{Q})$ es cíclico, así que la teoría de Galois
  implica que existe un subcampo cuadrático único $F \subset K$. Considerando
  la factorización de primos racionales en $\mathcal{O}_F$ y $\mathcal{O}_K$,
  demuestre que $F = \mathbb{Q} (\sqrt{p^*})$, donde $p^* = (-1)^{(p-1)/2} p$.

  \noindent (Sugerencia: si $p$ se ramifica en $\mathcal{O}_F$, entonces $p$ se
  ramifica en $\mathcal{O}_K$.)
\end{ejercicio}

\end{document}

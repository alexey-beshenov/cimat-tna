\documentclass{article}

\usepackage[utf8]{inputenc}
\usepackage[spanish]{babel}

\usepackage{amsmath,amssymb,amsthm}

\usepackage{tikz-cd}
\usetikzlibrary{babel}
\usetikzlibrary{calc}

\newcounter{tarea}
\setcounter{tarea}{7}
\theoremstyle{definition}
\newtheorem{ejercicio}{Ejercicio}[tarea]

\newenvironment{solucion}{\begin{proof}[Solución]}{\end{proof}}

\usepackage{framed}

\usepackage{mathspec}

\setmainfont{PT Serif}
\setsansfont{Montserrat}
\setmonofont{PT Mono}

\DeclareMathOperator{\Gal}{Gal}
\DeclareMathOperator{\Frob}{Frob}

\newcommand{\FF}{\mathbb{F}}
\newcommand{\ZZ}{\mathbb{Z}}
\newcommand{\QQ}{\mathbb{Q}}
\renewcommand{\O}{\mathcal{O}}
\newcommand{\CC}{\mathbb{C}}
\renewcommand{\Re}{\operatorname{Re}}

\newcommand{\legendre}[2]{\left(\frac{#1}{#2}\right)}

\title{Teoría de números algebraicos\\Tarea 7}
\author{Alexey Beshenov (alexey.beshenov@cimat.mx)}
\date{14 de octubre de 2020}

% \def\solutions{true}

\begin{document}

{\sffamily\bfseries\maketitle}

\ifdefined\solutions
\else
\thispagestyle{empty}
\fi

\vspace{1em}

\begin{ejercicio}
  Demuestre que para una extensión de Galois $L/K$, primos
  $\mathfrak{q} \subset \O_L$, $\mathfrak{p} \subset \O_K$,
  tales que
  $\mathfrak{q} \mid \mathfrak{p}$, y $\sigma \in \Gal (L/K)$ se tiene
  \[ D (\sigma (\mathfrak{q})|\mathfrak{p}) =
  \sigma \, D (\mathfrak{q}|\mathfrak{p}) \, \sigma^{-1}, \quad
  I (\sigma (\mathfrak{q})|\mathfrak{p}) =
  \sigma \, I (\mathfrak{q}|\mathfrak{p}) \, \sigma^{-1}. \]
  Además, si $\mathfrak{p}$ no se ramifica, entonces el Frobenius cumple
  \[ \Frob_{\sigma (\mathfrak{q})|\mathfrak{p}} =
  \sigma \, \Frob_{\mathfrak{q}|\mathfrak{p}} \, \sigma^{-1}. \]
\end{ejercicio}

\begin{ejercicio}
  Sea $F$ un campo de números, y $L/K/F$ una torre de extensiones tal que $L/K$
  es una extensión normal. Sean $\mathfrak{p} \subset \O_F$,
  $\mathfrak{q} \in \O_K$, $\mathfrak{Q} \subset \O_L$ ideales primos tales que
  $\mathfrak{Q} \mid \mathfrak{q}$ y $\mathfrak{q}\mid\mathfrak{p}$.

  \begin{enumerate}
  \item[1)] Demuestre que la restricción de automorfismos identifica
    $D (\mathfrak{Q}|\mathfrak{q})$ con un subgrupo de
    $D (\mathfrak{Q}|\mathfrak{p})$ e
    $I (\mathfrak{Q}|\mathfrak{q})$ con un subgrupo de
    $I (\mathfrak{Q}|\mathfrak{p})$.

  \item[2)] Si $\mathfrak{p}$ no se ramifica en $L$, demuestre que
    $\Frob_{\mathfrak{Q}|\mathfrak{q}} = (\Frob_{\mathfrak{Q}|\mathfrak{p}})^{f (\mathfrak{q}|\mathfrak{p})}$.

  \item[3)] Si la extensión $K/F$ es normal, demuestre que
    $\Frob_{\mathfrak{q}|\mathfrak{p}}$ es la restricción de
    $\Frob_{\mathfrak{Q}|\mathfrak{p}}$.
  \end{enumerate}
\end{ejercicio}

\begin{ejercicio}
  Sea $K$ el campo de descomposición del polinomio
  $$f = x^4 + 8x + 12.$$
  Calcule $\Gal (K/\QQ)$, las clases de conjugación, los tipos de descomposición
  que corresponden a cada $\Frob_{\mathfrak{p}\mid p}$, y las densidades que nos
  da el teorema de Chebotarëv.
\end{ejercicio}

\begin{ejercicio}
  Para $K = \QQ (\sqrt[4]{2})$ consideremos la cerradura de Galois
  $L = \QQ (\sqrt[4]{2},i)$.

  \begin{enumerate}
  \item[1)] Demuestre que el único primo racional $p$ que se ramifica en $L$ es
    $p = 2$.

  \item[2)] Para $p$ impar sea $\mathfrak{p} \subset \O_L$ un primo tal que
    $\mathfrak{p} \mid p$. Determine cómo el tipo de factorización de $p$ en
    $\O_K$ para toda posibilidad para $\Frob_{\mathfrak{p}|p}$.
  \end{enumerate}
\end{ejercicio}

\begin{ejercicio}
  Para la extensión ciclotómica $L = \QQ (\zeta_n)$ determine cómo los primos
  no ramificados $p\nmid n$ se descomponen en el subcampo
  $K = \QQ (\zeta_n + \zeta_n^{-1})$.
\end{ejercicio}

\end{document}

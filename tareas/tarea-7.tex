\documentclass{article}

\usepackage[utf8]{inputenc}
\usepackage[spanish]{babel}

\usepackage{amsmath,amssymb,amsthm}

\usepackage{multirow}

\usepackage{tikz-cd}
\usetikzlibrary{babel}
\usetikzlibrary{calc}
\usetikzlibrary{arrows}

\newcounter{tarea}
\setcounter{tarea}{7}
\theoremstyle{definition}
\newtheorem{ejercicio}{Ejercicio}[tarea]

\newenvironment{solucion}{\begin{proof}[Solución]}{\end{proof}}

\usepackage{framed}

\usepackage{mathspec}

\setmainfont{PT Serif}
\setsansfont{Montserrat}
\setmonofont{PT Mono}

\DeclareMathOperator{\Gal}{Gal}
\DeclareMathOperator{\Frob}{Frob}
\DeclareMathOperator{\Stab}{Stab}

\newcommand{\FF}{\mathbb{F}}
\newcommand{\ZZ}{\mathbb{Z}}
\newcommand{\QQ}{\mathbb{Q}}
\renewcommand{\O}{\mathcal{O}}
\newcommand{\CC}{\mathbb{C}}
\renewcommand{\Re}{\operatorname{Re}}

\newcommand{\legendre}[2]{\left(\frac{#1}{#2}\right)}

\title{Teoría de números algebraicos\\Tarea 7}
\author{Alexey Beshenov (alexey.beshenov@cimat.mx)}
\date{14 de octubre de 2020}

% \def\solutions{true}

\begin{document}

{\sffamily\bfseries\maketitle}

\ifdefined\solutions
\else
\thispagestyle{empty}
\fi

\vspace{1em}

\begin{ejercicio}
  Demuestre que para una extensión de Galois $L/K$, primos
  $\mathfrak{q} \subset \O_L$, $\mathfrak{p} \subset \O_K$,
  tales que
  $\mathfrak{q} \mid \mathfrak{p}$, y $\sigma \in \Gal (L/K)$ se tiene
  \[ D (\sigma (\mathfrak{q})|\mathfrak{p}) =
  \sigma \, D (\mathfrak{q}|\mathfrak{p}) \, \sigma^{-1}, \quad
  I (\sigma (\mathfrak{q})|\mathfrak{p}) =
  \sigma \, I (\mathfrak{q}|\mathfrak{p}) \, \sigma^{-1}. \]
  Además, si $\mathfrak{p}$ no se ramifica, entonces el Frobenius cumple
  \[ \Frob_{\sigma (\mathfrak{q})|\mathfrak{p}} =
     \sigma \, \Frob_{\mathfrak{q}|\mathfrak{p}} \, \sigma^{-1}. \]

  \ifdefined\solutions
  \begin{solucion}
    Recordemos que $D (\mathfrak{q}|\mathfrak{p})$ es el grupo estabilizador de
    $\mathfrak{q}$ respecto a la acción del grupo de Galois $\Gal (L/K)$ sobre
    los primos $\mathfrak{q}_1,\ldots,\mathfrak{q}_g$ que están sobre
    $\mathfrak{p}$. Es un cálculo general que para un $G$-conjunto $X$ se cumple
    $\Stab_{\sigma x} = \sigma \Stab_x \sigma^{-1}$.

    Para el grupo de inercia, podemos considerar el diagrama
    \[ \begin{tikzcd}
        1 \ar{r} & I (\sigma(\mathfrak{q})|\mathfrak{p}) \ar{r}\ar[equals]{d} & D (\sigma(\mathfrak{q})|\mathfrak{p}) \ar{r}\ar[equals]{d} & \Gal (\kappa(\sigma(\mathfrak{q}))/\kappa(\mathfrak{p})) \ar{r}\ar[equals]{d} & 1 \\
        1 \ar{r} & \sigma I (\mathfrak{q}|\mathfrak{p}) \sigma^{-1} \ar{r} & \sigma D (\mathfrak{q}|\mathfrak{p}) \sigma^{-1} \ar{r} & \overline{\sigma} \Gal (\kappa(\mathfrak{q})/\kappa(\mathfrak{p})) \overline{\sigma}^{-1} \ar{r} & 1
      \end{tikzcd} \]

    Aquí $\sigma \in \Gal (L/K)$ induce un isomorfismo
    $\overline{\sigma}\colon \kappa (\mathfrak{q}) \to \kappa (\sigma (\mathfrak{q}))$.

    De allí también se sigue la afirmación sobre el Frobenius.
    \[ \begin{tikzcd}
        D (\sigma(\mathfrak{q})|\mathfrak{p}) \ar{r}{\cong}\ar[equals]{d} & \Gal (\kappa(\sigma(\mathfrak{q}))/\kappa(\mathfrak{p})) \ar[equals]{d} \\
        \sigma D (\mathfrak{q}|\mathfrak{p}) \sigma^{-1} \ar{r}{\cong} & \overline{\sigma} \Gal (\kappa(\mathfrak{q})/\kappa(\mathfrak{p})) \overline{\sigma}^{-1}
      \end{tikzcd} \]
  \end{solucion}
  \fi
\end{ejercicio}

\begin{ejercicio}
  Sea $F$ un campo de números, y $L/K/F$ una torre de extensiones tal que $L/K$
  es una extensión normal. Sean $\mathfrak{p} \subset \O_F$,
  $\mathfrak{q} \in \O_K$, $\mathfrak{Q} \subset \O_L$ ideales primos tales que
  $\mathfrak{Q} \mid \mathfrak{q}$ y $\mathfrak{q}\mid\mathfrak{p}$.

  \begin{enumerate}
  \item[1)] Demuestre que $D (\mathfrak{Q}|\mathfrak{q})$ se identifica con un
    subgrupo de $D (\mathfrak{Q}|\mathfrak{p})$ e
    $I (\mathfrak{Q}|\mathfrak{q})$ con un subgrupo de
    $I (\mathfrak{Q}|\mathfrak{p})$.

  \item[2)] Si $\mathfrak{p}$ no se ramifica en $L$, demuestre que
    $\Frob_{\mathfrak{Q}|\mathfrak{q}} = (\Frob_{\mathfrak{Q}|\mathfrak{p}})^{f (\mathfrak{q}|\mathfrak{p})}$.

  \item[3)] Si la extensión $K/F$ es normal, demuestre que
    $\Frob_{\mathfrak{q}|\mathfrak{p}}$ es la restricción de
    $\Frob_{\mathfrak{Q}|\mathfrak{p}}$.
  \end{enumerate}

  \ifdefined\solutions
  \begin{solucion}
    Estamos en la siguiente situación:
    \[ \begin{tikzcd}
        \O_L \ar[->>]{r} \ar[-]{d} & \kappa (\mathfrak{Q}) \ar[-]{d} \\
        \O_K \ar[->>]{r} \ar[-]{d} & \kappa (\mathfrak{q}) \ar[-]{d} \\
        \O_F \ar[->>]{r} & \kappa (\mathfrak{p})
      \end{tikzcd} \]
    Tenemos inclusiones naturales
    $\Gal (L/K) \subset \Gal (L/F)$
    y
    $\Gal (\kappa(\mathfrak{Q})/\kappa(\mathfrak{q})) \subset \Gal (\kappa(\mathfrak{Q})/\kappa(\mathfrak{p}))$,
    y estas nos dan el diagrama conmutativo
    \[ \begin{tikzcd}
        1 \ar{r} & I (\mathfrak{Q}|\mathfrak{q}) \ar{r}\ar[dashed]{d} & D (\mathfrak{Q}|\mathfrak{q}) \ar{r}\ar[right hook->]{d} & \Gal (\kappa(\mathfrak{Q})/\kappa(\mathfrak{q})) \ar{r}\ar[right hook->]{d} & 1 \\
        1 \ar{r} & I (\mathfrak{Q}|\mathfrak{p}) \ar{r} & D (\mathfrak{Q}|\mathfrak{p}) \ar{r} & \Gal (\kappa(\mathfrak{Q})/\kappa(\mathfrak{p})) \ar{r} & 1
      \end{tikzcd} \]
    Aquí la flecha punteada existe por la conmutatividad del segundo cuadrado,
    y es un monomorfismo porque la flecha en el medio lo es. Si $\mathfrak{p}$
    no se ramifica, se obtiene
    \[ \begin{tikzcd}
        D (\mathfrak{Q}|\mathfrak{q}) \ar{r}{\cong}\ar[right hook->]{d} & \Gal (\kappa(\mathfrak{Q})/\kappa(\mathfrak{q}))\ar[right hook->]{d} \\
        D (\mathfrak{Q}|\mathfrak{p}) \ar{r}{\cong} & \Gal (\kappa(\mathfrak{Q})/\kappa(\mathfrak{p}))
      \end{tikzcd} \]
    Tenemos
    \[ \Frob_{\mathfrak{Q}|\mathfrak{p}}\colon \alpha \mapsto \alpha^{\# \kappa (\mathfrak{p})} = \alpha^{p^{f (\mathfrak{p}|p)}} \]
    y
    \[ \Frob_{\mathfrak{Q}|\mathfrak{q}}\colon \alpha \mapsto \alpha^{\# \kappa (\mathfrak{q})} = \alpha^{p^{f (\mathfrak{q}|p)}} = \Bigl(\alpha^{p^{f (\mathfrak{p}|p)}}\Bigr)^{f (\mathfrak{q}|\mathfrak{p})}. \]

    Si $K/F$ es también una extensión normal, entonces tiene sentido considerar
    el diagrama
    \[ \begin{tikzcd}
        D (\mathfrak{Q}|\mathfrak{p}) \ar{r}{\cong}\ar[->>]{d} & \Gal (\kappa(\mathfrak{Q})/\kappa(\mathfrak{p}))\ar[->>]{d} \\
        D (\mathfrak{q}|\mathfrak{p}) \ar{r}{\cong} & \Gal (\kappa(\mathfrak{q})/\kappa(\mathfrak{p}))
      \end{tikzcd} \]
    inducido por
    $\Gal (L/F) \twoheadrightarrow \Gal (K/F)$,
    y
    $\Gal (\kappa(\mathfrak{Q})/\kappa(\mathfrak{p})) \twoheadrightarrow \Gal (\kappa(\mathfrak{Q})/\kappa(\mathfrak{p}))$.
    Estas son las restricciones de automorfismos, con el núcleo
    $\Gal (L/K)$ y $\Gal (\kappa(\mathfrak{Q})/\kappa(\mathfrak{q}))$
    respectivamente.
  \end{solucion}
  \fi
\end{ejercicio}

\begin{ejercicio}
  Sea $K$ el campo de descomposición del polinomio
  $$f = x^4 + 8x + 12.$$
  Calcule $\Gal (K/\QQ)$, las clases de conjugación, los tipos de descomposición
  que corresponden a cada $\Frob_{\mathfrak{p}\mid p}$, y las densidades que nos
  da el teorema de Chebotarëv.

  \ifdefined\solutions
  \begin{solucion}
    Podemos ocupar la reducción módulo $p$. Para un polinomio mónico
    $f \in \ZZ [x]$ y un primo $p$, supongamos que el polinomio correspondiente
    $\overline{f} \in \FF_p [x]$ no tiene raíces múltiples en
    $\overline{\FF_p}$.  En este caso el grupo de Galois de $\overline{f}$ se
    encaja en el grupo de Galois de $f$. Para las pruebas, véanse por ejemplo
    \S\S VI.2 + VII.2 en el libro de Lang. Aquí me gustaría usar nuestro
    polinomio particular para explicar el uso de este resultado.

    El polinomio $f = x^4 + 8x + 12$ tiene factores múltiples al factorizarlo
    módulo $p = 2$ y $5$. La factorización módulo $p = 5$ nos da
    $$\overline{f} = (x+1)\,(x^3 + 4x^2 + x + 2).$$
    El grupo de Galois de $\overline{f}$ será
    $\Gal (\FF_{p^3}/\FF_p) \cong C_3$, lo que nos dice que el grupo de Galois
    de $f$ contiene un elemento de orden $3$.

    El grupo de Galois de $f$ se realiza como un subgrupo transitivo de $S_4$, y
    las únicas posibilidades para este son
    $$C_4, ~ V_4, ~ D_8, ~ A_4, ~ S_4.$$
    Sabiendo que tenemos un elemento de orden $3$, nos quedan solamente dos
    posibilidades: $A_4$ y $S_4$. Calculamos el discriminante
    $$\Delta (f) = \prod_{i < j} (\alpha_i - \alpha_j)^2.$$
    En este caso particular $\Delta (f) = 2^{12} \cdot 3^4$ es un
    cuadrado. Entonces,
    $$\prod_{i < j} (\alpha_i - \alpha_j) \in \QQ,$$
    y por lo tanto este número es fijo respecto a la acción del grupo de
    Galois. Notamos que la transposición que intercambia dos raíces $\alpha_i$ y
    $\alpha_j$ no está en el grupo de Galois: esta transposición cambia el signo
    de $\prod_{i < j} (\alpha_i - \alpha_j)$. Entonces, el grupo de Galois es
    $A_4$.

    Ahora bien, las clases de conjugación en $A_4$ son las siguientes.
    \begin{itemize}
    \item $C_1 = \{ id \}$. En este caso el Frobenius es trivial, así que $p$ se
      escinde completamente en $12$ ideales primos. La densidad correspondiente
      será $\frac{1}{12}$.

    \item $C_2 = \{ (1~2)\,(3~4), ~ (1~3)\,(2~4), ~ (1~4)\,(2~3) \}$. En este caso
      el Frobenius tiene orden $2$, lo que corresponde a descomposiciones de la
      forma $\mathfrak{p}_1\cdots\mathfrak{p}_6$. La densidad correspondiente es
      $\frac{1}{4}$.

    \item $C_3 = \{ (1~2~3), ~ (1~3~4), ~ (1~4~2), ~ (2~4~3) \}$. En este caso el
      Frobenius tiene orden $3$, lo que corresponde a las factorizaciones de la
      forma $\mathfrak{p}_1\,\mathfrak{p}_2\,\mathfrak{p}_3\,\mathfrak{p}_4$. La
      densidad correspondiente es $\frac{1}{3}$.

    \item $C_4 = \{ (1~2~4), ~ (1~3~2), ~ (1~4~3), ~ (2~3~4) \}$. Este caso es
      similar al anterior.
    \end{itemize}

    En particular, podemos concluir que hay tres tipos de factorizaciones: en
    $4$ ideales primos (densidad $\frac{2}{3}$), en $6$ ideales primos (densidad
    $\frac{1}{4}$), y en $12$ ideales primos (densidad $\frac{1}{12}$).
  \end{solucion}
  \fi
\end{ejercicio}

\begin{ejercicio}
  Para $K = \QQ (\sqrt[4]{2})$ consideremos la cerradura de Galois
  $L = \QQ (\sqrt[4]{2},i)$.

  \begin{enumerate}
  \item[1)] Demuestre que el único primo racional $p$ que se ramifica en $L$ es
    $p = 2$.

  \item[2)] Para $p$ impar sea $\mathfrak{p} \subset \O_L$ un primo tal que
    $\mathfrak{p} \mid p$. Determine cómo el tipo de factorización de $p$ en
    $\O_K$ para toda posibilidad para $\Frob_{\mathfrak{p}|p}$.
  \end{enumerate}

  \ifdefined\solutions
  \begin{solucion}
    Tenemos $L = KF$, donde $F = \QQ (i)$. La extensión $F/\QQ$ es de
    Galois. Dado un primo racional $p$, sean $\mathfrak{P} \subset \O_L$ un
    ideal primo tal que $\mathfrak{P}\mid p$ y
    $\mathfrak{p} = \mathfrak{P} \cap \O_K$ y
    $\mathfrak{q} = \mathfrak{P} \cap \O_F$.

    Se puede ver que en esta situación la restricción de automorfismos induce
    inclusiones
    $D (\mathfrak{P}|\mathfrak{p}) \hookrightarrow D (\mathfrak{q}|p)$ e
    $I (\mathfrak{P}|\mathfrak{p}) \hookrightarrow I (\mathfrak{q}|p)$.
    El único primo $p$ que se ramifica en $F$ es $p = 2$. Entonces, para todo
    $p$ impar se tiene $I (\mathfrak{P}|\mathfrak{p}) = I (\mathfrak{q}|p) = 1$.
    Esto implica que
    $$e (\mathfrak{P}|p) = e (\mathfrak{P}|\mathfrak{p})\,e (\mathfrak{p}|p) = e (\mathfrak{p}|p).$$
    Tenemos $\Delta (x^4 - 2) = -2^{11}$, así que en $K$ también se ramifica
    solamente $p = 2$. Entonces, $e (\mathfrak{P}|p) = 1$ para todo $p$ impar.
    Esto verifica la parte 1).

    Ahora $K$ es el subcampo fijo por la conjugación compleja, que bajo el
    isomorfismo $\Gal (L/\QQ) \cong D_4$ corresponde al subgrupo
    $H = \{ 1, f \}$. Las clases laterales correspondientes son
    \[ H, ~ H r, ~ H r^2, ~ H r^3. \]

    La siguiente tabla nos da las acciones del Frobenius
    $\Frob_{\mathfrak{p}|p}$ sobre las clases laterales y las descomposiciones
    correspondientes de $p$ en $K$.

    \begin{center}\def\arraystretch{1.75}
      \begin{tabular}{cccc}
        \hline
        $\Frob_{\mathfrak{p}|p}$ & órbitas & descomposición & densidad \\
        \hline
        $1$ & $\{ H \}, ~ \{ Hr \}, ~ \{ Hr^2 \}, ~ \{ Hr^3 \}$ & $\mathfrak{p}_1 \mathfrak{p}_2 \mathfrak{p}_3 \mathfrak{p}_4$ & $\frac{1}{8}$ \\
        \hline
        $r$ & $\{ H \to Hr \to Hr^2 \to Hr^3 \}$ & \multirow{2}{*}{inerte, $f = 4$} & \multirow{2}{*}{$\frac{1}{4}$} \\
        $r^3$ & $\{ H \to H r^3 \to H r^2 \to H r \}$ \\
        \hline
        $r^2$ & $\{ H \to H r^2 \}, ~ \{ H r \to H r^3 \}$ & \multirow{3}{*}{$\mathfrak{p}_1 \mathfrak{p}_2$, $f_1 = f_2 = 2$} & \multirow{3}{*}{$\frac{3}{8}$} \\
        $rf = fr^3$ & $\{ H \to H r^3 \}, ~ \{ H r \to H r^2 \}$ \\
        $r^3 f = fr$ & $\{ H \to Hr \}, ~ \{ H r^2 \to H r^3 \}$ \\
        \hline
        $f$ & $\{ H \}, ~ \{ H r \to H r^3 \}, ~ \{ H r^2 \}$ & \multirow{2}{*}{$\mathfrak{p}_1 \mathfrak{p}_2 \mathfrak{p}_3$, $f_1 = f_2 = 1$, $f_3 = 2$} & \multirow{2}{*}{$\frac{1}{4}$} \\
        $r^2 f = f r^2$ & $\{ H \to H r^2 \}, ~ \{ H r \}, ~ \{ H r^3 \}$ \\
        \hline
      \end{tabular}
    \end{center}
  \end{solucion}
  \fi
\end{ejercicio}

\begin{ejercicio}
  Para la extensión ciclotómica $L = \QQ (\zeta_n)$ determine cómo los primos
  no ramificados $p\nmid n$ se descomponen en el subcampo
  $K = \QQ (\zeta_n + \zeta_n^{-1})$.

  \ifdefined\solutions
  \begin{solucion}
    Para $p\nmid n$ el automorfismo de Frobenius sobre $\QQ (\zeta_n)$ viene
    dado por $\zeta_n \mapsto \zeta_n^p$. El grado de campos residuales $f$ es
    el orden del Frobenius; es decir, el orden de $p$ en el grupo multiplicativo
    $(\ZZ/n\ZZ)^\times \cong \Gal (\QQ (\zeta_n)/\QQ)$.

    Ahora $K = \QQ (\zeta_n + \zeta_n^{-1})$ es el subcampo fijo por la
    conjugación compleja $\zeta_n \mapsto \zeta_n^{-1}$ y
    $\Gal (K/\QQ) \cong (\ZZ/n\ZZ)^\times / \{ \pm 1 \}$. El Frobenius
    corresponde a la clase de $p$ en el cociente
    $(\ZZ/n\ZZ)^\times / \{ \pm 1 \}$, y el grado del campo residual será el
    orden de $p$ en ese grupo.

    He aquí un ejemplo particular para el campo $\QQ (\zeta_7)$.

    \begin{center}\def\arraystretch{1.75}
      \begin{tabular}{rcccccc}
        \hline
        $p$ mód $7\colon$ & $1$ & $2$ & $3$ & $4$ & $5$ & $6$ \\
        orden en $(\ZZ/7\ZZ)^\times\colon$ & $1$ & $3$ & $6$ & $3$ & $6$ & $2$ \\
        orden en $\frac{(\ZZ/7\ZZ)^\times}{\{ \pm 1 \}}\colon$ & $1$ & $3$ & $3$ & $3$ & $3$ & $1$ \\
        desc. en $\QQ (\zeta_7)\colon$ & $\mathfrak{P}_1\cdots\mathfrak{P}_6$ & $\mathfrak{P}_1\mathfrak{P}_2$ & inerte & $\mathfrak{P}_1\mathfrak{P}_2$ & inerte & $\mathfrak{P}_1\mathfrak{P}_2\mathfrak{P}_3$ \\
        desc. en $\QQ (\zeta_7 + \zeta_7^{-1})\colon$ & $\mathfrak{p}_1\mathfrak{p}_2\mathfrak{p}_3$ & inerte & inerte & inerte & inerte & $\mathfrak{p}_1\mathfrak{p}_2\mathfrak{p}_3$ \\
        \hline
      \end{tabular}
    \end{center}

    Otro ejemplo: $p = 2$ tiene orden $4$ en $(\ZZ/15\ZZ)^\times$ y también en
    $(\ZZ/15\ZZ)^\times / \{ \pm 1 \}$. Entonces, $p$ se descompone en
    $2 = \phi (15)/4$ ideales primos en $\QQ (\zeta_{15})$, y es inerte en el
    subcampo $\QQ (\zeta_{15} + \zeta_{15}^{-1})$.
  \end{solucion}
  \fi
\end{ejercicio}

\end{document}

\documentclass{article}

\usepackage[utf8]{inputenc}
\usepackage[spanish]{babel}

\usepackage{amsmath,amssymb,amsthm}

\usepackage{tikz-cd}
\usetikzlibrary{babel}

\newcounter{tarea}
\setcounter{tarea}{5}
\theoremstyle{definition}
\newtheorem{ejercicio}{Ejercicio}[tarea]

\newenvironment{solucion}{\begin{proof}[Solución]}{\end{proof}}

\usepackage{framed}

\usepackage{mathspec}

\setmainfont{PT Serif}
\setsansfont{Montserrat}
\setmonofont{PT Mono}

\newcommand{\ZZ}{\mathbb{Z}}
\newcommand{\QQ}{\mathbb{Q}}
\newcommand{\CC}{\mathbb{C}}
\renewcommand{\O}{\mathcal{O}}

\DeclareMathOperator{\Res}{Res}

\title{Teoría de números algebraicos\\Tarea 5}
\author{Alexey Beshenov (alexey.beshenov@cimat.mx)}
\date{23 de septiembre de 2020}

% \def\solutions{true}

\begin{document}

{\sffamily\bfseries\maketitle}

\noindent Fecha límite: viernes, 2 de octubre.

\ifdefined\solutions
\else
\thispagestyle{empty}
\fi

\vspace{1em}

\begin{ejercicio}
  Sea $X$ una matriz de $n\times n$ e $Y$ una matriz de $n'\times n'$.
  El \textbf{producto de Kronecker} $X\otimes Y$ es la matriz de
  $nn' \times nn'$ que consiste en bloques
  \[ \begin{pmatrix}
    x_{11} Y & \cdots & x_{1n} Y \\
    \vdots & \ddots & \vdots \\
    x_{n1} Y & \cdots & x_{nn} Y
  \end{pmatrix}. \]
  Demuestre que
  $$\det (X\otimes Y) = \det (X)^{n'} \cdot \det (Y)^n.$$

  \ifdefined\solutions
  \begin{solucion}
    Hay varios modos de verlo. Por ejemplo, si se trata de matrices complejas,
    recordamos que $X$ es diagonalizable si y solamente si el polinomio
    característico $p_X$ no tiene raíces múltiples. Esta condición corresponde
    a $\Delta (p_X) \ne 0$, donde $\Delta (p_X)$ es el discriminante, que será
    un polinomio en los coeficientes de $p_X$ (ocupando la fórmula
    $\Delta (f) = \pm\Res (f,f') = \det (\cdots)$), y entonces un polinomio
    en los coeficientes de $x_{ij}$. Entonces, el conjunto de matrices
    diagonalizables en $M_n (\CC)$ corresponde a la condición
    $\Delta (p_x)\ne 0$, y este conjunto es denso en $M_n (\CC)$. Entonces,
    bastaría probar la identidad
    $\det (X\otimes Y) = \det (X)^{n'} \cdot \det (Y)^n$ para las matrices
    diagonalizables.

    Ahora recordemos que los valores propios de $X\otimes Y$ son de la forma
    $\alpha\beta$, donde $\alpha$ es un valor propio de $X$ y $\beta$ es un
    valor propio de $Y$. De hecho, para los vectores propios correspondientes
    se tiene $X u = \alpha u$ e $Y v = \beta v$. Luego,
    $(X\otimes Y) (u\otimes v) = \alpha\beta\,u\otimes v$.

    Una vez sabemos esto, no hay que probar mucho. Al diagonalizar $X$ e $Y$,
    vemos que
    $$\det (X\otimes Y) = \prod_{i,j} \alpha_i \, \beta_j,$$
    donde $\alpha_i$ son los valores propios de $X$ y $\beta_j$ son los valores
    propios de $Y$. Ahora el último producto es precisamente
    \[ \Bigl(\prod_i \alpha_i\Bigr)^{n'} \cdot \Bigl(\prod_j \beta_j\Bigr)^n
    = \det (X)^{n'} \cdot \det (Y)^n. \]

    En general, la identidad se cumplirá para matrices con coeficientes en
    cualquier anillo conmutativo $R$ por la siguiente razón. Acabamos de probar
    que es cierta para $R = \CC$, pero luego es cierta para $R = \ZZ$, y podemos
    tratarla como una identidad de polinomios con coeficientes enteros en las
    variables $x_{ij}$ e $y_{k\ell}$. Tomando el cambio de base de $\ZZ$ a $R$,
    se concluye que la misma identidad se cumple para los polinomios con
    coeficientes en $R$.

    \vspace{1em}

    También podemos ocupar un cálculo directo. Notamos que la matriz
    $X\otimes Y$ es igual al producto
    $(X\otimes I_{n'})\,(I_n\otimes Y)$. Aquí la matriz $(I_n\otimes Y)$
    tiene forma
    \[ \begin{pmatrix}
      Y \\
      & \ddots \\
      & & Y
    \end{pmatrix} \]
    Además, después de un cambio de base, $X\otimes I_{n'}$ tiene la misma forma:
    si $e_1,\ldots,e_n$ es la base estándar de $R^n$
    y $f_1,\ldots,f_{n'}$ es la base estándar de $R^{n'}$, entonces
    basta escribir $X\otimes I_{n'}$ en la base
    \[ e_1\otimes f_1, \, e_1\otimes f_2, \ldots, e_1\otimes f_{n'}, \, e_2\otimes f_1, \, e_2\otimes f_2, \ldots \]
    Entonces,
    \[ \det (X\otimes Y) = \det (I_{n'}\otimes X)\cdot \det (I_n\otimes Y)
    = \det (X)^{n'} \cdot \det (Y)^n. \qedhere \]
  \end{solucion}
  \fi
\end{ejercicio}

\begin{ejercicio}
  Para el campo de números $K = \QQ (\sqrt{3},\zeta_5)$ calcule
  $\O_K$ y $\Delta_K$. ¿Cuáles primos racionales se ramifican en $K$?

  \ifdefined\solutions
  \begin{solucion}
    Tenemos $[\QQ (\sqrt{3}) : \QQ] = 2$ y
    $[\QQ (\zeta_5) : \QQ] = \phi (5) = 4$. Notamos que
    $\sqrt{3} \notin \QQ (\zeta_5)$: por ejemplo, si esto sería cierto,
    entonces $3$ (y también $2$) se ramificaría en $\QQ (\zeta_5)$, pero sabemos
    que allí se ramifica solo $5$. Estas consideraciones nos llevan a
    la conclusión que
    \[ [\QQ (\sqrt{3},\zeta_5) : \QQ] =
       [\QQ (\sqrt{3}) : \QQ]\cdot [\QQ (\zeta_5) : \QQ] = 8, \]
    y los campos son linealmente disjuntos. Ya sabemos cómo calcular los
    discriminantes:
    $$\Delta_{\QQ (\sqrt{3})} = 4\cdot 3, \quad \Delta_{\QQ (\zeta_5)} = 5^3.$$
    Los discriminantes son coprimos, así que una base de $\O_K$ es el
    producto de las bases:
    \[ 1, ~ \zeta_5, ~ \zeta_5^2, ~ \zeta_5^3, ~
       \sqrt{3}, ~ \sqrt{3}\zeta_5, ~ \sqrt{3}\zeta_5^2, ~ \sqrt{3}\zeta_5^3. \]
    En fin, el discriminante será
    $$\Delta_K = (2^2\cdot 3)^4\cdot (5^3)^2 = 2^8\cdot 3^4\cdot 5^6.$$
    Solo para confirmarlo, preguntémoslo a PARI/GP:
    \begin{framed}
\begin{verbatim}
? K=nfinit(t^2-3);
? L=rnfinit(K,polcyclo(5));
? nfinit(L).disc
% = 324000000
? factor (%)
% = 
[2 8]

[3 4]

[5 6]
\end{verbatim}
    \end{framed}
  \end{solucion}
  \fi
\end{ejercicio}

\begin{ejercicio}
  Consideremos los campos cuadráticos $K = \QQ (\sqrt{3})$ y
  $K' = \QQ (\sqrt{-5})$ y su compositum $KK' = \QQ (\sqrt{3},\sqrt{-5})$.
  Sea $\O = \ZZ \oplus \sqrt{3}\ZZ \oplus \sqrt{-5}\ZZ \oplus \sqrt{-15}\ZZ$.

  Calcule $\O_{KK'}$, $\Delta_{KK'}$ y el índice $[\O_{KK'} : \O]$.

  \ifdefined\solutions
  \begin{solucion}
    Tenemos $\Delta_K = 4\cdot 3$ y $\Delta_{K'} = 4\cdot 5$. De aquí sabemos
    que
    $$\Delta (\O) = (2^2\cdot 3)^2\cdot (2^2\cdot 5)^2 = 2^8\cdot 3^2\cdot 5^2.$$
    Sin embargo, los discriminantes no son coprimos, así que no podemos afirmar
    que $\O = \O_K$.

    De hecho, desde el principio se puede observar que se nos escapó un elemento
    entero bastante obvio: $3 \equiv -5 \equiv 3 \pmod{4}$, pero luego
    $-15\equiv 1\pmod{4}$, y el elemento $\frac{1+\sqrt{-15}}{2}$ es entero.
    Lo que no es tan obvio es que $\frac{\sqrt{3} + \sqrt{-5}}{2}$ es entero,
    pero lo podemos comprobar calculando directamente el polinomio mínimo.
    Primero, el polinomio mínimo de $\sqrt{3} + \sqrt{-5}$ es
    $$x^4 + 4x^2 + 64.$$

    Este es un caso particular del cálculo general: dados dos campos cuadráticos
    linealmente disjuntos $\QQ (\sqrt{d_1})$ y $\QQ (\sqrt{d_2})$, el polinomio
    mínimo de $\sqrt{d_1} + \sqrt{d_2}$ sobre $\QQ$ es su polinomio
    característico que viene dado por
    $$x^4 - 2\,(d_1+d_2)\,x^2 + (d_1 - d_2)^2.$$

    Ahora si $\sqrt{3} + \sqrt{-5}$ es una raíz de $x^4 + 4x^2 + 64$, entonces
    $\frac{\sqrt{3} + \sqrt{-5}}{2}$ es una raíz de $x^4 + x^2 + 4$. Esto
    demuestra la integralidad. Tiene sentido entonces considerar el anillo más
    grande
    $$\O' = \ZZ \oplus \sqrt{3}\ZZ \oplus \frac{\sqrt{3} + \sqrt{-5}}{2}\ZZ \oplus \frac{1 + \sqrt{-15}}{2}\ZZ.$$
    Para no equivocarnos y calcular bien el índice $[\O' : \O]$, notamos que la
    base de $\O$ se expresa en términos de la base de $\O'$ como
    \[ \begin{pmatrix}
      1 & 0 & 0 & -1 \\
      0 & 1 & -1 & 0 \\
      0 & 0 & 2 & 0 \\
      0 & 0 & 0 & 2
    \end{pmatrix}. \]
    El determinante de esta matriz es $4$, así que $[\O' : \O] = 4$ y luego
    $\Delta (\O') = \Delta (\O)/4^2 = 2^4\cdot 3^2\cdot 5^2$. Ahora
    $$2^4\cdot 3^2\cdot 5^2 = \Delta (\O') = [\O_{KK'} : \O']^2\cdot \Delta_{KK'}.$$
    Sabemos que $2$, $3$ se ramifican en $\QQ (\sqrt{3})$ y $2$, $5$ se
    ramifican en $\QQ (\sqrt{-5})$, así que necesariamente $2,3,5$ dividen a
    $\Delta_K$. Esto nos deja muy pocas posibilidades:
    $$[\O_{KK'} : \O'] = 1, 2.$$
    Entonces, bastaría considerar el cociente $\frac{1}{2}\O'/\O'$; es decir,
    ver cuáles elementos entre
    \begin{gather*}
      \frac{a_1}{2}\,\alpha_1 + \frac{a_2}{2}\,\alpha_2 + \frac{a_3}{2}\,\alpha_3 + \frac{a_4}{2}\,\alpha_4,\\
      \alpha_1 = 1, ~
      \alpha_2 = \sqrt{3}, ~
      \alpha_3 = \frac{\sqrt{3} + \sqrt{-5}}{2}, ~
      \alpha_4 = \frac{1 + \sqrt{-15}}{2}.
    \end{gather*}
    son enteros para $a_i = 0,1$.
    \begin{framed}\small
\begin{verbatim}
? K = nfinit(t^2-3);
? L = rnfinit(K,x^2+5);
? alpha2 = rnfeltreltoabs(L,t);
? alpha3 = rnfeltreltoabs(L,(t+x)/2);
? alpha4 = rnfeltreltoabs(L,(1+t*x)/2);

? { for(a1=0,1, for(a2=0,1, for(a3=0,1, for(a4=0,1,
        print([[a1,a2,a3,a4],
               minpoly((a1+a2*alpha2+a3*alpha3+a4*alpha4)/2)])
)))) };
[[0, 0, 0, 0], x]
[[0, 0, 0, 1], x^2 - 1/2*x + 1]
[[0, 0, 1, 0], x^4 + 1/4*x^2 + 1/4]
[[0, 0, 1, 1], x^4 - x^3 + 5/2*x^2 + 3/4*x + 9/16]
[[0, 1, 0, 0], x^2 - 3/4]
[[0, 1, 0, 1], x^4 - x^3 + 3/4*x^2 - 1/4*x + 23/8]
[[0, 1, 1, 0], x^4 - 11/4*x^2 + 4]
[[0, 1, 1, 1], x^4 - x^3 - 1/2*x^2 + 6*x + 6]
[[1, 0, 0, 0], x - 1/2]
[[1, 0, 0, 1], x^2 - 3/2*x + 3/2]
[[1, 0, 1, 0], x^4 - 2*x^3 + 7/4*x^2 - 3/4*x + 3/8]
[[1, 0, 1, 1], x^4 - 3*x^3 + 11/2*x^2 - 3*x + 1]
[[1, 1, 0, 0], x^2 - x - 1/2]
[[1, 1, 0, 1], x^4 - 3*x^3 + 15/4*x^2 - 9/4*x + 27/8]
[[1, 1, 1, 0], x^4 - 2*x^3 - 5/4*x^2 + 9/4*x + 27/8]
[[1, 1, 1, 1], x^4 - 3*x^3 + 5/2*x^2 + 21/4*x + 49/16]
\end{verbatim}
    \end{framed}

    No hay elementos enteros adicionales, y entonces podemos concluir que
    $\O_{KK'} = \O'$, $\Delta_{KK'} = 2^4\cdot 3^2\cdot 5^2$,
    $[\O_{KK'} : \O] = 4$.
  \end{solucion}
  \fi
\end{ejercicio}

\begin{ejercicio}
  Calcule que
  \[ \Delta (\ZZ [\zeta_{p^e}]) = \Delta (\Phi_{p^e}) = \pm\,p^s,
     \quad \text{donde } s = p^{e-1}\,(pe - e - 1). \]
  ¿Cuál es el signo?

  \ifdefined\solutions
  \begin{solucion}
    Tenemos
  \[ \Delta (1, \zeta, \ldots, \zeta^{d-1}) = \Delta (\Phi_{p^e})
     = \pm N_{K/\QQ} (\Phi_{p^e}' (\zeta)). \]
  Ahora
  $$x^{p^e} - 1 = (x^{p^{e-1}} - 1)\,\Phi_{p^e} (x),$$
  Tomando las derivadas,
  $$p^e\,x^{p^e-1} = p^{e-1}\,x^{p^{e-1}-1}\,\Phi_{p^e} (x) + (x^{p^{e-1}} - 1)\,\Phi_{p^e}' (x).$$
  Al sustituir $\zeta$ en lugar de $x$, nos queda
  $$p^e\,\zeta^{p^e-1} = (\zeta_p - 1)\,\Phi_{p^e}' (\zeta).$$
  Dado que $\zeta$ es invertible, tenemos $N_{K/\QQ} (\zeta) = \pm 1$.
  Ahora para calcular la norma de $\zeta_p - 1$, notamos que
  \[ [\QQ (\zeta_{p^e}) : \QQ (\zeta_p)] =
     \frac{[\QQ (\zeta_{p^e}) : \QQ]}{[\QQ (\zeta_p) : \QQ]} =
     \frac{\phi (p^e)}{\phi (p)} = p^{e-1}. \]
  (Recordamos que $\phi (p^e) = (p-1)\,p^{e-1}$.) Entonces,
  \[ N_{K/\QQ} (\zeta_p - 1) =
     N_{\QQ (\zeta_p)/\QQ} (\zeta_p - 1)^{p^{e-1}} = p^{p^{e-1}}. \]
  Esto nos lleva a la fórmula
  $$N_{K/\QQ} (\Phi_{p^e}' (\zeta)) = \pm \frac{(p^e)^{\phi (p^e)}}{p^{p^{e-1}}} = \pm p^s,$$
  donde
  $$s = e\,\phi (p^e) - p^{e-1} = p^{e-1} \, (pe- e - 1).$$
  El signo del discriminante sale del teorema de Brill: este es
  $(-1)^{\phi (p^e)/2}$.
  \end{solucion}
  \fi
\end{ejercicio}

\begin{ejercicio}
  Demuestre que si $n = m p^e$, donde $p \nmid m$, entonces se cumple
  la congruencia
  $$\Phi_n (x) \equiv \Phi_m (x)^{\phi (p^e)} \pmod{p}.$$

  \ifdefined\solutions
  \begin{solucion}
    Se trata de polinomios con coeficientes enteros, así que en teoría podríamos
    resolver el ejercicio ocupando fórmulas como
    $x^n - 1 = \prod_{d\mid n} \Phi_d (x)$ que definen los polinomios
    ciclotómicos. Procedamos por inducción sobre $m$ y $e$.
    Por ejemplo, si $m = 1$, entonces la fórmula sí se cumple:
    \[ \Phi_{p^e} (x) = \Phi_p (x^{p^{e-1}}) = \frac{x^{p^e} - 1}{x^{p^{e-1}} - 1} \equiv (x-1)^{p^e - p^{e-1}} = (x-1)^{\phi (p^e)} \pmod{p}. \]
    De la misma manera, si $e = 0$, entonces la fórmula se vuelve trivial.

    Para el paso inductivo, escribamos
    \[ \Phi_n (x) = (x^n-1)\left/\prod_{\substack{d \mid n \\ d < n}} \Phi_d (x)\right.
    = (x^n-1)\left/\Bigl(\prod_{\substack{d \mid m \\ d < m}}\prod_{0 \le k \le e} \Phi_{d p^k} (x)\prod_{0 \le k \le e-1} \Phi_{m p^k} (x)\Bigr)\right. \]
    Reduciendo módulo $p$, y usando la identidad
    $\sum_{0 \le k \le e} \phi (p^k) = p^e$,
    \[ \Phi_n (x) \equiv \cfrac{\left((x^m - 1)\left/\prod_{\substack{d \mid m \\ d < m}} \Phi_d (x)\right.\right)^{p^e}}{\Phi_m (x)^{p^{e-1}}} = \Phi_m (x)^{\phi (p^e)} \pmod{p}. \qedhere \]
  \end{solucion}
  \fi
\end{ejercicio}

\end{document}

\documentclass{article}

\usepackage[utf8]{inputenc}
\usepackage[spanish]{babel}

\usepackage{amsmath,amssymb,amsthm}

\usepackage{tikz-cd}
\usetikzlibrary{babel}

\newcounter{tarea}
\setcounter{tarea}{5}
\theoremstyle{definition}
\newtheorem{ejercicio}{Ejercicio}[tarea]

\newenvironment{solucion}{\begin{proof}[Solución]}{\end{proof}}

\usepackage{array}
\newcolumntype{x}[1]{>{\centering\hspace{0pt}}p{#1}}

\usepackage{mathspec}

\setmainfont{PT Serif}
\setsansfont{Montserrat}
\setmonofont{PT Mono}

\newcommand{\ZZ}{\mathbb{Z}}
\newcommand{\QQ}{\mathbb{Q}}
\renewcommand{\O}{\mathcal{O}}

\title{Teoría de números algebraicos\\Tarea 5}
\author{Alexey Beshenov (alexey.beshenov@cimat.mx)}
\date{23 de septiembre de 2020}

% \def\solutions{true}

\begin{document}

{\sffamily\bfseries\maketitle}

\noindent Fecha límite: viernes, 2 de octubre.

\ifdefined\solutions
\else
\thispagestyle{empty}
\fi

\vspace{1em}

\begin{ejercicio}
  Sea $X$ una matriz de $n\times n$ e $Y$ una matriz de $n'\times n'$.
  El \textbf{producto de Kronecker} $X\otimes Y$ es la matriz de
  $nn' \times nn'$ que consiste en bloques
  \[ \begin{pmatrix}
    x_{11} Y & \cdots & x_{1n} Y \\
    \vdots & \ddots & \vdots \\
    x_{n1} Y & \cdots & x_{nn} Y
  \end{pmatrix}. \]

  Demuestre que
  $$\det (X\otimes Y) = \det (X)^{n'} \cdot \det (Y)^n.$$
\end{ejercicio}

\begin{ejercicio}
  Para el campo de números $K = \QQ (\sqrt{-3},\zeta_5)$ calcule
  $\O_K$ y $\Delta_K$. ¿Cuáles primos racionales se ramifican en $K$?
\end{ejercicio}

\begin{ejercicio}
  Consideremos los campos cuadráticos $K = \QQ (\sqrt{3})$ y
  $K' = \QQ (\sqrt{5})$ y su compositum $KK' = \QQ (\sqrt{3},\sqrt{5})$.
  Sea $\O = \ZZ \oplus \sqrt{3}\ZZ \oplus \sqrt{5}\ZZ \oplus \sqrt{15}\ZZ$.

  \begin{enumerate}
  \item[1)] Calcule el índice $[\O_K : \O]$.

  \item[2)] Calcule $\O_K$ y $\Delta_K$.
  \end{enumerate}
\end{ejercicio}

\begin{ejercicio}
  Calcule que
  \[ \Delta (\ZZ [\zeta_{p^e}]) = \Delta (\Phi_{p^e}) = \pm\,p^s,
     \quad \text{donde } s = p^{e-1}\,(pe - e - 1). \]
\end{ejercicio}

\begin{ejercicio}
  Demuestre que si $n = m p^e$, donde $p \nmid m$, entonces se cumple
  la congruencia
  $$\Phi_n (x) \equiv \Phi_m (x)^{\phi (p^e)} \pmod{p}.$$
\end{ejercicio}

\end{document}

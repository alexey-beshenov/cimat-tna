\documentclass{article}

\usepackage[utf8]{inputenc}
\usepackage[spanish]{babel}

\usepackage{fullpage}

\usepackage[perpage,symbol]{footmisc}
\renewcommand{\thefootnote}{\ifcase\value{footnote}\or{*}\or{**}\or{***}\or{****}\fi}

\usepackage{amsmath,amssymb,amsthm}

\theoremstyle{definition}
\newtheorem{ejercicio}{Ejercicio}

\newenvironment{solucion}{\begin{proof}[Solución]}{\end{proof}}

\usepackage{framed}

\usepackage{mathspec}

\setmainfont{PT Serif}
\setsansfont{Montserrat}
\setmonofont{PT Mono}

\DeclareMathOperator{\Gal}{Gal}
\DeclareMathOperator{\Cl}{Cl}

\newcommand{\ZZ}{\mathbb{Z}}
\newcommand{\QQ}{\mathbb{Q}}
\renewcommand{\O}{\mathcal{O}}

\title{Teoría de números algebraicos\\Examen final}
\author{Alexey Beshenov (alexey.beshenov@cimat.mx)}
\date{9 de diciembre de 2020}

% \def\solutions{true}

\begin{document}

{\sffamily\bfseries\maketitle}

\ifdefined\solutions
\else
\thispagestyle{empty}
\fi

\noindent\emph{Fecha límite: 16 de diciembre de 2020.}

\begin{ejercicio}
  Consideremos el polinomio $f = x^3 + 5x^2 - x - 4$.

  \begin{enumerate}
  \item[0)] Demuestre que $f$ es irreducible en $\QQ [x]$.
    Sea $K = \QQ [x]/(f)$.

  \item[1)] Calcule el anillo de enteros $\O_K$ y discriminante $\Delta_K$.

  \item[2)] Demuestre que $u_1 = \alpha + 1$ y $u_2 = \alpha - 1$, donde
    $\alpha = x \mod f$, Son unidades en $\O_K^\times$. Asumiendo que $u_1$ y
    $u_2$ generan la parte libre de $\O_K^\times$, calcule el regulador.

  \item[3)] Calcule el grupo de clases $\Cl (K)$.

  \item[4)] Usando la fórmula analítica del número de clases\footnote{El residuo
      de $\zeta_K (s)$ en $s=1$ puede ser calculado en PARI/GP.}, compruebe que
    $u_1$ y $u_2$ son efectivamente unidades fundamentales.
  \end{enumerate}
\end{ejercicio}

\begin{ejercicio}
  Para un campo de números $K/\QQ$ demuestre que la cerradura de Galois $L/K$
  contiene subcampo $\QQ (\sqrt{\Delta_K})$. Dé un ejemplo particular cuando
  $\Delta_K$ no es un cuadrado y $K \ne \QQ (\sqrt{\Delta_K})$.
\end{ejercicio}

\begin{ejercicio}
  Sea $k > 0$ un entero positivo libre de cuadrados. Supongamos que
  $k \equiv 1,2 \pmod{4}$ y $k$ no tiene forma $3a^2 \pm 1$ para $a \in \ZZ$.
  Demuestre que si $3 \nmid h_{\QQ (\sqrt{-k})}$, entonces la ecuación
  $y^2 = x^3 - k$ no tiene soluciones enteras.

  Punto extra: encuentre un contraejemplo para $3 \mid h_{\QQ (\sqrt{-k})}$.
\end{ejercicio}

\begin{ejercicio}
  Dada una extensión ciclotómica $\QQ (\zeta_m)$, sean
  $X \subseteq \widehat{(\ZZ/m\ZZ)^\times}$ un grupo de caracteres de Dirichlet
  y $K \subseteq \QQ (\zeta_m)$ el subcampo correspondiente. Demuestre que $K$
  es un campo real (es decir, $r_2 = 0$) si y solamente si $\chi (-1) = +1$ para
  todo $\chi \in X$.
\end{ejercicio}

\begin{ejercicio}
  Consideremos el campo cuadrático real $K = \QQ (\sqrt{3})$.
  \begin{enumerate}
  \item[1)] Calcule el residuo de $\zeta_K (s)$ en $s = 1$.

  \item[2)] Exprese $\zeta_K (s)$ como un producto de series L de Dirichlet.

  \item[3)] Calcule los valores
    $\zeta_K (0)$, $\zeta_K (-1)$, $\zeta_K (-2)$, $\zeta_K (-3)$.

  \item[4)] Calcule los valores $\zeta_K (2)$ y $\zeta_K (4)$.
  \end{enumerate}
\end{ejercicio}

\end{document}

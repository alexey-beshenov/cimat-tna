\chapter{Función zeta de Dedekind}

Para un campo de números $K/\QQ$, la \textbf{función zeta de Dedekind}
correspondiente es una especie de función generatriz definida por la serie
$$\zeta_K (s) = \sum_{I \ne 0} \frac{1}{N_{K/\QQ} (I)^s},$$
donde la suma se toma sobre los ideales enteros no nulos $I \subseteq \O_K$.
Resulta que la función zeta de Dedekind codifica mucha información aritmética
sobre $K$. Nos interesará $\zeta_K (s)$ como un objeto analítico, y en
particular hay que ver la convergencia de la serie.

Primero notamos que si $K = \QQ$, entonces los ideales no nulos en $\O_K = \ZZ$
tienen forma $n\ZZ$ para $n = 1,2,3,\ldots$, así que $\zeta_\QQ (s) = \zeta (s)$
es la función zeta de Riemann.

\begin{lema}
  La serie $\zeta (s) = \sum_{n\ge 1} \frac{1}{n^s}$ converge absolutamente para
  $\Re s > 1$, y se tiene
  $$\lim_{s \to 1^+} (s - 1)\,\zeta (s) = 1.$$

  \begin{proof}
    Primero, notamos que $|1/n^s| = 1/n^{\Re s}$, así que será suficiente
    establecer la convergencia para $s > 1$ real. Tenemos
    $$\int_n^{n+1} \frac{dx}{x^s} \le \frac{1}{n^s} \le \int_{n-1}^n \frac{dx}{x^s},$$
    donde la primera desigualdad se cumple para $n \ge 1$ y la segunda para
    $n \ge 2$. Tomando la suma sobre $n \ge 1$, nos sale la desigualdad
    $$\int_1^\infty \frac{dx}{x^s} \le \zeta (s) \le 1 + \int_1^\infty \frac{dx}{x^s}.$$
    Ahora, dado que $\int_1^\infty \frac{dx}{x^s} = \frac{1}{s - 1}$, tenemos
    $$\frac{1}{s - 1} \le \zeta (s) \le \frac{s}{s - 1},$$
    y luego
    $$1 \le (s-1)\,\zeta (s) \le s.$$
    Esto establece la convergencia y también calcula $(s-1)\,\zeta (s)$ para
    $s \to 1^+$.
  \end{proof}
\end{lema}

Ahora podemos probar la convergencia de $\zeta_K (s)$.

\begin{proposicion}
  La serie $\zeta_K (s) = \sum_{I \ne 0} \frac{1}{N_{K/\QQ} (I)^s}$ converge
  absolutamente para $\Re s > 1$. Además, se cumple la
  \textbf{fórmula del producto de Euler}
  \[ \sum_{I \ne 0} \frac{1}{N_{K/\QQ} (I)^s} =
     \prod_{\mathfrak{p}} \frac{1}{1 - N_{K/\QQ} (\mathfrak{p})^{-s}}, \]
  donde el producto es sobre todos los ideales primos
  $\mathfrak{p} \subset \O_K$.

  \begin{proof}
    De nuevo, la convergencia absoluta para $\Re s > 1$ se sigue de la
    convergencia para $s > 1$ real. Será suficiente probar la convergencia
    absoluta del producto de Euler
    $$\prod_{\mathfrak{p}} \frac{1}{1 - N_{K/\QQ} (\mathfrak{p})^{-s}},$$
    y luego,
    \[ \prod_{\mathfrak{p}} \frac{1}{1 - N_{K/\QQ} (\mathfrak{p})^{-s}} =
       \prod_{\mathfrak{p}} \sum_{e \ge 1} \frac{1}{N_{K/\QQ} (\mathfrak{p}^e)^s} =
       \sum_{I\ne 0} \frac{1}{N_{K/\QQ} (I)^s}. \]
    Aquí hemos usado la serie geométrica, la multiplicatividad de la norma, y
    que todo ideal entero no nulo $I \subseteq \O_K$ tiene factorización única
    en ideales primos $I = \mathfrak{p}_1^{e_1}\cdots \mathfrak{p}_s^{e_s}$.

    Un producto $\prod_{n\ge 1} (1 + |x_n|)$ converge si y solamente si la serie
    $\sum_{n\ge 1} |x_n|$ converge. Entonces, la convergencia del producto de
    Euler se sigue de la convergencia de
    $\prod_{\mathfrak{p}} (1 - N_{K/\QQ} (\mathfrak{p})^{-s})$, y luego de
    $\sum_{\mathfrak{p}} \frac{1}{N_{K/\QQ} (\mathfrak{p})^s}$.

    Recordemos que para todo primo racional $p$, existen $\le [K:\QQ]$ ideales
    primos $\mathfrak{p} \mid p$, y luego para cada uno de estos se tiene
    $N_{K/\QQ} (\mathfrak{p}) = p^f \ge p$. Entonces,
    \[ \sum_{\mathfrak{p}} \frac{1}{N_{K/\QQ} (\mathfrak{p})^s} =
       \sum_p \sum_{\mathfrak{p} \mid p} \frac{1}{N_{K/\QQ} (\mathfrak{p})^s} \le
       \sum_p \frac{[K:\QQ]}{p^s} < [K : \QQ]\,\zeta (s). \]
    Aquí hemos usado la convergencia de la función zeta de Riemann.
  \end{proof}
\end{proposicion}

%%%%%%%%%%%%%%%%%%%%%%%%%%%%%%%%%%%%%%%%%%%%%%%%%%%%%%%%%%%%%%%%%%%%%%%%%%%%%%%%

\section{Ejemplo: la función zeta de \texorpdfstring{$\QQ (i)$}{ℚ(i)}}
\label{sec:funcion-zeta-de-Q(i)}

Tomemos $K = \QQ (i)$. Para entender cómo se ve la función zeta correspondiente
$\zeta_K (s)$, sería más fácil trabajar con el producto de Euler
$$\prod_{\mathfrak{p}} \frac{1}{1 - N_{K/\QQ} (\mathfrak{p})^{-s}}.$$
Los primos en $\O_K = \ZZ[i]$ son los siguientes.
\begin{itemize}
\item Al primo ramificado $p = 2$ corresponde un ideal primo $\mathfrak{p}_2$
  de norma $2$.

\item Si $p \equiv 1 \pmod{4}$, entonces $p$ se escinde en dos ideales primos
  $\mathfrak{p}$ y $\overline{\mathfrak{p}}$, cada uno de norma $p$.

\item Si $p \equiv 3 \pmod{4}$, entonces $p$ es inerte y corresponde a un ideal
  primo $\mathfrak{p}$ de norma $p^2$.
\end{itemize}

Entonces,
\[ \zeta_K (s) =
   \frac{1}{1 - 2^{-s}} \,
   \prod_{p \equiv 1 ~ (4)} \frac{1}{(1 - p^{-s})^2} \,
   \prod_{p \equiv 1 ~ (4)} \frac{1}{1 - p^{-2s}}. \]
Notamos que $\frac{1}{1 - p^{-2s}} = \frac{1}{1 - p^{-s}}\,\frac{1}{1 + p^{-s}}$,
así que
$$\zeta_K (s) = \prod_p \frac{1}{1 - p^{-s}}\,\prod_p \frac{1}{1 - \chi(p)\,p^{-s}},$$
donde
\[ \chi(p) = \begin{cases}
  +1, & p \equiv 1 \pmod{4},\\
  -1, & p \equiv 3 \pmod{4}.
\end{cases} \]
Este es un \textbf{carácter de Dirichlet} mód $4$, y a este se asocia la
\textbf{serie L de Dirichlet} correspondiente
\[ L (s,\chi) = \prod_p \frac{1}{1 - \chi(p)\,p^{-s}} =
   \sum_{n\ge 1} \frac{\chi (s)}{n^s}. \]
Entonces, se tiene
$$\zeta_K (s) = \zeta (s) \, L (s,\chi).$$
Invito que el lector revise el apéndice~\ref{ap:Dirichlet} para más detalles.

Recordamos que $\O_K = \ZZ [i]$ es un dominio de ideales principales y
$\O_K^\times = \{ \pm 1, \pm i \}$. Como consecuencia, todo ideal tiene forma
$I = (\alpha)$, y para $I \ne 0$ hay precisamente cuatro maneras de escoger un
generador $\alpha$. Si $\alpha = x + yi$, entonces $N_{K/\QQ} (I) = x^2 + y^2$.
Pongamos
$$C (n) = \# \{ (x,y) \in \ZZ^2 \mid x,y > 0, ~ x^2 + y^2 = n \}.$$
Esta función es responsable por contar los puntos enteros en el círculo de radio
$\sqrt{n}$.

Tenemos entonces una identidad con series de Dirichlet
\[ \zeta_K (s) = \sum_{n\ge 0} \frac{C (n)}{n^s} =
\Bigl(\sum_{n\ge 1} \frac{1}{n^s}\Bigr)\,\Bigl(\sum_{n\ge 1} \frac{\chi (n)}{n^s}\Bigr). \]
Comparando los coeficientes, nos sale
$$C (n) = \sum_{d \mid n} \chi (d).$$
De la multiplicatividad del carácter $\chi$ se sigue también que
$C (mn) = C (m)\,C (n)$ si $m$ y $n$ son coprimos. Calculando ahora $C (n)$ para
$n = p^e$, se obtiene el siguiente curioso resultado.

\begin{teorema}
  Supongamos que $n = p_1^{e_1}\cdots p_k^{e_k}$, donde los $p_i$ son diferentes
  primos.
  \begin{itemize}
  \item Si $p_i \equiv 3 \pmod{4}$ para algún $i$ y $e_i$ es impar, entonces
    el círculo de radio $\sqrt{n}$ no contiene ningún punto entero.

  \item En el caso contrario, el número de puntos enteros en el círculo será
    $4\,C (n)$, donde
    $$C (n) = \prod_{p_i \equiv 1~(4)} (e_i + 1).$$
  \end{itemize}
\end{teorema}

\begin{figure}
  \begin{center}
    \includegraphics{pic/circle-points.pdf}
  \end{center}

  \caption{$12$ puntos enteros en el círculo de radio $5$}
\end{figure}

\begin{comentario}
  En general, para cualquier extensión finita abeliana $K/\QQ$ se cumple
  una descomposición en ciertas series L de Dirichlet
  $$\zeta_K (s) = \prod_\chi L (s,\chi).$$
  En particular, para el carácter trivial sale el factor $\zeta (s)$.
  Más adelante veremos que $\zeta_K (s)$ determina la factorización de primos
  racionales en $\O_K$. La fórmula de arriba significa que
  \emph{en el caso abeliano} la factorización depende de $p$ mód $N$ para algún
  $N$. El caso no-abeliano es más complicado.

  Véase por ejemplo \cite[Chapter~4]{Washington-GTM83}.
\end{comentario}

%%%%%%%%%%%%%%%%%%%%%%%%%%%%%%%%%%%%%%%%%%%%%%%%%%%%%%%%%%%%%%%%%%%%%%%%%%%%%%%%

\section{Fórmula analítica del número de clases}

Nuestro próximo gran objetivo será calcular el residuo en $s = 1$:
\[ \lim_{s \to 1^+} (s - 1)\,\zeta_K (s) =
   \frac{2^{r_1}\,(2\pi)^{r_2}\,\Reg_K\,h_k}{\#\mu_K\,\sqrt{|\Delta_K|}}. \]
Esta fórmula contiene todos los invariantes básicos de $K$ que hemos
considerado.
\begin{itemize}
\item El número de encajes reales $r_1$ y el número de pares de encajes
  complejos $r_2$.

\item El número de clases $h_K = \# \Cl (K)$.

\item El número de las raíces de la unidad $\# \mu_K = \# (\O_K^\times)_{tors}$.

\item El discriminante $\Delta_K$. Específicamente, aparece $\sqrt{|\Delta_K|}$
  que es el covolumen de $\O_K$ realizado como un retículo en el espacio
  $K_\RR$.

\item En fin, $\Reg_K$ es el \textbf{regulador} que, salvo una normalización,
  corresponde al covolumen de la parte libre del grupo de unidades
  $\O_K^\times$, realizada como un retículo en el espacio $H$ de dimensión
  $r_1 + r_2 - 1$. En la siguiente sección vamos a dar una definición más
  precisa.
\end{itemize}

Para $K = \QQ$ se pone $\Reg_\QQ = 1$, y la fórmula se reduce a
$\lim_{s \to 1^+} (s - 1)\,\zeta (s) = 1$.

\vspace{1em}

La fórmula del número de clases fue descubierta por Dirichlet para el caso de
campos cuadráticos, y la versión general es de Dedekind.

%%%%%%%%%%%%%%%%%%%%%%%%%%%%%%%%%%%%%%%%%%%%%%%%%%%%%%%%%%%%%%%%%%%%%%%%%%%%%%%%

\section{Regulador}

Vamos a ver con más detalle qué es el regulador y cómo calcularlo.
Recordemos nuestra prueba del teorema de unidades en
\S\ref{sec:teorema-de-unidades} con el encaje logarítmico de $\O_K^\times$.
Será conveniente numerar diferentes encajes $K\hookrightarrow \CC$ por
$$\sigma_1,\ldots,\sigma_{r_1},\sigma_{r_1+1},\overline{\sigma_{r_1+1}},\ldots,\sigma_{r_1+r_2},\overline{\sigma_{r_1+r_2}},$$
donde los primeros $r_1$ encajes son reales y el resto son complejos.

Tenemos la aplicación
\[ \Phi\colon K^\times \hookrightarrow K_\RR^\times, \quad
   \alpha \mapsto (\sigma_i (\alpha))_i, \]
y el encaje logarítmico
\[ \ell\colon K_\RR^\times \to \RR^{r_1 + r_2}, \quad
   (z_{\sigma_i}) \mapsto (n_i\,\log |z_{\sigma_i}|)_i, \]
donde
\[ n_i = \begin{cases}
  1, & \text{si }1 \le i \le r_1,\\
  2, & \text{si }r_1 < i \le r_1 + r_2.
  \end{cases} \]
Esto nos da el diagrama conmutativo
\[ \begin{tikzcd}
  \O_K^\times\ar[right hook->]{d}\ar{rr}{L} & & H\ar[right hook->]{d} \\
  K^\times \ar[right hook->]{r}{\Phi}\ar{d}[swap]{N_{K/\QQ}} & K_\RR^\times \ar{r}{\ell} & \RR^{r_1 + r_2} \ar{d}{\sum} \\
  \QQ^\times \ar{rr}{\log |\cdot|} & & \RR
\end{tikzcd} \]

Hemos probado que la aplicación $L$ realiza la parte libre de $\O_K^\times$ como
un retículo de rango completo $r = r_1 + r_2 - 1$ en el subespacio
$$H = \{ x \in \RR^{r_1 + r_2} \mid \sum_i x_i = 0 \}.$$
Como consecuencia, $L (u_1), \ldots, L (u_r)$ forman una base
de $H$, y podemos completarla a una base de $\RR^{r_1+r_2}$ añadiendo el vector
$$L = \frac{1}{\sqrt{r_1 + r_2}} \, (1,\ldots,1) \in \RR^{r_1 + r_2}.$$
El vector $(1,\ldots,1)$ es ortogonal a $H$, y con la normalización de
arriba, la longitud de $L$ es $1$. Esto significa que el covolumen de
$L (\O_K^\times)$ en $H$ es igual al volumen del paralelepípedo en
$\RR^{r_1+r_2}$ generado por los vectores $L, L (u_1), \ldots, L (u_r)$; es
decir,
\[ \Reg_K = \covol L (\O_K^\times) = \pm \det \begin{pmatrix}
  L_1 & L_1 (u_1) & L_1 (u_2) & \cdots & L_1 (u_{r_1+r_2-1}) \\
  L_2 & L_2 (u_1) & L_2 (u_1) & \cdots & L_2 (u_{r_1+r_2-1}) \\
  \vdots & \vdots & \vdots & \ddots & \vdots \\ 
  L_{r_1 + r_2} & L_{r_1 + r_2} (u_1) & L_{r_1+r_2} (u_2) & \cdots & L_{r_1+r_2-1} (u_{r_1+r_2-1})
\end{pmatrix} \]
Ahora podemos sumar a la $i$-ésima fila de la matriz de arriba todas las filas.
Como resultado, en la $i$-ésima fila estará el vector
$$(1/\sqrt{r_1 + r_2}, 0, \ldots, 0).$$
Esto nos lleva al siguiente resultado.

\begin{proposicion-definicion}
  El covolumen del retículo $L (\O_K^\times)$ en $H$ es igual a
  $$\frac{1}{\sqrt{r_1 + r_2}} \, \Reg_K,$$
  donde $\Reg_K$ es el valor absoluto del determinante de cualquier menor
  de rango $r_1 + r_2 - 1$ de la matriz
  \[ (L_i (u_j))_{\substack{1 \le i \le r_1 + r_2 \\ 1 \le j \le r_1 + r_2 - 1}} =
     (n_i \log |\sigma_i (u_j)|)_{\substack{1 \le i \le r_1 + r_2 \\ 1 \le j \le r_1 + r_2 - 1}}. \]
\end{proposicion-definicion}

\begin{ejemplo}
  Si $K = \QQ (\sqrt{d})$ es un campo cuadrático real, entonces su regulador
  será $\log |u|$, donde $u$ es la unidad fundamental de $\O_K^\times$.
\end{ejemplo}

\begin{ejemplo}
  Para el campo $K = \QQ (\zeta_7)$ podemos calcular con ayuda de computadora
  que como unidades fundamentales, se pueden tomar
  $$u_1 = 1 + \zeta_7, \quad u_2 = \zeta_7 + \zeta_7^4.$$
  En este caso $r_1 = 0$ y $r_2 = 3$. Los encajes complejos, salvo conjugación,
  serán
  \[ \sigma_1\colon \zeta_7 \mapsto \zeta_7, \quad
     \sigma_2\colon \zeta_7 \mapsto \zeta_7^2, \quad
     \sigma_3\colon \zeta_7 \mapsto \zeta_7^3. \]
  Tenemos entonces
  \[ \Reg_K = \pm\det\begin{pmatrix}
    2\,\log |1 + \zeta_7| & 2\,\log |\zeta_7 + \zeta_7^4| \\
    2\,\log |1 + \zeta_7^2| & 2\,\log |\zeta_7^2 + \zeta_7| \\
  \end{pmatrix} = 2.101818\ldots \qedhere \]
\end{ejemplo}

%%%%%%%%%%%%%%%%%%%%%%%%%%%%%%%%%%%%%%%%%%%%%%%%%%%%%%%%%%%%%%%%%%%%%%%%%%%%%%%%

\section{Ejemplos de uso de la fórmula del número de clases}

Antes de probar la fórmula, podemos ver algunos ejemplos de su uso. Consideremos
el campo cuadrático real $K = \QQ (\sqrt{5})$. Argumentando de la misma manera
que en \S\ref{sec:funcion-zeta-de-Q(i)}, se demuestra la identidad
$$\zeta_K (s) = \zeta (s) \, L (s,\chi),$$
donde $\chi (n) = \legendre{n}{5}$.
Ahora $L (s,\chi)$ converge en $s = 1$ a un valor no nulo, mientras que
$\lim_{s\to 1^+} (s - 1)\,\zeta (s) = 1$, así que
$$\lim_{s \to 1^+} (s - 1)\,\zeta_K (s) = L (1,\chi).$$

En general, para el carácter cuadrático $\chi (n) = \legendre{n}{p}$ mód $p$
se tiene
$$\exp (g (\chi)\,L (1,\chi)) = \prod_n (1 - \zeta_p^n)\,\prod_r (1 - \zeta_p)^{-1},$$
donde
$$g (\chi) = \sum_{1 \le a \le p-1} \chi (a)\,\zeta_p^a$$
es la \textbf{suma de Gauss}, y los productos son sobre los no-residuos y
residuos cuadráticos mód $p$ respectivamente. Para la prueba, véase
el ejercicio~\ref{ejerc:L-series-caracter-cuadratico-mod-p}.

En nuestro caso calculamos que
$$g (\chi) = \zeta_5 - \zeta_5^2 - \zeta_5^3 + \zeta_5^4 = \sqrt{5}$$
y
$$(1 - \zeta_p^2)\,(1 - \zeta_5^3)\,(1 - \zeta_5)^{-1}\,(1 - \zeta_5^4)^{-1} = \frac{3 + \sqrt{5}}{2}.$$
Entonces,
\begin{equation}
  \label{eq:formula-de-clases-sqrt5-1}
  \lim_{s \to 1^+} (s - 1)\,\zeta_K (s) =
  L (1, \chi) = \frac{1}{\sqrt{5}}\,\log\frac{3 + \sqrt{5}}{2}.
\end{equation}

El regulador en este caso será igual a $\log u$, donde
$u = \frac{1+\sqrt{5}}{2}$ es la unidad fundamental de $\O_K^\times$.
Tenemos $r_1 = 2$, $r_2 = 0$ y $\mu_K = \{ \pm 1 \}$, $\Delta_K = 5$, y la
fórmula del número de clases nos da entonces
\begin{equation}
  \label{eq:formula-de-clases-sqrt5-2}
  \lim_{s \to 1^+} (s - 1)\,\zeta_K (s) =
  \frac{2^{r_1}\,(2\pi)^{r_2}\,\Reg_K\,h_k}{\#\mu_K\,\sqrt{|\Delta_K|}} =
  \frac{1}{\sqrt{5}}\,h_K\cdot 2\,\log \frac{1+\sqrt{5}}{2}.
\end{equation}
En fin,
\[ 2\,\log \frac{1+\sqrt{5}}{2} =
   \log \left(\frac{1+\sqrt{5}}{2}\right)^2 =
   \log \frac{3+\sqrt{5}}{2}. \]
Comparando \eqref{eq:formula-de-clases-sqrt5-1} y
\eqref{eq:formula-de-clases-sqrt5-2}, podemos concluir que $h_K = 1$.

\vspace{1em}

Si $K = \QQ[x]/(f)$, entonces para calcular el residuo de $\zeta_K (s)$ en
$s = 1$ en PARI/GP, basta digitar \texttt{lfun($f$, 1)}. Calculamos el residuo
para $K = \QQ (\sqrt{10})$.
\begin{shaded}
\begin{verbatim}
? lfun (x^2 - 10, 1)
% = 1.1500865228483708943221826442284221318*x^-1 + O(x^0)
\end{verbatim}
\end{shaded}

La unidad fundamental en este caso es $u = 3 + \sqrt{10}$, el discriminante
es $\Delta_K = 40$, así que
\[ \lim_{s \to 1^+} (s - 1)\,\zeta_K (s) =
   \frac{2^{r_1}\,(2\pi)^{r_2}\,\Reg_K\,h_k}{\#\mu_K\,\sqrt{|\Delta_K|}} =
   \frac{2}{\sqrt{40}}\,\log (3 + \sqrt{10})\cdot h_K. \]
\begin{shaded}
\begin{verbatim}
? 2 / sqrt (40) * log (3 + sqrt(10))
% = 0.57504326142418544716109132211421106589
? polcoef (lfun (x^2 - 10, 1), -1) / %
% = 2.0000000000000000000000000000000000000
\end{verbatim}
\end{shaded}
Esto nos permite concluir que $h_K = 2$.

De hecho, en este caso también se puede ocupar una descomposición
$\zeta_K (s) = \zeta (s) \, L (s,\chi)$ para cierto carácter de Dirichlet $\chi$
(véase ejercicio~\ref{ejerc:caracter-para-campo-cuadratico}), y luego obtener
el valor de $L (1,\chi)$, pero no quiero entrar en los detalles de este cálculo.

%%%%%%%%%%%%%%%%%%%%%%%%%%%%%%%%%%%%%%%%%%%%%%%%%%%%%%%%%%%%%%%%%%%%%%%%%%%%%%%%

\section{Número de clases de \texorpdfstring{$\QQ (\sqrt{-p})$}{ℚ(√−p)}}

Ahora vamos a aplicar la fórmula del número de clases a campos cuadráticos
imaginarios $K = \QQ (\sqrt{-d})$. En particular, tomemos $d = p$ primo, $p > 3$
y $p \equiv 3 \pmod{4}$. En este caso $\Delta_K = -p$, $\O_K^\times = \mu_K = \{ \pm 1 \}$,
y $Reg_K = 1$. Tenemos entonces
\[ \lim_{s \to 1^+} (s - 1)\,\zeta_K (s) =
   \frac{2^{r_1}\,(2\pi)^{r_2}\,\Reg_K\,h_k}{\#\mu_K\,\sqrt{|\Delta_K|}} =
   \frac{\pi}{\sqrt{p}}\,h_K. \]

Notamos que bajo nuestra hipótesis de que $p \equiv 3\pmod{4}$, la factorización
de un primo racional $q$ en $\O_K$ depende del símbolo de Legendre
$\legendre{q}{p}$, y como en \S\ref{sec:funcion-zeta-de-Q(i)} se demuestra la
identidad
$$\zeta_K (s) = \zeta (s)\,L (s,\chi),$$
donde $\chi (n) = \legendre{n}{p}$. Esto implica que
$$\lim_{s \to 1^+} (s - 1)\,\zeta_K (s) = L (1,\chi),$$
y luego
$$h_K = \frac{\sqrt{p}}{\pi}\,L (1,\chi).$$

Podemos de nuevo ocupar la fórmula
$$\exp (g (\chi)\,L(1,\chi)) = \prod_n (1 - \zeta_p^n)\,\prod_r (1 - \zeta_p)^{-1}.$$

El famoso cálculo de Gauss nos dice que
\[ g (\chi) = \begin{cases}
  \sqrt{p}, & p \equiv 1 \pmod{4},\\
  i\sqrt{p}, & p \equiv 3 \pmod{4}
\end{cases} \]
(véase \cite[Chapter~6]{Ireland-Rosen}).

Tomando los logaritmos, se obtiene (módulo $2\pi i\ZZ$)
$$L(1,\chi) = -\frac{1}{i\sqrt{p}}\,\sum_{1 \le a \le p-1} \chi (a)\,\log (1 - \zeta_p^a).$$
Nos interesa la suma
$$S_\chi = -\sum_{1 \le a \le p-1} \chi (a)\,\log (1 - \zeta_p^a).$$
Notamos que $\chi (-1) = -1$ (usando la hipótesis $p \equiv 3 \pmod{4}$),
así que podemos escribir
\[ 2 S_\chi = \sum_{1 \le a \le p-1} \chi (a)\,\Bigl(\log (1 - \zeta_p^{-a}) - \log (1 - \zeta_p^a)\Bigr) =
   \sum_{1 \le a \le p-1} \chi (a)\,\log \frac{1 - \zeta_p^{-a}}{1 - \zeta_p^a}. \]
Aquí
\[ \log \frac{1 - \zeta_p^{-a}}{1 - \zeta_p^a} =
   \log (-\zeta_p^{-a}) =
   \log \exp \left(\pi i - \frac{2\pi i a}{p}\right) =
   2\pi i\,\left(\frac{1}{2} - \frac{a}{p}\right). \]
Entonces,
$$S_\chi = -\frac{\pi i}{p}\,\sum_{1 \le a \le p-1} \chi (a)\,a,$$
y luego
\[ h_K = \frac{\sqrt{p}}{\pi}\,L(1,\chi) =
   \frac{\sqrt{p}}{\pi}\,\frac{1}{i\sqrt{p}}\,S_\chi =
   -\frac{1}{p}\,\sum_{1 \le a \le p-1} \chi (a)\,a. \]

Podemos escribir (recordamos que $p \equiv 3 \pmod{4}$, así que
$\chi (-1) = -1$)
\begin{multline}
  \label{eq:suma-de-caracteres-chi(a)-a-1}
  p h_K = -\sum_{1 \le a \le p-1} \chi (a)\,a =
  -\sum_{1 \le a < p/2} \chi (a)\,a - \sum_{1 \le a < p/2} \chi (p-a)\,(p-a) \\
  = -2\sum_{1 \le a < p/2} \chi (a)\,a + p\sum_{1 \le a < p/2} \chi (a).
\end{multline}
Por otra parte,
\begin{multline*}
  p h_K = -\sum_{1 \le a \le p-1} \chi (a)\,a =
  -\sum_{\substack{1 \le a \le p-1 \\ a \text{ par}}} \chi (a)\,a - \sum_{\substack{1 \le a \le p-1 \\ a \text{ par}}} \chi (p-a)\,(p-a) \\
= -4\,\chi(2)\,\sum_{1 \le a < p/2} \chi (a)\,a + p\,\chi(2)\,\sum_{1 \le a < p/2} \chi (a),
\end{multline*}
así que
\begin{equation}
  \label{eq:suma-de-caracteres-chi(a)-a-2}
  p\chi(2)\,h_K = -4\,\sum_{1 \le a < p/2} \chi (a)\,a + p\,\sum_{1 \le a < p/2} \chi (a),
\end{equation}

Ahora comparando \eqref{eq:suma-de-caracteres-chi(a)-a-1}
y \eqref{eq:suma-de-caracteres-chi(a)-a-2}, nos sale
$$h_K\,(2 - \chi (2)) = \sum_{1 \le a < p/2} \chi (a).$$
Recordamos que
\[ \chi (2) = \begin{cases}
  +1, & p \equiv 7 \pmod{8},\\
  -1, & p \equiv 3 \pmod{8}.
\end{cases} \]

Nuestros cálculos nos llevan entonces al siguiente resultado
(que ya fue mencionado en \S\ref{sec:campos-cuadraticos-imaginarios}).

\begin{teorema}[Dirichlet]
  Sea $p > 3$ un primo tal que $p \equiv 3 \pmod{4}$. Consideremos el campo
  cuadrático imaginario $K = \QQ (\sqrt{-p})$. Si $p \equiv 7 \pmod{8}$,
  entonces
  $$h_K = \sum_{1 \le a < p/2} \legendre{a}{p},$$
  y si $p \equiv 3 \pmod{8}$, entonces
  $$h_K = \frac{1}{3}\,\sum_{1 \le a < p/2} \legendre{a}{p}.$$
\end{teorema}

\begin{corolario}
  Si $p \equiv 3 \pmod{4}$, entonces el intervalo $[1, (p-1)/2]$ contiene más
  residuos que no-residuos cuadráticos.
\end{corolario}

\begin{comentario}
  En general, para \emph{cualquier} campo cuadrático $K = \QQ (\sqrt{d})$
  (imaginario o real) se puede definir un carácter $\chi$ módulo $|\Delta_K|$
  que gobierna la factorización de primos en $\O_K$ y nos lleva a la fórmula
  $$\zeta_K (s) = \zeta (s) \, L (s,\chi)$$
  ---véase ejercicio~\ref{ejerc:caracter-para-campo-cuadratico}.

  Luego los métodos parecidos a los de arriba nos permiten calcular $h_K$
  en términos de $L (1,\chi)$. Para los detalles, véase
  \cite[Chapter~5]{Borevich-Shafarevich}.
\end{comentario}

%%%%%%%%%%%%%%%%%%%%%%%%%%%%%%%%%%%%%%%%%%%%%%%%%%%%%%%%%%%%%%%%%%%%%%%%%%%%%%%%

\section{Demostración de la fórmula del número de clases}

El objetivo de esta sección será probar la fórmula analítica de clases
\[ \lim_{s \to 1^+} (s - 1)\,\zeta_K (s) =
   \frac{2^{r_1}\,(2\pi)^{r_2}\,\Reg_K\,h_k}{\#\mu_K\,\sqrt{|\Delta_K|}}. \]
La prueba es algo larga y técnica, así que al principio voy a explicar
la estrategia general. Mi referencia es \cite{Borevich-Shafarevich}.
Primero vamos a partir $\zeta_K (s)$ en $h_K$ series
$$\zeta_K (s) = \sum_{I\ne 0} \frac{1}{N_{K/\QQ} (I)^s} = \sum_{n\in \Cl (K)} \zeta_c (s),$$
donde
$$\zeta_c (s) = \sum_{[I] \in c} \frac{1}{N_{K/\QQ} (I)^s}.$$
A continuación veremos que para todo $c \in \Cl (K)$ se cumple
\[ \lim_{s \to 1^+} (s - 1)\,\zeta_c (s) =
   \frac{2^{r_1}\,(2\pi)^{r_2}\,\Reg_K}{\#\mu_K\,\sqrt{|\Delta_K|}}. \]
Esto claramente implica la fórmula del número de clases.

\vspace{1em}

Para cada clase $c \in \Cl (K)$ fijemos un ideal entero $I' \subseteq \O_K$
tal que $[I'] = c^{-1}$. Ahora para todo ideal entero $I \subseteq \O_K$ tal que
$[I] = c$ se tiene $I I' = \alpha \O_K$ para algún $\alpha \in \O_K$ no nulo.
Esto nos da una biyección
\[ \{ I \subseteq \O_K \mid [I] = c \} \leftrightarrow
\{ \text{ideales principales } \alpha\O_K \mid \alpha \in I' \}. \]
Tenemos $N_{K/\QQ} (I) \, N_{K/\QQ} (I') = |N_{K/\QQ} (\alpha)|$, y entonces
$$\zeta_c (s) = N_{K/\QQ} (I')^s \sum_{\substack{0 \ne (\alpha) \subseteq \O_K \\ \alpha \in I'}} \frac{1}{|N_{K/\QQ} (\alpha)|^s}.$$
La suma es sobre todos los ideales principales generados por los elementos
$\alpha \in I'$, que es lo mismo que la suma sobre $\alpha \in I'$ considerados
módulo la relación $\sim$.

\vspace{1em}

Consideremos unidades fundamentales $u_1, \ldots, u_r \in \O_K^\times$. Entonces
los vectores $L (u_1), \ldots, L (u_r)$ forman una base del espacio
$H \subset \RR^{r_1+r_2}$. Por otra parte, el vector
$$L = (\underbrace{1,\ldots,1}_{r_1}, \underbrace{2,\ldots,2}_{r_2})$$
no está en $H$, así que los $L (u_i)$ junto con $L$ forman una base de
$\RR^{r_1+r_2}$: todo $x \in \RR^{r_1+r_2}$ puede ser expresado como
\begin{equation}
  \label{eqn:base-de-Rr1+r2}
  \lambda_1 \, L (u_1) + \cdots + \lambda_r \, L (u_r) + \lambda L
\end{equation}
para algunos $\lambda_1,\ldots,\lambda_r,\lambda \in \RR$.
El grupo de unidades $\O_K^\times$ actúa sobre el espacio $K_\RR^\times$ mediante
$$u \cdot x = \Phi (u)\,x.$$
Esta es la multiplicación de dos vectores en $K_\RR^\times$ punto por punto.
No olvidemos que $K_\RR^\times$ es un subconjunto de $K_\RR$, que a su vez está
en el espacio complejo $K_\CC$. Entonces, aunque $K_\RR$ es un espacio
vectorial \emph{real} de dimensión $n = [K : \QQ]$, aquí es importante
interpretar sus elementos como números complejos, y la multiplicación es
la multiplicación compleja.

El dominio fundamental de esta acción tiene una buena descripción.  A partir de
ahora vamos a denotar por $m = \# \mu_K$ el número de las raíces de la unidad en
$K$.

\begin{teorema}
  \label{thm:X-dominio-fundamental-de-OK*}
  Consideremos el subconjunto $X \subset K_\RR^\times$ que satisface las
  siguientes condiciones.
  \begin{enumerate}
  \item[a)] Para todo $x \in X$ en la expresión de $\ell (x)$ en la base
    \eqref{eqn:base-de-Rr1+r2} se tiene $0 \le \lambda_i < 1$ para
    $i = 1,\ldots,r$.

  \item[b)] Se tiene $0 \le \arg x_1 < \frac{2\pi}{m}$.
  \end{enumerate}

  Entonces,
  \begin{enumerate}
  \item[1)] $X$ es un \textbf{cono}: si $x \in X$, entonces $\lambda x \in X$
    para todo $\lambda > 0$.

  \item[2)] $X$ es un dominio fundamental de la acción de $\O_K^\times$ sobre
    $K_\RR^\times$: para todo punto $y \in K_\RR^\times$ existen únicos
    $u \in \O_K^\times$ y $x \in X$ tales que $y = \Phi (u)\,x$.
  \end{enumerate}
\end{teorema}

La condición b) del teorema merece alguna explicación. Si $K$ tiene un encaje
real, entonces $\mu_K = \{ \pm 1 \}$ (otras raíces de la unidad no se encajan en
$\RR$), así que $m = 2$ y la condición b) nos dice simplemente que $x_1 > 0$.
Más adelante también vamos a dibujar $X$ en el caso de $[K : \QQ] = 2$ para
entender mejor qué está pasando.

Del último teorema se deduce fácilmente el siguiente resultado.

\begin{corolario}
  Para todo $\alpha \in \O_K$ no nulo existe único $\beta \sim \alpha$ tal que
  $\Phi (\beta) \in X$.

  \begin{proof}
    Primero, dado $\alpha \ne 0$, entonces para $\Phi (\alpha) \in K_\RR^\times$
    existen únicos $u \in \O_K^\times$ y $x \in X$ tales que
    $\Phi (\alpha) = \Phi (u)\,x$. Ahora $\Phi (u^{-1}\,\alpha) \in X$ y
    $\alpha \sim u^{-1}\,\alpha$.

    Para la unicidad, si $\alpha \sim \beta$ y $\Phi (\beta) \in X$, escribamos
    $\alpha = u\beta$ para $u \in \O_K^\times$. Tenemos
    $\Phi (\alpha) = \Phi (u)\,\Phi (\beta)$, pero el teorema dice que la unidad
    $u$ está definida de manera única.
  \end{proof}
\end{corolario}

Volvamos a nuestra expresión
$$\zeta_c (s) = N_{K/\QQ} (I')^s \sum_{\substack{0 \ne (\alpha) \subseteq \O_K \\ \alpha \in I'}} \frac{1}{|N_{K/\QQ} (\alpha)|^s}.$$
Denotemos por $\Lambda$ el retículo $\Phi (I') \subset K_\RR$. Denotemos por
$N\colon K_\RR \to \RR$ el producto de coordenadas $x \mapsto \prod_i x_i$.
En este caso $N_{K/\QQ} (\alpha) = N (\Phi (\alpha))$. Ocupando el último
corolario, podemos escribir entonces
$$\zeta_c (s) = N_{K/\QQ} (I')^s \sum_{\omega \in \Lambda \cap X} \frac{1}{|N (\omega)|^s}.$$

Para calcular esta serie, vamos a formular el siguiente teorema general.

\begin{teorema}
  \label{thm:residuo-de-Z(s)}
  Sean $X$ un cono en $\RR^n$ sin origen, $F\colon X\to \RR_{>0}$ una función
  positiva, y $\Lambda \subset X$ un retículo de rango completo. Supongamos que
  se cumplen las siguientes condiciones.
  \begin{enumerate}
  \item[a)] Para cualesquiera $x \in X$ y $\lambda > 0$ se cumple
    $F (\lambda x) = \lambda^n\,F (x)$.

  \item[b)] El subconjunto $T \subset X$ que consiste en los puntos $x \in X$
    con $F (x) \le 1$ es acotado y tiene volumen no nulo.
  \end{enumerate}
  Entonces, la serie
  $$Z (s) = \sum_{\omega \in \Lambda \cap X} \frac{1}{F (x)^s}$$
  converge para $s > 1$ y se tiene
  $$\lim_{s \to 1^+} (s-1)\,Z (s) = \frac{\vol T}{\covol \Lambda}.$$
\end{teorema}

En nuestra situación, aplicaremos el teorema a la función
$F\colon x \mapsto |N (x)|$ sobre $X \subset K_\RR$. Para el conjunto
$$T = \{ x \in X \mid |N (x)| \le 1 \}$$
vamos a calcular que
\begin{equation}
  \label{eq:vol-T}
  \vol T = \frac{2^{r_1}\,(2\pi)^{r_2}\,\Reg_K}{m}.
\end{equation}
Por otra parte, como ya sabemos,
$$\covol \Lambda = \sqrt{|\Delta_K|}\,N_{K/\QQ} (I').$$
Esto nos permite concluir que
$$\lim_{s\to 1^+} (s-1)\,\zeta_c (s) = \frac{2^{r_1}\,(2\pi)^{r_2}\,\Reg_K}{m\,\sqrt{|\Delta_K|}}.$$

Entonces, para terminar la prueba, nos falta lo siguiente:
\begin{itemize}
\item demostrar el teorema \ref{thm:X-dominio-fundamental-de-OK*} que nos da
  un dominio fundamental de la acción de $\O_K^\times$ sobre $K_\RR$,
\item demostrar el teorema \ref{thm:residuo-de-Z(s)} sobre el residuo de
  $Z (s) = \sum_{\omega \in \Lambda \cap X} \frac{1}{F (x)^s}$ en $s = 1$,
\item calcular el volumen \eqref{eq:vol-T}.
\end{itemize}

%%%%%%%%%%%%%%%%%%%%%%%%%%%%%%%%%%%%%%%%%%%%%%%%%%%%%%%%%%%%%%%%%%%%%%%%%%%%%%%%

\subsection{Dominio fundamental de la acción de unidades sobre \texorpdfstring{$K_\RR^\times$}{Kℝ*}}

Ahora vamos a demostrar el teorema \ref{thm:X-dominio-fundamental-de-OK*}.
Nos interesa el subconjunto $X \subset K_\RR^\times$ que satisface las
siguientes condiciones.
\begin{enumerate}
\item[a)] Para todo $x \in X$ en la expresión
  $$\ell (x) = \lambda_1 \, L (u_1) + \cdots + \lambda_r \, L (u_r) + \lambda L$$
  se tiene $0 \le \lambda_i < 1$ para $i = 1,\ldots,r$.

\item[b)] Se tiene $0 \le \arg x_1 < \frac{2\pi}{m}$.
\end{enumerate}

Primero vamos a verificar que $X$ es un cono. Para $x \in X$ y $\lambda > 0$
tenemos
$$\ell (\lambda x) = \log |\lambda|\,L + \ell (x),$$
y esto no afecta los coeficientes $\lambda_1,\ldots,\lambda_r$, así que la
condición a) se preserva. Por otra parte, $\arg (\lambda x_1) = \arg x_1$,
así que b) se preserva también. Notamos que $X \ne \emptyset$: por ejemplo,
el punto $\Phi (1)$ está en $X$.

Ahora nos gustaría probar que para todo $y \in K_\RR^\times$ existen únicos
$u \in \O_K^\times$ y $x \in X$ tales que $y = \Phi (u)\,x$.

Primero para la existencia, escribamos $\ell (y)$ en términos de nuestra base
de $\RR^{r_1+r_2}$:
$$\ell (y) = \lambda_1 \, L (u_1) + \cdots + \lambda_r \, L (u_r) + \lambda L.$$
Para $i = 1,\ldots,r$ pongamos
$$\lambda_i = a_i + \mu_i,$$
donde
$$a_i \in \ZZ, \quad 0 \le \mu_i < 1.$$
Consideremos la unidad $v = u_1^{a_1}\cdots u_r^{a_r}$ y el punto
$z = \Phi (v^{-1})\,y$. Ahora
\[ \ell (z) = L (v^{-1}) + \ell (y) =
   - a_1\,L (u_1) - \cdots - a_r\,L(u_r)
   + \lambda_1 \, L (u_1) + \cdots + \lambda_r \, L (u_r) + \lambda L =
   \mu_1 \, L (u_1) + \cdots + \mu_r \, L (u_r) + \lambda L. \]
Esto nos asegura la condición a) de la definición de $X$, y falta analizar
la condición b). Si $\arg z_1 = \phi$, entonces para algún $k$ se tiene
$$\frac{2\pi k}{m} \le \phi < \frac{2\pi {k+1}}{m}.$$
Sea $\zeta \in \mu_K$ la $m$-ésima raíz de la unidad tal que
$\sigma_1 (\zeta) = \exp \Bigl(\frac{2\pi i}{m}\Bigr)$. En este caso el punto
$x = \Phi (\zeta^{-k})\,z$ está en $X$: primero tenemos
$$\ell (x) = L (\zeta^{-k}) + \ell (z) = \ell (z),$$
dado que $\mu_K = \ker L$, así que la condición a) se preserva. Por otra parte,
$$\arg x_1 = \arg z_1 - \frac{2\pi k}{m} = \phi - \frac{2\pi k}{m},$$
y luego
$$0 \le \arg x_1 < \frac{2\pi}{m}.$$
Entonces,
$$y = \Phi (v)\,z = \Phi (\zeta^k v)\,x$$
es la representación que estábamos buscando.

Ahora para ver que la representación es única, supongamos que
$$y = \Phi (u)\,x = \Phi (u')\,x'$$
para algunos $u,u' \in \O_K^\times$, $x,x' \in X$. Tomando los logaritmos, se
obtiene
$$L (u) + \ell (x) = L (u') + \ell (x').$$
Aquí por nuestra hipótesis
\begin{align*}
  \ell (x) & = \lambda_1 \, L (u_1) + \cdots + \lambda_r \, L (u_r) + \lambda L, \\
  \ell (x') & = \lambda_1' \, L (u_1) + \cdots + \lambda_r' \, L (u_r) + \lambda' L,
\end{align*}
donde $0 \le \lambda_i, \lambda_i' < 1$. Por otra parte,
\begin{align*}
  L (u) & = a_1 \, L (u_1) + \cdots + a_r \, L (u_r), \\
  L (u') & = a_1' \, L (u_1) + \cdots + a_r' \, L (u_r),
\end{align*}
donde $a_i,a_i' \in \ZZ$. Esto nos permite concluir que $L (u) = L (u')$, así
que $u' = \zeta u$ para alguna raíz de la unidad $\zeta \in \mu_K = \ker L$.
Ahora $\Phi (u') = \Phi (\zeta)\,\Phi (u)$, y entonces
$x = \Phi (\zeta)\,x'$, y en particular $x_1 = \sigma_1 (\zeta)\,x_1'$.
La condición b) nos dice que
$$0 \le \arg x_1, \arg x_1' < \frac{2\pi}{m}.$$
Tenemos
$$0 \le |\sigma_1 (\zeta)| < \frac{2\pi}{m},$$
pero $\sigma_1 (\zeta)$ es una raíz $m$-ésima compleja, así que la única opción
es $\sigma_1 (\zeta) = 1$, y luego $\zeta = 1$. Podemos concluir que $u = u'$ y
$x = x'$. \qed

\vspace{1em}

\begin{ejemplo}
  Consideremos un campo cuadrático imaginario, por ejemplo
  $K = \QQ (\sqrt{-3})$. Tenemos un isomorfismo
  \[ K_\RR \cong \RR^2, \quad
     (z_\sigma, z_{\overline{\sigma}}) \mapsto (\Re z_\sigma, \Im z_\sigma). \]
  El grupo de unidades es $\O_K^\times = \mu_6 (\CC)$. La condición a) en la
  definición de $X$ es vacía porque $r_1 + r_2 - 1 = 0$, y no hay unidades
  fundamentales, mientras que la condición b) dice que nos interesan los puntos
  de $\CC \cong \RR^2$ tales que $0 \le \arg z < \frac{\pi}{3}$. El conjunto $T$
  es la intersección de $X$ con el círculo definido por $|z| = 1$.

  \begin{center}
    \includegraphics{pic/X-T-sqrt-minus-3.pdf}
  \end{center}
\end{ejemplo}

\begin{ejemplo}
  Consideremos el campo cuadrático real $K = \QQ (\sqrt{3})$. En este caso
  $K_\RR = \RR^2$, y la unidad fundamental es $u = 2 + \sqrt{3}$. Entonces,
  la condición a) significa que $X$ consiste en los puntos $(x,y) \in \RR^2$
  tales que
  \begin{align*}
    \log |x| & = \lambda_1\,\log (2 + \sqrt{3}) + \lambda,\\
    \log |y| & = \lambda_1\,\log (2 - \sqrt{3}) + \lambda,
  \end{align*}
  donde $0 \le \lambda_1 < 1$, mientras que la condición b) nos dice que
  $x > 1$. De aquí es fácil ver que $X$ será la unión de dos conos: uno generado
  por los vectores $v_1 = (1,1)$, $v_2 = (2 + \sqrt{3}, 2 - \sqrt{3})$,
  y el otro generado por $v_1 = (1,-1)$ y $v_2 = (2 + \sqrt{3}, -2 + \sqrt{3})$.
  El conjunto $T$ está acotado por la curva $xy = \pm 1$.

  \begin{center}
    \includegraphics{pic/X-T-sqrt-plus-3.pdf}
  \end{center}
\end{ejemplo}

%%%%%%%%%%%%%%%%%%%%%%%%%%%%%%%%%%%%%%%%%%%%%%%%%%%%%%%%%%%%%%%%%%%%%%%%%%%%%%%%

\subsection{Conos, retículos y residuo en $s = 1$}

Ahora continuando con nuestro programa, vamos a probar el
teorema~\ref{thm:residuo-de-Z(s)}. Tenemos un retículo de rango completo
$\Lambda \in \RR^n$ y un cono sin origen $X \subset \RR^n$ junto con una función
$F\colon X\to \RR_{>0}$ que es homogénea en el sentido de que
$$F (\lambda x) = \lambda^n\,F (x).$$
Consideremos el conjunto
$$T = \{ x \in X \mid F (x) \le 1 \}$$
que es acotado y de volumen finito no nulo. Nos interesa probar que la serie
$$Z (s) = \sum_{\omega \in \Lambda \cap X} \frac{1}{F (x)^s}$$
converge para $s > 1$ y tiene residuo $\frac{\vol T}{\covol \Lambda}$ en
$s = 1$.

\vspace{1em}

Recordamos que a fin de cuentas, nos interesa un cono en el espacio $K_\RR$ que
no es precisamente $\RR^n$, sino fue tiene estructura euclidiana ligeramente
distinta, respecto a cual $\vol = 2^{r_2}\,\vol_{Leb.}$, pero esto será
irrelevante en la prueba porque nos interesa el cociente de dos volúmenes.

\vspace{1em}

El punto clave es una interpretación de volúmenes en términos del conteo de
puntos en retículos. Para un parámetro $r > 0$ vamos a considerar el retículo
$\frac{1}{r}\Lambda$. Notamos que
$\covol\Bigl(\frac{1}{r}\Lambda\Bigr) = \frac{1}{r^n}\,\covol (\Lambda)$.
Definamos
\[ C (r) = \# \{ \omega \in \frac{1}{r}\Lambda \mid \omega \in T \}
         = \# \{ \omega \in \Lambda \mid \omega \in rT \}
         = \# \{ \omega \in \Lambda \cap X \mid F (\omega) \le r^n \}. \]
Tenemos entonces
\[ \vol T = \lim_{r\to\infty} C (r) \, \covol \Bigl(\frac{1}{r}\,\Lambda\Bigr)
          = \covol \Lambda \, \lim_{r\to\infty} \frac{C(r)}{r^n}. \]
Podemos ordenar los puntos $\omega \in \Lambda \cap X$ de tal manera que
$$0 < F (\omega_1) \le F (\omega_2) \le F (\omega_3) \le \cdots$$

Pongamos $r_k = \sqrt[n]{F (\omega_k)}$. Tenemos
$\{ \omega_1, \ldots, \omega_k \} \subseteq r_k\,T$, así que $C (r_k) \ge k$.
Por otra parte, para todo $\epsilon > 0$ se tiene
$x_k \notin (r_k - \epsilon)\,T$, así que $C (r_k - \epsilon) < k$. Esto nos
da la desigualdad
$$C (r_k - \epsilon) < k \le C (r_k),$$
y luego
$$\frac{C (r_k - \epsilon)}{(r_k-\epsilon)^n}\,\frac{(r_k - \epsilon)^n}{r_k^n} < \frac{k}{r_k^n} \le \frac{C (r_k)}{r_k^n}.$$
Pasando al límite $k \to \infty$, tenemos $r_k \to \infty$, y luego
$$\lim_{k\to\infty} \frac{k}{F (\omega_k)} = \frac{\vol T}{\covol \Lambda}.$$
Esto significa que para todo $\epsilon > 0$ se tiene
\[ \Bigl(\frac{\vol T}{\covol \Lambda} - \epsilon\Bigr)\frac{1}{k}
   < \frac{1}{F (\omega_k)}
   < \Bigl(\frac{\vol T}{\covol \Lambda} + \epsilon\Bigr)\frac{1}{k} \]
para todo $k$ suficientemente grande, digamos $k \ge k_0$. Elevando todo a $s > 1$
y sumando sobre $k\ge k_0$, tenemos
\[ \Bigl(\frac{\vol T}{\covol \Lambda} - \epsilon\Bigr)^s\,\sum_{k\ge k_0} \frac{1}{k^s}
   < \sum_{k\ge k_0} \frac{1}{F (\omega_k)^s}
   < \Bigl(\frac{\vol T}{\covol \Lambda} + \epsilon\Bigr)^s\,\sum_{k\ge k_0} \frac{1}{k^s}. \]
Aquí
$$\sum_{k\ge k_0} \frac{1}{k^s} = \zeta (s) - \sum_{1\le k < k_0} \frac{1}{k^s},$$
donde $\zeta (s)$ converge para $s > 1$. Por otra parte, nuestra serie es
$$Z (s) = \sum_{k\ge 1} \frac{1}{F (\omega_k)^s}.$$
Las desigualdades de arriba que se cumplen para todo $\epsilon > 0$ establecen
la convergencia de $Z (s)$ para $s > 1$.

Ahora multiplicando la desigualdad por $(s - 1)$ y pasando al límite
$s \to 1^+$, en vista de la fórmula
$$\lim_{s\to 1^+} (s-1)\,\sum_{k\ge k_0} \frac{1}{k^s} = \lim_{s\to 1^+} (s-1)\,\zeta(s) = 1,$$
se obtiene
\[ \frac{\vol T}{\covol \Lambda} - \epsilon
   < \liminf_{s\to 1^+} (s-1)\,Z(s) \le \limsup_{s\to 1^+} (s-1)\,Z(s)
   < \frac{\vol T}{\covol \Lambda} + \epsilon. \]
Esto se cumple para todo $\epsilon > 0$, así que podemos concluir que
$$\lim_{s\to 1^+} (s-1)\,Z (s) = \frac{\vol T}{\covol \Lambda}.$$
Hemos entonces probado el teorema~\ref{thm:residuo-de-Z(s)}.

\subsection{Cálculo del volumen de T}

Para terminar la prueba de la fórmula del número de clases, tenemos que ver que
$$T = \{ x \in X \mid |N (x)| \le 1 \}$$
es un conjunto acotado y calcular su volumen
\begin{equation}
  \label{eq:vol-T}
  \vol T = \frac{2^{r_1}\,(2\pi)^{r_2}\,\Reg_K}{m}.
\end{equation}
El cálculo esencialmente consiste en una reducción a ciertas integrales
iteradas, lo que no suena muy interesante. Sin embargo, este es precisamente
el punto donde aparece el regulador y $m = \# \mu_K$.

**TODO**

%%%%%%%%%%%%%%%%%%%%%%%%%%%%%%%%%%%%%%%%%%%%%%%%%%%%%%%%%%%%%%%%%%%%%%%%%%%%%%%%

\section{Perspectiva: Prolongación analítica}

**TODO** STATE WITHOUT PROOFS FOR $\zeta (s)$, $\zeta_K (s)$, $L (s,\chi)$

%%%%%%%%%%%%%%%%%%%%%%%%%%%%%%%%%%%%%%%%%%%%%%%%%%%%%%%%%%%%%%%%%%%%%%%%%%%%%%%%

\section{Perspectiva: Valores especiales}

**TODO**
stuff like
the case of $\zeta (s)$ and $B_n$ (with proofs),
$\zeta (2k+1)$ (no proofs :-)
Siegel--Klingen (without proof),
calculations for the quadratic case via $L (s,\chi)$ and $B_{n,\chi}$ (with proofs),
etc.

%%%%%%%%%%%%%%%%%%%%%%%%%%%%%%%%%%%%%%%%%%%%%%%%%%%%%%%%%%%%%%%%%%%%%%%%%%%%%%%%

\section{Equivalencia aritmética}

Hemos visto hasta el momento que la función zeta $\zeta_K (s)$ trae mucha
información aritmética sobre el campo de números $K$. Sin embargo, $\zeta_K (s)$
no sabe \emph{todo} de $K$: podemos tener $\zeta_K (s) = \zeta_{K'} (s)$ para
dos campos no isomorfos $K \not\cong K'$.

**TODO** \cite{Perlis-1977}.

%%%%%%%%%%%%%%%%%%%%%%%%%%%%%%%%%%%%%%%%%%%%%%%%%%%%%%%%%%%%%%%%%%%%%%%%%%%%%%%%

\pagebreak

\phantomsection

\addcontentsline{toc}{section}{Ejercicios}
\section*{Ejercicios}

\begin{ejercicio}
  ¿Cuántos puntos enteros están en la elipse $x^2 - xy + y^2 = n$?

  \noindent (Considere la función zeta de $\QQ (\zeta_3)$.)
\end{ejercicio}

\begin{ejercicio}
  \label{ejerc:caracter-para-campo-cuadratico}
  Sea $K = \QQ (\sqrt{d})$ un campo cuadrático. Para $n$ coprimo con $\Delta_K$
  definamos mediante el \textbf{símbolo de Jacobi} $\legendre{n}{m}$
  \[ \chi (n) = \begin{cases}
    \legendre{n}{|d|}, & \text{si }d \equiv 1 \pmod{4},\\
    (-1)^{\frac{n-1}{2}}\,\legendre{n}{|d|}, & \text{si }d \equiv 3 \pmod{4},\\
    (-1)^{\frac{n^2-1}{8} + \frac{n-1}{2}\,\frac{d'-1}{2}}\,\legendre{n}{|d'|}, & \text{si }d = 2d'.
  \end{cases} \]
  Si $\gcd (n,\Delta_K) \ne 1$, pongamos $\chi (n) = 0$.

  \begin{enumerate}
  \item[1)] Demuestre que $\chi$ es un carácter de Dirichlet mód $|\Delta_K|$.

  \item[2)] Demuestre que $\chi$ determina la factorización de primos
    racionales en $\O_K$:
    \[ p\O_K = \begin{cases}
      \mathfrak{p}\,\overline{\mathfrak{p}}, & \text{si } \chi (p) = +1,\\
      \mathfrak{p}, & \text{si } \chi (p) = -1,\\
      \mathfrak{p}^2, & \text{si } \chi (p) = 0.
    \end{cases} \]

  \item[3)] Demuestre que $\zeta_K (s) = \zeta (s) \, L (s,\chi)$.
  \end{enumerate}
\end{ejercicio}

\begin{ejercicio}
  \label{ejerc:L-series-caracter-cuadratico-mod-p}
  Sea $p$ un número primo y $\chi$ el carácter de Dirichlet de orden $2$ mód
  $p$, definido por el símbolo de Legendre $\chi (n) = \legendre{n}{p}$.

  \begin{enumerate}
  \item[1)] Demuestre que
    $$\exp (g (\chi)\,L (1,\chi)) = \prod_n (1 - \zeta_p^n)\,\prod_r (1 - \zeta_p^r)^{-1},$$
    donde $g (\chi) = \sum_{1 \le a
      \le p-1} \chi (a)\,\zeta_p^a$, y los productos son sobre los no-residuos y
    residuos cuadráticos mód $p$ respectivamente.

  \item[2)] Use la parte anterior para calcular $L (1,\chi)$, donde $\chi$ es el
    carácter de orden $2$ mód $5$.  (Para el valor numérico en PARI/GP, basta
    digitar \texttt{lfun(5,1)})
  \end{enumerate}
\end{ejercicio}

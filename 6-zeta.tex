\chapter{Función zeta de Dedekind}

Para un campo de números $K/\QQ$, la \textbf{función zeta de Dedekind}
correspondiente es una especie de serie generatriz definida por la serie
$$\zeta_K (s) = \sum_{I \ne 0} \frac{1}{N_{K/\QQ} (I)^s},$$
donde la suma se toma sobre los ideales enteros no nulos $I \subseteq \O_K$.
Resulta que la función zeta de Dedekind codifica mucha información aritmética
sobre $K$. Nos interesará $\zeta_K (s)$ como un objeto analítico, y en
particular hay que ver la convergencia de la serie.

Primero notamos que si $K = \QQ$, entonces los ideales no nulos en $\O_K = \ZZ$
tienen forma $n\ZZ$ para $n = 1,2,3,\ldots$, así que $\zeta_\QQ (s) = \zeta (s)$
es la función zeta de Riemann.

\begin{lema}
  La serie $\zeta (s) = \sum_{n\ge 1} \frac{1}{n^s}$ converge absolutamente para
  $\Re s > 1$, y se tiene
  $$\lim_{s \to 1^+} (s - 1)\,\zeta (s) = 1.$$

  \begin{proof}
    Primero, notamos que $|1/n^s| = 1/n^{\Re s}$, así que será suficiente
    establecer la convergencia para $s > 1$ real. Tenemos
    $$\int_n^{n+1} \frac{dx}{x^s} \le \frac{1}{n^s} \le \int_{n-1}^n \frac{dx}{x^s},$$
    donde la primera desigualdad se cumple para $n \ge 1$ y la segunda para
    $n \ge 2$. Tomando la suma sobre $n \ge 1$, nos sale la desigualdad
    $$\int_1^\infty \frac{dx}{x^s} \le \zeta (s) \le 1 + \int_1^\infty \frac{dx}{x^s}.$$
    Ahora, dado que $\int_1^\infty \frac{dx}{x^s} = \frac{1}{s - 1}$, tenemos
    $$\frac{1}{s - 1} \le \zeta (s) \le \frac{s}{s - 1},$$
    y luego
    $$1 \le (s-1)\,\zeta (s) \le s.$$
    Esto establece la convergencia y también calcula $(s-1)\,\zeta (s)$ para
    $s \to 1^+$.
  \end{proof}
\end{lema}

Ahora podemos probar la convergencia de $\zeta_K (s)$.

\begin{proposicion}
  La serie $\zeta_K (s) = \sum_{I \ne 0} \frac{1}{N_{K/\QQ} (I)^s}$ converge
  absolutamente para $\Re s > 1$. Además, se cumple la
  \textbf{fórmula del producto de Euler}
  \[ \sum_{I \ne 0} \frac{1}{N_{K/\QQ} (I)^s} =
     \prod_{\mathfrak{p}} \frac{1}{1 - N_{K/\QQ} (\mathfrak{p})^{-s}}, \]
  donde el producto es sobre todos los ideales primos
  $\mathfrak{p} \subset \O_K$.

  \begin{proof}
    De nuevo, la convergencia absoluta para $\Re s > 1$ se sigue de la
    convergencia para $s > 1$ real. Será suficiente probar la convergencia
    absoluta del producto de Euler
    $$\prod_{\mathfrak{p}} \frac{1}{1 - N_{K/\QQ} (\mathfrak{p})^{-s}},$$
    y luego,
    \[ \prod_{\mathfrak{p}} \frac{1}{1 - N_{K/\QQ} (\mathfrak{p})^{-s}} =
       \prod_{\mathfrak{p}} \sum_{e \ge 1} \frac{1}{N_{K/\QQ} (\mathfrak{p}^e)^s} =
       \sum_{I\ne 0} \frac{1}{N_{K/\QQ} (I)^s}. \]
    Aquí hemos usado la serie geométrica, la multiplicatividad de la norma, y
    que todo ideal entero no nulo $I \subseteq \O_K$ tiene factorización única
    en ideales primos $I = \mathfrak{p}_1^{e_1}\cdots \mathfrak{p}_s^{e_s}$.

    Un producto $\prod_{n\ge 1} (1 + |x_n|)$ converge si y solamente si la serie
    $\sum_{n\ge 1} |x_n|$ converge. Entonces, la convergencia del producto de
    Euler se sigue de la convergencia de
    $\prod_{\mathfrak{p}} (1 - N_{K/\QQ} (\mathfrak{p})^{-s})$, y luego de
    $\sum_{\mathfrak{p}} \frac{1}{N_{K/\QQ} (\mathfrak{p})^s}$.

    Recordemos que para todo primo racional $p$, existen $\le [K:\QQ]$ ideales
    primos $\mathfrak{p} \mid p$, y luego para cada uno de estos se tiene
    $N_{K/\QQ} (\mathfrak{p}) = p^f \ge p$. Entonces,
    \[ \sum_{\mathfrak{p}} \frac{1}{N_{K/\QQ} (\mathfrak{p})^s} =
       \sum_p \sum_{\mathfrak{p} \mid p} \frac{1}{N_{K/\QQ} (\mathfrak{p})^s} \le
       \sum_p \frac{[K:\QQ]}{p^s} < [K : \QQ]\,\zeta (s). \]
    Aquí hemos usado la convergencia de la función zeta de Riemann.
  \end{proof}
\end{proposicion}

%%%%%%%%%%%%%%%%%%%%%%%%%%%%%%%%%%%%%%%%%%%%%%%%%%%%%%%%%%%%%%%%%%%%%%%%%%%%%%%%

\section{Ejemplo: la función zeta de \texorpdfstring{$\QQ (i)$}{ℚ(i)}}
\label{sec:funcion-zeta-de-Q(i)}

Tomemos $K = \QQ (i)$. Para entender cómo se ve la función zeta correspondiente
$\zeta_K (s)$, sería más fácil trabajar con el producto de Euler
$$\prod_{\mathfrak{p}} \frac{1}{1 - N_{K/\QQ} (\mathfrak{p})^{-s}}.$$
Los primos en $\O_K = \ZZ[i]$ son los siguientes.
\begin{itemize}
\item Al primo ramificado $p = 2$ corresponde un ideal primo $\mathfrak{p}_2$
  de norma $2$.

\item Si $p \equiv 1 \pmod{4}$, entonces $p$ se escinde en dos ideales primos
  $\mathfrak{p}$ y $\overline{\mathfrak{p}}$, cada uno de norma $p$.

\item Si $p \equiv 3 \pmod{4}$, entonces $p$ es inerte y corresponde a un ideal
  primo $\mathfrak{p}$ de norma $p^2$.
\end{itemize}

Entonces,
\[ \zeta_K (s) =
   \frac{1}{1 - 2^{-s}} \,
   \prod_{p \equiv 1 ~ (4)} \frac{1}{(1 - p^{-s})^2} \,
   \prod_{p \equiv 1 ~ (4)} \frac{1}{1 - p^{-2s}}. \]
Notamos que $\frac{1}{1 - p^{-2s}} = \frac{1}{1 - p^{-s}}\,\frac{1}{1 + p^{-s}}$,
así que
$$\zeta_K (s) = \prod_p \frac{1}{1 - p^{-s}}\,\prod_p \frac{1}{1 - \chi(p)\,p^{-s}},$$
donde
\[ \chi(p) = \begin{cases}
  +1, & p \equiv 1 \pmod{4},\\
  -1, & p \equiv 3 \pmod{4}.
\end{cases} \]
Este es un \textbf{carácter de Dirichlet} mód $4$, y a este se asocia la
\textbf{serie L de Dirichlet} correspondiente
\[ L (s,\chi) = \prod_p \frac{1}{1 - \chi(p)\,p^{-s}} =
   \sum_{n\ge 1} \frac{\chi (s)}{n^s}. \]
Entonces, se tiene
$$\zeta_K (s) = \zeta (s) \, L (s,\chi).$$
Invito que el lector revise el apéndice~\ref{ap:Dirichlet} para más detalles.

Recordamos que $\O_K = \ZZ [i]$ es un dominio de ideales principales y
$\O_K^\times = \{ \pm 1, \pm i \}$. Como consecuencia, todo ideal tiene forma
$I = (\alpha)$, y para $I \ne 0$ hay precisamente cuatro maneras de escoger un
generador $\alpha$. Si $\alpha = x + yi$, entonces $N_{K/\QQ} (I) = x^2 + y^2$.
Pongamos
$$C (n) = \# \{ (x,y) \in \ZZ^2 \mid x,y > 0, ~ x^2 + y^2 = n \}.$$
Esta función es responsable por contar los puntos enteros en el círculo de radio
$\sqrt{n}$.

Tenemos entonces una identidad con series de Dirichlet
\[ \zeta_K (s) = \sum_{n\ge 0} \frac{C (n)}{n^s} =
\Bigl(\sum_{n\ge 1} \frac{1}{n^s}\Bigr)\,\Bigl(\sum_{n\ge 1} \frac{\chi (n)}{n^s}\Bigr). \]
Comparando los coeficientes, nos sale
$$C (n) = \sum_{d \mid n} \chi (d).$$
De la multiplicatividad del carácter $\chi$ se sigue también que
$C (mn) = C (m)\,C (n)$ si $m$ y $n$ son coprimos. Calculando ahora $C (n)$ para
$n = p^e$, se obtiene el siguiente curioso resultado.

\begin{teorema}
  Supongamos que $n = p_1^{e_1}\cdots p_k^{e_k}$, donde los $p_i$ son diferentes
  primos.
  \begin{itemize}
  \item Si $p_i \equiv 3 \pmod{4}$ para algún $i$ y $e_i$ es impar, entonces
    el círculo de radio $\sqrt{n}$ no contiene ningún punto entero.

  \item En el caso contrario, el número de puntos enteros en el círculo será
    $4\,C (n)$, donde
    $$C (n) = \prod_{p_i \equiv 1~(4)} (e_i + 1).$$
  \end{itemize}
\end{teorema}

\begin{figure}
  \begin{center}
    \includegraphics{pic/circle-points.pdf}
  \end{center}

  \caption{$12$ puntos enteros en el círculo de radio $5$}
\end{figure}

\begin{comentario}
  En general, para cualquier extensión finita abeliana $K/\QQ$ se cumple
  una descomposición en ciertas series L de Dirichlet
  $$\zeta_K (s) = \prod_\chi L (s,\chi).$$
  En particular, para el carácter trivial sale el factor $\zeta (s)$.
  Más adelante veremos que $\zeta_K (s)$ determina la factorización de primos
  racionales en $\O_K$. La fórmula de arriba significa que
  \emph{en el caso abeliano} la factorización depende de $p$ mód $N$ para algún
  $N$. El caso no-abeliano es más complicado.

  Véase por ejemplo \cite[Chapter~4]{Washington-GTM83}.
\end{comentario}

%%%%%%%%%%%%%%%%%%%%%%%%%%%%%%%%%%%%%%%%%%%%%%%%%%%%%%%%%%%%%%%%%%%%%%%%%%%%%%%%

\section{Fórmula analítica del número de clases}

Aunque no lo vamos a probar, la función $\zeta_K (s)$ admite una
\textbf{prolongación meromorfa} a todo $s \in \CC$, con el único polo en
$s = 1$. Nos interesará más que todo el residuo en $s = 1$.

Nuestro próximo gran objetivo será probar que
\[ \lim_{s \to 1^+} (s - 1)\,\zeta_K (s) =
   \frac{2^{r_1}\,(2\pi)^{r_2}\,\Reg_K\,h_k}{\#\mu_K\,\sqrt{|\Delta_K|}}. \]
Esta fórmula contiene todos los invariantes básicos de $K$ que hemos
considerado:
\begin{itemize}
\item el número de encajes reales y complejos,
\item el número de clases $h_K = \# \Cl (K)$,
\item el discriminante $\Delta_K$,
\item el número de las raíces de la unidad $\# \mu_K = \# (\O_K^\times)_{tors}$,
\item el \textbf{regulador} $\Delta_K$ que es el covolumen del retículo
  $\Lambda = L (\O_K^\times) \subset H$. En términos específicos, si
  $u_1, \ldots, u_{r_1 + r_2 - 1}$ son unidades fundamentales de $\O_K^\times$,
  entonces
  $$\Reg_K = |\det (n_i\,\log |\tau_i (u_j)|)_{i,j = 1,\ldots,r_1+r_2-1}|.$$
  Aquí $\tau_1,\tau_2,\ldots,\tau_{r_1+r_2-1}$ son los encajes no conjugados
  $K \hookrightarrow \CC$, con un encaje quitado, dado que
  $\prod_\tau \tau (u_j) = 1$. El coeficiente $n_i$ viene dado por
  \[ n_i = \begin{cases}
    1, & \text{si }\tau_i\text{ es real},\\
    2, & \text{si }\tau_i\text{ es complejo}.
  \end{cases} \]
\end{itemize}

La fórmula del número de clases fue descubierta por Dirichlet para el caso de
campos cuadráticos, y la versión general es de Dedekind.

%%%%%%%%%%%%%%%%%%%%%%%%%%%%%%%%%%%%%%%%%%%%%%%%%%%%%%%%%%%%%%%%%%%%%%%%%%%%%%%%

\section{Ejemplos de uso de la fórmula del número de clases}

Antes de probar la fórmula, podemos ver algunos ejemplos de su uso. Consideremos
el campo cuadrático real $K = \QQ (\sqrt{5})$. Argumentando de la misma manera
que en \S\ref{sec:funcion-zeta-de-Q(i)}, se demuestra la identidad
$$\zeta_K (s) = \zeta (s) \, L (s,\chi),$$
donde $\chi (n) = \legendre{n}{5}$.
Ahora $L (s,\chi)$ converge en $s = 1$ a un valor no nulo, mientras que
$\lim_{s\to 1^+} (s - 1)\,\zeta (s) = 1$, así que
$$\lim_{s \to 1^+} (s - 1)\,\zeta_K (s) = L (1,\chi).$$

En general, para el carácter cuadrático $\chi (n) = \legendre{n}{p}$ mód $p$
se tiene
$$\exp (g (\chi)\,L (1,\chi)) = \prod_n (1 - \zeta_p^n)\,\prod_r (1 - \zeta_p)^{-1},$$
donde
$$g (\chi) = \sum_{1 \le a \le p-1} \chi (a)\,\zeta_p^a$$
es la \textbf{suma de Gauss}, y los productos son sobre los no-residuos y
residuos cuadráticos mód $p$ respectivamente. Para la prueba, véase
el ejercicio~\ref{ejerc:L-series-caracter-cuadratico-mod-p}.

En nuestro caso calculamos que
$$g (\chi) = \zeta_5 - \zeta_5^2 - \zeta_5^3 + \zeta_5^4 = \sqrt{5}$$
y
$$(1 - \zeta_p^2)\,(1 - \zeta_5^3)\,(1 - \zeta_5)^{-1}\,(1 - \zeta_5^4)^{-1} = \frac{3 + \sqrt{5}}{2}.$$
Entonces,
\begin{equation}
  \label{eq:formula-de-clases-sqrt5-1}
  \lim_{s \to 1^+} (s - 1)\,\zeta_K (s) =
  L (1, \chi) = \frac{1}{\sqrt{5}}\,\log\frac{3 + \sqrt{5}}{2}.
\end{equation}

El regulador en este caso será igual a $\log u$, donde
$u = \frac{1+\sqrt{5}}{2}$ es la unidad fundamental de $\O_K^\times$.
Tenemos $r_1 = 2$, $r_2 = 0$ y $\mu_K = \{ \pm 1 \}$, $\Delta_K = 5$, y la
fórmula del número de clases nos da entonces
\begin{equation}
  \label{eq:formula-de-clases-sqrt5-2}
  \lim_{s \to 1^+} (s - 1)\,\zeta_K (s) =
  \frac{2^{r_1}\,(2\pi)^{r_2}\,\Reg_K\,h_k}{\#\mu_K\,\sqrt{|\Delta_K|}} =
  \frac{1}{\sqrt{5}}\,h_K\cdot 2\,\log \frac{1+\sqrt{5}}{2}.
\end{equation}
En fin,
\[ 2\,\log \frac{1+\sqrt{5}}{2} =
   \log \left(\frac{1+\sqrt{5}}{2}\right)^2 =
   \log \frac{3+\sqrt{5}}{2}. \]
Comparando \eqref{eq:formula-de-clases-sqrt5-1} y
\eqref{eq:formula-de-clases-sqrt5-2}, podemos concluir que $h_K = 1$.

\vspace{1em}

Si $K = \QQ[x]/(f)$, entonces para calcular el residuo de $\zeta_K (s)$ en
$s = 1$ en PARI/GP, basta digitar \texttt{lfun($f$, 1)}. Calculamos el residuo
para $K = \QQ (\sqrt{10})$.
\begin{shaded}
\begin{verbatim}
? lfun (x^2 - 10, 1)
% = 1.1500865228483708943221826442284221318*x^-1 + O(x^0)
\end{verbatim}
\end{shaded}

La unidad fundamental en este caso es $u = 3 + \sqrt{10}$, el discriminante
es $\Delta_K = 40$, así que
\[ \lim_{s \to 1^+} (s - 1)\,\zeta_K (s) =
   \frac{2^{r_1}\,(2\pi)^{r_2}\,\Reg_K\,h_k}{\#\mu_K\,\sqrt{|\Delta_K|}} =
   \frac{2}{\sqrt{40}}\,\log (3 + \sqrt{10})\cdot h_K. \]
\begin{shaded}
\begin{verbatim}
? 2 / sqrt (40) * log (3 + sqrt(10))
% = 0.57504326142418544716109132211421106589
? polcoef (lfun (x^2 - 10, 1), -1) / %
% = 2.0000000000000000000000000000000000000
\end{verbatim}
\end{shaded}
Esto nos permite concluir que $h_K = 2$.

De hecho, en este caso también se puede ocupar una descomposición
$\zeta_K (s) = \zeta (s) \, L (s,\chi)$ para cierto carácter de Dirichlet $\chi$
(véase ejercicio~\ref{ejerc:caracter-para-campo-cuadratico}), y luego obtener
el valor de $L (1,\chi)$, pero no quiero entrar en los detalles de este cálculo.

%%%%%%%%%%%%%%%%%%%%%%%%%%%%%%%%%%%%%%%%%%%%%%%%%%%%%%%%%%%%%%%%%%%%%%%%%%%%%%%%

\section{Número de clases de \texorpdfstring{$\QQ (\sqrt{-p})$}{ℚ(√−p)}}

Ahora vamos a aplicar la fórmula del número de clases a campos cuadráticos
imaginarios $K = \QQ (\sqrt{-d})$. En particular, tomemos $d = p$ primo, $p > 3$
y $p \equiv 3 \pmod{4}$. En este caso $\Delta_K = -p$, $\O_K^\times = \mu_K = \{ \pm 1 \}$,
y $Reg_K = 1$. Tenemos entonces
\[ \lim_{s \to 1^+} (s - 1)\,\zeta_K (s) =
   \frac{2^{r_1}\,(2\pi)^{r_2}\,\Reg_K\,h_k}{\#\mu_K\,\sqrt{|\Delta_K|}} =
   \frac{\pi}{\sqrt{p}}\,h_K. \]

Notamos que bajo nuestra hipótesis de que $p \equiv 3\pmod{4}$, la factorización
de un primo racional $q$ en $\O_K$ depende del símbolo de Legendre
$\legendre{q}{p}$, y como en \S\ref{sec:funcion-zeta-de-Q(i)} se demuestra la
identidad
$$\zeta_K (s) = \zeta (s)\,L (s,\chi),$$
donde $\chi (n) = \legendre{n}{p}$. Esto implica que
$$\lim_{s \to 1^+} (s - 1)\,\zeta_K (s) = L (1,\chi),$$
y luego
$$h_K = \frac{\sqrt{p}}{\pi}\,L (1,\chi).$$

Podemos de nuevo ocupar la fórmula
$$\exp (g (\chi)\,L(1,\chi)) = \prod_n (1 - \zeta_p^n)\,\prod_r (1 - \zeta_p)^{-1}.$$

El famoso cálculo de Gauss nos dice que
\[ g (\chi) = \begin{cases}
  \sqrt{p}, & p \equiv 1 \pmod{4},\\
  i\sqrt{p}, & p \equiv 3 \pmod{4}
\end{cases} \]
(véase \cite[Chapter~6]{Ireland-Rosen}).

Tomando los logaritmos, se obtiene (módulo $2\pi i\ZZ$)
$$L(1,\chi) = -\frac{1}{i\sqrt{p}}\,\sum_{1 \le a \le p-1} \chi (a)\,\log (1 - \zeta_p^a).$$
Nos interesa la suma
$$S_\chi = -\sum_{1 \le a \le p-1} \chi (a)\,\log (1 - \zeta_p^a).$$
Notamos que $\chi (-1) = -1$, así que podemos escribir
\[ 2 S_\chi = \sum_{1 \le a \le p-1} \chi (a)\,\Bigl(\log (1 - \zeta_p^{-a}) - \log (1 - \zeta_p^a)\Bigr) =
   \sum_{1 \le a \le p-1} \chi (a)\,\log \frac{1 - \zeta_p^{-a}}{1 - \zeta_p^a}. \]
Aquí
\[ \log \frac{1 - \zeta_p^{-a}}{1 - \zeta_p^a} =
   \log (-\zeta_p^{-a}) =
   \log \exp \left(\pi i - \frac{2\pi i a}{p}\right) =
   2\pi i\,\left(\frac{1}{2} - \frac{a}{p}\right). \]
Entonces,
$$S_\chi = -\frac{\pi i}{p}\,\sum_{1 \le a \le p-1} \chi (a)\,a,$$
y luego
\[ h_K = \frac{\sqrt{p}}{\pi}\,L(1,\chi) =
   \frac{\sqrt{p}}{\pi}\,\frac{1}{i\sqrt{p}}\,S_\chi =
   -\frac{1}{p}\,\sum_{1 \le a \le p-1} \chi (a)\,a. \]

Podemos escribir (recordamos que $p \equiv 3 \pmod{4}$, así que
$\chi (-1) = -1$)
\begin{multline}
  \label{eq:suma-de-caracteres-chi(a)-a-1}
  p h_K = -\sum_{1 \le a \le p-1} \chi (a)\,a =
  -\sum_{1 \le a < p/2} \chi (a)\,a - \sum_{1 \le a < p/2} \chi (p-a)\,(p-a) \\
  = -2\sum_{1 \le a < p/2} \chi (a)\,a + p\sum_{1 \le a < p/2} \chi (a).
\end{multline}
Por otra parte,
\begin{multline*}
  p h_K = -\sum_{1 \le a \le p-1} \chi (a)\,a =
  -\sum_{\substack{1 \le a \le p-1 \\ a \text{ par}}} \chi (a)\,a - \sum_{\substack{1 \le a \le p-1 \\ a \text{ par}}} \chi (p-a)\,(p-a) \\
= -4\,\chi(2)\,\sum_{1 \le a < p/2} \chi (a)\,a + p\,\chi(2)\,\sum_{1 \le a < p/2} \chi (a),
\end{multline*}
así que
\begin{equation}
  \label{eq:suma-de-caracteres-chi(a)-a-2}
  p\chi(2)\,h_K = -4\,\sum_{1 \le a < p/2} \chi (a)\,a + p\,\sum_{1 \le a < p/2} \chi (a),
\end{equation}

Ahora comparando \eqref{eq:suma-de-caracteres-chi(a)-a-1}
y \eqref{eq:suma-de-caracteres-chi(a)-a-2}, nos sale
$$h_K\,(2 - \chi (2)) = \sum_{1 \le a < p/2} \chi (a).$$
Recordamos que
\[ \chi (2) = \begin{cases}
  +1, & p \equiv 7 \pmod{8},\\
  -1, & p \equiv 3 \pmod{8}.
\end{cases} \]

Nuestros cálculos nos llevan entonces al siguiente resultado
(que ya fue mencionado en \S\ref{sec:campos-cuadraticos-imaginarios}).

\begin{teorema}[Dirichlet]
  Sea $p > 3$ un primo tal que $p \equiv 3 \pmod{4}$. Consideremos el campo
  cuadrático imaginario $K = \QQ (\sqrt{-p})$. Si $p \equiv 7 \pmod{8}$,
  entonces
  $$h_K = \sum_{1 \le a < p/2} \legendre{a}{p},$$
  y si $p \equiv 3 \pmod{8}$, entonces
  $$h_K = \frac{1}{3}\,\sum_{1 \le a < p/2} \legendre{a}{p}.$$
\end{teorema}

\begin{corolario}
  Si $p \equiv 3 \pmod{4}$, entonces el intervalo $[1, (p-1)/2]$ contiene más
  residuos que no-residuos cuadráticos.
\end{corolario}

\begin{comentario}
  En general, para \emph{cualquier} campo cuadrático $K = \QQ (\sqrt{d})$
  (imaginario o real) se puede definir un carácter $\chi$ módulo $|\Delta_K|$
  que gobierna la factorización de primos en $\O_K$ y nos lleva a la fórmula
  $$\zeta_K (s) = \zeta (s) \, L (s,\chi)$$
  ---véase ejercicio~\ref{ejerc:caracter-para-campo-cuadratico}.

  Luego los métodos parecidos a los de arriba nos permiten calcular $h_K$
  en términos de $L (1,\chi)$. Para los detalles, véase
  \cite[Chapter~5]{Borevich-Shafarevich}.
\end{comentario}

%%%%%%%%%%%%%%%%%%%%%%%%%%%%%%%%%%%%%%%%%%%%%%%%%%%%%%%%%%%%%%%%%%%%%%%%%%%%%%%%

\section{Demostración de la fórmula del número de clases}

**TODO**

%%%%%%%%%%%%%%%%%%%%%%%%%%%%%%%%%%%%%%%%%%%%%%%%%%%%%%%%%%%%%%%%%%%%%%%%%%%%%%%%

\section{Equivalencia aritmética}

**TODO** \cite{Perlis-1977}.

%%%%%%%%%%%%%%%%%%%%%%%%%%%%%%%%%%%%%%%%%%%%%%%%%%%%%%%%%%%%%%%%%%%%%%%%%%%%%%%%

\pagebreak

\phantomsection

\addcontentsline{toc}{section}{Ejercicios}
\section*{Ejercicios}

\begin{ejercicio}
  ¿Cuántos puntos enteros están en la elipse $x^2 - xy + y^2 = n$?

  \noindent (Considere la función zeta de $\QQ (\zeta_3)$.)
\end{ejercicio}

\begin{ejercicio}
  \label{ejerc:caracter-para-campo-cuadratico}
  Sea $K = \QQ (\sqrt{d})$ un campo cuadrático. Para $n$ coprimo con $\Delta_K$
  definamos mediante el \textbf{símbolo de Jacobi} $\legendre{n}{m}$
  \[ \chi (n) = \begin{cases}
    \legendre{n}{|d|}, & \text{si }d \equiv 1 \pmod{4},\\
    (-1)^{\frac{n-1}{2}}\,\legendre{n}{|d|}, & \text{si }d \equiv 3 \pmod{4},\\
    (-1)^{\frac{n^2-1}{8} + \frac{n-1}{2}\,\frac{d'-1}{2}}\,\legendre{n}{|d'|}, & \text{si }d = 2d'.
  \end{cases} \]
  Si $\gcd (n,\Delta_K) \ne 1$, pongamos $\chi (n) = 0$.

  \begin{enumerate}
  \item[1)] Demuestre que $\chi$ es un carácter de Dirichlet mód $|\Delta_K|$.

  \item[2)] Demuestre que $\chi$ determina la factorización de primos
    racionales en $\O_K$:
    \[ p\O_K = \begin{cases}
      \mathfrak{p}\,\overline{\mathfrak{p}}, & \text{si } \chi (p) = +1,\\
      \mathfrak{p}, & \text{si } \chi (p) = -1,\\
      \mathfrak{p}^2, & \text{si } \chi (p) = 0.
    \end{cases} \]

  \item[3)] Demuestre que $\zeta_K (s) = \zeta (s) \, L (s,\chi)$.
  \end{enumerate}
\end{ejercicio}

\begin{ejercicio}
  \label{ejerc:L-series-caracter-cuadratico-mod-p}
  Sea $p$ un número primo y $\chi$ el carácter de Dirichlet de orden $2$ mód
  $p$, definido por el símbolo de Legendre $\chi (n) = \legendre{n}{p}$.

  \begin{enumerate}
  \item[1)] Demuestre que
    $$\exp (g (\chi)\,L (1,\chi)) = \prod_n (1 - \zeta_p^n)\,\prod_r (1 - \zeta_p^r)^{-1},$$
    donde $g (\chi) = \sum_{1 \le a
      \le p-1} \chi (a)\,\zeta_p^a$, y los productos son sobre los no-residuos y
    residuos cuadráticos mód $p$ respectivamente.

  \item[2)] Use la parte anterior para calcular $L (1,\chi)$, donde $\chi$ es el
    carácter de orden $2$ mód $5$.  (Para el valor numérico en PARI/GP, basta
    digitar \texttt{lfun(5,1)})
  \end{enumerate}
\end{ejercicio}

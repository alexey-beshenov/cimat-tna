\chapter{Teorema de Dirichlet sobre primos en progresiones aritméticas}
\label{ap:Dirichlet}

\pdfbookmark{Clase 16 (07/10/20)}{clase-16}
\marginpar{\small Clase 16 \\ 07/10/20}

Uno de los temas principales de nuestro curso es la descomposición de primos
racionales $p \in \ZZ$ en ideales primos en el anillo de enteros $\O_K$ de un
campo de números $K/\QQ$. Cuando el grupo de Galois $\Gal (K/\QQ)$ es abeliano,
el tipo de descomposición depende del resto de $p$ módulo $m$ para cierto $m$.
En particular, lo observamos en el caso cuando $K = \QQ (\zeta_m)$ es un campo
ciclotómico (véase \ref{thm:descomposicion-en-campos-ciclotomicos}).
Esto hace particularmente interesante el
\textbf{teorema de Dirichlet sobre primos en progresiones aritméticas} que
afirma que si $\gcd (a,m) = 1$, entonces hay un número infinito de primos tales
que $p \equiv a \pmod{m}$. Además, en cierto sentido técnico, hay precisamente
$1/\phi(m)$ primos con esta propiedad. Este apéndice está dedicado a un bosquejo
de la prueba que usa ideas importantes de la teoría analítica de números.

%%%%%%%%%%%%%%%%%%%%%%%%%%%%%%%%%%%%%%%%%%%%%%%%%%%%%%%%%%%%%%%%%%%%%%%%%%%%%%%%

\section{Caso de p = 1 (m)}

Primero veremos un resultado particular cuando $a = 1$. En este caso existe una
prueba elemental basada en las propiedades de polinomios ciclotómicos.

\begin{lema}
  Si $p$ es un primo tal que $p \nmid m$ y $p \mid \Phi_m (N)$ para algún
  $N \in \ZZ$, entonces $p \equiv 1 \pmod{m}$.

  \begin{proof}
    Las raíces de $\Phi_m (x)$ en $\FF_p$ corresponden a los elementos de orden
    $m$ en $\FF_p^\times$. Entonces, si $\Phi_m (N) \equiv 0 \pmod{p}$, esto
    significa que $N$ tiene orden $m$ en $\FF_p^\times$, y luego
    $p\equiv 1\pmod{m}$.
  \end{proof}
\end{lema}

\begin{ejemplo}
  Tenemos $\Phi_5 (3) = \frac{3^5 - 1}{3 - 1} = 242/2 = 121 = 11^2$.
  Ahora $11 \equiv 1 \pmod{5}$.
\end{ejemplo}

\begin{teorema}
  Para $m$ fijo, existe un número infinito de primos $p \equiv 1 \pmod{m}$.

  \begin{proof}
    Supongamos que $p_1,\ldots,p_s$ son los únicos primos tales que
    $p_i\equiv 1 \pmod{m}$. Pongamos
    $$N = C m p_1\cdots p_s,$$
    donde $C \gg 1$ se escoge de tal manera que $\Phi_m (N) > 1$.
    El término constante de cualquier polinomio ciclotómico es igual a $\pm 1$
    (véase \eqref{eqn:termino-constante-de-Phi-n}), así que
    $$\Phi_m (N) \equiv \pm 1 \pmod{m}, \quad \Phi_m (N) \equiv \pm 1 \pmod{p_i}.$$
    Ahora si $p$ es un factor primo de $\Phi_m (N)$, entonces las congruencias
    de arriba demuestran que $p \nmid m$ y $p \notin \{ p_1, \ldots, p_s \}$.
    Pero luego el lema de arriba implica que
    $p \equiv 1 \pmod{m}$. Contradicción.
  \end{proof}
\end{teorema}

Notamos que aunque el argumento de arriba está formulado como una prueba por
contradicción, en realidad este nos da una manera de obtener a partir de un
primo $p \equiv 1 \pmod{m}$ un número infinito de $p$ con la misma propiedad,
aunque estos $p$ serán bastante grandes y muy pronto el número $\Phi_m (N)$
se volverá demasiado grande para buscar sus factores primos en práctica.

\begin{ejemplo}
  Consideremos $m = 3$. El primo $p_1 = 7$ claramente cumple la condición
  $p_1 \equiv 1 \pmod{3}$. Luego
  $$p_2 = \Phi_3 (3\cdot p_1) = 463$$
  casualmente es también primo. Luego
  $$p_3 = \Phi_3 (3\cdot p_1 p_2) = 94546453$$
  es también primo. En fin,
  $$\Phi_3 (3\cdot p_1 p_2 p_3) = 845066824425253137587881 = 2059573\cdot 410311663837724197$$
  tiene dos factores primos $p_4$ y $p_5$. Todos los primos $p_i$ de arriba
  cumplen la condición $p_i \equiv 1 \pmod{3}$.
\end{ejemplo}

El argumento con polinomios ciclotómicos funciona para $a = 1$. Nos interesa
el caso general de $\gcd (a,m)=1$, y allí ya no existe una prueba elemental;
todas las pruebas conocidas tienen origen analítico.

%%%%%%%%%%%%%%%%%%%%%%%%%%%%%%%%%%%%%%%%%%%%%%%%%%%%%%%%%%%%%%%%%%%%%%%%%%%%%%%%

\section{Series de Dirichlet}

A partir de ahora vamos a ocupar varios resultados de la teoría analítica de
números. Como referencia, recomiendo los apuntes \cite{Everiste-ANT}. Otro texto
útil es \cite{Hlawka-Taschner-Schoissengeier}. Allí se encuentran demostraciones
de todas las afirmaciones de abajo.

\begin{definicion}
  Para una función $f\colon \ZZ_{>0} \to \CC$ la \textbf{serie de Dirichlet}
  correspondiente viene dada por
  $$F (s) = \sum_{n\ge 1} \frac{f(n)}{n^s},$$
  donde $s$ es una variable compleja.
\end{definicion}

Un caso muy particular de las series de Dirichlet es la
\textbf{función zeta de Riemann}
$$\zeta (s) = \sum_{n\ge 1} \frac{1}{n^s}.$$

Las series de Dirichlet cumplen las siguientes propiedades generales.

\begin{enumerate}
\item Existe un número $-\infty \le \sigma_0 \le +\infty$, llamado
  la \textbf{abscisa de convergencia}, tal que $F (s)$ converge para
  $\Re s > \sigma_0$. Este es un análogo del radio de convergencia que se tiene
  para las series de potencias habituales.

\item De la misma manera, existe la \textbf{abscisa de convergencia absoluta}
  $\sigma_a$ tal que $F (s)$ converge absolutamente para $\Re s > \sigma_a$.
  Siempre se tiene $\sigma_0 \le \sigma_a \le \sigma_0 + 1$.

\item Si existe una constante $C$ tal que
  $\left|\sum_{1 \le n\le N}\right| \le C$ para todo $N$, entonces $F (s)$
  converge para $\Re s > 0$.

\item Si $f$ es una función multiplicativa en el sentido fuerte, es decir,
  $f (ab) = f (a)\,f (b)$ para cualesquiera $a, b \in \ZZ_{> 0}$, entonces
  para $\Re s > \sigma_a$ se cumple la \textbf{fórmula del producto de Euler}
  $$F (s) = \prod_p \frac{1}{1 - f (p)\,p^{-s}},$$
  donde el producto es sobre todos los primos.

  Esta fórmula tiene sus orígenes en el teorema fundamental de la aritmética.
  De hecho, manipulando con las series de manera
  formal\footnote{¿O informal? :-)}  y ocupando la multiplicatividad de $f$,
  \[ \prod_p \frac{1}{1 - f (p)\,p^{-s}} =
     \prod_p \sum_{e \ge 0} \frac{f (p^e)}{p^{es}} =
     \sum_{n\ge 1} \frac{f (n)}{n^s}. \]
\end{enumerate}

Un homomorfismo $\chi\colon (\ZZ/m\ZZ)^\times \to \CC^\times$ se llama un
\textbf{carácter de Dirichlet} mód $m$. Notamos que los valores de $\chi$
son las raíces $m$-ésimas de la unidad. Un carácter $\chi$ se levanta a una
función $\widetilde{\chi}\colon \ZZ \to \CC$ mediante
\[ \widetilde{\chi} (a) = \begin{cases}
  \chi (a), & \text{si }\gcd (a,m) = 1, \\
  0, & \text{si }\gcd (a,m) \ne 1.
\end{cases} \]
A continuación vamos a escribir simplemente «$\chi$» en lugar de
«$\widetilde{\chi}$».

Los caracteres de Dirichlet mód $m$ forman un grupo abeliano respecto a
la multiplicación punto por punto. El elemento neutro en este grupo es
el \textbf{carácter principal} (o \textbf{trivial}) $\chi_0$ definido mediante
\[ \chi_0 (a) = \begin{cases}
  1, & \text{si }\gcd (a,m) = 1,\\
  0, & \text{si }\gcd (a,m) \ne 1.
\end{cases} \]

En general, para cualquier grupo abeliano finito $G$ podemos considerar su grupo
de caracteres $\widehat{G} = \Hom (G, \CC^\times)$, y este cumple la
\textbf{relación de ortogonalidad}
\[ \sum_{\chi \in \widehat{G}} \chi (g)\,\chi^{-1} (h) = \begin{cases}
    |G|, & \text{si }g = h,\\
    0, & \text{si }g \ne h.
\end{cases} \]
(Este cálculo se ve en la teoría de representación de grupos finitos.)
En particular, para $G = (\ZZ/m\ZZ)^\times$ y $\gcd (a,m) = \gcd (b,m) = 1$ se
cumple
\begin{equation}
  \label{eqn:relacion-de-ortogonalidad}
  \sum_\chi \chi (a)\,\chi^{-1} (b) = \begin{cases}
    \phi (m), & \text{si }a \equiv b \pmod{m},\\
    0, & \text{si }a \not\equiv b \pmod{m}.
  \end{cases}
\end{equation}

\begin{definicion}
  Dado un carácter de Dirichlet $\chi$, la \textbf{serie L} correspondiente
  se define mediante
  $$L (s,\chi) = \sum_{n\ge 1} \frac{\chi (n)}{n^s}.$$
\end{definicion}

La función zeta de Riemann converge para $\Re s > 1$ y también cumple la
fórmula del producto (descubierta por Euler)
$$\zeta (s) = \prod_p \frac{1}{1 - p^{-s}}. \quad (\Re s > 1)$$
Notamos que $L (s,\chi)$ también converge absolutamente para $\Re s > 1$
(los coeficientes $\chi (n)$ son nulos o cumplen con $|\chi (n)| = 1$),
así que gracias a la multiplicatividad de $\chi$,
$$L (s,\chi) = \prod_p \frac{1}{1 - \chi (p)\,p^{-s}}. \quad (\Re s > 1)$$

Por otra parte, para $s = 1$ se obtiene la \textbf{serie armónica}
$\sum_{n\ge 1} \frac{1}{n}$ que es divergente. No es difícil calcular que (véase \ref{lema:convergencia-y-residuo-de-zeta-de-Riemann})
\begin{equation}
  \label{eq:residuo-de-zeta-en-1}
  \lim_{s\to 1^+} (s-1) \, \zeta (s) = 1.
\end{equation}
Para las series $L$, si $\chi = \chi_0$ es el carácter principal mód $m$,
entonces se obtiene
\[ L (s, \chi_0) = \prod_{p \nmid m} \frac{1}{1 - p^{-s}}
   = \prod_{p \mid m} (1 - p^{-s}) \, \zeta (s). \]
En particular, $L (s, \chi_0)$ es divergente en $s = 1$, y se tiene
\[ \lim_{s\to 1^+} (s-1)\,L (s, \chi_0) = \prod_{p \mid m} (1 - p^{-1})
   = \phi (m)/m. \]

Por otra parte, si $\chi \ne \chi_0$ no es un carácter principal, entonces
$\sum_{a \in (\ZZ/m\ZZ)^\times} \chi (a) = 0$, y de allí es fácil ver que
para cualquier $N$ se cumple $\Bigl|\sum_{1 \le n \le N} \chi (n)\Bigr| < m$.
Esto implica que $L (s,\chi)$ converge para $\Re s > 0$ en el caso cuando
$\chi \ne \chi_0$. El resultado clave que contiene todas las dificultades
técnicas de la prueba de Dirichlet es el siguiente.

\begin{teorema}
  \label{thm:L(1,chi)}
  Si $\chi$ no es un carácter principal, entonces $L (1,\chi) \ne 0$.

  \begin{proof}
    Una prueba mediante la teoría algebraica de números es
    \ref{cor:L(1,chi)-ne-0}. Para una prueba concisa y elemental, véase por
    ejemplo \cite[\S 16.5]{Ireland-Rosen}. En general, $L (s,\chi) \ne 0$ para
    $\Re s = 1$. Esto se demuestra, por ejemplo, en
    \cite[Chapter~5]{Everiste-ANT} y
    \cite[Chapter~6]{Hlawka-Taschner-Schoissengeier}.
  \end{proof}
\end{teorema}

\begin{ejemplo}
  Sea $\chi$ el carácter no principal mód $3$. Este viene dado por
  $\chi (1) = +1$ y $\chi (2) = -1$. Ahora
  $$L (1,\chi) = \sum_{n \ge 1} \frac{\chi (n)}{n} = \sum_{n \ge 0} \left(\frac{1}{3n + 1} - \frac{1}{3n+2}\right) = \sum_{n \ge 0} \frac{1}{(3n+1)\,(3n+2)}.$$
  Para evaluar la suma, notamos que
  $$\int_0^1 t^{3n} \, (1-t) \, dt = \frac{1}{(3n+1)\,(3n+2)}.$$
  Ahora
  \begin{multline*}
    \sum_{n \ge 0} \frac{1}{(3n+1)\,(3n+2)} = \sum_{n \ge 0} \int_0^1 t^{3n} \, (1-t) \, dt = \int_0^1 \left(\sum_{n \ge 0} t^{3n} \, (1-t)\right) \, dt =\\
    \int_0^1 \frac{1-t}{1-t^3}\,dt = \int_0^1 \frac{1}{1 + t + t^2}\,dt = \frac{2}{\sqrt{3}}\,\left[\arctan \left(\frac{2t+1}{\sqrt{3}}\right)\right]_0^1 = \frac{2}{\sqrt{3}}\,\left(\frac{\pi}{3} - \frac{\pi}{6}\right) = \frac{\pi}{3\sqrt{3}}. \qedhere
  \end{multline*}
\end{ejemplo}

%%%%%%%%%%%%%%%%%%%%%%%%%%%%%%%%%%%%%%%%%%%%%%%%%%%%%%%%%%%%%%%%%%%%%%%%%%%%%%%%

\section{Densidad de primos}

\begin{lema}
  \label{lema:log-zeta-log-sm1}
  Se tiene
  $$\lim_{s\to 1^+} \log \zeta (s)\left/\log \frac{1}{s-1}\right. = 1.$$

  \begin{proof}
    Denotemos $\psi (s) = (s-1)\,\zeta (s)$. Tenemos entonces
    $$\log \psi (s) = \log (s-1) + \log \zeta (s).$$
    Luego,
    \[ \log \zeta (s)\left/\log \frac{1}{s-1}\right. =
       1 + \log \psi (s)\left/\log \frac{1}{s-1}\right.. \]
    Pasando al límite $s \to 1^+$, notamos que $\psi (s) \to 1$
    según \eqref{eq:residuo-de-zeta-en-1}.
  \end{proof}
\end{lema}

\begin{lema}
  \label{lema:log-zeta}
  Se tiene
  $$\log \zeta (s) = \sum_p \frac{1}{p^s} + R (s),$$
  donde la suma es sobre todos los primos, y la función $R (s)$ es acotada para
  $s \to 1^+$.

  \begin{proof}
    La fórmula del producto nos dice que
    $$\zeta (s) = \prod_{p \le N} \frac{1}{1 - p^{-s}}\,R_N (s),$$
    donde $\lim_{N\to\infty} R_N (s) = 1$. Tomando el logaritmo de ambas partes,
    se obtiene\footnote{Usando la serie del logaritmo
      $-\log (1-x) = \sum_{n\ge 1} \frac{x^n}{n}$ para $-1 < x < 1$.}
    \[ \log \zeta (s) = \sum_{p \le N} -\log (1 - p^{-s}) + \log R_N (s) =
    \sum_{p\le N} \sum_{n\ge 1} \frac{p^{-ns}}{n} + \log R_N (s). \]
    Al pasar al límite $N \to \infty$, nos queda
    \[ \log \zeta (s) =
       \sum_p \frac{1}{p^s} + \sum_p \sum_{n \ge 2} \frac{p^{-ns}}{n}. \]
    Usando la fórmula de progresión aritmética, escribamos el segundo
    término como
    $$R (s) = \sum_p \sum_{n \ge 2} \frac{p^{-ns}}{n} = \sum_p p^{-2s}\,\frac{1}{1 - p^{-s}}.$$
    Ahora $\frac{1}{1 - p^{-s}} \le 2$ para todo $p$ y $s > 1$, así que
    \[ R (S) \le 2\,\sum_p \frac{1}{p^{2s}} \le 2\,\sum_{n\ge 1} \frac{1}{n^{2s}} \le 2\,\zeta (2). \qedhere \]
  \end{proof}
\end{lema}

Notamos que lo que acabamos de probar, junto con la divergencia de $\zeta (s)$
en $s=1$, implica que la serie $\sum_p \frac{1}{p}$ es divergente. Esto nos da
una prueba curiosa de la infinitud de números primos.

\vspace{1em}

Las propiedades que acabamos de ver motivan la siguiente definición.

\begin{definicion}
  Sea $X$ un conjunto que consiste en números primos.
  Su \textbf{densidad de Dirichlet} (o \textbf{densidad analítica})
  viene dada por
  $$d (X) = \lim_{s \to 1^+} \sum_{p \in X} \frac{1}{p^s}\left/\log \frac{1}{s-1}\right.,$$
  si el límite correspondiente existe.
\end{definicion}

Los lemas de arriba nos dicen que la densidad es una especie de medida sobre los
conjuntos de números primos. En partucular, se cumplen las siguientes
propiedades.

\begin{enumerate}
\item[1)] Si $X$ es un conjunto finito, entonces $d (X) = 0$.

\item[2)] Si $X$ consiste en todos los primos, salvo un número finito de ellos,
  entonces $d (X) = 1$.

\item[3)] Si $X = Y \cup Z$ es una unión disjunta y las densidades $d (Y)$ y
  $d (Z)$ existen, entonces $d (X) = d (Y) + d (Z)$.
\end{enumerate}

Dirichlet probó lo siguiente.

\begin{teorema}
  Para $\gcd (a,m) = 1$ la densidad de primos tales que $p \equiv a \pmod{m}$
  es igual a $1/\phi(m)$.
\end{teorema}

Notamos que puesto que la densidad no es nula, esto implica que hay un número
infinito de primos con esta propiedad. Sin embargo, el resultado de Dirichlet
es más fuerte: es \emph{cuantitativo}.

Una generalización del teorema de Dirichlet en la teoría algebraica de números
es el teorema de densidad de Chebotarëv, explicada en \S\ref{sec:frobenius}.

%%%%%%%%%%%%%%%%%%%%%%%%%%%%%%%%%%%%%%%%%%%%%%%%%%%%%%%%%%%%%%%%%%%%%%%%%%%%%%%%

\section{Bosquejo de demostración del teorema de Dirichlet}

Dado un carácter de Dirichlet $\chi$ mód $m$, consideremos la serie
$$G (s,\chi) = \sum_p \sum_{k\ge 1} \frac{(\chi (p)\,p^{-s})^k}{k}.$$
Aquí $\left|\frac{(\chi (p)\,p^{-s})^k}{k}\right| \le \frac{1}{p^{ks}}$,
y la serie $\zeta (s)$ converge para $s > 1$ y converge uniformemente para
$s \ge 1 + \delta$, así que $G (s,\chi)$ define una función continua para
$s > 1$. Notamos que
\[ \exp \left(\sum_{k\ge 1} \frac{(\chi (p)\,p^{-s})^k}{k}\right) =
\exp (-\log (1 - \chi (p)\,p^{-s})) = \frac{1}{1 - \chi(p)\,p^{-s}}. \]
Entonces, la fórmula del producto para las series $L$ nos dice que
$$\exp (G (s,\chi)) = L (s,\chi) \quad\text{para }s > 1.$$

\begin{proposicion}
  Tenemos
  \[ \lim_{s\to 1^+} G (s,\chi)\left/\log \frac{1}{s-1}\right. =
  \begin{cases}
    1, & \text{si } \chi = \chi_0, \\
    0, & \text{si } \chi \ne \chi_0.
  \end{cases} \]

  \begin{proof}
    Primero, si $\chi = \chi_0$, entonces para $s > 1$ se tiene
    \[ G (s,\chi) = \log L (s,\chi) =
    \sum_{p\mid m} \log (1 - p^{-s}) + \log \zeta (s), \]
    y podemos concluir usando el lema \ref{lema:log-zeta-log-sm1}.

    Por otra parte, si $\chi \ne \chi_0$, entonces, como fue mencionado en
    \ref{thm:L(1,chi)}, tenemos $L (1,\chi) \ne 0$, y luego
    $\lim_{s\to 1^+} G (s,\chi) = \log L (1,\chi)$. En este caso el valor
    $L (1,\chi)$ será un número complejo, y su logaritmo está bien definido solo
    módulo $2\pi i \ZZ$, pero de todas maneras, $G (s,\chi)$ es acotado para
    $s \to 1^+$.
  \end{proof}
\end{proposicion}

Argumentando como en la prueba del lema \ref{lema:log-zeta}, llegamos a la
conclusión que
$$G (s,\chi) = \sum_{p \nmid m} \chi (p)\,p^{-s} + R_\chi (s),$$
donde $R_\chi (s)$ es acotado para $s \to 1^+$. Dado un número $a$ tal que
$\gcd (a,m) = 1$, multiplicamos ambas partes de la última ecuación por
$\chi^{-1} (a)$ y sumamos sobre todos los caracteres de Dirichlet mód $m$:
$$\sum_\chi \chi^{-1} (a)\,G (s,\chi) = \sum_{p \nmid m} p^{-s} \sum_\chi \chi (p)\,\chi^{-1} (a) + \sum_\chi \chi^{-1} (a)\,R_\chi (s).$$
Usando la relación de ortogonalidad \eqref{eqn:relacion-de-ortogonalidad},
podemos escribir esta expresión como
$$\sum_\chi \chi^{-1} (a)\,G (s,\chi) = \sum_{p \equiv a ~ (m)} p^{-s} + \sum_\chi \chi^{-1} (a)\,R_\chi (s).$$
Al dividir esta expresión por $\log \frac{1}{s-1}$ y tomar el límite
$s \to 1^+$, en la parte izquierda, gracias a la proposición de arriba, nos
quedará $1$. Por otra parte, en la parte derecha estará
$$\phi (m)\,\lim_{s\to 1^+} \sum_{p \equiv a ~ (m)} \frac{1}{p} \left/\log \frac{1}{s-1}\right. = \phi (m) \cdot d (\{ p \mid p \equiv a ~ (m) \}).$$
Esto establece el teorema de Dirichlet. \qed

\vspace{1em}

Voy a reiterar que la parte clave que no fue probada es el teorema
\ref{thm:L(1,chi)} que afirma que $L (1,\chi) \ne 0$ para $\chi \ne \chi_0$.
Una posible prueba es \ref{cor:L(1,chi)-ne-0}.

%%%%%%%%%%%%%%%%%%%%%%%%%%%%%%%%%%%%%%%%%%%%%%%%%%%%%%%%%%%%%%%%%%%%%%%%%%%%%%%%

\section{Densidad natural}

La definición de densidad $d (X)$ parece algo rara al principio, pero después
de ver la prueba, se entiende cuál es su propósito. Hay otra noción de densidad,
llamada la \textbf{densidad natural}:
$$d_{nat} (X) = \lim_{N\to\infty} \frac{\# \{ p \mid p\in X, ~ p \le N \}}{\# \{ p \mid p \le N \}}.$$
Sería interesante investigar cuál es la densidad natural de los primos tales que
$p \equiv a \pmod{m}$. Se puede probar que para la función
$$\pi (a,m,N) = \# \{ p \mid p \equiv a ~ (m), ~ p \le N \}$$
se tiene la asintótica
$$\pi (a,m,N) \sim \frac{1}{\phi (m)}\cdot \frac{N}{\log N}.$$
(Aquí la notación $f (N) \sim g (N)$ significa que
$\lim_{N\to\infty} f(N)/g(N) = 1$.) Véase por ejemplo
\cite[Chapter~7]{Everiste-ANT} o
\cite[Chapter~6]{Hlawka-Taschner-Schoissengeier}.
Este resultado generaliza el célebre \textbf{teorema de los números primos}
que nos dice que
$$\pi (N) \sim \frac{N}{\log N},$$
donde $\pi (N)$ denota el número de primos $\le N$. Como consecuencia,
la densidad natural de los primos $p \equiv a \pmod{m}$ es también igual a
$1/\phi(m)$. En general, ocupando el teorema de los números primos, se puede
probar que si un conjunto de primos $X$ tiene densidad natural $d_{nat} (X)$,
entonces este también tiene densidad de Dirichlet $d (X)$, y las dos coinciden.
Esto no funciona en la otra dirección: existen conjuntos que tienen densidad
de Dirichlet, pero no tienen densidad natural. En general, es más fácil trabajar
con la densidad de Dirichlet.

También cabe mencionar que el teorema de los números primos es un resultado más
complicado, que fue probado por Hadamard y de la Vallée Poussin en 1896,
mientras que Dirichlet probó su teorema en 1837.

%%%%%%%%%%%%%%%%%%%%%%%%%%%%%%%%%%%%%%%%%%%%%%%%%%%%%%%%%%%%%%%%%%%%%%%%%%%%%%%%

\section{Aplicación: irreducibilidad de polinomios ciclotómicos}
\label{sec:irreducibilidad-de-Phi-Dirichlet}

Como una aplicación curiosa del teorema de Dirichlet, podemos probar
la irreducibilidad de polinomios ciclotómicos $\Phi_m \in \ZZ [x]$. El argumento
es el siguiente.

Supongamos que $\Phi_m = fg$ para algunos polinomios $f,g \in \ZZ[x]$.
Sin pérdida de generalidad, $f$ y $g$ son mónicos y $f$ es irreducible. Entonces
alguna raíz $n$-ésima primitiva $\zeta$ es una raíz de $f$. Nos gustaría
probar que cualquier otra raíz primitiva $\zeta^a$, donde $\gcd (a,m) = 1$ es
una raíz de $f$, y por lo tanto $g = 1$.

Si $p$ es un primo tal que $p \equiv a \pmod{m}$, entonces en el anillo
$\ZZ [\zeta_m]$ se cumple
$$p \mid f (\zeta^p) = f (\zeta^a).$$
En efecto, la fórmula del binomio en característica $p$ nos dice que
$f (\zeta^p) \equiv f (\zeta)^p \pmod{p}$.

Ahora gracias al teorema de Dirichlet, existe un número infinito de primos tales
que $p \equiv a \pmod{m}$, y que entonces dividen a $f (\zeta^a)$. Pero
esto implica que $f (\zeta^a) = 0$: un elemento no nulo de $\ZZ [\zeta_m]$ tiene
solo un número finito de divisores primos. \qed

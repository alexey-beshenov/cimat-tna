\documentclass{beamer}

\usetheme{boxes}
\definecolor{beamer@structure@color}{rgb}{0,0,0}

\usecolortheme{structure}

\setbeamertemplate{footline}[frame number]
\setbeamertemplate{frametitle}{\color{black}
\def\myhrulefill{\leavevmode\leaders\hrule height 1pt\hfill\kern 0pt}
\headingfont\insertframetitle\par\vskip-8pt\myhrulefill}

\newcommand{\ZZ}{\mathbb{Z}}
\newcommand{\FF}{\mathbb{F}}
\newcommand{\QQ}{\mathbb{Q}}
\renewcommand{\O}{\mathcal{O}}

\usepackage{amsmath,amssymb}
\DeclareMathOperator{\Cl}{Cl}
\DeclareMathOperator{\GL}{GL}

\setbeamertemplate{navigation symbols}{}

\usepackage{mathspec}
\setmainfont{Montserrat}
\setsansfont{Montserrat}
\setmonofont{PT Mono}
\newfontfamily\headingfont[]{Montserrat Bold}

\usepackage{xcolor}
\usepackage{framed}
\definecolor{shadecolor}{rgb}{0.945, 0.902, 0.698}

\usepackage{tikz-cd}

\usepackage{array}
\newcolumntype{x}[1]{>{\centering\hspace{0pt}}p{#1}}

\begin{document}

%%%%%%%%%%%%%%%%%%%%%%%%%%%%%%%%%%%%%%%%%%%%%%%%%%%%%%%%%%%%%%%%%%%%%%%%%%%%%%%%

\begin{frame}[noframenumbering]
  \headingfont
  \begin{center}
    {\huge Teoría de números

      algebraicos

      en PARI/GP

    }

    \vspace{3em}

    {\large Parte II: ideales en el anillo de enteros}

    \vspace{3em}

    28/09/2020

  \end{center}
\end{frame}

%%%%%%%%%%%%%%%%%%%%%%%%%%%%%%%%%%%%%%%%%%%%%%%%%%%%%%%%%%%%%%%%%%%%%%%%%%%%%%%%

\begin{frame}[plain]
  \headingfont

  \begin{center}
    {\huge ¿Cómo especificar un ideal?}
  \end{center}
\end{frame}

%%%%%%%%%%%%%%%%%%%%%%%%%%%%%%%%%%%%%%%%%%%%%%%%%%%%%%%%%%%%%%%%%%%%%%%%%%%%%%%%

\begin{frame}
  \frametitle{Ideales principales}

  $\alpha\O_K$ se especifica por su generador:

  \begin{itemize}
  \item polinomio en $x$

    = $\alpha$ en la base $1,x,x^2,\ldots,x^{n-1}$ de $\QQ[x]/(f)$

  \item vector \texttt{[$a_1$,\ldots,$a_n$]\textasciitilde}

    = $\alpha$ en la $\ZZ$-base de $\O_K$ calculada en \texttt{$K$.zk}
  \end{itemize}

\end{frame}

%%%%%%%%%%%%%%%%%%%%%%%%%%%%%%%%%%%%%%%%%%%%%%%%%%%%%%%%%%%%%%%%%%%%%%%%%%%%%%%%

\begin{frame}
  \frametitle{Ideales en general}

  \begin{itemize}
  \item $I \subseteq \O_K$ --- $\ZZ$-submódulo libre.

  \item $I$ $\longleftrightarrow$ matriz de una $\ZZ$-base de $I$

    en términos de la base de $\O_K$.

  \item \textbf{Forma normal de Hermite} (HNF).

    \texttt{mathnf($M$)} en PARI/GP.

    ¡Canónica!
  \end{itemize}
\end{frame}

%%%%%%%%%%%%%%%%%%%%%%%%%%%%%%%%%%%%%%%%%%%%%%%%%%%%%%%%%%%%%%%%%%%%%%%%%%%%%%%%

\begin{frame}
  \frametitle{Forma normal de Hermite}

  \begin{itemize}
  \item $H \in M_{n\times n} (\ZZ)$ triangular superior con elementos $\ge 0$.

  \item Coeficiente mayor de la fila = primer coeficiente no nulo.

  \item Coeficiente mayor está a la derecha del coeficiente mayor de la fila
    anterior.

  \item Elementos arriba del coeficiente mayor son estrictamente menor.

  \item Elementos abajo del coeficiente mayor son nulos.
  \end{itemize}

  Para toda $A \in M_{n\times n} (\ZZ)$ existe única $U \in \GL_n (\ZZ)$ tal que
  $H = UA$ está en la HNF.

  Se calcula mediante LLL (= Lenstra–Lenstra–Lovász).
\end{frame}

%%%%%%%%%%%%%%%%%%%%%%%%%%%%%%%%%%%%%%%%%%%%%%%%%%%%%%%%%%%%%%%%%%%%%%%%%%%%%%%%

\begin{frame}
  \frametitle{Ejemplo}

  \[ \underbrace{\begin{pmatrix}
     0 & +1  & +1 \\
    -1 & +1 & -1 \\
    -1 & +1 & +2
  \end{pmatrix}}_U \cdot \underbrace{\begin{pmatrix}
      +3 & -2 &  0 \\
      +4 & +3 & -3 \\
       0 & -2 & +2
  \end{pmatrix}}_A = \underbrace{\begin{pmatrix}
      2 & 1 & 1 \\
      0 & 4 & 1 \\
      0 & 0 & 2
  \end{pmatrix}}_H \]

\end{frame}

%%%%%%%%%%%%%%%%%%%%%%%%%%%%%%%%%%%%%%%%%%%%%%%%%%%%%%%%%%%%%%%%%%%%%%%%%%%%%%%%

\begin{frame}[fragile]
  \frametitle{Pasando a la HNF}

  \begin{shaded}\small
\begin{verbatim}
? K = nfinit (x^2-5);
? K.zk
% = [1, 1/2*x - 1/2]
? a = idealhnf(K, 4+x)
% =
[11 8]

[ 0 1]
\end{verbatim}
  \end{shaded}

  Interpretación:
  \[ \O_K = \alpha_1\ZZ \oplus \alpha_2\ZZ, \quad
     \alpha_1 = 1, \quad
     \alpha_2 = \frac{\sqrt{5} - 1}{2}, \]
  $$(4 + \sqrt{5})\O_K = 11\alpha_1\ZZ \oplus (8\alpha_1 + \alpha_2)\ZZ.$$
\end{frame}

%%%%%%%%%%%%%%%%%%%%%%%%%%%%%%%%%%%%%%%%%%%%%%%%%%%%%%%%%%%%%%%%%%%%%%%%%%%%%%%%

\begin{frame}[fragile]
  \frametitle{Ejemplo bien conocido}

  $K = \QQ (i)$,

  $\O_K = \ZZ[i]$ es un DIP.

  \begin{shaded}\small
\begin{verbatim}
? K = nfinit(x^2+1);
? K.zk
% = [1, x]
\end{verbatim}
  \end{shaded}
\end{frame}

%%%%%%%%%%%%%%%%%%%%%%%%%%%%%%%%%%%%%%%%%%%%%%%%%%%%%%%%%%%%%%%%%%%%%%%%%%%%%%%%

\begin{frame}[fragile]
  \frametitle{Ideales en $\ZZ [i]$ como $\ZZ$-módulos}

  \begin{align*}
    (m+ni) & = \{(c + di)\,(m + ni) \mid c, d \in \ZZ \} \\
    & = \{ c \cdot (m + ni) + d \cdot (-n + mi) \mid c, d \in \ZZ \} \\
    & = (m + ni) \ZZ \oplus (-n + mi) \ZZ.
  \end{align*}

  \[ (m+ni) \longleftrightarrow \begin{pmatrix}
    m & -n \\
    n & m
  \end{pmatrix} \quad \text{(¡no es HNF!)} \]

  \begin{shaded}\small
\begin{verbatim}
? a = idealhnf(K, 2+3*x)
% = 
[13 5]
[ 0 1]

? mathnf ([2, -3; 3, 2])
% = 
[13 5]
[ 0 1]
\end{verbatim}
  \end{shaded}
\end{frame}

%%%%%%%%%%%%%%%%%%%%%%%%%%%%%%%%%%%%%%%%%%%%%%%%%%%%%%%%%%%%%%%%%%%%%%%%%%%%%%%%

\begin{frame}[fragile]
  \frametitle{Igualdad de ideales}

  \[ \mathfrak{a} = \mathfrak{b} \iff \text{\tt idealhnf}(K,\mathfrak{a}) = \text{\tt idealhnf}(K,\mathfrak{b}) \]

  \begin{shaded}
\begin{verbatim}
? a = idealhnf(K, 2+3*x)
% = 
[13 5]
[ 0 1]

? b = idealhnf(K, -3+2*x)
% =
[13 5]
[ 0 1]

? Mod ((2+3*x)/(-3+2*x), K.pol)
% = Mod(-x, x^2 + 1)
\end{verbatim}
  \end{shaded}
\end{frame}

%%%%%%%%%%%%%%%%%%%%%%%%%%%%%%%%%%%%%%%%%%%%%%%%%%%%%%%%%%%%%%%%%%%%%%%%%%%%%%%%

\begin{frame}[fragile]
  \frametitle{Enumeración de ideales}

  \texttt{ideallist($K$,$N$)} = ideales $I \subseteq \O_K$ tales que $N_{K/\QQ} (I) \le N$


  Salida: vector

  \begin{center}
    [ideales de norma 0,\\
      ideales de norma 1,\\
      $\ldots$,\\
      ideales de norma $N$]
  \end{center}
\end{frame}

%%%%%%%%%%%%%%%%%%%%%%%%%%%%%%%%%%%%%%%%%%%%%%%%%%%%%%%%%%%%%%%%%%%%%%%%%%%%%%%%

\begin{frame}[fragile]
  \frametitle{Ejemplo}

  \begin{shaded}\small
\begin{verbatim}
? K = nfinit(x^2+1);
? L = ideallist(K,10)
% = [[[1, 0; 0, 1]],  /* norma 1: I=O_K */
     [[2, 1; 0, 1]],
     [],              /* no hay de norma 3 */
     [[2, 0; 0, 2]],
     [[5, 3; 0, 1], [5, 2; 0, 1]],
     [],              /* no hay de norma 6 */
     [],              /* no hay de norma 7 */ 
     [[4, 2; 0, 2]],
     [[3, 0; 0, 3]],
     [[10, 3; 0, 1], [10, 7; 0, 1]]]

? vector (#L,i,#L[i])
% = [1, 1, 0, 1, 2, 0, 0, 1, 1, 2]
? vecsum (%)
% = 9
\end{verbatim}
  \end{shaded}
\end{frame}

%%%%%%%%%%%%%%%%%%%%%%%%%%%%%%%%%%%%%%%%%%%%%%%%%%%%%%%%%%%%%%%%%%%%%%%%%%%%%%%%

\begin{frame}[fragile]
  \frametitle{Problema de la tarea}

  ¿Cuántos ideales de norma $\le 10$ hay en $\O_K$ para
  $K = \QQ (\sqrt[3]{17})$?

  \begin{shaded}\small
\begin{verbatim}
? K = nfinit(x^3 - 17);
? L = ideallist (K,10);
? vector (#L,i,#L[i])
% = [1, 1, 2, 2, 1, 2, 0, 2, 3, 1]
? vecsum(%)
% = 15
\end{verbatim}
  \end{shaded}

  Ejemplo: no hay ideales de norma $7$:

  $\mathfrak{p} = 7\O_K$ es primo (= $7$ es inerte).

  $17 \equiv 3~(7)$; los cubos mód $7$ son $1$ y $6 \equiv (-1)^3$.
\end{frame}

%%%%%%%%%%%%%%%%%%%%%%%%%%%%%%%%%%%%%%%%%%%%%%%%%%%%%%%%%%%%%%%%%%%%%%%%%%%%%%%%

\begin{frame}[plain]
  \headingfont

  \begin{center}
    {\huge Operaciones con ideales}
  \end{center}
\end{frame}

%%%%%%%%%%%%%%%%%%%%%%%%%%%%%%%%%%%%%%%%%%%%%%%%%%%%%%%%%%%%%%%%%%%%%%%%%%%%%%%%

\begin{frame}[fragile]
  \frametitle{Operaciones aritméticas}

  \begin{itemize}
  \item \texttt{idealadd($K$,$\mathfrak{a}$,$\mathfrak{b}$)} = $\mathfrak{a}+\mathfrak{b}$
  \item \texttt{idealmul($K$,$\mathfrak{a}$,$\mathfrak{b}$)} = $\mathfrak{a}\mathfrak{b}$
  \item \texttt{idealpow($K$,$\mathfrak{a}$,$n$)} = $\mathfrak{a}^n$
  \item \texttt{idealinv($K$,$\mathfrak{a}$)} = $\mathfrak{a}^{-1}$
  \item \texttt{idealintersect($K$,$\mathfrak{a}$,$\mathfrak{b}$)} = $\mathfrak{a} \cap \mathfrak{b}$
  \end{itemize}
\end{frame}

%%%%%%%%%%%%%%%%%%%%%%%%%%%%%%%%%%%%%%%%%%%%%%%%%%%%%%%%%%%%%%%%%%%%%%%%%%%%%%%%

\begin{frame}[fragile]
  \frametitle{Ideal «abajo»}

  \[ \begin{tikzcd}
    \mathfrak{a} &[-3em] \subseteq &[-3em] K\ar[-]{d} & \O_K\text{-módulo} \\
    \mathfrak{a}\cap \QQ & \subseteq & \QQ & \ZZ\text{-módulo}
    \end{tikzcd} \]

  \begin{center}
    \texttt{idealdown($K$,$\mathfrak{a}$)} = el $\ZZ$-ideal $\mathfrak{a} \cap \QQ$
  \end{center}

  \begin{shaded}
\begin{verbatim}
? K = nfinit(x^2+1);
? idealdown(K,1+x)
% = 2
? idealdown(K,3+3*x)
% = 6
\end{verbatim}
  \end{shaded}
\end{frame}

%%%%%%%%%%%%%%%%%%%%%%%%%%%%%%%%%%%%%%%%%%%%%%%%%%%%%%%%%%%%%%%%%%%%%%%%%%%%%%%%

\begin{frame}[fragile]
  \frametitle{Norma}

  \begin{center}
    \texttt{idealnorm($K$,$\mathfrak{a}$)} = $N_{K/\QQ} (\mathfrak{a})$
  \end{center}

  \begin{shaded}
\begin{verbatim}
? K = nfinit(x^2+1);
? idealnorm(K,1+x)
% = 2
? idealnorm(K,3)
% = 9
? idealnorm(K,3+3*x)
% = 18
\end{verbatim}
  \end{shaded}
\end{frame}

%%%%%%%%%%%%%%%%%%%%%%%%%%%%%%%%%%%%%%%%%%%%%%%%%%%%%%%%%%%%%%%%%%%%%%%%%%%%%%%%

\begin{frame}[fragile]
  \frametitle{Maximalidad}

  \begin{shaded}
\begin{verbatim}
? K = nfinit(x^2+1);
? idealismaximal(K,1+x)
% = [2, [1, 1]~, 2, 1, [1, -1; 1, 1]]
? idealismaximal(K,3+3*x)
% = 0

? L = nfinit(x^3-17);
? idealismaximal(L,7)
% = [7, [7, 0, 0]~, 1, 3, 1]
\end{verbatim}
  \end{shaded}
\end{frame}

%%%%%%%%%%%%%%%%%%%%%%%%%%%%%%%%%%%%%%%%%%%%%%%%%%%%%%%%%%%%%%%%%%%%%%%%%%%%%%%%

\begin{frame}[fragile]
  \frametitle{Generación por dos elementos}

  Recordatorio: todo ideal en $\O_K$ tiene forma $ (\alpha,\beta)$

  \begin{shaded}\small
\begin{verbatim}
? K = nfinit(x^3 - 2);
? a = [3,1,2; 0,1,0; 0,0,1]
% = 
[3 1 2]
[0 1 0]
[0 0 1]

? idealtwoelt(K,a)
% = [3, [1, 1, 0]~]
? nfbasistoalg(K,%[2])
% = Mod(x + 1, x^3 - 2)
\end{verbatim}
  \end{shaded}

  Significado: $\mathfrak{a} = (3, 1 + \sqrt[3]{2})$.
\end{frame}

%%%%%%%%%%%%%%%%%%%%%%%%%%%%%%%%%%%%%%%%%%%%%%%%%%%%%%%%%%%%%%%%%%%%%%%%%%%%%%%%

\begin{frame}[fragile]
  \frametitle{¿Ideales principales?}

  \begin{itemize}
  \item $\mathfrak{a}$ principal $\iff$ $[\mathfrak{a}] = [\O_K] = 0$ en $\Cl (K)$.

  \item $\O_K$ es un DFU $\iff$ $\Cl (K) = 0$.

  \item Veremos después\dots
  \end{itemize}
\end{frame}

%%%%%%%%%%%%%%%%%%%%%%%%%%%%%%%%%%%%%%%%%%%%%%%%%%%%%%%%%%%%%%%%%%%%%%%%%%%%%%%%

\begin{frame}[plain]
  \headingfont

  \begin{center}
    {\huge Factorización de ideales}
  \end{center}
\end{frame}

%%%%%%%%%%%%%%%%%%%%%%%%%%%%%%%%%%%%%%%%%%%%%%%%%%%%%%%%%%%%%%%%%%%%%%%%%%%%%%%%

\begin{frame}[fragile]
  \frametitle{Factorización}

  \begin{itemize}
  \item \texttt{idealfactor($K$,$\mathfrak{a}$)}: factorizar $\mathfrak{a} = \mathfrak{p}_1^{e_1}\cdots\mathfrak{p}_s^{e_s}$

  \item \texttt{idealprimedec($K$,$p$)}: factorizar $\mathfrak{a} = p\O_K$
  \end{itemize}
\end{frame}

%%%%%%%%%%%%%%%%%%%%%%%%%%%%%%%%%%%%%%%%%%%%%%%%%%%%%%%%%%%%%%%%%%%%%%%%%%%%%%%%

\begin{frame}[fragile]
  \frametitle{Factorización de primos racionales}

  $$p\O_K = \mathfrak{p}_1^{e_1} \cdots \mathfrak{p}_s^{e_s},$$

  \begin{center}
    \texttt{idealprimedec($K$,$p$) = [$P_1$,\ldots,$P_s$]},
  \end{center}

  \begin{itemize}
  \item \texttt{$P_i$.e} = índice de ramificación

  \item \texttt{$P_i$.f} = grado del campo residual

  \item \texttt{$P_i$.gen = [$p$, $\alpha$]}, donde $\mathfrak{p}_i = (p,\alpha)$
  \end{itemize}
\end{frame}

%%%%%%%%%%%%%%%%%%%%%%%%%%%%%%%%%%%%%%%%%%%%%%%%%%%%%%%%%%%%%%%%%%%%%%%%%%%%%%%%

\begin{frame}[fragile]
  \frametitle{Ejemplo: $K = \QQ (\sqrt{5})$}

  \begin{shaded}\footnotesize
\begin{verbatim}
? K = nfinit(x^2 - 5);

? decK = idealprimedec(K,11)
% = [[11, [-3, 2]~, 1, 1, [5, 2; 2, 3]],
     [11, [5, 2]~, 1, 1, [-3, 2; 2, -5]]]
? #decK
% 2               /* dos factores */

? [decK[1].e, decK[1].f]
% = [1, 1]
? decK[1].gen
% = [11, [-3, 2]~]
? nfbasistoalg (K,%[2])
% = Mod(x - 4, x^2 - 5)
\end{verbatim}
  \end{shaded}

  \[ 11\O_K = \mathfrak{p}_1 \mathfrak{p}_2, \quad
     \mathfrak{p}_1 = (11, \sqrt{5}-4), ~
     \mathfrak{p}_2 = (11, \sqrt{5}+4), ~
     f_1 = f_2 = 1 \]
\end{frame}

%%%%%%%%%%%%%%%%%%%%%%%%%%%%%%%%%%%%%%%%%%%%%%%%%%%%%%%%%%%%%%%%%%%%%%%%%%%%%%%%

\begin{frame}[fragile]
  \frametitle{Ejemplo: $L = \QQ (\zeta_5) \supset \QQ (\sqrt{5})$}

  \begin{shaded}\footnotesize
\begin{verbatim}
? L = nfinit(polcyclo(5));
? decL = idealprimedec(L,11);
? #decL
# 4               /* 4 factores */
? vector (#decL,i, [decL[i].e, decL[i].f])
% = [[1, 1], [1, 1], [1, 1], [1, 1]]
\end{verbatim}
  \end{shaded}

  \[ 11\O_L = \mathfrak{p}_1\,\mathfrak{p}_2\,\mathfrak{p}_3\,\mathfrak{p}_4 \]
\end{frame}

%%%%%%%%%%%%%%%%%%%%%%%%%%%%%%%%%%%%%%%%%%%%%%%%%%%%%%%%%%%%%%%%%%%%%%%%%%%%%%%%

\begin{frame}[fragile]
  \frametitle{Ramificación}

  \begin{shaded}\footnotesize
\begin{verbatim}
? idealprimedec(K,5)
% = [[5, [1, 2]~, 2, 1, [1, 2; 2, -1]]]
? %[1].e
% = 2
? idealprimedec(L,5)
% = [[5, [-1, 1, 0, 0]~, 4, 1, [.....]]]
? %[1].e
% = 4
\end{verbatim}
  \end{shaded}

  Significado: $5\O_K = \mathfrak{p}^2$, $5\O_L = \mathfrak{q}^4$
\end{frame}

%%%%%%%%%%%%%%%%%%%%%%%%%%%%%%%%%%%%%%%%%%%%%%%%%%%%%%%%%%%%%%%%%%%%%%%%%%%%%%%%

\begin{frame}[fragile]
  \frametitle{Ejemplo de Kummer}

  \begin{shaded}\footnotesize
\begin{verbatim}
? K = nfinit(polcyclo(23));
? dec = idealprimedec(K,47);
? #dec
% = 22
\end{verbatim}
  \end{shaded}

  \[ 47 \ZZ [\zeta_{23}] = \mathfrak{p}_1\cdots \mathfrak{p}_{22} \]
\end{frame}

%%%%%%%%%%%%%%%%%%%%%%%%%%%%%%%%%%%%%%%%%%%%%%%%%%%%%%%%%%%%%%%%%%%%%%%%%%%%%%%%

\begin{frame}[plain]
  \headingfont

  \begin{center}
    {\huge *Pausa para el café*

    }
  \end{center}
\end{frame}

%%%%%%%%%%%%%%%%%%%%%%%%%%%%%%%%%%%%%%%%%%%%%%%%%%%%%%%%%%%%%%%%%%%%%%%%%%%%%%%%

\begin{frame}[plain]
  \headingfont

  \begin{center}
    {\huge Experimento:

      Ideales de norma $\le N$

    }
  \end{center}
\end{frame}

%%%%%%%%%%%%%%%%%%%%%%%%%%%%%%%%%%%%%%%%%%%%%%%%%%%%%%%%%%%%%%%%%%%%%%%%%%%%%%%%

\begin{frame}[fragile]
  \frametitle{Pregunta}

  Cómo se comporta la función
  $$N \mapsto \# \{ I \subseteq \O_K \mid N_{K/\QQ} (I) \le N \}$$
\end{frame}

%%%%%%%%%%%%%%%%%%%%%%%%%%%%%%%%%%%%%%%%%%%%%%%%%%%%%%%%%%%%%%%%%%%%%%%%%%%%%%%%

\begin{frame}[fragile]
  \frametitle{Ejemplo: $K = \QQ (i)$}

  \begin{center}
    \includegraphics[width=10cm]{../pic/ideal-count-Qi.pdf}
  \end{center}
\end{frame}

%%%%%%%%%%%%%%%%%%%%%%%%%%%%%%%%%%%%%%%%%%%%%%%%%%%%%%%%%%%%%%%%%%%%%%%%%%%%%%%%

\begin{frame}[fragile]
  \frametitle{Explicación}

  $$N_{K/\QQ} (\alpha\ZZ[i]) = N_{K/\QQ} ((a+bi)\ZZ[i]) = a^2 + b^2$$

  $$(\alpha) = (-\alpha) = (i\alpha) = (-i\alpha)$$

  \[ \#\{ I \subseteq \ZZ [i] \mid N (I) \le N \} =
  \frac{1}{4}\cdot \# \{ (a,b) \in \ZZ^2 \mid a^2 + b^2 \le N \} \]

  \begin{center}
    \includegraphics[width=5cm]{../pic/points-in-circle.pdf}
  \end{center}
\end{frame}

%%%%%%%%%%%%%%%%%%%%%%%%%%%%%%%%%%%%%%%%%%%%%%%%%%%%%%%%%%%%%%%%%%%%%%%%%%%%%%%%

\begin{frame}[fragile]
  \frametitle{Explicación}

  \begin{center}
    \includegraphics[width=10cm]{../pic/ideal-count-Qi.pdf}

    $\# \{ \text{puntos dentro del círculo de radio }r \} \sim \pi r^2$

    $\# \{ \text{ideales de norma }\le N \} \sim \frac{\pi}{4} N$
  \end{center}
\end{frame}

%%%%%%%%%%%%%%%%%%%%%%%%%%%%%%%%%%%%%%%%%%%%%%%%%%%%%%%%%%%%%%%%%%%%%%%%%%%%%%%%

\begin{frame}[fragile]
  \frametitle{Algunos cálculos}

  \begin{shaded}
\begin{verbatim}
? K = nfinit(x^2+1);
? L = ideallist(K,20);

? vector (#L,s, sum(i=1,s,#L[i]))
% = [1,2,2,3,5,5,5,6,7,9,9, 9,11,11,11,12,14...]

? vector (20,i, ceil(Pi/4*i))
% = [1,2,3,4,4,5,6,7,8,8,9,10,11,11,12,13,14...]
\end{verbatim}
  \end{shaded}

  * Relacionado: \textbf{problema del círculo de Gauss}
\end{frame}

%%%%%%%%%%%%%%%%%%%%%%%%%%%%%%%%%%%%%%%%%%%%%%%%%%%%%%%%%%%%%%%%%%%%%%%%%%%%%%%%

\begin{frame}[fragile]
  \frametitle{Ejemplo: $K = \QQ (\zeta_5)$, $\Cl (K) = 0$, $|\O_K^\times| = \infty$}

  \begin{center}
    \includegraphics[width=10cm]{../pic/ideal-count-zeta-5.pdf}

    $\sim C\cdot N$, donde $C$ = ?????
  \end{center}
\end{frame}

%%%%%%%%%%%%%%%%%%%%%%%%%%%%%%%%%%%%%%%%%%%%%%%%%%%%%%%%%%%%%%%%%%%%%%%%%%%%%%%%

\begin{frame}[fragile]
  \frametitle{$K = \QQ (\sqrt{10})$, $\Cl (K) \ne 0$, $|\O_K^\times| = \infty$}

  \begin{center}
    \includegraphics[width=10cm]{../pic/ideal-count-x2-10.pdf}

    $\sim C\cdot N$, donde $C$ = ?????
  \end{center}
\end{frame}

%%%%%%%%%%%%%%%%%%%%%%%%%%%%%%%%%%%%%%%%%%%%%%%%%%%%%%%%%%%%%%%%%%%%%%%%%%%%%%%%

\begin{frame}[fragile]
  \frametitle{$K = \QQ (\sqrt[3]{19})$, $\Cl (K)\ne 0$, $|\O_K^\times| = \infty$}

  \begin{center}
    \includegraphics[width=10cm]{../pic/ideal-count-x3-19.pdf}

    $\sim C\cdot N$, donde $C$ = ?????
  \end{center}
\end{frame}

%%%%%%%%%%%%%%%%%%%%%%%%%%%%%%%%%%%%%%%%%%%%%%%%%%%%%%%%%%%%%%%%%%%%%%%%%%%%%%%%

\begin{frame}[fragile]
  \frametitle{Explicación breve:\\
    función zeta de Dedekind}

  \begin{itemize}
  \item Función meromorfa
    $$\zeta_K (s) = \sum_{\substack{ I \subseteq \O_K \\ I \ne 0 }} \frac{1}{N_{K/\QQ} (I)^s}$$
    responsable por contar los ideales.

  \item $C$ = residuo en el polo $s = 1$

  \item $C$ trae información aritmética

    (fórmula del número de clase)
  \end{itemize}
\end{frame}

%%%%%%%%%%%%%%%%%%%%%%%%%%%%%%%%%%%%%%%%%%%%%%%%%%%%%%%%%%%%%%%%%%%%%%%%%%%%%%%%

\begin{frame}[plain]
  \headingfont

  \begin{center}
    {\huge Experimento:

      estadística sobre descomposiciones

    }
  \end{center}
\end{frame}

%%%%%%%%%%%%%%%%%%%%%%%%%%%%%%%%%%%%%%%%%%%%%%%%%%%%%%%%%%%%%%%%%%%%%%%%%%%%%%%%

\begin{frame}[fragile]
  \frametitle{Pregunta}

  \begin{itemize}
  \item Si $p$ no se ramifica en $K/\QQ$:
    $$p\O_K = \mathfrak{p}_1\cdots\mathfrak{p}_s.$$

  \item Número finito de ramificaciones ($p \mid \Delta_K$).

  \item $f_i = [\O_K/\mathfrak{p}_i : \FF_p]$,
    $$f_1 + \cdots + f_s = [K : \QQ].$$

  \item ¿Con qué frecuencia surgen diferentes particiones de $[K : \QQ]$?
  \end{itemize}
\end{frame}

%%%%%%%%%%%%%%%%%%%%%%%%%%%%%%%%%%%%%%%%%%%%%%%%%%%%%%%%%%%%%%%%%%%%%%%%%%%%%%%%

\begin{frame}[fragile]
  \frametitle{Ejemplo: $K = \QQ (\sqrt[3]{2}, \zeta_3)$}

  \begin{itemize}
  \item $p = 2, 3$ se ramifican

    (ya se ramifican en $\QQ (\sqrt[3]{2})$ y $\QQ (\zeta_3)$)

  \item Algunos ejemplos
    \begin{center}\small
      \renewcommand{\arraystretch}{1.5}
      \begin{tabular}{cc|cc|cc}
        $p$ & \text{partición} & $p$ & \text{partición} & $p$ & \text{partición} \\
        \hline
        $5$ & $2 + 2 + 2$ & $41$ & $2 + 2 + 2$ & $83$ & $2 + 2 + 2$ \\
        $7$ & $3 + 3$ & $43$ & $1 + \cdots + 1$ & $89$ & $2 + 2 + 2$ \\
        $11$ & $2 + 2 + 2$ & $47$ & $2 + 2 + 2$ & $97$ & $3 + 3$ \\
        $13$ & $3 + 3$ & $53$ & $2 + 2 + 2$ & $101$ & $2 + 2 + 2$ \\
        $17$ & $2 + 2 + 2$ & $59$ & $2 + 2 + 2$ & $103$ & $3 + 3$ \\
        $19$ & $3 + 3$ & $61$ & $3 + 3$ & $107$ & $2 + 2 + 2$ \\
        $23$ & $2 + 2 + 2$ & $67$ & $3 + 3$ & $109$ & $1 + \cdots + 1$ \\
        $29$ & $2 + 2 + 2$ & $71$ & $2 + 2 + 2$ & $113$ & $2 + 2 + 2$ \\
        $31$ & $1 + \cdots + 1$ & $73$ & $3 + 3$ & $127$ & $1 + \cdots + 1$ \\
        $37$ & $3 + 3$ & $79$ & $3 + 3$ & $131$ & $2 + 2 + 2$ \\
      \end{tabular}
    \end{center}
  \end{itemize}
\end{frame}

%%%%%%%%%%%%%%%%%%%%%%%%%%%%%%%%%%%%%%%%%%%%%%%%%%%%%%%%%%%%%%%%%%%%%%%%%%%%%%%%

\begin{frame}[fragile]
  \frametitle{Ejemplo: $K = \QQ (\sqrt[3]{2}, \zeta_3)$}

  \begin{itemize}
  \item Surge solo $1 + \cdots + 1$, $2 + 2 + 2$, $3 + 3$.

  \item ¿Estadística para los primeros $N$ primos?
    \begin{center}
      \renewcommand{\arraystretch}{1.5}
      \begin{tabular}{lx{1.5cm}x{1.5cm}x{1.5cm}}
        $N$      & $1+\cdots+1$ & $2+2+2$ & $3+3$ \tabularnewline
        \hline
        $10$     & $0.1000$ & $0.5000$ & $0.4000$ \tabularnewline
        $100$    & $0.1500$ & $0.5200$ & $0.3300$ \tabularnewline
        $1000$   & $0.1570$ & $0.5080$ & $0.3350$ \tabularnewline
        $10000$  & $0.1635$ & $0.5011$ & $0.3354$ \tabularnewline
        $100000$ & $0.1659$ & $0.5004$ & $0.3337$ \tabularnewline
      \end{tabular}
    \end{center}

  \item Converge a $\frac{1}{6}$, $\frac{1}{2}$, $\frac{1}{3}$.
  \end{itemize}
\end{frame}

%%%%%%%%%%%%%%%%%%%%%%%%%%%%%%%%%%%%%%%%%%%%%%%%%%%%%%%%%%%%%%%%%%%%%%%%%%%%%%%%

\begin{frame}[fragile]
  \frametitle{Ejemplo que podemos entender:\\
    caso ciclotómico}

  \begin{itemize}
  \item Consideremos $K = \QQ (\zeta_7)$

  \item Factorización depende solo de $p \pmod{7}$:
    \begin{center}
      \renewcommand{\arraystretch}{1.5}
      \begin{tabular}{lx{3cm}c}
        $p~(7)$ & \text{factorización} & \text{partición} \tabularnewline
        \hline
        $1$ & $\mathfrak{p}_1\cdots\mathfrak{p}_6$ & $1+\cdots+1$ \tabularnewline
        $6$ & $\mathfrak{p}_1\,\mathfrak{p}_2\,\mathfrak{p}_3$ & $2+2+2$ \tabularnewline
        $2,4$ & $\mathfrak{p}_1\,\mathfrak{p}_2$ & $3+3$ \tabularnewline
        $3,5$ & $\mathfrak{p}$ & $6$ \tabularnewline
      \end{tabular}
    \end{center}

  \item Dirichlet: $\frac{1}{6}$ primos cumplen $p \equiv a \pmod{7}$
    para $a = 1,2,3,4,5,6$ fijo.
  \end{itemize}
\end{frame}

%%%%%%%%%%%%%%%%%%%%%%%%%%%%%%%%%%%%%%%%%%%%%%%%%%%%%%%%%%%%%%%%%%%%%%%%%%%%%%%%

\begin{frame}[fragile]
  \frametitle{Estadística}

  \begin{center}
    \renewcommand{\arraystretch}{1.5}
    \begin{tabular}{lx{1.5cm}x{1.5cm}x{1.5cm}x{1.5cm}}
      $N$      & $1+\cdots+1$ & $2+2+2$ & $3+3$ & $6$ \tabularnewline
      \hline
      $10$     & $0.2000$ & $0.1000$ & $0.3000$ & $0.4000$ \tabularnewline
      $100$    & $0.1700$ & $0.1600$ & $0.3200$ & $0.3500$ \tabularnewline
      $1000$   & $0.1660$ & $0.1660$ & $0.3300$ & $0.3380$ \tabularnewline
      $10000$  & $0.1662$ & $0.1663$ & $0.3324$ & $0.3351$ \tabularnewline
      $100000$ & $0.1668$ & $0.1669$ & $0.3328$ & $0.3336$ \tabularnewline
    \end{tabular}
  \end{center}

  Converge a $\frac{1}{6}$, $\frac{1}{6}$, $\frac{1}{3}$, $\frac{1}{3}$.
\end{frame}

%%%%%%%%%%%%%%%%%%%%%%%%%%%%%%%%%%%%%%%%%%%%%%%%%%%%%%%%%%%%%%%%%%%%%%%%%%%%%%%%

\begin{frame}[fragile]
  \frametitle{Explicación muy breve}

  \textbf{«Teorema de densidad de Chebotarëv»}
\end{frame}

%%%%%%%%%%%%%%%%%%%%%%%%%%%%%%%%%%%%%%%%%%%%%%%%%%%%%%%%%%%%%%%%%%%%%%%%%%%%%%%%

\begin{frame}[plain]
  \headingfont

  \begin{center}
    {\large Videos:

      \vspace{2em}

      \href{https://cadadr.org/cimat-tna/chebotarev.html}{cadadr.org/cimat-tna/chebotarev.html}

    }
  \end{center}
\end{frame}

%%%%%%%%%%%%%%%%%%%%%%%%%%%%%%%%%%%%%%%%%%%%%%%%%%%%%%%%%%%%%%%%%%%%%%%%%%%%%%%%

\begin{frame}[plain]
  \headingfont

  \begin{center}
    {\huge ¡Gracias por su atención!}
  \end{center}
\end{frame}

\end{document}

\chapter{Teoría de Galois}
\label{cap:teoria-de-galois}

Ya hemos usado ciertos argumentos de la teoría de Galois, y en este capítulo
veremos de manera más sistemática algunas propiedades de los campos de números
$K/\QQ$ que son extensiones de Galois.

%%%%%%%%%%%%%%%%%%%%%%%%%%%%%%%%%%%%%%%%%%%%%%%%%%%%%%%%%%%%%%%%%%%%%%%%%%%%%%%%

\pdfbookmark{Clase 15 (05/10/20)}{clase-15}
\section{Breve recordatorio sobre la teoría de Galois}
\marginpar{\small Clase 15 \\ 05/10/20}

En esta sección vamos a revisar rápidamente la teoría de Galois. El apéndice
\ref{ap:teoria-de-Galois} contiene la mayoría de los resultados necesarios.

\vspace{1em}

La teoría de Galois considera las extensiones de Galois que son extensiones
separables y normales. En la característica nula cualquier extensión es
separable, así que la condición que nos interesa para los campos de números es
la normalidad.\footnote{También en este curso nos interesan extensiones finitas
  de campos finitos, pero estas son siempre extensiones de Galois.}

Dado un campo de números $K/\QQ$, gracias al teorema del elemento primitivo,
podemos escribirlo como $K = \QQ (\alpha)$ para algún número algebraico
$\alpha$. Sea $f = f^\alpha_\QQ$ el polinomio mínimo de $\alpha$. Consideremos
sus raíces complejas
$$f = (x - \alpha_1) \cdots (x - \alpha_n).$$
Por la separabilidad, se tiene $\alpha_i \ne \alpha_j$ para $i \ne j$.
El campo de descomposición de $f$ viene dado por
$L = \QQ (\alpha_1, \ldots, \alpha_n)$, y esta es una extensión normal.
El grupo
$$G = \Gal (L/\QQ) = \Aut (L/\QQ)$$
se llama el \textbf{grupo de Galois}. Se tiene $|G| = [L : \QQ]$.
Hay una acción fiel y transitiva sobre las raíces
$$G \curvearrowright \{ \alpha_1, \ldots, \alpha_n \}.$$
Fijando una numeración de las raíces (como ya hicimos implícitamente),
se obtiene un homomorfismo inyectivo $G \hookrightarrow S_n$. Si $K = L$,
entonces $K/\QQ$ es una extensión de Galois. En el caso contrario, $L$ es la
\textbf{cerradura de Galois} de $K$.

\begin{ejemplo}
  Ya hemos usado en varias ocasiones que las extensiones ciclotómicas
  $\QQ (\zeta_n)/\QQ$ son de Galois: el polinomio mínimo de $\zeta_n$
  es el polinomio ciclotómico $\Phi_n$ y sus raíces son las raíces
  $n$-ésimas primitivas que están en $\QQ (\zeta_n)$.

  Todo automorfismo $\sigma\colon \QQ (\zeta_n) \to \QQ (\zeta_n)$ debe mandar
  $\zeta_n$ a otra raíz $n$-ésima primitiva, así que los automorfismos son
  $$\sigma_a\colon \zeta_n \mapsto \zeta_n^a, \quad \gcd (a,n) = 1.$$
  Tenemos un isomorfismo
  \[ \Gal (\QQ (\zeta_n)/\QQ) \cong (\ZZ/n\ZZ)^\times,
     \quad \sigma_a \mapsto \overline{a}. \]
  Todo el trabajo duro consiste en probar que $\Phi_n$ es el polinomio mínimo
  de $\zeta_n$; es decir, probar la irreducibilidad de $\Phi_n$. Véase el
  apéndice \ref{ap:polinomios-ciclotomicos}.
\end{ejemplo}

\begin{ejemplo}
  La extensión $\QQ (\sqrt[3]{2})/\QQ$ no es de Galois. Tenemos
  \[ f = x^3 - 2 =
     (x - \sqrt[3]{2})\,(x - \zeta_3\sqrt[3]{2})\,(x - \zeta_3^2\sqrt[3]{2}). \]
  El campo de descomposición de $f$ es $\QQ (\sqrt[3]{2}, \zeta_3)$,
  y su grupo de Galois es el grupo simétrico $S_3$. Específicamente, hay dos
  automorfismos
  \begin{alignat*}{2}
    \sigma\colon \sqrt[3]{2} & \mapsto \zeta_3\sqrt[3]{2}, \quad & \zeta_3 & \mapsto \zeta_3,\\
    \tau\colon \sqrt[3]{2} & \mapsto \sqrt[3]{2}, & \zeta_3 & \mapsto \zeta_3^2.
  \end{alignat*}
  Aquí el orden de $\sigma$ es $3$ y el orden de $\tau$ es $2$. Tenemos
  $\sigma\tau = \tau\sigma^2 \ne \tau\sigma$. Estos dos elementos generan
  el grupo de Galois que es isomorfo al grupo simétrico $S_3$.
\end{ejemplo}

El problema con el campo de números $K = \QQ [\alpha]/(\alpha^3 - 2)$
es el siguiente: este tiene tres diferentes encajes $K \hookrightarrow \CC$:
un encaje real con imagen $\QQ (\sqrt[3]{2})$ y dos encajes complejos con imagen
$\QQ (\zeta_3\sqrt[3]{2})$ y $\QQ (\zeta_3^2\sqrt[3]{2})$. Esto no puede pasar
con extensiones de Galois.

\begin{proposicion}
  Sea $K/\QQ$ una extensión de Galois. Entonces, todo encaje
  $\sigma\colon K \hookrightarrow \CC$ tiene la misma imagen. Como consecuencia,
  todos los encajes son reales ($r_1 = [K : \QQ]$) o todos los encajes son
  complejos ($r_2 = \frac{1}{2} [K : \QQ]$).

  \begin{proof}
    En general, una extensión finita $K/F$ es normal si y solamente si para todo
    $F$-homomorfismo $\sigma\colon K \to \overline{K}$ se cumple
    $\sigma (K) = K$ (véase \ref{prop-dfn:extensiones-normales}).
  \end{proof}
\end{proposicion}

Este no es un curso de la teoría de Galois, pero nuestra discusión sería
incompleta sin el siguiente resultado.

\begin{teorema}[Correspondencia de Galois]
  Dada una extensión finita de Galois $K/\QQ$, consideremos el grupo de Galois
  $G = \Gal (K/\QQ)$.  A una subextensión $\QQ \subseteq F \subseteq K$ se puede
  asociar un subgrupo $H = \Gal (K/F) \subseteq G$. Viceversa, dado un subgrupo
  $H \subseteq G$, se obtiene una subextensión
  $$F = K^H = \{ \alpha \in K \mid \sigma (\alpha) = \alpha \text{ para }\sigma\in H \}.$$
  Esto nos da una biyección
  \[ \begin{tikzcd}[column sep=4em]
    \{ \text{ subcampos }F \subseteq K \}
    \ar[shift left=0.25em]{r}{F \mapsto \Gal (K/F)} &
    \{ \text{ subgrupos }H \subseteq G \}
    \ar[shift left=0.25em]{l}{K^H \mapsfrom H}
  \end{tikzcd} \]

  Esta correspondencia satisface las siguientes propiedades.
  \begin{itemize}
  \item La correspondencia invierte las inclusiones.
    Si $F \subseteq F'$, entonces $\Gal (K/F') \subseteq \Gal (K/F)$.
    Si $H \subseteq H' \subseteq G$, entonces $K^{H'} \subseteq K^H$.

  \item $[K:F] = |H|$ y $[F:\QQ] = [G:H]$.

  \item La extensión $F/\QQ$ es normal (y entonces Galois) si y solamente si el
    subgrupo $H \subseteq G$ es normal. En este caso la restricción de
    automorfismos $\Gal (K/\QQ) \to \Gal (F/\QQ)$ es sobreyectiva y tiene $H$
    como su núcleo, así que $\Gal (F/\QQ) \cong G/H$.

  \item Para dos subextensiones $F$ y $F'$ se tiene $F\cong F'$ si y solamente
    si los subgrupos correspondientes $H, H' \subseteq G$ son conjugados por un
    elemento de $G$.
  \end{itemize}

  \begin{proof}
    Véase \ref{thm:correspondencia-de-Galois}.
  \end{proof}
\end{teorema}

\begin{ejemplo}
  En $K = \QQ (\sqrt[3]{2}, \zeta_3)$ hay un subcampo cuadrático $\QQ (\zeta_3)$
  y tres subcampos cúbicos $\QQ(\sqrt[3]{2})$, $\QQ(\zeta_3\sqrt[3]{2})$,
  $\QQ(\zeta_3^2\sqrt[3]{2})$ isomorfos entre sí. La correspondencia con los
  subgrupos de $\Gal (K/\QQ) = \langle\sigma,\tau\rangle \cong S_3$ es la
  siguiente:
  \[ \begin{tikzcd}[row sep=1em,column sep=1em]
    & \QQ (\sqrt[3]{2}, \zeta_3)\ar[-]{dd}[description]{2}\ar[-]{ddr}[description]{2}\ar[-]{ddrr}[description]{2}\ar[-]{dddl}[description]{3} & & & & & 1\ar[-]{dd}\ar[-]{ddr}\ar[-]{ddrr}\ar[-]{dddl} \\
    \\
    & \QQ (\sqrt[3]{2})\ar[-]{dd}[description]{3} & \QQ (\zeta_3\sqrt[3]{2})\ar[-]{ddl}[description]{3} & \QQ (\zeta_3^2\sqrt[3]{2})\ar[-]{ddll}[description]{3} & & & \langle\tau\rangle\ar[-]{dd} & \,\sigma\,\langle\tau\rangle\,\sigma^{-1}\ar[-]{ddl} & \sigma^2\,\langle\tau\rangle\,\sigma^{-2}\ar[-]{ddll} \\
    \QQ (\zeta_3)\ar[-]{dr}[description]{2} & & & & & \langle\sigma\rangle\ar[-]{dr} \\
    & \QQ & & & & & \langle\sigma,\tau\rangle
  \end{tikzcd} \]
\end{ejemplo}

Uno de los problemas abiertos más importantes de la aritmética es el
\textbf{problema inverso de Galois} que pregunta si todo grupo finito es
isomorfo a $\Gal (K/\QQ)$ para alguna extensión de Galois $K/\QQ$.

\begin{ejemplo}
  Según un teorema de Selmer \cite{Selmer-1956}, el polinomio
  $x^n - x - 1 \in \QQ [x]$ es irreducible para todo $n$. Su campo de
  descomposición tiene $S_n$ como su grupo de Galois; véase \cite{Osada-1987}
  o la exposición \cite{KCd-Selmer}.
\end{ejemplo}

Para los grupos abelianos, el problema se resuelve fácilmente de la siguiente
manera.

\begin{proposicion}
  Cualquier grupo abeliano finito puede ser realizado como un grupo de Galois.

  \begin{proof}
    Primero notamos que para todo primo $p$ la extensión ciclotómica
    $\QQ (\zeta_p)/\QQ$ es una extensión de Galois, con el grupo de Galois
    cíclico
    $$\Gal (\QQ (\zeta_p)/\QQ) \xrightarrow{\cong} (\ZZ/p\ZZ)^\times$$
    A saber, los automorfismos son
    $$\sigma\colon \zeta_p \mapsto \zeta_p^a, ~ \gcd (a,p) = 1.$$

    Todo grupo abeliano finito se expresa como producto de grupos cíclicos
    $$C_{n_1} \times C_{n_2} \times \cdots \times C_{n_s}.$$
    Ocupando el teorema de Dirichlet sobre primos en progresiones aritméticas
    (véase el apéndice \ref{ap:Dirichlet}), podemos encontrar diferentes primos
    $p_1,\ldots,p_s$ tales que $p_i \equiv 1 \pmod{n_i}$. De hecho, el teorema
    afirma que para cada $n_i$ existe un número infinito de primos $p_i$ con
    esta propiedad.

    Ahora consideremos el campo ciclotómico
    $$K = \QQ (\zeta_{p_1\cdots p_s}).$$
    Su grupo de Galois es un producto de grupos cíclicos
    \[ G \cong (\ZZ/p_1\cdots p_s\ZZ)^\times
       \cong (\ZZ/p_1\ZZ)^\times \times \cdots \times (\ZZ/p_s\ZZ)^\times. \]
    Por nuestra elección de $p_i$, existe subgrupo
    $H_i \subset (\ZZ/p_i\ZZ)^\times$ de índice $n_i$, y luego
    \[ G/(H_1\times \cdots \times H_s) \cong C_{n_1} \times \cdots C_{n_s}. \qedhere \]
  \end{proof}
\end{proposicion}

\begin{ejemplo}
  Para hacerlo más específico, si buscamos una extensión con el grupo de Galois
  $C_3$, podemos tomar $p = 7$. Nos interesa entonces la extensión ciclotómica
  $\QQ (\zeta_7)/\QQ$ y el grupo de Galois
  $$G = \Gal (\QQ (\zeta_7)/\QQ) \cong (\ZZ/7\ZZ)^\times.$$
  La conjugación compleja $\zeta_7 \to \zeta_7^{-1}$ tiene orden $2$.
  El subcampo cúbico real fijo por la conjugación compleja es
  $\QQ (\zeta_7 + \zeta_7^{-1})$.
  También hay automorfismo de orden $3$ dado por $\zeta_7 \mapsto \zeta_7^2$.
  Este fija el subcampo cuadrático $\QQ (\sqrt{-7})$, donde
  $$\sqrt{-7} = 1 + 2\zeta_7 + 2\zeta_7^2 + 2\zeta_7^4.$$

  Hemos descrito todas las posibles subextensiones:
  \[ \begin{tikzcd}[row sep=1em,column sep=1em]
    & \QQ (\zeta_7)\ar[-]{dl}[description]{2}\ar[-]{ddr}[description]{3} & & & 1\ar[-]{dl}\ar[-]{ddr} \\
    \QQ (\zeta_7 + \zeta_7^{-1})\ar[-]{ddr}[description]{3} & & & \{ 1, 6 \}\ar[-]{ddr} \\
    & & \QQ (\sqrt{-7}) \ar[-]{dl}[description]{2} & & & \{ 1,2,4 \}\ar[-]{dl} \\
     & \QQ & & & (\ZZ/7\ZZ)^\times
    \end{tikzcd} \]
\end{ejemplo}

\begin{ejemplo}
  Si queremos encontrar una extensión con el grupo abeliano $C_2\times C_2$
  usando este método, podemos tomar el campo ciclotómico $\QQ (\zeta_{15})$
  con el grupo de Galois
  $$G \cong (\ZZ/15\ZZ)^\times \cong (\ZZ/3\ZZ)^\times \times (\ZZ/5\ZZ)^\times,$$
  y tomar adentro el subgrupo
  $$H = \{ 1, 4 \} \subset (\ZZ/15\ZZ)^\times.$$
  Ahora el subcampo $\QQ (\zeta_{15})^H$ es el campo bicuadrático
  $\QQ (\sqrt{-3}, \sqrt{5})$.
  \[ \begin{tikzcd}
    & \QQ (\zeta_{15})\ar[-]{d}[description]{2}\ar[-]{ddl}[description]{4}\ar[-]{dr}[description]{2}\ar[-]{drr}[description]{2} \\
    & \QQ (\sqrt{-3},\sqrt{5})\ar[-]{dd}[description]{2}\ar[-]{dl}[description]{2}\ar[-]{ddr}[description]{2} & \QQ (\zeta_5) \ar[-]{dd}[description]{2} & \QQ (\zeta_{15} + \zeta_{15}^{-1}) \ar[-]{ddl}[description]{2} \\
    \QQ (\zeta_3) \ar[equals]{d} \\
    \QQ (\sqrt{-3})\ar[-]{ddr}[description]{2} & \QQ (\sqrt{-15})\ar[-]{dd}[description]{2} & \QQ (\sqrt{5})\ar[-]{ddl}[description]{2} \\
    \\
    & \QQ
  \end{tikzcd} \]
  El diagrama de arriba contiene todos los subcampos de $\QQ (\zeta_{15})$. Los
  subgrupos correspondientes $H \subseteq G$ son los siguientes:
  \[ \begin{tikzcd}
    & 1\ar[-]{d}\ar[-]{dr}\ar[-]{ddl}\ar[-]{drr} \\
    & \{ 1, 4 \} \ar[-]{d}\ar[-]{dl}\ar[-]{dr} & \{ 1, -4 \} \ar[-]{d} & \{ \pm 1 \}\ar[-]{dl} \\
    \{ 1, 4, 7, 13 \}\ar[-]{ddr} & \{ 1, 2, 4, 8 \}\ar[-]{dd} & \{ \pm 1, \pm 4 \}\ar[-]{ddl} \\
    \\
    & (\ZZ/15\ZZ)^\times
  \end{tikzcd} \]
\end{ejemplo}

Entonces, cualquier grupo de Galois abeliano se realiza mediante una
subextensión de algún campo ciclotómico. Esta no es una coincidencia:
se cumple el siguiente resultado mucho más fuerte.

\begin{teorema}[Kronecker--Weber]
  Sea $K/\QQ$ une extensión con el grupo $\Gal (K/\QQ)$ abeliano. Entonces, para
  algún $n$ se tiene $K \subseteq \QQ (\zeta_n)$.

  \begin{proof}
    La prueba requiere bastante trabajo y nos llevaría lejos de los objetivos
    de este curso\dots{} El lector interesado puede consultar
    \cite[Chapter~14]{Washington-GTM83}.
  \end{proof}
\end{teorema}

\begin{ejemplo}
  Para un campo cuadrático $K = \QQ (\sqrt{d})$ es fácil encontrar $n$ tal que
  $K \subset \QQ (\zeta_n)$: use que para un primo impar $p$ se tiene
  $\sqrt{p^*} \in \QQ (\zeta_p)$, donde
  $p^* = (-1)^{\frac{p-1}{2}}\,p$ (véase ejercicio
  \ref{ejerc:subcampo-cuadratico-en-ciclotomico}), y que
  $\QQ (\sqrt{-1}) = \QQ (\zeta_4)$ y
  $\sqrt{\pm 2} \in \QQ (\zeta_8)$. Para el caso general, basta
  factorizar $d$ en números primos. Dejo los detalles como un ejercicio.
\end{ejemplo}

%%%%%%%%%%%%%%%%%%%%%%%%%%%%%%%%%%%%%%%%%%%%%%%%%%%%%%%%%%%%%%%%%%%%%%%%%%%%%%%%

\section{Acción del grupo de Galois sobre los ideales}

A partir de ahora supongamos que $K/\QQ$ es una extensión finita de Galois y
denotemos $G = \Gal (K/\QQ)$. Primero notamos que la acción de $G$ sobre $K$
induce acción de $G$ sobre $\O_K$ y los ideales en $\O_K$.

\begin{proposicion}
  \label{prop:accion-sobre-los-ideales}
  Consideremos un elemento $\sigma \in \Gal (K/\QQ)$.

  \begin{enumerate}
  \item[1)] Si $\alpha \in \O_K$, entonces $\sigma (\alpha) \in \O_K$.

  \item[2)] Dado un ideal $I \subseteq \O_K$, el conjunto
    $\sigma (I) = \{ \sigma (\alpha) \mid \alpha \in I \}$
    es también un ideal en $\O_K$. En términos de generadores, si
    $I = (\alpha_1, \ldots, \alpha_n)$, entonces
    $\sigma (I) = (\sigma(\alpha_1), \ldots, \sigma(\alpha_n))$.

  \item[3)] Hay isomorfismo natural
    $\O_K/I \cong \O_K/\sigma(I)$.

  \item[4)] Si $\mathfrak{p} \subset \O_K$ es un ideal primo, entonces
    el ideal $\sigma (\mathfrak{p}) \subset \O_K$ es también primo.
    Además, si $\mathfrak{p} \mid p$ para un primo racional $p \in \ZZ$,
    entonces $\sigma(\mathfrak{p}) \mid p$, y los grados de campos residuales
    coinciden.
  \end{enumerate}

  \begin{proof}
    En la parte 1), si $\alpha$ es una raíz de un polinomio mónico
    $f \in \ZZ [x]$, entonces $f (\sigma (\alpha)) = \sigma (f (\alpha)) = 0$,
    así que $\sigma (\alpha) \in \O_K$. La parte 2) se verifica fácilmente
    usando el hecho de que $\sigma$ preserva sumas y productos.

    Para la parte 3), basta notar que el homomorfismo
    \[ \O_K \twoheadrightarrow \O_K/\sigma(I), \quad
       \alpha \mapsto \sigma (\alpha) + \sigma (I) \]
    tiene $I$ como su núcleo y entonces induce el isomorfismo deseado.

    En particular, $\O_K/\mathfrak{p}$ es un dominio si y solamente si
    $\O_K/\sigma(\mathfrak{p})$ es un dominio, y esto demuestra que
    para $\mathfrak{p}$ primo el ideal $\sigma (\mathfrak{p})$ es también
    primo. Ahora si $p \in \mathfrak{p}$, entonces
    $p = \sigma (p) \in \sigma (\mathfrak{p})$. En fin, el isomorfismo
    $\O_K/\mathfrak{p} \cong \O_K/\sigma(\mathfrak{p})$ nos dice que
    los grados del campo residual son iguales. Esto establece la parte 4).
  \end{proof}
\end{proposicion}

Entonces, si $p\O_K = \mathfrak{p}_1^{e_1}\cdots\mathfrak{p}_s^{e_s}$,
el grupo $G$ actúa de alguna manera sobre el conjunto
$\{ \mathfrak{p}_1, \ldots, \mathfrak{p}_s \}$. Esta acción es el objeto
principal de estudio del presente capítulo.

\begin{ejemplo}
  Si $K = \QQ (\sqrt{d})$ es una extensión cuadrática, entonces para
  $\legendre{d}{p} = +1$ se obtiene
  $$p\O_K = \mathfrak{p} \, \sigma (\mathfrak{p}),$$
  donde $\sigma\colon \sqrt{d} \mapsto -\sqrt{d}$ es el automorfismo no trivial
  de $K/\QQ$.
\end{ejemplo}

Consideremos alguna extensión un poco más interesante que cuadrática.

\begin{ejemplo}
  Consideremos algún campo ciclotómico, por ejemplo $K = \QQ (\zeta_5)$.
  Tenemos $\Gal (K/\QQ) \cong (\ZZ/5\ZZ)^\times$, donde como generador se puede
  tomar $\sigma\colon \zeta_5 \mapsto \zeta_5^2$. La descomposición de un primo
  racional $p$ depende de su resto módulo $5$.

  \begin{itemize}
  \item Si $p \equiv 1 \pmod{5}$, entonces
    $p\O_K = \mathfrak{p}_1\,\mathfrak{p}_2\,\mathfrak{p}_3\,\mathfrak{p}_4$,
    donde según Krull--Dedekind, $\mathfrak{p}_i = (p, \zeta_5 - a^i)$, y
    $a$ es una quinta raíz primitiva de la unidad mód $p$.
\iffalse
    Tenemos,
    por ejemplo,
    $$\sigma \mathfrak{p}_1 = (p, \zeta_5^2 - a) = \mathfrak{p}_3 = (p, \zeta_5 - a^3).$$
    De hecho,
    $$(\zeta_5 + a^3)\,(\zeta_5 - a^3) \equiv \zeta_5^2 - a \pmod{p},$$
    así que $\zeta_5^2 - a \in \mathfrak{p}_3$. Entonces,
    $\sigma\mathfrak{p}_1 \subseteq \mathfrak{p}_3$, y luego
    $\sigma\mathfrak{p}_1 = \mathfrak{p}_3$ por la maximalidad.
    De manera similar podemos ver qué sucede con otros ideales, y concluir que
\fi
    Se puede calcular que la acción de $\sigma$ sobre los ideales primos viene
    dada por
    \[ \begin{tikzcd}
      \mathfrak{p}_1\ar[bend left=35]{rr} & \mathfrak{p}_2\ar[bend left=35]{l} & \mathfrak{p}_3\ar[bend left=35]{r} & \mathfrak{p}_4\ar[bend left=35]{ll}
    \end{tikzcd} \]
    Nos conviene entonces escribir la factorización como
    $p\O_K = \mathfrak{p}\,\sigma(\mathfrak{p})\,\sigma^2(\mathfrak{p})\,\sigma^3(\mathfrak{p})$.

  \item Si $p \equiv 4 \pmod{5}$, entonces la factorización tiene forma
    $p\O_K = \mathfrak{p}_1\,\mathfrak{p}_2$. En este caso se puede calcular
    que $\sigma (\mathfrak{p}_1) = \mathfrak{p}_2$ y viceversa,
    $\sigma (\mathfrak{p}_2) = \mathfrak{p}_1$. Entonces, la factorización
    toma forma $p\O_K = \mathfrak{p}\,\sigma(\mathfrak{p})$.

\iffalse
    Esto se debe al hecho de que
    en $\FF_p [x]$
    $$\Phi_5 (x) = (x^2 - (a + a^4)\,x + 1)\,(x^2 - (a^2 + a^3)\,x + 1),$$
    donde $a$ es una raíz quinta primitiva en $\FF_{p^2}$. En este caso podemos
    ver que
    \[ (\zeta_5^2 + (a^2 + a^3)\,\zeta_5 + 1) \,
       (\zeta_5^2 - (a^2 + a^3)\,\zeta_5 + 1) \equiv
       \zeta_5^4 - (a + a^4)\,\zeta_5^2 + 1 \pmod{p}, \]
    y esto de manera similar al caso anterior implica que
    $\sigma (\mathfrak{p}_1) = \mathfrak{p}_2$. Esto implica que
    $\sigma (\mathfrak{p}_2) = \mathfrak{p}_1$.
\fi

  \item Si $p \equiv 2,3 \pmod{5}$, entonces $p$ es inerte: el ideal $p\O_K$
    es primo.

  \item Si $p = 5$, entonces tenemos ramificación $p\O_K = \mathfrak{p}^4$,
    donde $\mathfrak{p} = (1 - \zeta_5)$, y no es difícil comprobar a mano que
    $\sigma (\mathfrak{p}) = \mathfrak{p}$ (aunque ya lo sabemos: $\sigma$
    permuta los primos $\mathfrak{p} \mid p$, y en este caso hay un solo
    primo sobre $p$). \qedhere
  \end{itemize}
\end{ejemplo}

Resulta que la acción de $G$ sobre los primos $\mathfrak{p} \mid p$ es siempre
transitiva. Esto puede ser probado usando el siguiente resultado general.

\begin{lema}[Tate]
  Sean $A$ un anillo conmutativo y $G$ un grupo finito que actúa sobre $A$
  mediante automorfismos. Consideremos los elementos fijos respecto a esta
  acción:
  $$A^G = \{ a \in A \mid \sigma (a) = a \text{ para todo }\sigma\in G \}.$$
  Sean $R$ un dominio y $\phi,\psi$ dos homomorfismos
  \[ \begin{tikzcd}
    A^G &[-3em] \subset &[-3em] A \ar[shift left=0.25em]{r}{\phi}\ar[shift right=0.25em]{r}[swap]{\psi} & R
  \end{tikzcd} \]
  tales que $\left.\phi\right|_{A^G} = \left.\psi\right|_{A^G}$. Entonces,
  $\phi = \psi\circ\sigma$ para algún $\sigma \in G$.
\end{lema}

Antes de probar el lema, vamos a sacar un corolario.

\begin{corolario}
  Para una extensión de Galois $K/\QQ$, si
  $\mathfrak{p}_1, \mathfrak{p}_2 \subset \O_K$ son dos primos tales que
  $\mathfrak{p}_1, \mathfrak{p}_2 \mid p$, entonces existe $\sigma \in G$
  tal que $\sigma (\mathfrak{p}_1) = \mathfrak{p}_2$.

  \begin{proof}
    Tenemos $K^G = \QQ$, y luego $(\O_K)^G = \ZZ$. Cada $\mathfrak{p}_i$ es
    el núcleo de algún homomorfismo $\phi_i\colon \O_K \to \overline{\FF_p}$.
    Estamos en la siguiente situación:
    \[ \begin{tikzcd}
      \ZZ &[-3em] \subset &[-3em] \O_K \ar[shift left=0.25em]{r}{\phi_1}\ar[shift right=0.25em]{r}[swap]{\phi_2} & \overline{\FF_p}
    \end{tikzcd} \]
    Aquí $\left.\phi_1\right|_{\ZZ} = \left.\phi_2\right|_{\ZZ}$, así que
    el lema de Tate implica que $\phi_1 = \phi_2\circ\sigma$ para algún
    $\sigma \in G$. Ahora $\sigma (\ker (\phi_1)) = \ker (\phi_2)$.
  \end{proof}
\end{corolario}

\begin{proof}[Demostración del lema de Tate]
  Todo homomorfismo $\phi\colon A\to R$ se extiende a
  $\phi\colon A[x] \to R[x]$. Tenemos
  \[ \tag{*} \begin{tikzcd}
    A^G [x] &[-3em] \subset &[-3em] A [x] \ar[shift left=0.25em]{r}{\phi}\ar[shift right=0.25em]{r}[swap]{\psi} & R [x]
  \end{tikzcd} \]
  Para un elemento $a \in A$ definamos un polinomio
  $$f = \prod_{g\in G} (x - \sigma (a)).$$
  Notamos que los coeficientes de este polinomio son invariantes respecto
  a la acción de $G$, así que $f \in A^G [x]$, y por nuestra hipótesis
  se tiene $\phi (f) = \psi (f)$ en $R [x]$. El elemento $\phi (a)$ es una raíz
  de $\phi (f) = \psi (f)$:
  \[ \phi (f) = \prod_{\sigma \in G} = (x - \phi \sigma (a)) =
     \psi (f) = \prod_{\sigma \in G} = (x - \psi \sigma (a)). \]
  En particular, $\phi (a) = \psi \sigma (a)$ para algún $\sigma \in G$.

  Ahora para cada $\sigma \in G$ consideremos
  $$A_\sigma = \{ a \in A \mid \phi (a) = \psi \sigma (a) \}.$$
  Por lo que acabamos de probar,
  $$A = \bigcup_{\sigma \in G} A_\sigma.$$
  Afirmamos que se tiene $A = A_\sigma$ para algún $\sigma \in G$. Supongamos
  que esto no es cierto y para todo $\sigma \in G$ existe $a_\sigma \in A$
  tal que $a_\sigma \notin A_\sigma$. Consideremos el polinomio
  $$g = \sum_{\sigma \in G} a_\sigma x^{d_\sigma} \in A [x].$$
  donde los $d_\sigma$ son diferentes. El mismo argumento de arriba aplicado
  a (*) demuestra que
  \[ A [x] = \bigcup_{\sigma \in G} (A [x])_\sigma
           = \bigcup_{\sigma \in G} A_\sigma [x]. \]
  Tenemos $g \in A [x]$, pero $g \notin A_\sigma [x]$ para todo $\sigma$.
  Contradicción.
\end{proof}

La transitividad de la acción del grupo de Galois sobre los primos
$\mathfrak{p} \mid p$ tiene la siguiente consecuencia importante.

\begin{proposicion}
  Sea $K/\QQ$ una extensión finita de Galois. Para un primo racional $p$
  consideremos la factorización
  \[ \tag{*} p \O_K = \mathfrak{p}_1^{e_1} \cdots \mathfrak{p}_s^{e_s}. \]
  Los grados de campos residuales e índices de ramificación coinciden:
  $$f_1 = \cdots = f_s, \quad e_1 = \cdots = e_s.$$
  Entonces, si $f_p$ denota los grados de campos residuales,
  $e_p$ denota los índices de ramificación y $g_p = s$ es el número de primos,
  se tiene
  $$e_p\,f_p\,g_p = [K : \QQ].$$

  \begin{proof}
    Para la igualdad de los $f_i$, ya notamos que $\mathfrak{p}$ y
    $\sigma (\mathfrak{p})$ tienen el mismo grado del campo residual,
    y basta usar que la acción de $G$ sobre los $\mathfrak{p}_i$ es transitiva.
    Para los índices de ramificación, aplicando $\sigma$ a la expresión (*)
    se obtiene
    $$p\O_K = \sigma (\mathfrak{p}_1)^{e_1} \cdots \sigma (\mathfrak{p}_s)^{e_s},$$
    y luego si $\sigma (\mathfrak{p}_i) = \mathfrak{p}_j$, entonces
    $e_i = e_j$ por la unicidad de factorización en ideales primos. De nuevo,
    la transitividad de la acción de $G$ sobre los $\mathfrak{p}_i$ implica que
    todos los $e_i$ coinciden.
  \end{proof}
\end{proposicion}

\begin{ejemplo}
  Consideremos el campo ciclotómico $K = \QQ (\zeta_n)$, donde
  $n = \prod_p p^{v_p}$. Hemos visto que para un primo racional $p$
  se tiene $e_p = \phi (p^{v_p})$ y $f_p$ es el orden de $p$ módulo
  $n/p^{v_p}$. Notamos que estos números dependen solamente del resto de $p$
  módulo $n$.
\end{ejemplo}

\begin{ejemplo}
  Consideremos el campo de números $K = \QQ (\sqrt[3]{19})$ y su cerradura de
  Galois $L = \QQ (\sqrt[3]{19},\zeta_3)$. También tenemos un subcampo
  cuadrático $F = \QQ (\zeta_3)$.

  \[ \begin{tikzcd}[row sep=1em, column sep=1em]
    & L\ar[-]{dl}[swap]{2}\ar[-]{ddr}{3} \\
    K\ar[-]{ddr}[swap]{3} \\
    & & F\ar[-]{dl}{2} \\
    & \QQ
    \end{tikzcd} \]

  Las siguientes consideraciones son útiles. Para un primo racional $p$ y
  $\mathfrak{p} \mid p$ en $\O_K$, sea $\mathfrak{q}$ un primo en $\O_L$ tal que
  $\mathfrak{p} \subset \mathfrak{q}$. En este caso tenemos la siguiente
  situación:
  \[ \begin{tikzcd}
    \mathfrak{q} &[-3em] \subset &[-3em] \O_L\ar[-]{d}\ar[->>]{r} & \O_L/\mathfrak{q}\ar[-]{d}\ar[-,bend left=45]{dd}{f (\mathfrak{q})} \\
    \mathfrak{p} & \subset & \O_K\ar[-]{d}\ar[->>]{r} & \O_K/\mathfrak{p}\ar[-]{d}[swap]{f (\mathfrak{p})} \\
    p & \subset & \ZZ\ar[->>]{r} & \FF_p
  \end{tikzcd} \]
  En particular, $f (\mathfrak{p}) \mid f (\mathfrak{q})$.

  \vspace{1em}

  Para analizar los primos ramificados, podemos calcular los discriminantes:
  \[ \Delta_F = -3, \quad
     \Delta_K = -3\cdot 19^2, \quad
     \Delta_L = -3^3\cdot 19^4. \]
  En particular, los primos ramificados en $\O_L$ son los mismos que en $\O_K$.

  \begin{itemize}
  \item Para $p = 3$ se tiene $p\O_F = \mathfrak{r}^2$. El ideal
    $\mathfrak{r}\O_L$ se factoriza de alguna manera en ideales primos en
    $\O_L$ que tal vez pueden ramificarse más, pero de todos modos, tenemos
    $2 \mid e_3$. Por otra parte,
    $3\O_K = \mathfrak{p}^2 \, \mathfrak{p}'$, así que $g_3 \ge 2$. Dado
    que $e_3\,f_3\,g_3 = 6$, esto nos deja la única posibilidad
    $(e_3,f_3,g_3) = (2,1,3)$. Entonces,
    $$3\O_L = \mathfrak{q}^2\mathfrak{q}'^2\mathfrak{q}''^2,$$
    donde $f_3 = 1$.
    
  \item Para $p = 19$ se tiene $19\O_K = \mathfrak{p}^3$. Por otra parte,
    $19\O_F = \mathfrak{p}'\mathfrak{p}''$. Entonces, $e_p \ge 3$ y
    $g_p \ge 2$. Pero esto nos deja con la única posibilidad
    $e_p = 3$, $g_p = 2$, $f_{19} = 1$:
    $$19\O_L = \mathfrak{q}^3\mathfrak{q}'^3.$$
  \end{itemize}
  Ahora para los primos no ramificados, recordemos que $p$ se escinde en
  $F$ si y solamente si
  $$\legendre{-3}{p} = +1 \iff p \equiv 1 \pmod{3}.$$

  \begin{itemize}
  \item Si $p \equiv 2 \pmod{3}$, entonces
    \[ p\O_K = \mathfrak{p}\,\mathfrak{p}', \quad
       f (\mathfrak{p}) = 1, ~ f (\mathfrak{p}') = 2. \]
    Esto implica que $2 \mid f_p$ y $g_p > 1$, pero dado que $f_p\,g_p = 6$,
    la única posibilidad es $(f_p,g_p) = (2,3)$.
    $$p\O_L = \mathfrak{q}\,\mathfrak{q}'\,\mathfrak{q}''.$$

  \item Si $p \equiv 1 \pmod{3}$ y $19$ no es un cubo módulo $p$, entonces
    $p$ es inerte en $\O_K$, lo cuál implica que $3 \mid f_p$. Por otra parte,
    $p$ se escinde en $F$, y luego $g_p \ge 2$. Esto nos deja la única
    posibilidad $f_p = 3$ y $g_p = 2$:
    $$p\O_L = \mathfrak{q}\,\mathfrak{q}'.$$

  \item Si $p \equiv 1\pmod{3}$ y $19$ es un cubo módulo $p$, entonces
    $p\O_K = \mathfrak{p} \mathfrak{p}' \mathfrak{p}''$.
    Hay dos posibilidades:
    $p\O_L = \mathfrak{q}_1 \mathfrak{q}_2 \mathfrak{q}_3$,
    o
    $p\O_L = \mathfrak{q}_1 \mathfrak{q}_2 \cdots \mathfrak{q}_6$.

    Pero sabemos que
    $$p\O_L \cap \O_F = p\O_F = \mathfrak{r}\,\sigma (\mathfrak{r}),$$
    donde $\sigma$ es la conjugación compleja. Entonces, $\mathfrak{r}\O_L$
    y $\sigma (\mathfrak{r})\O_L$ se factorizan de la misma manera en
    ideales primos en $\O_L$, y la única posibilidad es
    $$p\O_L = \mathfrak{q}_1 \mathfrak{q}_2 \cdots \mathfrak{q}_6.$$
  \end{itemize}

  La figura \ref{fig:cerradura-de-sqrt-19} demuestra las factorizaciones en
  $\O_F$, $\O_K$ y $\O_K$. Los primeros primos $p \equiv 1 \pmod{3}$ tales que
  $19$ es un cubo mód $p$ son
  \[ p = 97, 109, 127, 151, 181, 271, 277, 283, 307, 313, \ldots \qedhere \]
\end{ejemplo}

Note que en el último ejemplo la factorización de $p$ no depende del resto de
$p$ módulo algún $N$, sino de una condición misteriosa
«$19$ es un cubo módulo $p$». Para las extensiones abelianas (con el grupo
$\Gal (K/\QQ)$ abeliano), el comportamiento de primos sí depende del resto de
$p$ módulo algún $N$. La razón detrás de este fenómeno es el teorema de
Kronecker--Weber.

\begin{figure}
  \begin{center}
    \renewcommand{\arraystretch}{1.5}
    \begin{tabular}{x{0.75cm}x{1cm}x{1.25cm}x{2.5cm}x{0.75cm}x{0.75cm}x{1cm}x{1.25cm}x{2.5cm}x{0.75cm}}
      $p$ & $p\O_F$ & $p\O_K$ & $p\O_L$ & $p~(3)$ & $p$ & $p\O_F$ & $p\O_K$ & $p\O_L$ & $p~(3)$ \tabularnewline
      \hline
      $2$ & $\mathfrak{r}$ & $\mathfrak{p}_1\,\mathfrak{p}_2$ & $\mathfrak{q}_1\,\mathfrak{q}_2\,\mathfrak{q}_3$ & $2$ & $127$ & $\mathfrak{r}_1\,\mathfrak{r}_2$ & $\mathfrak{p}_1\,\mathfrak{p}_2\,\mathfrak{p}_3$ & $\mathfrak{q}_1\,\mathfrak{q}_2\,\mathfrak{q}_3\,\mathfrak{q}_4\,\mathfrak{q}_5\,\mathfrak{q}_6$ & $1$ \tabularnewline
      \hline
      $3$ & $\mathfrak{r}^2$ & $\mathfrak{p}_1\,\mathfrak{p}_2^2$ & $\mathfrak{q}_1^2\,\mathfrak{q}_2^2\,\mathfrak{q}_3^2$ & $0$ & $131$ & $\mathfrak{r}$ & $\mathfrak{p}_1\,\mathfrak{p}_2$ & $\mathfrak{q}_1\,\mathfrak{q}_2\,\mathfrak{q}_3$ & $2$ \tabularnewline
      \hline
      $5$ & $\mathfrak{r}$ & $\mathfrak{p}_1\,\mathfrak{p}_2$ & $\mathfrak{q}_1\,\mathfrak{q}_2\,\mathfrak{q}_3$ & $2$ & $137$ & $\mathfrak{r}$ & $\mathfrak{p}_1\,\mathfrak{p}_2$ & $\mathfrak{q}_1\,\mathfrak{q}_2\,\mathfrak{q}_3$ & $2$ \tabularnewline
      \hline
      $7$ & $\mathfrak{r}_1\,\mathfrak{r}_2$ & $\mathfrak{p}$ & $\mathfrak{q}_1\,\mathfrak{q}_2$ & $1$ & $139$ & $\mathfrak{r}_1\,\mathfrak{r}_2$ & $\mathfrak{p}$ & $\mathfrak{q}_1\,\mathfrak{q}_2$ & $1$ \tabularnewline
      \hline
      $11$ & $\mathfrak{r}$ & $\mathfrak{p}_1\,\mathfrak{p}_2$ & $\mathfrak{q}_1\,\mathfrak{q}_2\,\mathfrak{q}_3$ & $2$ & $149$ & $\mathfrak{r}$ & $\mathfrak{p}_1\,\mathfrak{p}_2$ & $\mathfrak{q}_1\,\mathfrak{q}_2\,\mathfrak{q}_3$ & $2$ \tabularnewline
      \hline
      $13$ & $\mathfrak{r}_1\,\mathfrak{r}_2$ & $\mathfrak{p}$ & $\mathfrak{q}_1\,\mathfrak{q}_2$ & $1$ & $151$ & $\mathfrak{r}_1\,\mathfrak{r}_2$ & $\mathfrak{p}_1\,\mathfrak{p}_2\,\mathfrak{p}_3$ & $\mathfrak{q}_1\,\mathfrak{q}_2\,\mathfrak{q}_3\,\mathfrak{q}_4\,\mathfrak{q}_5\,\mathfrak{q}_6$ & $1$ \tabularnewline
      \hline
      $17$ & $\mathfrak{r}$ & $\mathfrak{p}_1\,\mathfrak{p}_2$ & $\mathfrak{q}_1\,\mathfrak{q}_2\,\mathfrak{q}_3$ & $2$ & $157$ & $\mathfrak{r}_1\,\mathfrak{r}_2$ & $\mathfrak{p}$ & $\mathfrak{q}_1\,\mathfrak{q}_2$ & $1$ \tabularnewline
      \hline
      $19$ & $\mathfrak{r}_1\,\mathfrak{r}_2$ & $\mathfrak{p}^3$ & $\mathfrak{q}_1^3\,\mathfrak{q}_2^3$ & $1$ & $163$ & $\mathfrak{r}_1\,\mathfrak{r}_2$ & $\mathfrak{p}$ & $\mathfrak{q}_1\,\mathfrak{q}_2$ & $1$ \tabularnewline
      \hline
      $23$ & $\mathfrak{r}$ & $\mathfrak{p}_1\,\mathfrak{p}_2$ & $\mathfrak{q}_1\,\mathfrak{q}_2\,\mathfrak{q}_3$ & $2$ & $167$ & $\mathfrak{r}$ & $\mathfrak{p}_1\,\mathfrak{p}_2$ & $\mathfrak{q}_1\,\mathfrak{q}_2\,\mathfrak{q}_3$ & $2$ \tabularnewline
      \hline
      $29$ & $\mathfrak{r}$ & $\mathfrak{p}_1\,\mathfrak{p}_2$ & $\mathfrak{q}_1\,\mathfrak{q}_2\,\mathfrak{q}_3$ & $2$ & $173$ & $\mathfrak{r}$ & $\mathfrak{p}_1\,\mathfrak{p}_2$ & $\mathfrak{q}_1\,\mathfrak{q}_2\,\mathfrak{q}_3$ & $2$ \tabularnewline
      \hline
      $31$ & $\mathfrak{r}_1\,\mathfrak{r}_2$ & $\mathfrak{p}$ & $\mathfrak{q}_1\,\mathfrak{q}_2$ & $1$ & $179$ & $\mathfrak{r}$ & $\mathfrak{p}_1\,\mathfrak{p}_2$ & $\mathfrak{q}_1\,\mathfrak{q}_2\,\mathfrak{q}_3$ & $2$ \tabularnewline
      \hline
      $37$ & $\mathfrak{r}_1\,\mathfrak{r}_2$ & $\mathfrak{p}$ & $\mathfrak{q}_1\,\mathfrak{q}_2$ & $1$ & $181$ & $\mathfrak{r}_1\,\mathfrak{r}_2$ & $\mathfrak{p}_1\,\mathfrak{p}_2\,\mathfrak{p}_3$ & $\mathfrak{q}_1\,\mathfrak{q}_2\,\mathfrak{q}_3\,\mathfrak{q}_4\,\mathfrak{q}_5\,\mathfrak{q}_6$ & $1$ \tabularnewline
      \hline
      $41$ & $\mathfrak{r}$ & $\mathfrak{p}_1\,\mathfrak{p}_2$ & $\mathfrak{q}_1\,\mathfrak{q}_2\,\mathfrak{q}_3$ & $2$ & $191$ & $\mathfrak{r}$ & $\mathfrak{p}_1\,\mathfrak{p}_2$ & $\mathfrak{q}_1\,\mathfrak{q}_2\,\mathfrak{q}_3$ & $2$ \tabularnewline
      \hline
      $43$ & $\mathfrak{r}_1\,\mathfrak{r}_2$ & $\mathfrak{p}$ & $\mathfrak{q}_1\,\mathfrak{q}_2$ & $1$ & $193$ & $\mathfrak{r}_1\,\mathfrak{r}_2$ & $\mathfrak{p}$ & $\mathfrak{q}_1\,\mathfrak{q}_2$ & $1$ \tabularnewline
      \hline
      $47$ & $\mathfrak{r}$ & $\mathfrak{p}_1\,\mathfrak{p}_2$ & $\mathfrak{q}_1\,\mathfrak{q}_2\,\mathfrak{q}_3$ & $2$ & $197$ & $\mathfrak{r}$ & $\mathfrak{p}_1\,\mathfrak{p}_2$ & $\mathfrak{q}_1\,\mathfrak{q}_2\,\mathfrak{q}_3$ & $2$ \tabularnewline
      \hline
      $53$ & $\mathfrak{r}$ & $\mathfrak{p}_1\,\mathfrak{p}_2$ & $\mathfrak{q}_1\,\mathfrak{q}_2\,\mathfrak{q}_3$ & $2$ & $199$ & $\mathfrak{r}_1\,\mathfrak{r}_2$ & $\mathfrak{p}$ & $\mathfrak{q}_1\,\mathfrak{q}_2$ & $1$ \tabularnewline
      \hline
      $59$ & $\mathfrak{r}$ & $\mathfrak{p}_1\,\mathfrak{p}_2$ & $\mathfrak{q}_1\,\mathfrak{q}_2\,\mathfrak{q}_3$ & $2$ & $211$ & $\mathfrak{r}_1\,\mathfrak{r}_2$ & $\mathfrak{p}$ & $\mathfrak{q}_1\,\mathfrak{q}_2$ & $1$ \tabularnewline
      \hline
      $61$ & $\mathfrak{r}_1\,\mathfrak{r}_2$ & $\mathfrak{p}$ & $\mathfrak{q}_1\,\mathfrak{q}_2$ & $1$ & $223$ & $\mathfrak{r}_1\,\mathfrak{r}_2$ & $\mathfrak{p}$ & $\mathfrak{q}_1\,\mathfrak{q}_2$ & $1$ \tabularnewline
      \hline
      $67$ & $\mathfrak{r}_1\,\mathfrak{r}_2$ & $\mathfrak{p}$ & $\mathfrak{q}_1\,\mathfrak{q}_2$ & $1$ & $227$ & $\mathfrak{r}$ & $\mathfrak{p}_1\,\mathfrak{p}_2$ & $\mathfrak{q}_1\,\mathfrak{q}_2\,\mathfrak{q}_3$ & $2$ \tabularnewline
      \hline
      $71$ & $\mathfrak{r}$ & $\mathfrak{p}_1\,\mathfrak{p}_2$ & $\mathfrak{q}_1\,\mathfrak{q}_2\,\mathfrak{q}_3$ & $2$ & $229$ & $\mathfrak{r}_1\,\mathfrak{r}_2$ & $\mathfrak{p}$ & $\mathfrak{q}_1\,\mathfrak{q}_2$ & $1$ \tabularnewline
      \hline
      $73$ & $\mathfrak{r}_1\,\mathfrak{r}_2$ & $\mathfrak{p}$ & $\mathfrak{q}_1\,\mathfrak{q}_2$ & $1$ & $233$ & $\mathfrak{r}$ & $\mathfrak{p}_1\,\mathfrak{p}_2$ & $\mathfrak{q}_1\,\mathfrak{q}_2\,\mathfrak{q}_3$ & $2$ \tabularnewline
      \hline
      $79$ & $\mathfrak{r}_1\,\mathfrak{r}_2$ & $\mathfrak{p}$ & $\mathfrak{q}_1\,\mathfrak{q}_2$ & $1$ & $239$ & $\mathfrak{r}$ & $\mathfrak{p}_1\,\mathfrak{p}_2$ & $\mathfrak{q}_1\,\mathfrak{q}_2\,\mathfrak{q}_3$ & $2$ \tabularnewline
      \hline
      $83$ & $\mathfrak{r}$ & $\mathfrak{p}_1\,\mathfrak{p}_2$ & $\mathfrak{q}_1\,\mathfrak{q}_2\,\mathfrak{q}_3$ & $2$ & $241$ & $\mathfrak{r}_1\,\mathfrak{r}_2$ & $\mathfrak{p}$ & $\mathfrak{q}_1\,\mathfrak{q}_2$ & $1$ \tabularnewline
      \hline
      $89$ & $\mathfrak{r}$ & $\mathfrak{p}_1\,\mathfrak{p}_2$ & $\mathfrak{q}_1\,\mathfrak{q}_2\,\mathfrak{q}_3$ & $2$ & $251$ & $\mathfrak{r}$ & $\mathfrak{p}_1\,\mathfrak{p}_2$ & $\mathfrak{q}_1\,\mathfrak{q}_2\,\mathfrak{q}_3$ & $2$ \tabularnewline
      \hline
      $97$ & $\mathfrak{r}_1\,\mathfrak{r}_2$ & $\mathfrak{p}_1\,\mathfrak{p}_2\,\mathfrak{p}_3$ & $\mathfrak{q}_1\,\mathfrak{q}_2\,\mathfrak{q}_3\,\mathfrak{q}_4\,\mathfrak{q}_5\,\mathfrak{q}_6$ & $1$ & $257$ & $\mathfrak{r}$ & $\mathfrak{p}_1\,\mathfrak{p}_2$ & $\mathfrak{q}_1\,\mathfrak{q}_2\,\mathfrak{q}_3$ & $2$ \tabularnewline
      \hline
      $101$ & $\mathfrak{r}$ & $\mathfrak{p}_1\,\mathfrak{p}_2$ & $\mathfrak{q}_1\,\mathfrak{q}_2\,\mathfrak{q}_3$ & $2$ & $263$ & $\mathfrak{r}$ & $\mathfrak{p}_1\,\mathfrak{p}_2$ & $\mathfrak{q}_1\,\mathfrak{q}_2\,\mathfrak{q}_3$ & $2$ \tabularnewline
      \hline
      $103$ & $\mathfrak{r}_1\,\mathfrak{r}_2$ & $\mathfrak{p}$ & $\mathfrak{q}_1\,\mathfrak{q}_2$ & $1$ & $269$ & $\mathfrak{r}$ & $\mathfrak{p}_1\,\mathfrak{p}_2$ & $\mathfrak{q}_1\,\mathfrak{q}_2\,\mathfrak{q}_3$ & $2$ \tabularnewline
      \hline
      $107$ & $\mathfrak{r}$ & $\mathfrak{p}_1\,\mathfrak{p}_2$ & $\mathfrak{q}_1\,\mathfrak{q}_2\,\mathfrak{q}_3$ & $2$ & $271$ & $\mathfrak{r}_1\,\mathfrak{r}_2$ & $\mathfrak{p}_1\,\mathfrak{p}_2\,\mathfrak{p}_3$ & $\mathfrak{q}_1\,\mathfrak{q}_2\,\mathfrak{q}_3\,\mathfrak{q}_4\,\mathfrak{q}_5\,\mathfrak{q}_6$ & $1$ \tabularnewline
      \hline
      $109$ & $\mathfrak{r}_1\,\mathfrak{r}_2$ & $\mathfrak{p}_1\,\mathfrak{p}_2\,\mathfrak{p}_3$ & $\mathfrak{q}_1\,\mathfrak{q}_2\,\mathfrak{q}_3\,\mathfrak{q}_4\,\mathfrak{q}_5\,\mathfrak{q}_6$ & $1$ & $277$ & $\mathfrak{r}_1\,\mathfrak{r}_2$ & $\mathfrak{p}_1\,\mathfrak{p}_2\,\mathfrak{p}_3$ & $\mathfrak{q}_1\,\mathfrak{q}_2\,\mathfrak{q}_3\,\mathfrak{q}_4\,\mathfrak{q}_5\,\mathfrak{q}_6$ & $1$ \tabularnewline
      \hline
      $113$ & $\mathfrak{r}$ & $\mathfrak{p}_1\,\mathfrak{p}_2$ & $\mathfrak{q}_1\,\mathfrak{q}_2\,\mathfrak{q}_3$ & $2$ & $281$ & $\mathfrak{r}$ & $\mathfrak{p}_1\,\mathfrak{p}_2$ & $\mathfrak{q}_1\,\mathfrak{q}_2\,\mathfrak{q}_3$ & $2$ \tabularnewline
      \hline
    \end{tabular}
  \end{center}

  \caption{Factorización de primos racionales en $F = \QQ (\zeta_3)$, $K = \QQ (\sqrt[3]{19})$ y $L = \QQ (\sqrt[3]{19}, \zeta_3)$}
  \label{fig:cerradura-de-sqrt-19}
\end{figure}

%%%%%%%%%%%%%%%%%%%%%%%%%%%%%%%%%%%%%%%%%%%%%%%%%%%%%%%%%%%%%%%%%%%%%%%%%%%%%%%%

\pdfbookmark{Clase 17 (12/10/20)}{clase-17}
\section{Descomposición e inercia}
\marginpar{\small Clase 17 \\ 12/10/20}

El último ejemplo con la descomposición de primos en $\QQ (\sqrt[3]{19})$
sugiere que es útil considerar extensiones de campos de números $L/K/\QQ$.
Vamos a resumir brevemente qué sucede en este caso. Dado un ideal primo
$\mathfrak{p} \subset \O_K$, tenemos su factorización en ideales primos
$\mathfrak{q} \subset \O_L$
$$\mathfrak{p}\O_L = \prod_{\mathfrak{q} \mid \mathfrak{p}} \mathfrak{q}^{e (\mathfrak{q}|\mathfrak{p})}.$$
Pongamos
\[ f (\mathfrak{q}|\mathfrak{p})
   = [\O_L/\mathfrak{q} : \O_K/\mathfrak{p}]
   = [\kappa (\mathfrak{q}) : \kappa (\mathfrak{p})]. \]
Se cumple entonces
$$\sum_{\mathfrak{q} \mid \mathfrak{p}} e (\mathfrak{q}|\mathfrak{p}) \, f (\mathfrak{q}|\mathfrak{p}) = [L : K].$$

Ahora si $\mathfrak{p} \mid p$ para un primo racional $p$, entonces se cumple
\[ f (\mathfrak{q}|p) = f (\mathfrak{q}|\mathfrak{p}) \cdot f (\mathfrak{p}|p),
   \quad
   e (\mathfrak{q}|p) = e (\mathfrak{q}|\mathfrak{p}) \cdot e (\mathfrak{p}|p). \]

\[ \begin{tikzcd}
  \mathfrak{q}\ar[-]{d} &[-3em] \subset &[-3em] \O_L\ar[-]{d}\ar[->>]{r} & \kappa (\mathfrak{q}) \ar[-]{d}[swap]{f (\mathfrak{q}|\mathfrak{p})}\ar[-,bend left=45]{dd}{f (\mathfrak{q}|p)} \\
  \mathfrak{p}\ar[-]{d} & \subset & \O_K\ar[-]{d}\ar[->>]{r} & \kappa (\mathfrak{p}) \ar[-]{d}[swap]{f (\mathfrak{p}|p)} \\
  (p) & \subset & \ZZ \ar[->>]{r} & \FF_p
\end {tikzcd} \]

Si $L/K$ es una extensión de Galois, el grupo $\Gal (L/K)$ induce una acción
sobre $\O_L$, y luego para todo primo $\mathfrak{p} \subset \O_K$ una acción
transitiva sobre los primos $\mathfrak{q} \mid \mathfrak{p}$ en $\O_L$. De la
transitividad de esta acción se deduce que los números
$f (\mathfrak{q}|\mathfrak{p})$ y $e (\mathfrak{q}|\mathfrak{p})$
coinciden para todo $\mathfrak{q} \mid \mathfrak{p}$, y entonces si
$g_\mathfrak{p}$ es el número de ideales primos en $\O_L$ que están sobre
$\mathfrak{p}$, se cumple
$$e (\mathfrak{q}|\mathfrak{p}) \, f (\mathfrak{q}|\mathfrak{p}) \, g_\mathfrak{p} = [L:K].$$

\begin{definicion}
  En la situación de arriba, para $\mathfrak{q} \mid \mathfrak{p}$ el
  \textbf{grupo de descomposición} es el estabilizador de $\mathfrak{q}$
  respecto a la acción de $\Gal (L/K)$ sobre los primos sobre $\mathfrak{p}$:
  \[ D (\mathfrak{q}|\mathfrak{p})
     = \{ \sigma \in \Gal (L/K) \mid \sigma (\mathfrak{q}) = \mathfrak{q} \}. \]
\end{definicion}

El grupo de descomposición tiene el siguiente significado: todo elemento
$\sigma \in D (\mathfrak{q}|\mathfrak{p})$ induce un automorfismo
$\overline{\sigma} \in \Gal (\kappa(\mathfrak{q})/\kappa(\mathfrak{p}))$.
\[ \begin{tikzcd}
  & \O_K \ar[->>]{dd}\ar[left hook->]{dl}\ar[right hook->]{dr} \\
  \O_L \ar[crossing over]{rr}[near end]{\sigma}\ar[->>]{dd} & & \O_L \ar[->>]{dd} \\
  & \kappa (\mathfrak{p}) \ar[left hook->]{dl}\ar[right hook->]{dr} \\
  \kappa (\mathfrak{q}) \ar{rr}{\overline{\sigma}} && \kappa (\mathfrak{q}) \\
\end{tikzcd} \]
De esta manera se obtiene un homomorfismo de grupos
\[ D (\mathfrak{q}|\mathfrak{p}) \to \Gal (\kappa(\mathfrak{q}) / \kappa(\mathfrak{p})),
   \quad \sigma \mapsto \overline{\sigma}. \]


\begin{definicion}
  Para un primo $\mathfrak{q} \mid \mathfrak{p}$ el \textbf{grupo de inercia}
  viene dado por
  \[ I (\mathfrak{q}|\mathfrak{p})
  = \ker \Bigl(D (\mathfrak{q}|\mathfrak{p}) \to \Gal (\kappa(\mathfrak{q}) / \kappa(\mathfrak{p}))\Bigr)
  = \{ \sigma \in \Gal (L/K) \mid \sigma (\alpha) \equiv \alpha \pmod{\mathfrak{q}}
  \text{para todo }\alpha \in \O_L \}. \]
\end{definicion}

\noindent (Note que $\sigma (\alpha) \equiv \alpha \pmod{\mathfrak{q}}$ implica
que $\sigma (\mathfrak{q}) = \mathfrak{q}$, así que
$\sigma \in D (\mathfrak{q}|\mathfrak{p})$.)

\begin{definicion}
  Para los grupos $D = D (\mathfrak{q}|\mathfrak{p})$ e
  $I = I (\mathfrak{q}|\mathfrak{p})$ los campos fijos correspondientes
  $L^D$ y $L^I$ se llaman los \textbf{campos de descomposición e inercia}
  respectivamente.
\end{definicion}

Notamos que para $\sigma \in \Gal (L/K)$ se tiene
\[ D (\sigma (\mathfrak{q})|\mathfrak{p}) =
\sigma \, D (\mathfrak{q}|\mathfrak{p}) \, \sigma^{-1}, \quad
I (\sigma (\mathfrak{q})|\mathfrak{p}) =
\sigma \, I (\mathfrak{q}|\mathfrak{p}) \, \sigma^{-1}. \]
Esto implica que el campo de descomposición e inercia, salvo isomorfismo,
depende solo de $\mathfrak{p}$.

\vspace{1em}

En general, dado un subgrupo $H \subseteq \Gal (L/K)$, consideremos el subcampo
fijo $K \subseteq L^H \subseteq L$ y el anillo
$$(\O_L)^H = L^H \cap \O_L.$$
Para un primo $\mathfrak{q} \subset \O_L$ consideremos el primo correspondiente
en $(\O_K)^H$:
$$\mathfrak{q}^H = \mathfrak{q} \cap (\O_L)^H.$$
Tenemos entonces la siguiente situación para $\mathfrak{q} \mid \mathfrak{p}$:
\[ \begin{tikzcd}
  \mathfrak{q}\ar[-]{d} &[-3em] \subset &[-3em] \O_L\ar[-]{d}\ar[->>]{r} & \kappa (\mathfrak{q}) \ar[-]{d} \\
  \mathfrak{q}^H\ar[-]{d} & \subset & (\O_L)^H \ar[-]{d}\ar[->>]{r} & \kappa (\mathfrak{q}^H) \ar[-]{d} \\
  \mathfrak{p} & \subset & \O_K \ar[->>]{r} & \kappa (\mathfrak{p})
\end{tikzcd} \]

\begin{teorema}
  \label{thm:campo-de-descomposicion-e-inercia}
  Para una extensión de Galois de campos de números $L/K$, sean
  $\mathfrak{p} \subset \O_K$ y $\mathfrak{q} \subset \O_K$ primos tales que
  $\mathfrak{q}\mid\mathfrak{p}$ y $D = D (\mathfrak{q}|\mathfrak{p})$
  e $I = I (\mathfrak{q}|\mathfrak{p})$ los grupos de descomposición
  e inercia correspondientes. Denotemos por $g_\mathfrak{p}$ el número de primos
  en $\O_L$ sobre $\mathfrak{p}$.

  \begin{enumerate}
  \item[1)] Tenemos los siguientes grados de extensiones e índices de
    ramificación.

    \[ \begin{tikzcd}
      L \ar[-]{d}{e (\mathfrak{q}|\mathfrak{p})} & \mathfrak{q} \ar[-]{d} & e (\mathfrak{q}|\mathfrak{q}^I) &[-3em] = &[-3em] e (\mathfrak{q}|\mathfrak{p}) & f (\mathfrak{q}|\mathfrak{q}^I) &[-3em] = &[-3em] 1 \\
      L^I \ar[-]{d}{f (\mathfrak{q}|\mathfrak{p})} & \mathfrak{q}^I \ar[-]{d} & e (\mathfrak{q}^I|\mathfrak{q}^D) & = & 1 & f (\mathfrak{q}^I|\mathfrak{q}^D) & = & f (\mathfrak{q}|\mathfrak{p}) \\
      L^D \ar[-]{d}{g_\mathfrak{p}} & \mathfrak{q}^D \ar[-]{d} & e (\mathfrak{q}^D|\mathfrak{p}) & = & 1 & f (\mathfrak{q}^D|\mathfrak{p}) & = & 1 \\
      K & \mathfrak{p}
    \end{tikzcd} \]

  \item[2)] Se tiene $[G : D] = g_\mathfrak{p}$ e
    $|I| = e (\mathfrak{q}|\mathfrak{p})$.

  \item[3)] Tenemos una sucesión exacta corta de grupos
    \[ 1 \to I (\mathfrak{q}|\mathfrak{p}) \to
    D (\mathfrak{q}|\mathfrak{p}) \to
    \Gal (\kappa(\mathfrak{q})/\kappa(\mathfrak{p})) \to 1 \]
    En particular, si $\mathfrak{p}$ no se ramifica en $L$, entonces $I = 1$
    y se tiene un isomorfismo
    \[ D (\mathfrak{q}|\mathfrak{p}) \cong
    \Gal (\kappa(\mathfrak{q})/\kappa(\mathfrak{p})), \quad
    \sigma \mapsto \overline{\sigma}. \]
  \end{enumerate}

  \begin{proof}
    Denotemos $G = \Gal (L/K)$.

    \begin{itemize}
      \item Primero vamos a probar que $[L^D : K] = g_\mathfrak{p}$. Por la
        teoría de Galois, se tiene $[L^D : K] = [G : D]$. Recordemos el
        teorema de órbitas y estabilizadores: si $G$ actúa sobre un conjunto
        $X$, entonces para $x \in X$ hay una biyección entre la órbita $Gx$ y
        las clases laterales del estabilizador $G/G_x$. En nuestro caso la
        acción es transitiva, así que $[G : D] = g_\mathfrak{p}$.

      \item Ahora veremos que
        $e (\mathfrak{q}^D|\mathfrak{p}) = f (\mathfrak{q}^D|\mathfrak{p}) = 1$.
        La acción de $\Gal (L/L^D)$ sobre los primos sobre $\mathfrak{q}^D$ es
        transitiva, pero $\Gal (L/L^D) \cong D$ deja $\mathfrak{q}$ fijo, así que
        podemos concluir que $\mathfrak{q}$ es el único primo que está sobre
        $\mathfrak{q}^D$. Como consecuencia, tenemos
        $$[L : L^D] = e (\mathfrak{q}|\mathfrak{q}^D) \, f (\mathfrak{q}|\mathfrak{q}^D).$$
        Ahora
        $$[L : K] = e (\mathfrak{q}|\mathfrak{p}) \, f (\mathfrak{q}|\mathfrak{p}) \, g_\mathfrak{p}$$
        junto con $[L^D : K] = g_\mathfrak{p}$ nos da
        \[ e (\mathfrak{q}|\mathfrak{q}^D) = e (\mathfrak{q}|\mathfrak{p}), \quad
           f (\mathfrak{q}|\mathfrak{q}^D) = f (\mathfrak{q}|\mathfrak{p}), \]
        y luego
        $$e (\mathfrak{q}^D|\mathfrak{p}) = f (\mathfrak{q}^D|\mathfrak{p}) = 1.$$

      \item Vamos a ver que $f (\mathfrak{q}|\mathfrak{q}^I) = 1$. Esto
        equivale a probar que el grupo
        $\Gal (\kappa (\mathfrak{q})/\kappa (\mathfrak{q}^I))$ es trivial.

        Para $\alpha \in \O_K$ consideremos el polinomio
        $$f (x) = \prod_{\sigma \in I} (x - \sigma (\alpha)).$$
        Se ve que los coeficientes de $f (x)$ están en $(\O_L)^I$. Reduciendo módulo
        $\mathfrak{q}$, se obtiene un polinomio
        $\overline{f} (x) \in \kappa (\mathfrak{q}) [x]$, y como acabamos de ver, sus
        coeficientes están en $\kappa (\mathfrak{q}^I)$.

        Para un elemento $\overline{\alpha} \in \kappa (\mathfrak{q})$, dado que
        $\sigma (\alpha) \equiv \overline{\alpha} \pmod{\mathfrak{q}}$ para todo
        $\sigma \in I$, tenemos $\overline{f} (x) = (x - \overline{\alpha})^n$,
        donde $n = |I|$, y el polinomio $\overline{f} (x)$ tiene coeficientes en
        $\kappa (\mathfrak{q}^I)$. Esto implica que cualquier automorfismo de
        $\kappa (\mathfrak{q})/\kappa (\mathfrak{q}^I)$ debe mandar
        $\overline{\alpha}$ a otra raíz de $\overline{f} (x)$, pero la única
        raíz de $\overline{f} (x)$ es $\overline{\alpha}$. Entonces, $\kappa
        (\mathfrak{q})/\kappa (\mathfrak{q}^I)$ no tiene automorfismos no
        triviales.

      \item De $f (\mathfrak{q}^D|\mathfrak{p}) = 1$ y
        $f (\mathfrak{q}|\mathfrak{q}^I) = 1$ se sigue que
        $f (\mathfrak{q}^I|\mathfrak{q}^D) = f (\mathfrak{q}|\mathfrak{p})$.

      \item Tenemos $[L^I : L^D] = [D : I]$. Por una parte,
        $[L^I : L^D] \ge f (\mathfrak{q}^I|\mathfrak{q}^D) = f (\mathfrak{q}|\mathfrak{p})$.
        Además, tenemos una sucesión exacta
        \[ 1 \to I (\mathfrak{q}|\mathfrak{p}) \to
        D (\mathfrak{q}|\mathfrak{p}) \to
        \Gal (\kappa(\mathfrak{q})/\kappa(\mathfrak{p})), \]
        de donde
        $[D : I] \le |\Gal (\mathfrak{q}/\mathfrak{p})| = f (\mathfrak{q}|\mathfrak{p})$.
        Entonces, podemos concluir que
        $[L^I : L^D] = [D : I] = f (\mathfrak{q}|\mathfrak{p})$, y además que
        el último homomorfismo en la sucesión exacta es sobreyectivo.

        Esto también implica que $e (\mathfrak{q}^I|\mathfrak{q}^D) = 1$.

      \item De lo que hemos calculado se sigue que
        $[L : L^I] = e (\mathfrak{q}|\mathfrak{p})$ y
        $e (\mathfrak{q}|\mathfrak{q}^I) = e (\mathfrak{q}|\mathfrak{p})$. \qedhere
    \end{itemize}
    \end{proof}
\end{teorema}

El resultado que acabamos de probar explica los términos
«campo de descomposición» y «campo de inercia».

\begin{corolario}
  Si $D = D (\mathfrak{q}|\mathfrak{p})$ es un subgrupo normal en
  $G = \Gal (K/\QQ)$, entonces $\mathfrak{p}$ se descompone en $g_\mathfrak{p}$
  diferentes primos en $L^D$. Además, si
  $I = I (\mathfrak{q}|\mathfrak{p})$ es también un subgrupo normal en $G$,
  entonces cada uno de esos primos es también primo en $L^I$ (es decir, inerte).
  En fin, estos primos se ramifican en $L$, volviendo
  $e = e (\mathfrak{q}|\mathfrak{p})$-ésimas potencias.

  \begin{proof}
    Si $D \subseteq G$ es un subgrupo normal, entonces $L^D/K$ es una extensión
    de Galois. En este caso todos los primos en $L_D$ que están sobre
    $\mathfrak{p}$ tendrán $e = 1$ y $f = 1$, así que la factorización tiene
    forma $\mathfrak{p}\,(\O_L)^D = \mathfrak{p}'_1\cdots\mathfrak{p}_g'$.

    De la misma manera, si $I \subseteq G$ es un subgrupo normal, entonces
    $L^I/K$ es una extensión de Galois y los índices de ramificación serán
    $e = 1$ y los grados de campos residual
    $f = f (\mathfrak{q}|\mathfrak{p})$, y habrá $g$ primos. La factorización
    tiene forma
    $\mathfrak{p}\,(\O_L)^I = \mathfrak{p}''_1\cdots\mathfrak{p}_g''$,
    lo que significa precisamente que los primos $\mathfrak{p}_i'$ son inertes
    en $L^I$.
  \end{proof}
\end{corolario}

Los cálculos de \ref{thm:campo-de-descomposicion-e-inercia} de hecho
caracterizan el campo de descomposición e inercia. Usando la misma notación,
consideremos un subcampo $K \subseteq K' \subseteq L$ y un primo
$\mathfrak{p} = \mathfrak{q} \cap \O_{K'}$. Tenemos $K' = L^H$ para algún
subgrupo $H \subseteq \Gal (L/K)$, y la extensión $L/K'$ es Galois.
Notamos que $\mathfrak{p}' \mid \mathfrak{p}$. De las definiciones del grupo
de descomposición e inercia se ve que
\[ D (\mathfrak{q}|\mathfrak{p}') = D\cap H, \quad
   I (\mathfrak{q}|\mathfrak{p}') = I\cap H, \]
y los campos de descomposición e inercia correspondientes son los compositums
\[ L^{D (\mathfrak{q}|\mathfrak{p}')} = L^D\,K', \quad
   L^{I (\mathfrak{q}|\mathfrak{p}')} = L^I\,K'. \]
El teorema \ref{thm:campo-de-descomposicion-e-inercia} nos da el siguiente
diagrama de extensiones

\[ \begin{tikzcd}[row sep=1em, column sep=1em]
  & L\ar[-]{dddl}[swap]{e (\mathfrak{q}|\mathfrak{p})}\ar[-]{ddr}{e (\mathfrak{q}|\mathfrak{p}')} \\
  \\
  & & L^I\,K'\ar[-]{dll}\ar[-]{dd}{f (\mathfrak{q}|\mathfrak{p}')} \\
  L^I\ar[-]{dd}{f (\mathfrak{q}|\mathfrak{p})} \\
  & & L^D\,K'\ar[-]{dll}\ar[-]{dd}{g_{\mathfrak{p}'}}\\
  L^D\ar[-]{dd}{g_\mathfrak{p}} \\
  & & K'\ar[-]{dll}\\
  K
\end{tikzcd} \]

\marginpar{\footnotesize Añadí este\\resultado\\después}

\begin{proposicion}
  \label{prop:caracterizacion-de-campo-de-descomposicion-e-inercia}
  En la situación de \ref{thm:campo-de-descomposicion-e-inercia}, consideremos
  un subcampo $K \subseteq K' \subseteq L$. El campo de descomposición $L^D$
  e inercia $L^I$ se caracterizan por las siguientes propiedades.

  \begin{enumerate}
  \item[a)] $L^D$ es el $K'$ más grande tal que
    $e (\mathfrak{p}'|\mathfrak{p}) = f (\mathfrak{p}'|\mathfrak{p}) = 1$.

  \item[b)] $L^D$ es el $K'$ más pequeño tal que $g_{\mathfrak{p}'} = 1$
    (es decir, $\mathfrak{q}$ es el único primo en $\O_L$ tal que
    $\mathfrak{q} \mid \mathfrak{p}'$).

  \item[c)] $L^I$ es el $K'$ más grande tal que
    $e (\mathfrak{p}'|\mathfrak{p}) = 1$.

  \item[d)] $L^I$ es el $K'$ más pequeño tal que
    $e (\mathfrak{q}|\mathfrak{p}') = [L : K']$.
  \end{enumerate}

  \begin{proof}
    Primero observamos que \ref{thm:campo-de-descomposicion-e-inercia}
    nos dice que $L^D$ y $L^I$ cumplen las propiedades deseadas.

    En a), si tenemos
    $e (\mathfrak{p}'|\mathfrak{p}) = f (\mathfrak{p}'|\mathfrak{p}) = 1$,
    entonces del diagrama de arriba se deduce que
    \[ [L : L^D\,K'] =
       e (\mathfrak{q}|\mathfrak{p}')\,f (\mathfrak{q}|\mathfrak{p}') =
       e (\mathfrak{q}|\mathfrak{p}')\,e (\mathfrak{p}'|\mathfrak{p})\,
       f (\mathfrak{q}|\mathfrak{p}')\,f (\mathfrak{p}'|\mathfrak{p}) =
       e (\mathfrak{q}|\mathfrak{p})\,f (\mathfrak{q}|\mathfrak{p}) =
       [L : L^D]. \]
    Entonces, $L^D\,K' = L^D$, y por lo tanto $K' \subseteq L^D$.

    En b), si $g_{\mathfrak{p}'} = [L^D\,K' : K'] = 1$, entonces
    $L^D\,K' = K'$, y por ende $L^D \subseteq K'$.

    En c), si $e (\mathfrak{p}'|\mathfrak{p}) = 1$, entonces de la misma manera
    calculamos
    \[ [L : L^I\,K'] = e (\mathfrak{q}|\mathfrak{p}') =
       e (\mathfrak{q}|\mathfrak{p}')\,e (\mathfrak{p}'|\mathfrak{p}) =
       e (\mathfrak{q}|\mathfrak{p}) = [L : L^I], \]
    y entonces $K' \subseteq L^I$.

    En fin, en d), si $e (\mathfrak{q}|\mathfrak{p}') = [L:K']$, entonces del
    diagrama se ve que $L^I\,K' = K'$, y luego $L^I \subseteq K'$.
  \end{proof}
\end{proposicion}

\begin{ejemplo}
  Consideremos el campo ciclotómico $K = \QQ (\zeta_{28})$.
  El primo $p = 2$
  se ramifica. Tenemos
  $$\Phi_{28} \equiv (x^3 + x + 1)^2\,(x^3 + x^2 + 1)^2 \pmod{2},$$
  lo que nos da la factorización
  $p\O_K = \mathfrak{p}_1^2 \, \mathfrak{p}_2^2$,
  donde $f_1 = f_2 = 3$, y
  \[ \mathfrak{p}_1 = (2, 1 + \zeta_{28} + \zeta_{28}^3),
  \quad
  \mathfrak{p}_2 = (2, 1 + \zeta_{28}^2 + \zeta_{28}^3). \]
  Escribiendo $K = \QQ (i, \zeta_7)$, se ve que
  $\Gal (K/\QQ) \cong (\ZZ/4\ZZ)^\times \times (\ZZ/7\ZZ)^\times$.
  Como los generadores, podemos tomar
  $$\sigma\colon i \mapsto -i, ~ \zeta_7 \mapsto \zeta_7$$
  de orden $2$ y
  $$\tau\colon i \mapsto i, ~ \zeta_7 \mapsto \zeta_7^3$$
  de orden $6$. En términos de la raíz $\zeta_{28}$, tenemos
  $\sigma\colon \zeta_{28} \mapsto \zeta_{28}^{15}$ y
  $\tau\colon \zeta_{28} \mapsto \zeta_{28}^{17}$.
  Calculamos que
  \[ \sigma (\mathfrak{p}_1) = \mathfrak{p}_1, \quad
  \tau (\mathfrak{p}_1) = \mathfrak{p}_2, \quad
  \sigma (\mathfrak{p}_2) = \mathfrak{p}_2, \quad
  \tau (\mathfrak{p}_2) = \mathfrak{p}_1. \]

  De aquí se ve que el grupo de descomposición viene dado por
  $D (\mathfrak{p}_1|p) = \langle\sigma,\tau^2\rangle$.
  Aquí $\sigma$ y $\tau^2$ inducen algunos automorfismos del campo finito
  $$\ZZ [\zeta_{28}]/\mathfrak{p}_1 \cong \FF_2 [x]/(x^3 + x + 1) \cong \FF_8.$$
  Podemos notar, por ejemplo, que $\sigma$ actúa de manera trivial, dado que
  $|\FF_8^\times| = 7$ y $15 \equiv 1 \pmod{7}$, y entonces
  $\overline{\sigma}\colon \overline{\zeta_{28}} \mapsto \overline{\zeta_{28}}$.
  Por otra parte, la acción de $\overline{\tau^2}$ no es trivial y viene dada
  por $\overline{\zeta_{28}} \mapsto \overline{\zeta_{28}}^2$. (En efecto,
  $17^2 \equiv 3^2 \equiv 2 \pmod{7}$.)  De estas consideraciones se sigue que
  $I (\mathfrak{p}_1|p) = \langle\sigma\rangle$.

  Ahora el campo de inercia será $K^\sigma = \QQ (\zeta_7)$, y el campo de
  descomposición es $K^{\langle\sigma,\tau^2\rangle} = \QQ (\sqrt{-7})$.
  \[ \begin{tikzcd}
    \QQ (\zeta_{28}) \ar[-]{d}{e = 2} \\
    \QQ (\zeta_7) \ar[-]{d}{f = 3} \\
    \QQ (\sqrt{-7}) \ar[-]{d}{g = 2} \\
    \QQ
  \end{tikzcd} \]

  La factorización de $p = 2$ en $\QQ (\sqrt{-7})$ toma forma
  $\mathfrak{r}_1\,\mathfrak{r}_2$, y la factorización en $\QQ (\zeta_7)$
  toma forma $\mathfrak{r}_1'\,\mathfrak{r}_2'$, donde $f_1 = f_2 = 3$:
  factorizando el polinomio ciclotómico correspondiente,
  $$\Phi_7 \equiv (x^3 + x + 1)\,(x^3 + x^2 + 1) \pmod{2}.$$

  Ahora podemos considerar un primo no ramificado, por ejemplo $p = 3$.
  Tenemos entonces
  $$\Phi_{28} \equiv (x^6 + x^5 + x^3 + x + 1)\,(x^6 - x^5 - x^3 - x + 1) \pmod{3},$$
  lo que nos da factorización $p\O_K = \mathfrak{p}_1\,\mathfrak{p}_2$, donde
  \[ \mathfrak{p}_1 = (3, \zeta_{28}^6 + \zeta_{28}^5 + \zeta_{28}^3 + \zeta_{28} + 1), \quad
     \mathfrak{p}_2 = (3, \zeta_{28}^6 - \zeta_{28}^5 - \zeta_{28}^3 + \zeta_{28} + 1), \]
  y
  $f_1 = f_2 = 3$. Podemos calcular el grupo de descomposición:
  \[ \sigma (\mathfrak{p}_1) = \mathfrak{p}_2, \quad
     \tau (\mathfrak{p}_1) = \mathfrak{p}_1, \quad
     \sigma (\mathfrak{p}_2) = \mathfrak{p}_1, \quad
     \tau (\mathfrak{p}_2) = \mathfrak{p}_1. \]
  Se ve que $\sigma\tau$ y $\tau^2 = (\sigma\tau)^2$ están en el grupo de
  descomposición, y este será generado por
  $$\sigma\tau\colon \zeta_{28} \mapsto \zeta_{28}^{15\cdot 17} = \zeta_{28}^3.$$
  Al reducir $\sigma\tau$ módulo $\mathfrak{p}_1$, se obtiene el automorfismo
  de Frobenius $x \mapsto x^3$ sobre $\kappa (\mathfrak{p}_1) \cong \FF_{3^3}$,
  y tenemos un isomorfismo
  \[ D (\mathfrak{p}_1|p) = \langle\sigma\tau\rangle
     \cong \Gal (\kappa (\mathfrak{p}_1)/\FF_p). \]
  El grupo de inercia $I (\mathfrak{p}_1|p)$ será trivial, y esto sucede
  precisamente porque $p = 3$ no se ramifica.
\end{ejemplo}

\subsection{Reciprocidad cuadrática}

Como una aplicación de la teoría anterior, podemos dar otra prueba más de
la ley de reciprocidad cuadrática. Dado un primo impar $p$, consideremos
el campo ciclotómico $L = \QQ (\zeta_p)$ y el subcampo cuadrático
$K = \QQ (\sqrt{p^*})$, donde $p^* = (-1)^{\frac{p-1}{2}}\,p$
(véase ejercicio \ref{ejerc:subcampo-cuadratico-en-ciclotomico}).
Primero notamos el siguiente resultado.

\begin{lema}
  Un primo impar $q \ne p$ se escinde en $K$ si y solamente si
  $q$ se factoriza en $L$ en un número par de ideales primos.

  \begin{proof}
    Si $q$ se escinde en $K$, entonces
    $q\O_K = \mathfrak{q} \, \sigma (\mathfrak{q})$ para algún
    $\sigma \in \Gal (K/\QQ) \subset \Gal (L/\QQ)$. Luego, $\sigma$ induce una
    biyección entre los primos $\mathfrak{Q} \subset \O_L$ tales que
    $\mathfrak{Q} \mid \mathfrak{q}$ y los primos
    $\mathfrak{Q} \mid \sigma (\mathfrak{q})$. Como consecuencia, el número
    $\# \{ \mathfrak{Q} \mid q \} = 2\cdot \# \{ \mathfrak{Q} \mid \mathfrak{q} \}$
    es par.

    Viceversa, supongamos que el número $g$ de primos
    $\mathfrak{Q} \subset \O_L$ tales que $\mathfrak{Q} \mid q$ es par.
    Para el grupo de descomposición $D = D (\mathfrak{Q}|q)$ se cumple
    $[\Gal (L/\QQ) : D] = g$. El grupo $G = \Gal (L/\QQ)$ es cíclico de orden
    $p-1$. Para subgrupo $H$ tal que $[G : H]$ se tiene $L^H = K$, pero luego,
    puesto que $[G : D]$ es par, se cumple $D \subset H$, y entonces
    $K \subset L^D$. Ahora $f (\mathfrak{Q}^D|q) = 1$, y luego
    $f (\mathfrak{Q}\cap K|q) = 1$. Esto significa que $q$ se
    escinde en $K$.
  \end{proof}
\end{lema}

Ahora sabemos que $q$ se escinde en $K = \QQ (\sqrt{p^*})$ si y solamente si
$\legendre{p^*}{q} = 1$.

Por otra parte, el número de ideales primos en $\O_L = \ZZ [\zeta_p]$ sobre $q$
es igual a $g = \frac{p-1}{f}$, donde $f$ es el orden de $q$ módulo $p$
(véase \ref{prop:factorizacion-en-Q-zeta-p} y
\ref{thm:descomposicion-en-campos-ciclotomicos}). Notamos que este número es
par si y solamente si $f \mid \frac{p-1}{2}$; es decir, si y solamente si
$q^{\frac{p-1}{2}} \equiv 1 \pmod{p}$.

Recodando la congruencia de Euler
$q^{\frac{p-1}{2}} \equiv \legendre{q}{p} \pmod{p}$, estas consideraciones
demuestran que
$$\legendre{q}{p} = \legendre{p^*}{q}.$$

%%%%%%%%%%%%%%%%%%%%%%%%%%%%%%%%%%%%%%%%%%%%%%%%%%%%%%%%%%%%%%%%%%%%%%%%%%%%%%%%

\pdfbookmark{Clase 18 (14/10/20)}{clase-18}
\section{El Frobenius}
\marginpar{\small Clase 18 \\ 14/10/20}
\label{sec:frobenius}

\epigraph{\dots{}the Galois group $\Gal(L/K)$ contains an essential element
  called Frobenius substitution. This element <\dots>{} governs over the
  decomposition of $\mathfrak{p}$ in $L$: We might even claim that it has
  “the soul of $\mathfrak{p}$”. It glimmers like a firefiy in $\Gal (L/K)$
  for each and every prime ideal.}{\cite[p.~18]{Kato-NT-2}}

Como antes, consideremos una extensión de Galois de campos de números $L/K$
y primos $\mathfrak{p} \subset \O_K$, $\mathfrak{q} \subset \O_L$ tales que
$\mathfrak{q}\mid\mathfrak{p}$. En este caso si $\mathfrak{p}$ no se ramifica
en $L$, el grupo de inercia es trivial y
tenemos un isomorfismo
\[ D (\mathfrak{q}|\mathfrak{p}) \cong
   \Gal (\kappa(\mathfrak{q})/\kappa(\mathfrak{p})), \quad
   \sigma \mapsto \overline{\sigma}. \]
El grupo $\Gal (\kappa(\mathfrak{q})/\kappa(\mathfrak{p}))$ es cíclico, generado
por el automorfismo de Frobenius $x \mapsto x^{\# \kappa(\mathfrak{p})}$.
Gracias al isomorfismo de arriba, podemos considerar el elemento correspondiente
en el grupo de descomposición.

\begin{definicion}
  En la situación de arriba, el elemento
  $\Frob_{\mathfrak{q}|\mathfrak{p}} \in D (\mathfrak{q}|\mathfrak{p})$
  que se reduce mód $\mathfrak{q}$ a $x \mapsto x^{\# \kappa(\mathfrak{p})}$
  se llama el \textbf{automorfismo de Frobenius}\footnote{Ferdinand Georg
    Frobenius (1849--1917), matemático alemán.}, o simplemente
  \textbf{el Frobenius}.
\end{definicion}

\begin{comentario}
  Hagamos algunas observaciones básicas.

  \begin{enumerate}
  \item[1)] En otras palabras, $\Frob_{\mathfrak{q}|\mathfrak{p}}$ es el
    único elemento de $\Gal (L/K)$ que cumple
    \[ \tag{*} \Frob_{\mathfrak{q}|\mathfrak{p}} (\alpha) \equiv
               \alpha^{\# \kappa (\mathfrak{p})} \pmod{\mathfrak{q}} \]
    para todo $\alpha \in \O_L$. (Esta condición implica que
    $\Frob_{\mathfrak{q}|\mathfrak{p}} \in D (\mathfrak{q}|\mathfrak{p})$.)

  \item[2)] Al pasar de $\mathfrak{q}\mid\mathfrak{p}$ a otro primo
    $\sigma(\mathfrak{q})\mid\mathfrak{p}$ se obtiene
    $$\Frob_{\sigma(\mathfrak{q})|\mathfrak{p}} = \sigma \circ \Frob_{\mathfrak{q}|\mathfrak{p}} \circ \sigma^{-1}$$
    (ejercicio para el lector). Entonces, la clase de conjugación del Frobenius
    depende solo de $\mathfrak{p}$.

  \item[3)] En particular, para una extensión abeliana $L/K$, el Frobenius
    depende solo de $\mathfrak{p}$, y podemos denotarlo por
    $\Frob_\mathfrak{p}$. En este caso la condición (*) se cumple para todo
    $\mathfrak{q}\mid\mathfrak{p}$, y ocupando el teorema chino del resto
    (y que $e (\mathfrak{q}|\mathfrak{p}) = 1$ por nuestra hipótesis),
    podemos concluir que el Frobenius está definido por la condición
    \[ \Frob_\mathfrak{p} (\alpha) \equiv
       \alpha^{\# \kappa (\mathfrak{p})} \pmod{\mathfrak{p}\O_L}. \]
  \end{enumerate}
\end{comentario}

\begin{ejemplo}
  Consideremos un campo cuadrático $K = \QQ (\sqrt{d})$. Si $p$ es un primo no
  ramificado, entonces hay dos posibilidades:
  \begin{itemize}
  \item $p$ se escinde en dos ideales en $K$, y luego
    $D (\mathfrak{p}|p) = 1$, y $\Frob_p$ es el automorfismo trivial;

  \item $p$ es inerte en $K$, y luego $D (\mathfrak{p}|p) = \Gal (K/\QQ)$,
    y $\Frob_p\colon \sqrt{d}\mapsto -\sqrt{d}$ es el automorfismo no trivial.
  \end{itemize}

  Para $p$ impar tal que $p \nmid d$, podemos escribir entonces
  $\Frob_p\colon \sqrt{d} \mapsto \legendre{d}{p}\sqrt{d}$.
\end{ejemplo}

\begin{ejemplo}
  Consideremos un campo ciclotómico $K = \QQ (\zeta_n)$. Los primos $p \nmid n$
  no se ramifican en $K$. En este caso
  $\Frob_p\colon \zeta_n \mapsto \zeta_n^p$. En efecto, para un elemento
  $\alpha \in \O_K = \ZZ [\zeta_n] = \sum_i a_i \, \zeta_n^i$, donde
  $a_i \in \ZZ$, calculamos que
  \[ \alpha^p \equiv \sum_i a_i^p \, \zeta_n^p \equiv
     \sum_i a_i \, \zeta_n^p \pmod{p}. \qedhere \]
\end{ejemplo}

El orden de $\Frob_{\mathfrak{q}|\mathfrak{p}}$ es igual a
$f (\mathfrak{q}|\mathfrak{p})$, y entonces determina el tipo de
descomposición de $\mathfrak{p}$ en $L$: tenemos
$\mathfrak{p} \O_L = \mathfrak{q}_1\cdots \mathfrak{q}_g$, donde $f$
es el orden de $\Frob_{\mathfrak{q}|\mathfrak{p}}$ y $g = [L:K]/f$.
En particular, $\mathfrak{p}$ se descompone en $[L:K]$ ideales primos
si y solamente si $\Frob_{\mathfrak{q}|\mathfrak{p}} = 1$. En este caso se
dice que $\mathfrak{p}$ \textbf{se escinde completamente} en $L$.

\vspace{1em}

El siguiente resultado forma una parte importante de la teoría de campos de
clases y se conoce como el \textbf{teorema de densidad de Chebotarëv}.
Solo para simplificar el enunciado, vamos a formular la versión para extensiones
$K/\QQ$ en lugar del caso general $L/K$\footnote{Para las extensiones $L/K$ hay
  que definir la noción de densidad para ideales primos en $\O_K$ que está
  relacionada con la \textbf{función zeta de Dedekind} $\zeta_K (s)$ de la misma
  manera que la densidad de primos racionales está relacionada con la función
  zeta de Riemann $\zeta (s)$.}.

\begin{teorema}
  Para una extensión de Galois $K/\QQ$, sea $C$ una clase de conjugación en
  el grupo $G = \Gal (K/\QQ)$. El conjunto de primos racionales
  $X = \{ p \mid \Frob_p \in C \}$ tiene densidad $d (X) = \#C/\#G$.
\end{teorema}

En particular, el teorema nos dice que salvo conjugación, cualquier elemento del
grupo de Galois puede ser realizado como $\Frob_p$ para un número infinito de
$p$.

Para más detalles (matemáticos e históricos), recomiendo el artículo
\cite{Lenstra-Stevenhagen-1996}. Para la definición de densidad $d (X)$,
véase el apéndice \ref{ap:Dirichlet}. El teorema es cierto para la densidad de
Dirichlet (analítica) y también para la densidad natural\footnote{Véase
  \url{https://mathoverflow.net/questions/302390/}}.

\begin{corolario}
  Hay un número infinito de primos racionales $p$ que se escinden completamente
  en $K$, y su densidad es igual a $1/\#G$.
\end{corolario}

\begin{ejemplo}
  Para una extensión ciclotómica $K = \QQ (\zeta_n)$ y un primo $p \nmid n$,
  el Frobenius $\Frob_p$ está determinado por el resto de $p$ módulo $n$.
  En este caso el teorema de densidad de Chebotarëv se reduce al teorema de
  Dirichlet sobre los primos en progresiones aritméticas (véase el apéndice
  \ref{ap:Dirichlet}). Se puede pensar en el teorema de Chebotarëv como una
  generalización del teorema de Dirichlet.
\end{ejemplo}

\begin{ejemplo}
  Consideremos el campo $K = \QQ (\sqrt[3]{19}, \zeta_3)$. El grupo de Galois
  es isomorfo a $S_3$, donde hay tres clases de conjugación:
  \[ C = \{ 1 \}, \quad
     C' = \{ (1~2), \, (1~3), \, (2~3) \}, \quad
     C'' = \{ (1~2~3), \, (1~3~2) \}. \]
  El teorema de Chebotarëv nos dice lo siguiente.
  \begin{itemize}
  \item Los primos con $\Frob_p \in C$ corresponden a las factorizaciones
    de la forma $\mathfrak{p}_1\mathfrak{p}_2\cdots \mathfrak{p}_6$. Su densidad
    es $\frac{1}{6}$. Estos son los primos $p \equiv 1 \pmod{3}$ tales que
    $19$ es un cubo mód $p$.

  \item Los primos con $\Frob_p \in C'$ corresponden a las factorizaciones
    de la forma $\mathfrak{p}_1\mathfrak{p}_2\mathfrak{p}_2$.
    Su densidad es $\frac{1}{2}$. Estos son los primos tales que
    $p \equiv 2\pmod{3}$.

  \item Los primos con $\Frob_p \in C''$ corresponden a las factorizaciones
    de la forma $\mathfrak{p}_1\mathfrak{p}_2$. Su densidad es
    $\frac{1}{3}$. Estos son los primos $p \equiv 1 \pmod{3}$ tales que
    $19$ no es un cubo mód $p$.
  \end{itemize}

  La densidad de los primos $p \equiv 2\pmod{3}$ viene dada por el teorema de
  Dirichlet sobre primos en progresiones aritméticas. Sin embargo, para
  $p \equiv 1\pmod{3}$ la condición «$19$ es un cubo mód $p$» es más sofisticada
  y no está claro por qué la densidad correspondiente debe ser $\frac{1}{6}$.

  He aquí alguna estadística calculada con PARI/GP. En la primera columna está
  el número de primos y en la segunda la fracción de primos que cumplen la
  condición de que $p \equiv 1 \pmod{3}$ y $19$ es un cubo mód $p$.
  \begin{align*}
    10^2\colon & 0.140000000 \\
    10^3\colon & 0.160000000 \\
    10^4\colon & 0.163500000 \\
    10^5\colon & 0.166410000 \\
    10^6\colon & 0.166575000 \\
    10^7\colon & 0.166609100 \\
    10^8\colon & 0.166655150 \\
    10^9\colon & 0.166663355 \qedhere
  \end{align*}
\end{ejemplo}

Notamos que el Frobenius contiene más información que el tipo de descomposición
de $\mathfrak{p}$: dos elementos $\sigma,\tau\in \Gal (L/K)$ pueden tener el
mismo orden pero estar en diferentes clases de conjugación. (Por ejemplo, en el
grupo alternante $A_4$ los $3$-ciclos $(1~2~3)$ y $(1~2~4)$ no son conjugados.)

%%%%%%%%%%%%%%%%%%%%%%%%%%%%%%%%%%%%%%%%%%%%%%%%%%%%%%%%%%%%%%%%%%%%%%%%%%%%%%%%

\section{Caso de extensiones no Galois}

Aunque el Frobenius fue definido para extensiones de Galois $L/K$, en general es
posible tomar la cerradura de Galois y determinar el tipo de descomposición
mediante el Frobenius. Consideremos una torre de campos de números
\[ \begin{tikzcd}
  L\ar[-]{d}[swap]{H}\ar[-,bend left=45]{dd}{G} \\
  K\ar[-]{d} \\
  F
\end{tikzcd} \]
donde $L/F$ es una extensión de Galois con $\Gal (L/F) = G$ y $K = L^H$.
Para un primo $\mathfrak{p} \subset \O_F$ que no se ramifica en $L$, sea
$\mathfrak{Q} \subset \O_L$ un primo tal que $\mathfrak{Q} \mid \mathfrak{p}$.
El grupo de descomposición $D (\mathfrak{Q}|\mathfrak{p})$ actúa sobre
las clases laterales $H\sigma$ mediante la multiplicación por la derecha.
Puesto que $D (\mathfrak{Q}\mid\mathfrak{p})$ está generado por el Frobenius
$F = \Frob_{\mathfrak{Q}|\mathfrak{p}}$, las órbitas de esta acción vienen
dadas por
\begin{gather*}
  \{ H \sigma_1, H \sigma_1 F, H \sigma_1 F^2, \ldots, H \sigma_1 F^{n_1 - 1} \}, \\
  \dots \\
  \{ H \sigma_s, H \sigma_s F, H \sigma_s F^2, \ldots, H \sigma_s F^{n_s - 1} \},
\end{gather*}
donde $n_i$ es el mínimo número tal que $H \sigma_i F^{n_i} = H \sigma_i$.
Lo que tenemos es una partición de las clases laterales, así que
$\sum_i n_i = [G : H] = [K : F]$.

\begin{teorema}
  En la situación de arriba se tiene
  $\mathfrak{p} \O_K = \mathfrak{q}_1\cdots \mathfrak{q}_s$, donde
  $\mathfrak{q}_i = \sigma_i (\mathfrak{Q}) \cap \O_K$ y
  $f (\mathfrak{q}_i|\mathfrak{p}) = n_i$.

  \begin{proof}
    Está claro que $\mathfrak{q}_i \mid \mathfrak{p}$. Primero vamos a probar
    que $\mathfrak{q}_i \ne \mathfrak{q}_j$ para $i \ne j$. Supongamos que
    $\mathfrak{q}_i = \mathfrak{q}_j$. En este caso
    $\sigma_i (\mathfrak{Q})$ y $\sigma_j (\mathfrak{Q})$ son primos en $\O_L$
    sobre el mismo primo en $\O_K$, y luego por la transitividad de la acción
    de Galois sobre ideales primos, existe $\sigma \in H$ tal que
    $\sigma_j (\mathfrak{Q}) = \tau \sigma_i (\mathfrak{Q})$. Luego tenemos
    $\sigma_j^{-1} \tau \sigma_i \in D (\mathfrak{Q}|\mathfrak{p})$.
    El grupo de descomposición es cíclico, generado por $F$, así que
    $\sigma_j^{-1} \tau \sigma_i = F^k$ para algún $k$. Esto implica que
    $H \sigma_i = H \sigma_j F^k$ están en la misma órbita, pero luego
    $i = j$.

    Falta ver que los $\mathfrak{q}_i$ son todos los primos en $\O_K$ tales que
    $\mathfrak{q}_i \mid \mathfrak{p}$. Notamos que
    $$\sum_i n_i = \sum_{\mathfrak{q}\mid\mathfrak{p}} f (\mathfrak{q}|\mathfrak{p}) = [K:F],$$
    así que bastaría probar que para todo $i$ se tiene
    $f (\mathfrak{q}_i|\mathfrak{p}) \ge n_i$.
    Estamos en la siguiente situación:
    \[ \begin{tikzcd}
      \O_L \ar[-]{d}\ar[->>]{r} & \kappa (\sigma_i (\mathfrak{Q}))\ar[-]{d} \\
      \O_K \ar[-]{d}\ar[->>]{r} & \kappa (\mathfrak{q}_i)\ar[-]{d} \\
      \O_F \ar[->>]{r} & \kappa (\mathfrak{p})
    \end{tikzcd} \]
    y se ve que (¡ejercicio!)
    \[ \Frob_{\sigma_i (\mathfrak{Q}) | \mathfrak{q}_i} =
       (\Frob_{\sigma_i (\mathfrak{Q}) | \mathfrak{p}})^{f (\mathfrak{q}_i|\mathfrak{p})} =
       (\sigma_i\,F\,\sigma_i^{-1})^{f (\mathfrak{q}_i|\mathfrak{p})} =
       \sigma_i\,F^{f (\mathfrak{q}_i|\mathfrak{p})}\,\sigma_i^{-1}. \]
    Ahora $\Frob_{\sigma_i (\mathfrak{Q}) | \mathfrak{q}_i} \in H$, y entonces
    $\sigma_i\,F^{f (\mathfrak{q}_i|\mathfrak{p})}\,\sigma_i^{-1} \in H$.
    Esto implica que
    $H \sigma_i = H \sigma_i F^{f (\mathfrak{q}_i|\mathfrak{p})}$,
    de donde $f (\mathfrak{q}_i|\mathfrak{p}) \ge n_i$.
  \end{proof}
\end{teorema}

Como mencionamos, el teorema de Kronecker--Weber nos dice que para toda
extensión $K/\QQ$ con el grupo $\Gal (K/\QQ)$ abeliano existe $n$ tal que
$K \subseteq \QQ (\zeta_n)$. Entonces, la descomposición de primos en $K$
depende solamente del resto de $p$ módulo $n$. El caso de extensiones no
abelianas es mucho más complicado.

\begin{ejemplo}
  Consideremos el campo de números $K = \QQ (\sqrt[4]{2})$. La extensión $K/\QQ$
  no es normal, y podemos pasar a la cerradura de Galois
  $L = \QQ (\sqrt[4]{2},i)$. Dejo al lector comprobar que el grupo de Galois de
  $L/\QQ$ está generado por los automorfismos
  $\sigma\colon \sqrt[4]{2} \mapsto i\sqrt[4]{2}$ de orden $4$ y
  $\tau\colon i \mapsto -i$ de orden $2$. Se cumple
  $\sigma\tau = \tau\sigma^3 \ne \tau\sigma$. El grupo de Galois es isomorfo al
  grupo diédrico $D_4$ (también conocido como $D_8$):
  \[ \Gal (L/\QQ) = \{ 1,\sigma,\sigma^2,\sigma^3,
                       \tau,\tau\sigma,\tau\sigma^2,\tau\sigma^3 \}. \]
  Tenemos $K = L^{\langle\tau\rangle}$, y las clases laterales de
  $H = \langle\tau\rangle$ en $G$ son
  $$H, H\sigma, H\sigma^2, H\sigma^3.$$

  En este caso el único primo que se ramifica en $L$ es $p = 2$. (¡Ejercicio!)
  Par $p$ impar sea $\mathfrak{p} \subset \O_L$ un primo tal que
  $\mathfrak{p}\mid p$, y consideremos el Frobenius correspondiente
  $F = \Frob_{\mathfrak{p}|p}$. Supongamos por ejemplo que $F = \tau$.
  En este caso calculamos que $HF = H$, $H\sigma F = H\sigma^3$,
  y $H\sigma^2 F = H\sigma^2$. Esto nos da tres órbitas
  $$\{ H \}, \quad \{ H\sigma, H\sigma^3 \}, \quad \{ H\sigma^2 \}.$$
  Entonces, el teorema nos dice que
  $p\O_K = \mathfrak{p}_1\mathfrak{p}_2\mathfrak{p}_3$, donde
  $f_1 = f_2 = 1$ y $f_3 = 2$.

  Un ejercicio instructivo sería determinar el tipo de factorización para
  el resto de posibilidades para el Frobenius $F$.
\end{ejemplo}

Con esto terminamos nuestra breve investigación de la teoría de Galois para los
campos de números.

%%%%%%%%%%%%%%%%%%%%%%%%%%%%%%%%%%%%%%%%%%%%%%%%%%%%%%%%%%%%%%%%%%%%%%%%%%%%%%%%

\pagebreak

\phantomsection

\addcontentsline{toc}{section}{Ejercicios}
\section*{Ejercicios}

\begin{ejercicio}
  Para un campo cuadrático $\QQ (\sqrt{d})$ encuentre $n$ tal que
  $\QQ (\sqrt{d}) \subset \QQ (\zeta_n)$.
\end{ejercicio}

\begin{ejercicio}
  Describa los subcampos de $\QQ (\sqrt[4]{2},i)$.
\end{ejercicio}

\begin{ejercicio}
  Demuestre que para una extensión de Galois $L/K$, primos
  $\mathfrak{q} \subset \O_L$, $\mathfrak{p} \subset \O_K$,
  tales que
  $\mathfrak{q} \mid \mathfrak{p}$, y $\sigma \in \Gal (L/K)$ se tiene
  \[ D (\sigma (\mathfrak{q})|\mathfrak{p}) =
  \sigma \, D (\mathfrak{q}|\mathfrak{p}) \, \sigma^{-1}, \quad
  I (\sigma (\mathfrak{q})|\mathfrak{p}) =
  \sigma \, I (\mathfrak{q}|\mathfrak{p}) \, \sigma^{-1}. \]
  Además, si $\mathfrak{p}$ no se ramifica, entonces el Frobenius cumple
  \[ \Frob_{\sigma (\mathfrak{q})|\mathfrak{p}} =
  \sigma \, \Frob_{\mathfrak{q}|\mathfrak{p}} \, \sigma^{-1}. \]
\end{ejercicio}

\begin{ejercicio}
  Sea $F$ un campo de números, y $L/K/F$ una torre de extensiones tal que $L/K$
  es una extensión normal. Sean $\mathfrak{p} \subset \O_F$,
  $\mathfrak{q} \in \O_K$, $\mathfrak{Q} \subset \O_L$ ideales primos tales que
  $\mathfrak{Q} \mid \mathfrak{q}$ y $\mathfrak{q}\mid\mathfrak{p}$.

  \begin{enumerate}
  \item[1)] Demuestre que la restricción de automorfismos identifica
    $D (\mathfrak{Q}|\mathfrak{q})$ con un subgrupo de
    $D (\mathfrak{Q}|\mathfrak{p})$ e
    $I (\mathfrak{Q}|\mathfrak{q})$ con un subgrupo de
    $I (\mathfrak{Q}|\mathfrak{p})$.

  \item[2)] Si $\mathfrak{p}$ no se ramifica en $L$, demuestre que
    $$\Frob_{\mathfrak{Q}|\mathfrak{q}} = (\Frob_{\mathfrak{Q}|\mathfrak{p}})^{f (\mathfrak{q}|\mathfrak{p})}.$$

  \item[3)] Si la extensión $K/F$ es normal, demuestre que
    $\Frob_{\mathfrak{q}|\mathfrak{p}}$ es la restricción de
    $\Frob_{\mathfrak{Q}|\mathfrak{p}}$.
  \end{enumerate}
\end{ejercicio}

\begin{ejercicio}
  Sea $K$ el campo de descomposición del polinomio $f = x^4 + 8x + 12$. Calcule
  $\Gal (K/\QQ)$, las clases de conjugación, los tipos de descomposición que
  corresponden a cada $\Frob_{\mathfrak{p}\mid p}$, y las densidades que nos da
  el teorema de Chebotarëv.
\end{ejercicio}

\begin{ejercicio}
  Para $K = \QQ (\sqrt[4]{2})$ consideremos la cerradura de Galois
  $L = \QQ (\sqrt[4]{2},i)$.

  \begin{enumerate}
  \item[1)] Demuestre que el único primo racional $p$ que se ramifica en $L$ es
    $p = 2$.

  \item[2)] Para $p$ impar sea $\mathfrak{p} \subset \O_L$ un primo tal que
    $\mathfrak{p} \mid p$. Determine cómo el tipo de factorización de $p$ en
    $\O_K$ para toda posibilidad para $\Frob_{\mathfrak{p}|p}$.
  \end{enumerate}
\end{ejercicio}

\begin{ejercicio}
  Para la extensión ciclotómica $L = \QQ (\zeta_n)$ determine cómo los primos
  no ramificados $p\nmid n$ se descomponen en el subcampo
  $K = \QQ (\zeta_n + \zeta_n^{-1})$.
\end{ejercicio}

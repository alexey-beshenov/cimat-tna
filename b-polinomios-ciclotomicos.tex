\chapter{Polinomios y campos ciclotómicos}
\label{ap:polinomios-ciclotomicos}

En este apéndice vamos a revisar brevemente los polinomios ciclotómicos $\Phi_n$
y probar su irreducibilidad.

\section{Definición y propiedades básicas}

\begin{definicion}
  Consideremos las raíces $n$-ésimas de la unidad
  $$1, \zeta_n, \zeta_n^2, \ldots, \zeta_n^{n-1} \in \CC,$$
  donde $\zeta_n = \exp (2\pi i/n)$. Se dice que $\zeta_n^a$ es una raíz
  \term{primitiva} si $\gcd (a,n) = 1$.
\end{definicion}

El número de las raíces $n$-ésimas primitivas coincide entonces con la función
de Euler $\phi (n)$.

\begin{lema}
  \label{lema:descomposicion-en-raices-primitivas}
  Para todo $a = 0,1,\ldots,n-1$ el número $\zeta_n^a$ es una raíz $d$-ésima
  primitiva para algún $d \mid n$. Este $d$ está definido de modo único.

  Si $\zeta$ es una raíz $d$-ésima primitiva, entonces, todas las raíces
  $d$-ésimas primitivas son de la forma $\zeta^a$ para $\gcd (a,d) = 1$.

  \begin{proof}
    Teoría de números elemental. Si denotamos por $\mathcal{S}_d$ el conjunto de
    las raíces primitivas de orden $d$, entonces
    \[ \{ 1, \zeta_n, \zeta_n^2, \ldots, \zeta_n^{n-1} \} =
        \bigcup_{d\mid n} \mathcal{S}_d \]
    es una partición gracias a la identidad $\sum_{d\mid n} \phi (n) = n$.
    La última afirmación también está clara.
  \end{proof}
\end{lema}

\begin{figure}
  \begin{center}
    \includegraphics{pic/sixth-roots.pdf}
  \end{center}

  \caption{Raíces sextas de la unidad}
\end{figure}

\begin{definicion}
  El $n$-ésimo \term{polinomio ciclotómico}\footnote{La palabra «ciclotomia»
    significa «división del círculo» en griego y se refiere al hecho de que
    las $n$-ésimas raíces de la unidad son vértices de un $n$-ágono regular
    inscrito en el circulo unitario.} es el polinomio mónico que tiene como sus
  raíces las raíces $n$-ésimas primitivas de la unidad:
  $$\Phi_n = \prod_{\substack{0 \le a < n \\ \gcd (a,n) = 1}} (x - \zeta_n^a).$$
\end{definicion}

Esta claro que $\Phi_n$ es un poliomio mónico de grado $\phi (n)$.

\begin{ejemplo}
  Los primeros polinomios ciclotómicos son
  \begin{align*}
    \Phi_1 & = x - 1,\\
    \Phi_2 & = x + 1,\\
    \Phi_3 & = (x - \zeta_3)\,(x - \zeta_3^2) = x^2 + x + 1,\\
    \Phi_4 & = (x - \zeta_4)\,(x - \zeta_4^3) = x^2 + 1. \qedhere
  \end{align*}
\end{ejemplo}

Lo que no está tan claro de la definición es que los coeficientes de $\Phi_n$
son números enteros.

\begin{proposicion}
~

  \begin{enumerate}
  \item[1)] Para todo primo $p$ se tiene
    $$\Phi_p = \frac{x^p-1}{x-1} = x^{p-1} + x^{p-2} + \cdots + x^2 + x + 1.$$

  \item[2)] Para todo primo $p$ y $k \ge 1$ se tiene
    \[ \Phi_{p^k} = \frac{x^{p^k} - 1}{x^{p^{k-1}} - 1} =
       \Phi_p (x^{p^{k-1}}) =
       x^{(p-1)\,p^{k-1}} + x^{(p-2)\,p^{k-1}} + \cdots + x^{2p^{k-1}} + x^{p^{k-1}} + 1. \]

  \item[3)] Para todo $n$ se tiene
    $$\prod_{d\mid n} \Phi_d = x^n - 1.$$

  \item[4)] Todos los polinomios $\Phi_n$ tienen coeficientes enteros.
  \end{enumerate}

  \begin{proof}
    Observamos que
    $$\prod_{0 \le a < n} (x - \zeta_n^a) = x^n - 1.$$

    En la parte 1), basta notar que entre las raíces $p$-ésimas, todas son
    primitivas, salvo la raíz trivial $1$, así que
    \[ \Phi_p = \prod_{1 \le a < p} (x - \zeta_p^a) =
       \left.\prod_{0 \le a < p} (x - \zeta_p^a)\right/(x-1) =
       \frac{x^p-1}{x-1}. \]

    De la misma manera, en 2) notamos que un número $0 \le a < p^k$ tal que
    $\gcd (a, p^k) \ne 1$ es necesariamente divisible por $p$, así que las
    raíces de orden $p^k$ que no son primitivas tienen forma
    $\zeta_{p^k}^{pb} = \zeta_{p^{k-1}}^b$ y son precisamente todas las
    raíces de orden $p^{k-1}$:
    \[ \Phi_{p^k} =
       \prod_{\substack{0 \le a < p^k \\ \gcd (a,p^k) = 1}} (x - \zeta_{p^k}^a) =
       \left.\prod_{0 \le a < p^k} (x - \zeta_{p^k}^a)\right/\prod_{0 \le b < p^{k-1}} (x - \zeta_{p^{k-1}}^b) =
       \frac{x^{p^k} - 1}{x^{p^{k-1}}-1}. \]

    En la parte 3), basta notar que
    \[ x^n - 1 = \prod_{0\le a < n} (x - \zeta_n^a) = \prod_{d\mid n}
       \prod_{\substack{0 \le a < n \\ \gcd (a,d) = 1}} (x - \zeta_n^a) =
       \prod_{d\mid n} \Phi_d, \]
    usando la observación que hicimos en
    \ref{lema:descomposicion-en-raices-primitivas}.

    La parte 4) se demuestra por inducción sobre $n$. Esto es cierto, por
    ejemplo, para $\Phi_1 = x-1$. Luego, si $\Phi_m \in \ZZ [x]$ para todo
    $m < n$, entonces podemos considerar el polinomio
    $$g = \prod_{\substack{d\mid n \\ d\ne n}} \Phi_d \in \ZZ [x].$$
    Este es mónico, siendo un producto de polinomios mónicos. La división con
    resto en el anillo $\ZZ [x]$ nos da
    $$x^n - 1 = q\,g + r, \quad \deg (r) < \deg (g)$$
    para algunos $q, r \in \ZZ [x]$, mientras que en el anillo más grande
    $\QQ [x] \supset \ZZ [x]$ se cumple
    $$x^n - 1 = \Phi_n\,g.$$
    Pero para la división con resto en $\QQ [x]$ el cociente y el resto están
    definidos de modo único, así que $r = 0$ y $\Phi_n = q \in \ZZ[x]$.
  \end{proof}
\end{proposicion}

\begin{ejemplo}
  Tenemos
  \begin{align*}
    \Phi_4 & = \Phi_2 (x^2) = x^2 + 1,\\
    \Phi_5 & = x^4 + x^3 + x^2 + x + 1,\\
    \Phi_6 & = \frac{x^6 - 1}{\Phi_1\,\Phi_2\,\Phi_3} = \frac{(x^3-1)\,(x^3+1)}{\Phi_1\,\Phi_2\,\Phi_3} = \frac{x^3+1}{\Phi_2} = \frac{x^3+1}{x+1} = x^2 - x + 1,\\
    \Phi_7 & = x^6 + x^5 + x^4 + x^3 + x^2 + x + 1,\\
    \Phi_8 & = \Phi_2 (x^4) = x^4 + 1,\\
    \Phi_9 & = \Phi_3 (x^2) = x^6 + x^3 + 1,\\
    \Phi_{10} & = \frac{x^{10}-1}{\Phi_1\,\Phi_2\,\Phi_5} = \frac{(x^5+1)\,(x^5-1)}{(x^5-1)\,\Phi_2} = \frac{x^5+1}{x+1} = x^4 - x^3 + x^2 - x + 1.
  \end{align*}

  Notamos que
  \begin{gather*}
    \Phi_3 = x^2 + x + 1, \quad \Phi_6 = x^2 - x + 1 = \Phi_3 (-x),\\
    \Phi_5 = x^4 + x^3 + x^2 + x + 1, \quad \Phi_{10} = x^4 - x^3 + x^2 - x + 1 = \Phi_5 (-x).
  \end{gather*}
  Esta no es una coincidencia: en general, $\Phi_{2m} = \Phi_m (-x)$ para todo
  $m > 1$ impar (este es un buen ejercicio para el lector).
\end{ejemplo}

El resultado de arriba nos permite calcular de manera bastante eficaz los
polinomios ciclotómicos. En PARI/GP la función \texttt{polcylo($n$)} devuelve
$\Phi_n$.

\begin{shaded}
\begin{verbatim}
? polcyclo(105)
% = x^48 + x^47 + x^46 - x^43 - x^42 - 2*x^41 - x^40 - x^39 + x^36
    + x^35 + x^34 + x^33 + x^32 + x^31 - x^28 - x^26 - x^24 - x^22
    - x^20 + x^17 + x^16 + x^15 + x^14 + x^13 + x^12 - x^9 - x^8 - 2*x^7
    - x^6 - x^5 + x^2 + x + 1

? polcyclo(1)*polcyclo(3)*polcyclo(5)*polcyclo(15)
% = x^15 - 1
\end{verbatim}
\end{shaded}

\begin{comentario}
  Una prueba mucho más lista de que $\Phi_n \in \ZZ [x]$ es la siguiente.
  El grupo de Galois $\Gal (\QQ (\zeta_n)/\QQ)$ es isomorfo a
  $(\ZZ/n\ZZ)^\times$ y consiste en automorfismos
  $\sigma\colon \zeta_n \mapsto \zeta_n^a$ con $\gcd (a,n) = 1$. Se ve que cada
  $\sigma$ deja fijos los coeficientes de $\Phi_n$, y la teoría de Galois nos
  dice entonces que los coeficientes están en $\QQ \cap \ZZ [\zeta_n] = \ZZ$.

  Sin embargo, este argumento me parece un poco tramposo: uno que sabe calcular
  $\Gal (\QQ (\zeta_n)/\QQ)$ normalmente también sabrá manejar los polinomios
  ciclotómicos\dots
\end{comentario}

%%%%%%%%%%%%%%%%%%%%%%%%%%%%%%%%%%%%%%%%%%%%%%%%%%%%%%%%%%%%%%%%%%%%%%%%%%%%%%%%

\section{Irreducibilidad}

El objetivo de esta sección es probar que los polinomios ciclotómicos $\Phi_n$
son irreducibles en $\ZZ [x]$ (y luego en $\QQ [x]$ gracias al lema de
Gauss). Para tratar primero el caso de $n = p^k$ donde $p$ es primo, recordemos
el siguiente criterio de irreducibilidad.

\begin{proposicion}[Eisenstein]
  Sea
  $$f = x^n + a_{n-1} x^{n-1} + \cdots + a_1 x + a_0 \in \ZZ [x]$$
  un polinomio mónico con coeficientes enteros. Supongamos que existe un primo
  $p$ tal que $p \mid a_i$ para todo $i = 0,1,\ldots,n-1$, pero $p^2 \nmid
  a_0$. Entonces, $f$ es irreducible.
\end{proposicion}

La prueba es muy breve, así que será más fácil recordarla por completo que citar
un libro de texto.

\begin{proof}
  Si $f$ es reducible, entonces $f = gh$, donde $1 \le \deg (g), \deg (h) < n$.
  Reduciendo módulo $p$, se obtiene la identidad
  \[ \overline{x^n} = \overline{f} =
     \overline{g}\,\overline{h}\quad\text{en }\FF_p [x] \]
  por la hipótesis sobre los coeficientes de $f$. Esto implica que
  $$\overline{g} = c\,x^k, \quad \overline{h} = c^{-1}\,x^\ell,$$
  para algún $c\in \FF_p^\times$ y $k + \ell = n$, donde $k \le \deg (g)$ y $\ell
  \le \deg (h)$, así que $k,\ell > 0$.

  Sin embargo, si ambos $g$ y $h$ se reducen a un polinomio sin término
  constante, esto significa que los términos constantes de $g$ y $h$ son
  divisibles por $p$. Esto implicaría que el término constante de $f$ es
  divisible por $p^2$, pero no es el caso por nuestra hipótesis.
\end{proof}

Una aplicación típica del criterio de Eisenstein es la irreducibilidad de los
polinomios ciclotómicos $\Phi_{p^k}$. Notamos que el término constante de
cualquier polinomio ciclotómico viene dado por
\[ \Phi_n (0) =
   (-1)^{\phi (n)} \prod_{\substack{0 \le a < n \\ \gcd (a,n) = 1}} \zeta_n^a. \]

Excluyendo el caso excepcional de $\Phi_1 = x-1$ y $\Phi_2 = x+1$, notamos que
para $n > 2$ el número $\phi (n)$ es par y cada $\zeta_n^a$ en el producto se
cancela con su inverso $\zeta_n^{-a}$, así que $\Phi_n (0) = 1$.

Esto significa que el criterio de Eisenstein nunca se aplica directamente a
$\Phi_n$, pero para $n = p^k$ funciona la sustitución de $x+1$ en lugar de
$x$. Por ejemplo,
$$\Phi_8 (x+1) = (x+1)^4 + 1 = x^4 + 4x^3 + 6x^2 + 4x + 2,$$
y el criterio de Eisenstein sí funciona para $p = 2$.

Notamos que un polinomio no constante $f \in \ZZ [x]$ es irreducible si y solo
si $f (x + a)$ es irreducible para algún $a \in \ZZ$: esta sustitución no cambia
el grado y una factorización no trivial $f (x + a) = g (x)\,h(x)$ corresponde a
una factorización no trivial $f (x) = g (x-a)\,h (x-a)$.

\begin{proposicion}
  Para todo primo $p$ el polinomio $\Phi_p$ es irreducible.

  \begin{proof}
    Al sustituir $x+1$ en lugar de $x$, nos salen los coeficientes
    $a_i = {p\choose i}$, y a estos se aplica el criterio de Eisenstein:
    \begin{multline*}
      \Phi_p (x+1) = \frac{(x+1)^p - 1}{(x+1) - 1} = \frac{1}{x}\,\sum_{1 \le i \le p} {p\choose i} x^i \\
      = {p\choose p}\,x^{p-1} + {p\choose p-1}\,x^{p-2} + \cdots + {p\choose 3}\,x^2 + {p\choose 2}\,x + {p \choose 1}. \qedhere
    \end{multline*}
  \end{proof}
\end{proposicion}

\begin{proposicion}
  Para todo primo $p$ y $k \ge 1$ el polinomio $\Phi_{p^k}$ es irreducible.

  \begin{proof}
    Ya vimos el caso de $k = 1$. Podemos asumir entonces que $k \ge 2$. De
    nuevo, consideremos la sustitución
    \[ \Phi_{p^k} (x+1) =
       \frac{(x+1)^{p^k} - 1}{(x+1)^{p^{k-1}} - 1} =
       \sum_{0 \le i \le p-1} (x+1)^{i\,p^{k-1}}. \]
    Tenemos para todo $k \ge 2$
    $$(x+1)^{p^{k-1}} \equiv x^{p^{k-1}} + 1 \pmod{p},$$
    y luego
    \begin{multline*}
      \Phi_{p^k} (x+1) \equiv \sum_{0 \le i \le p-1} (x^{p^{k-1}}+1)^i =
      \frac{(x^{p^{k-1}}+1)^p - 1}{(x^{p^{k-1}}+1) - 1} \\
      = \frac{(x^{p^{k-1}}+1)^p - 1}{x^{p^{k-1}}} \equiv \frac{x^{p^k}}{x^{p^{k-1}}} =
      x^{p^{k-1}\,(p-1)} \pmod{p}.
    \end{multline*}
    Esto significa que todos los coeficientes menores de $\Phi_{p^k} (x+1)$ son
    divisibles por $p$. El coeficiente constante es igual a
    $$\Phi_{p^k} (1) = \Phi_p (1^{p^{k-1}}) = \Phi_p (1) = p,$$
    y de nuevo podemos aplicar el criterio de Eisenstein.
  \end{proof}
\end{proposicion}

La irreducibilidad de $\Phi_n$ para todo $n$ es un resultado más difícil.

\begin{teorema}[Gauss]
  Para cualquier $n = 1,2,3,\ldots$ el polinomio ciclotómico $\Phi_n$ es
  irreducible.
\end{teorema}

Un poco de la historia: el 1808 Gauss anotó en su diario matemático que había
establecido la irreducibilidad de $\Phi_n$ para cualquier $n$. Su argumento
original se considera perdido, pero ya que se trata de Gauss, es muy probable
que él disponía de una prueba correcta y completa. Sin embargo, la primera
demostración publicada pertenece a Kronecker (1854).

\vspace{1em}

Empecemos por un pequeño lema.

\begin{lema}
  \label{lema:factor-repetido-en-xn-1}
  Si $p \nmid n$, entonces en la factorización del polinomio ciclotómico
  $\Phi_n$ en $\FF_p [x]$ no hay factores repetidos.

  \begin{proof}
    Gracias a la fórmula $x^n - 1 = \prod_{d\mid n} \Phi_n$, sería suficiente
    probar que en la factorización de $f = x^n - 1$ en $\FF_p [x]$ no hay
    factores repetidos. Para esto basta calcular que
    $\gcd (f,f') = \gcd (x^n-1, n x^{n-1}) = 1$. Por ejemplo, usando $p\nmid n$,
    podemos escribir laa identidad de Bézout
    \[ \frac{x}{n}\cdot (n\,x^{n-1}) - (x^n - 1) = 1. \qedhere \]
  \end{proof}
\end{lema}

\begin{proof}[Demostración del teorema]
  Escribamos
  $$\Phi_n = f\,g$$
  para algunos polinomios $f,g\in \ZZ[x]$, donde $f$ es irreducible. Dado que
  $\Phi_n$ es mónico, el coeficiente mayor de $f$ y $g$ es $\pm 1$, y podemos
  asumir que son también mónicos. Sea $\zeta$ una raíz $n$-ésima
  primitiva. Tenemos entonces
  $$\Phi_n (\zeta) = f (\zeta)\,g (\zeta) = 0.$$
  Esto implica que $f (\zeta) = 0$ o $g (\zeta) = 0$. Puesto que $f$ no es
  constante, alguna raíz $n$-ésima primitiva $\zeta$ debe ser una raíz de $f$, y
  nuestro objetivo es probar que todas las raíces primitivas
  $$\zeta^a, \quad \gcd (a,n) = 1$$
  son raíces de $f$.

  \vspace{1em}

  Asumamos entonces que $f (\zeta) = 0$. Por la irreducibilidad de $f$, esto
  significa que $f$ es el polinomio mínimo de $\zeta$. Sea $p$ un número primo
  tal que $p \nmid n$. Entonces, $\zeta^p$ es también una raíz $n$-ésima
  primitiva y
  $$\Phi_n (\zeta^p) = f (\zeta^p)\,g (\zeta^p) = 0.$$
  Asumamos que $g (\zeta^p) = 0$. En este caso $f \mid g (x^p)$ en $\ZZ [x]$, ya
  que $f$ es el polinomio mínimo de $\zeta$. Reduciendo módulo $p$, se
  obtiene\footnote{Usando el teorema del binomio en característica $p$ y $a^p =
    a$ en $\FF_p$, notamos que
    $$g(x)^p = (a_n\,x^n + a_{n-1}\,x^{n-1} + \cdots + a_1\,x + a_0)^p = a_n\,(x^p)^n + a_{n-1}\,(x^p)^{n-1} + \cdots + a_1\,x^p + a_0 = g (x^p).$$}
  $\overline{f} \mid \overline{g} (x^p) = \overline{g}^p$ en $\FF_p [x]$. Pero
  esto implica que $\overline{\Phi_n} = \overline{f}\,\overline{g}$ tiene un
  factor repetido en su factorización en $\FF_p [x]$, lo que contradice el lema
  \ref{lema:factor-repetido-en-xn-1}. Entonces, $f (\zeta^p) = 0$.

  Esto demuestra que para cualquier primo $p$ tal que $p\nmid n$ se tiene
  $$f (\zeta) = 0 \Longrightarrow f (\zeta^p) = 0.$$
  Ahora todas las raíces $n$-ésimas primitivas son de la forma $\zeta^a$ donde
  $\gcd (a,n) = 1$. Podemos factorizar entonces $a = p_1\cdots p_s$ donde $p_i$
  son primos (no necesariamente diferentes) tales que $p_i\nmid n$, y luego
  $$\zeta^a = (((\zeta^{p_1})^{p_2})^{\cdots})^{p_s}.$$
  El argumento de arriba nos dice que $f (\zeta^{p_1}) = 0$. Luego, el mismo
  argumento aplicado a $\zeta^{p_1}$ demuestra que
  $f ((\zeta^{p_1})^{p_2}) = 0$, etcétera, y en fin $f (\zeta^a) = 0$. Entonces,
  todas las raíces $n$-ésimas primitivas son raíces de $f$ y por ende $g = 1$.
\end{proof}

\begin{comentario}
  La idea principal del argumento de arriba es probar que \emph{si para un
    polinomio irreducible $f \in \ZZ [x]$ se tiene $f (\zeta) = 0$ para una raíz
    $n$-ésima primitiva $\zeta$, entonces $f (\zeta^a) = 0$ para todo $a$ tal
    que $\gcd (a,n) = 1$}.

  He aquí otro modo de establecerlo. Primero notamos que para cualquier primo
  $p$ se tiene $f (\zeta^p) \equiv f (\zeta)^p \pmod{p}$ en el anillo $\ZZ
  [\zeta_n]$. Ahora, el \term{teorema de Dirichlet sobre primos en progresiones
    aritméticas} (!) afirma que para todo $a$ tal que $\gcd (a,n) = 1$ hay un
  número infinito de primos $p$ que cumplen $p \equiv a \pmod{n}$. Entonces,
  $$p \mid f (\zeta^p) = f (\zeta^a) \quad \text{en }\ZZ [\zeta_n]$$
  para un número infinito de $p$, lo que implica que $f (\zeta^a) = 0$.

  Sin embargo, no es tan fácil demostrar el teorema de Dirichlet: la prueba se
  basa en la teoría analítica de números; el lector interesado puede consultar
  \cite[Chapter~16]{Ireland-Rosen} o \cite[Chapter~7]{Apostol-analytic}.
\end{comentario}

%%%%%%%%%%%%%%%%%%%%%%%%%%%%%%%%%%%%%%%%%%%%%%%%%%%%%%%%%%%%%%%%%%%%%%%%%%%%%%%%

\section{Campos ciclotómicos}
\label{sec:campos-ciclotomicos}

\begin{definicion}
  El $n$-ésimo campo ciclotómico es el campo $\QQ (\zeta_n)$.
\end{definicion}

De la irreducibilidad de $\Phi_n$ se sigue que $\Phi_n$ es el polinomo mínimo de
$\zeta_n$ sobre $\QQ$, y por ende $\QQ (\zeta_n) \cong \QQ [x] / (\Phi_n)$, y en
particular $[\QQ (\zeta_n) : \QQ] = \deg (\Phi_n) = \phi (n)$. El campo
ciclotómico es el campo de descomposición de $\Phi_n$, así que
$\QQ (\zeta_n)/\QQ$ es una extensión de Galois.

\begin{proposicion}
  El grupo de Galois $\Gal (\QQ(\zeta_n)/\QQ)$ consiste en automorfismos
  $\sigma_k\colon \zeta_n \mapsto \zeta_n^k$, donde $\gcd (k,n) = 1$.
  Hay isomorfismo $\Gal (\QQ (\zeta_n)/\QQ) \cong (\ZZ/n\ZZ)^\times$.

  \begin{proof}
    Un automorfismo $\sigma\colon \QQ(\zeta_n) \to \QQ(\zeta_n)$ debe mandar
    $\zeta_n$ a otra raíz de $\Phi_n$, y de allí surgen todos los automorfismos
    $\sigma_k$. La aplicación $\sigma_k \mapsto k\operatorname{~mod~}n$
    establece el isomorfismo $\Gal (\QQ (\zeta_n)/\QQ) \cong (\ZZ/n\ZZ)^\times$.
  \end{proof}
\end{proposicion}

Notamos que si $m \mid n$, entonces $\zeta_m = \zeta_n^{n/m} \in \QQ (\zeta_n)$,
y luego $\QQ (\zeta_m) \subseteq \QQ (\zeta_n)$.

Ahora si $m$ es un número impar, entonces $\QQ (\zeta_{2m}) = \QQ (\zeta_m)$.
De hecho, tenemos la inclusión obvia $\zeta_m \in \QQ (\zeta_{2m})$, y por otro
lado, escribiendo $m = 2k + 1$,
\[ \zeta_{2m} = \zeta_{2m}^{(2k+1) - 2k}
             = \zeta_{2m}^{m}\,(\zeta_{2m}^2)^{-k}
             = \zeta_2\,\zeta_m^{-k}
             = -\zeta_m^{-k} \in \QQ (\zeta_m). \]

Esta propiedad se cumle por la razón banal de que
$\zeta_2 = -1 \in \QQ$. Resulta que en otras situaciones los campos ciclotómicos
no coinciden. Para probarlo, podemos investigar cuáles raíces de la unidad están
en $\QQ (\zeta_m)$.

\begin{lema}
  \label{lema:phi-de-Euler}
  Si $m$ es par y $m\mid r$, entonces $\phi (r) \le \phi (m)$ implica $r = m$.

  \begin{proof}
    Primero, notamos que para cualesquiera $a,m \ge 1$ se cumple
    $$\phi (am) = \frac{\phi (a)\,\phi (m)\,\gcd (a,m)}{\phi (\gcd (a,m))}$$
    ---esto se sigue de las fórmulas
    \begin{align*}
      \phi (a) & = a\,\prod_{p\mid a} \left(1 - \frac{1}{p}\right),\\
      \phi (m) & = m\,\prod_{p\mid m} \left(1 - \frac{1}{p}\right),\\
      \phi (am) & = am\,\prod_{p\mid am} \left(1 - \frac{1}{p}\right),\\
      \phi (\gcd (a,m)) & = \gcd (a,m)\,\prod_{p \mid a, p \mid m} \left(1 - \frac{1}{p}\right).
    \end{align*}
    (Notamos que cuando $a$ y $m$ son coprimos, se tiene
    $\gcd (a,m) = \phi (\gcd (a,m)) = 1$ y se recupera la fórmula conocida.)
    Ahora para $m$ par y $m\mid r$, asumamos que $m < r$, así que $r = am$ para
    algún $a > 1$. Tenemos
    \[ \phi (r) = \phi (am) =
       \frac{\phi (a)\,\phi(m)\,\gcd (a,m)}{\phi (\gcd (a,m))}. \]
    Si $a = 2$, entonces $\phi (a) = \phi (2) = 1$ y $\gcd (a,m) = 2$. Luego,
    \[ \frac{\phi (a)\,\phi(m)\,\gcd (a,m)}{\phi (\gcd (a,m))} =
       2\,\phi(m) > \phi (m). \]
    Si $a > 2$, entonces $\phi (a) \ge 2$, y luego
    \[ \frac{\phi (a)\,\phi(m)\,\gcd (a,m)}{\phi (\gcd (a,m))} \ge
       \phi (a)\,\phi(m) > \phi (m). \]
    En ambos casos, $m < r$ implica $\phi (m) < \phi (r)$.
  \end{proof}
\end{lema}

\begin{proposicion}
  Las raíces de la unidad en el campo $\QQ (\zeta_m)$ son precisamente
  \[ \mu_\infty (\CC) \cap \QQ (\zeta_m)^\times = \begin{cases}
    \mu_m (\CC), & \text{si }m\text{ es par},\\
    \mu_{2m} (\CC), & \text{si }m\text{ es impar}.
  \end{cases} \]

  \begin{proof}
    Si $m = 2k+1$ es un número impar, entonces ya notamos que
    $\QQ (\zeta_m) = \QQ (\zeta_{2m})$. Por esto sería suficiente considerar
    el caso cuando $m$ es un número par.

    Tenemos $\zeta_m \in \QQ (\zeta_m)$, y por ende todas las raíces $m$-ésimas
    de la unidad, siendo potencias de $\zeta_m$, están en $\QQ (\zeta_m)$:
    $$\mu_m (\CC) \subseteq \mu_\infty (\CC) \cap \QQ (\zeta_m)^\times.$$
    Hay que ver que en $\QQ (\zeta_m)$ no hay raíces de la unidad de orden
    $k\nmid m$. Bastaría considerar las raíces $k$-ésimas primitivas.

    Supongamos que $\zeta_k^\ell \in \QQ (\zeta_m)$ donde $\zeta_k^\ell$ es una
    raíz $k$-ésima primitiva; es decir, $\gcd (k,\ell) = 1$. Pongamos
    $$r = \lcm (k,m) = \frac{km}{d}, \quad d = \gcd (k,m).$$
    Luego,
    $$\gcd (k,\ell m) = \gcd (k,m) = d,$$
    lo que significa que existen $a,b \in \ZZ$ tales que
    $$d = a k + b\ell m.$$
    Ahora,
    \[ \zeta_r = \zeta_{km}^d
               = \zeta_{km}^{ak + b\ell m}
               = \zeta_{km}^{ak}\,\zeta_{km}^{b\ell m}
               = \zeta_m^a\,(\zeta_k^\ell)^b \in \QQ (\zeta_m) \]
    y
    $$\phi (r) \le \phi (m), \quad m\text{ es par}, \quad m \mid r,$$
    así que el lema \ref{lema:phi-de-Euler} nos permite concluir que
    $$r = \gcd (k,m) = m,$$
    lo que significa que $k \mid m$.
  \end{proof}
\end{proposicion}

\begin{corolario}
  Si $\QQ (\zeta_m) = \QQ (\zeta_n)$ para $m < n$, entonces $m$ es impar y
  $n = 2m$.

  \begin{proof}
    Si $m$ es par, entonces las raíces de la unidad en $\QQ (\zeta_m)$ son de
    orden $m$, mientras que las raíces de la unidad en $\QQ (\zeta_n)$ son de
    orden $n$ o $2n$, dependiendo de la paridad de $n$. Pero en ambos casos la
    hipótesis $m < n$ nos lleva a una contradicción.

    Entonces, $m$ es impar y las raíces de la unidad en $\QQ (\zeta_m)$ son de
    orden $m$. La única posibilidad es $n = 2m$.
  \end{proof}
\end{corolario}

Entonces, para enumerar los campos ciclotómicos sin redundancias, basta
considerar $\QQ (\zeta_n)$ donde $n \not\equiv 2\pmod{4}$.

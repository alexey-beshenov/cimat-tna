\chapter{Teoría de Minkowski}

\pdfbookmark{Clase 19 (19/10/20)}{clase-19}
\marginpar{\small Clase 19 \\ 19/10/20}

El objetivo de este capítulo es probar los siguientes teoremas.

\begin{enumerate}
\item[1)] Para una extensión no trivial $K/\QQ$ se tiene $|\Delta_K| > 1$.

\item[2)] \textbf{Teorema de Hermite}: para $C$ fijo, existe un número finito de
  campos de números $K/\QQ$, salvo isomorfismo, con discriminante
  $|\Delta_K| < C$.

\item[3)] Dado un campo de números $K/\QQ$, el grupo de clases
  $\Cl (K) = \Pic (\O_K)$ es finito.

  Este resultado es bastante sutil y no se sigue del álgebra conmutativa:
  si en lugar de $\O_K$ se toma otro dominio de Dedekind $R$, el grupo
  $\Pic (R)$ ya no tiene por qué ser finito.

\item[4)] \textbf{Teorema de unidades de Dirichlet}: el grupo $\O_K^\times$ es
  finitamente generado de rango $r_1 + r_2 - 1$, donde $r_1$ es el número de
  encajes reales $K\hookrightarrow \RR$ y $2r_2$ es el número de encajes
  complejos $K\hookrightarrow \CC$. En otras palabras, existen unidades
  $\epsilon_1, \ldots, \epsilon_{r_1 + r_2 - 1} \in \O_K^\times$,
  llamadas \textbf{unidades fundamentales}, tales que
  $$\O_K^\times \cong \mu_K \times \langle\epsilon_1\rangle \times \cdots \times \langle\epsilon_{r_1+r_2-1}\rangle.$$
  Aquí $\mu_K$ es el subgrupo de torsión que consiste en las raíces de unidad
  en $K$, mientras que $\epsilon_i$ son generadores de diferentes componentes
  cíclicas infinitas.
\end{enumerate}

La herramienta principal en las pruebas será el teorema de Minkowski sobre
puntos de retículos en conjuntos convexos simétricos. El término clásico para
los resultados de Minkowski es «geometría de números»
(\emph{Geometrie der Zahlen}), pero hoy en día el punto de vista geométrico a
la teoría de números está contenido más bien en la teoría de esquemas (véase por
ejemplo \cite{Eisenbud-Harris} y \cite{Gortz-Wedhorn}). Por esto un nombre más
adecuado para este capítulo sería «teoría de Minkowski».

%%%%%%%%%%%%%%%%%%%%%%%%%%%%%%%%%%%%%%%%%%%%%%%%%%%%%%%%%%%%%%%%%%%%%%%%%%%%%%%%

\section{Retículos y el teorema de Minkowski}

\begin{definicion}
  Sea $V$ un espacio vectorial real.
  Un \textbf{retículo}\footnote{\emph{lattice} en inglés; no confundir con los
    retículos que estudian lógicos y «algebristas universales».}
  en $V$ es un subgrupo aditivo $\Lambda \subset V$ de la forma
  $$\Lambda = \ZZ\,\omega_1 + \cdots + \ZZ\,\omega_n,$$
  donde $\omega_1,\ldots,\omega_n \in V$ son vectores linealmente
  independientes. En este caso se dice que $\omega_1,\ldots,\omega_n$ es una
  \textbf{base} de $\Lambda$. Si $n = \dim_\RR V$, se dice que $\Lambda$
  \textbf{tiene rango completo} en $V$. El conjunto
  $$\Pi = \Bigl\{ \sum_i \lambda_i\,\omega_i \Bigm| 0 \le \lambda_i < 1 \Bigr\}$$
  se llama un \textbf{dominio fundamental} de $\Lambda$.
\end{definicion}

\begin{comentario}
  Si $\Lambda$ tiene rango completo, entonces $V$ puede ser recubierto por el
  dominio fundamental $\Pi$ trasladado por los elementos de $\Lambda$:
  $$V = \bigsqcup_{\omega \in \Lambda} \Pi + \omega.$$
  Esta unión es disjunta. El dominio fundamental se identifica con el cociente
  $V/\Lambda$.
\end{comentario}

\begin{comentario}
  Vamos a considerar $V$ como un espacio vectorial topológico, dotado de la
  topología real estándar. Para las aplicaciones que nos interesan, se puede
  pensar que $V = \RR^n$.
\end{comentario}

\begin{ejemplo}
  Consideremos los enteros de Eisenstein
  $\ZZ [\zeta_3] \subset \CC$. Identificando $\CC$ con $\RR^2$ de la manera
  habitual, podemos ver $\ZZ [\zeta_3]$ como un retículo generado por los
  vectores $\omega_1 = (1,0)$ y
  $\omega_2 = \left(-\frac{1}{2}, \, \frac{\sqrt{3}}{2}\right)$.

  \begin{center}
    \includegraphics{pic/eisenstein-integers-lattice.pdf}
  \end{center}
\end{ejemplo}

\begin{ejemplo}
  El subgrupo $\Lambda = \ZZ\cdot 1 + \ZZ\cdot \sqrt{2}$ no es un retículo en
  $\RR$: los vectores $1$ y $\sqrt{2}$ no son linealmente independientes.
\end{ejemplo}

\begin{lema}
  Un retículo $\Lambda \subset V$ tiene rango completo si y solamente si existe
  un conjunto acotado $X \subseteq V$ tal que
  \[ \tag{*} V = \bigcup_{\omega \in \Lambda} X + \omega. \]

  \begin{proof}
    Si $\Lambda = \ZZ\,\omega_1 + \cdots + \ZZ\,\omega_n$ tiene rango completo,
    entonces podemos tomar como $X$ el dominio fundamental $\Pi$.

    Viceversa, si existe un conjunto acotado $X$ tal que se cumple
    (*), denotemos por $V_0$ el subespacio vectorial generado por los elementos
    de $\Lambda$. Nos gustaría ver que $V_0 = V$. Para todo $v \in V$ y
    $k \in \NN$ podemos escribir $k v = x_k + \omega_k$ para algunos $x_k \in X$
    y $\omega_k \in \Lambda$. Puesto que $X$ es acotado,
    $$\lim_{k\to\infty} \frac{1}{k}\,x_k = 0.$$
    Ahora tenemos
    $$v = \lim_{k\to\infty} \frac{1}{k}\,\omega_k \in V_0,$$
    usando que $V_0$ es un subespacio cerrado de $V$.
  \end{proof}
\end{lema}

Aunque nuestra definición de retículos menciona explícitamente una $\ZZ$-base de
$\Lambda$, hay otra caracterización más canónica.

\begin{lema}
  Un subgrupo aditivo $\Lambda \subset V$ es un retículo si y solamente
  si $\Lambda$ es discreto.
\end{lema}

Recordemos que $\Lambda \subset V$ es un subespacio \textbf{discreto} si para
todo $\omega \in \Lambda$ existe un entorno abierto $U \ni \omega$ en $V$ tal
que $\Lambda \cap U = \{ \omega \}$.

\begin{ejemplo}
  Para $\Lambda = \ZZ\cdot 1 + \ZZ\cdot \sqrt{2}$, usando la irracionalidad de
  $\sqrt{2}$, se puede ver que para cualquier $\epsilon > 0$ existe un número
  infinito de $a + b\sqrt{2} \in \ZZ [\sqrt{2}]$ tales que
  $|a + b\sqrt{2}| < \epsilon$.
\end{ejemplo}

\begin{proof}
  Primero, si $\Lambda = \ZZ\,\omega_1 + \cdots + \ZZ\,\omega_n$ es un retículo en
  $V$, entonces para todo punto $\omega = \sum_i a_i\,\omega_i \in \Lambda$
  podemos tomar el entorno abierto
  $$U = \Bigl\{ \sum_i \lambda_i\,\omega_i \Bigm| |a_i - \lambda_i| < 1 \Bigr\},$$
  y se cumple $\Lambda \cap U = \{ \omega \}$.

  Viceversa, supongamos que $\Lambda \subset V$ es un subgrupo discreto.
  En general, si $G$ es un grupo topológico de Hausdorff, entonces cualquier
  subgrupo discreto $H \subset G$ es cerrado. En nuestro caso particular,
  $\Lambda$ será cerrado. Sea $V_0$ el subespacio de $V$ generado por los
  elementos de $\Lambda$. Podemos entonces escoger una base de $V_0$ que
  consiste en elementos $\omega_1,\ldots,\omega_n \in \Lambda$. Esta base nos da
  un subretículo
  $$\Lambda_0 = \ZZ\,\omega_1 + \cdots + \ZZ\,\omega_n \subseteq \Lambda.$$
  Dado que $\Lambda_0$ tiene rango completo en $V_0$, tenemos
  $$V_0 = \bigcup_{\omega \in \Lambda_0} \Pi_0 + \omega,$$
  donde $\Pi_0$ es el dominio fundamental que corresponde a la base
  $\omega_1,\ldots,\omega_n$. Vamos a probar que el cociente $\Lambda/\Lambda_0$
  es finito. Sean $\omega_i \in \Lambda$ representantes de diferentes elementos
  en $\Lambda/\Lambda_0$. Escribamos
  $$\omega_i = x_i + \omega_{0i},$$
  donde $x_i \in \Pi_0$ y $\omega_{0i} \in \Lambda_0$. Aquí
  $x_i = \omega_i - \omega_{0i} \in \Lambda \cap \Pi_0$. El espacio
  $\Lambda \cap \Pi_0$ es discreto y acotado, así que es finito. Se sigue que
  el cociente $\Lambda/\Lambda_0$ es finito. Ahora
  $$\Lambda_0 \subseteq \Lambda \subseteq \frac{1}{[\Lambda : \Lambda_0]}\,\Lambda_0,$$
  y $\Lambda$ es también un grupo abeliano de rango $n$, así que admite una
  $\ZZ$-base finita $\omega_1', \ldots, \omega_n'$. Estos vectores son
  linealmente independientes sobre $\RR$ porque generan el espacio $V_0$ de
  dimensión $n$.
\end{proof}

Ahora supongamos que $V$ tiene estructura de espacio euclidiano; es decir, viene
con una forma bilineal definida positiva
$$\langle\cdot,\cdot\rangle\colon V\times V \to \RR.$$

\begin{definicion}
  Para un retículo $\Lambda = \ZZ\,\omega_1 + \cdots + \ZZ\,\omega_n$ el
  \textbf{covolumen} viene dado por
  $$\covol (\Lambda) = \vol (\Pi) = |\det (\langle\omega_i,\omega_j\rangle)|^{1/2}.$$
\end{definicion}

Notamos que el covolumen no depende de una base particular: para otra base
$\omega_1', \ldots, \omega_n'$ la matriz de cambio de base tiene determinante
$\pm 1$ (recuerde también nuestra discusión del discriminante en el
capítulo~\ref{ch:algebra-z-lineal}).

\begin{definicion}
  Sea $X \subseteq V$ un subconjunto.

  \begin{itemize}
  \item Se dice que $X$ es \textbf{simétrico} (respecto al origen) si para todo
    $x \in X$ se tiene $-x \in X$.

  \item Se dice que $X$ es \textbf{convexo} si para cualesquiera $x,y\in X$ la
    recta entre $x$ e $y$ también está en $X$:
    $$\{ \lambda\,y + (1-\lambda)\,x \mid 0 \le \lambda \le 1 \} \subseteq X.$$
  \end{itemize}
\end{definicion}

Todo conjunto convexo simétrico no vacío $X \subseteq V$ necesariamente contiene
el punto $0$. Ahora dado un retículo $\Lambda \subset V$, si $X$ es
suficientemente grande, entonces $X$ contiene otro punto de $\Lambda$ a parte de
$0$. Este es el contenido del siguiente teorema.

\begin{teorema}[{Minkowski\footnote{Hermann Minkowski (1864--1909)}}]
  Sean $\Lambda \subset V$ un retículo de rango completo y $X \subseteq V$
  un conjunto convexo simétrico tal que
  $$\vol X > 2^n\,\covol \Lambda.$$
  Entonces, $X$ contiene un punto no nulo de $\Lambda$.
\end{teorema}

En algún sentido, este resultado es una versión continua del principio
del palomar.

\begin{center}
  \includegraphics{pic/minkowski.pdf}
\end{center}

\begin{comentario}
  Para entender el significado del múltiplo $2^n$ en la cota del teorema,
  podemos considerar el hipercubo abierto con $2^n$ vértices en
  $(\pm 1, \pm 1, \ldots, \pm 1)$. Consideremos el retículo
  $\Lambda = \ZZ^n \subset \RR^n$. El volumen del cubo es
  $2^n = \covol \Lambda$, pero el cubo no contiene ningún punto de $\Lambda$
  salvo $0$.

  \begin{center}
    \includegraphics{pic/minkowski-2n.pdf}
  \end{center}

  Dejo como un ejercicio probar que si $X$ es compacto, entonces también
  funcionaría $\vol (X) = 2^n \, \covol \Lambda$.
\end{comentario}

Para demostrar el teorema, necesitamos el siguiente resultado auxiliar.

\begin{lema}[{Blichfeldt\footnote{Hans Frederick Blichfeldt (1873--1945), matemático danés.}}]
  Dado un conjunto medible $X\subset V$ tal que $\vol X > \covol \Lambda$,
  existen dos diferentes puntos $x,x'\in X$ tales que $x-x'\in \Lambda$.

  \begin{proof}
    Puesto que
    $$V = \bigsqcup_{\omega\in \Lambda} \Pi + \omega,$$
    tenemos
    $$X = \bigsqcup_{\omega\in \Lambda} X \cap (\Pi + \omega),$$
    así que
    $$\vol X = \sum_{\omega\in \Lambda} \vol (X \cap (\Pi + \omega)) = \sum_{\omega\in \Lambda} \vol ((X-\omega) \cap \Pi).$$
    Aquí los conjuntos $(X-\omega) \cap \Pi$ están en el dominio fundamental
    $\Pi$, y por nuestra hipótesis $\vol X > \vol \Pi$, así que podemos
    deducir que existen $\omega, \omega'\in \Lambda$ tales que
    $$(X-\omega) \cap (X-\omega') \ne \emptyset.$$
    Tomando $y \in (X-\omega) \cap (X-\omega')$, se obtiene
    \[ x = y + \omega, \quad x' = y + \omega' \in X, \quad
       x - x' = \omega-\omega' \in \Lambda. \qedhere \]
  \end{proof}
\end{lema}

Notamos que en realidad el lema no usa la covexidad de $X$ y se aplica a
cualquier conjunto medible, pero en particular, los conjuntos convexos son
medibles.

\vspace{1em}

Ahora estamos listos para demostrar el teorema de Minkowski, y aquí será
importante la hipótesis de que $X$ es convexo y simétrico. Consideremos
el conjunto
$$\frac{1}{2}\,X = \Bigl\{ \frac{1}{2}\,x \Bigm| x \in X \Bigr\}.$$
Tenemos
$$\vol \left(\frac{1}{2}\,X\right) = \frac{1}{2^n} \vol X > \covol \Lambda,$$
así que por el lema de Blichfeldt existen dos puntos distintos
$x,x' \in \frac{1}{2}\,X$ tales que $x-x' \in \Lambda$. Para terminar la
demostración, sería suficiente ver que este punto pertenece a $X$. Por la
hipótesis que $X$ es simétrico, $-x'\in \frac{1}{2}\,X$, y luego
$$x = \frac{1}{2}\,y, \quad -x' = \frac{1}{2}\,y' \quad\text{para algunos }y,y'\in X.$$
El punto
$$x - x' = \frac{1}{2}\,y + \frac{1}{2}\,y'$$
pertenece a $X$, siendo una combinación convexa de dos puntos $y,y'\in X$. \qed

\vspace{1em}

Notamos que el argumento no es constructivo y no nos da una manera eficaz de
obtener un punto no nulo en $X \cap \Lambda$.

%%%%%%%%%%%%%%%%%%%%%%%%%%%%%%%%%%%%%%%%%%%%%%%%%%%%%%%%%%%%%%%%%%%%%%%%%%%%%%%%

\section{Aplicación: teorema de cuatro cuadrados}

Prácticamente todo este capítulo será dedicado a varias aplicaciones del teorema
de Minkowski, pero primero sería instructivo ver algún ejemplo más sencillo de
su uso. En esta sección vamos a probar el siguiente famoso resultado, conocido
como el \textbf{teorema de cuatro cuadrados}.

\begin{teorema}[Lagrange]
  Para todo entero $n \ge 0$ existen $a,b,c,d \in \ZZ$ tales que
  $n = a^2 + b^2 + c^2 + d^2$.
\end{teorema}

Primero, gracias a la identidad
\begin{multline*}
  (a^2 + b^2 + c^2 + d^2)\cdot (x^2 + y^2 + z^2 + w^2) \\
  = (a x - b y - c z - d w)^2 +
    (a y + b x + c w - d z)^2 +
    (a z - b w + c x + d y)^2 +
    (a w + b z - c y + d x)^2,
\end{multline*}
descubierta por Euler, notamos que es suficiente probar el teorema para el caso
cuando $n = p$ es un número primo.

\begin{comentario}
  He aquí una breve explicación. La identidad para las sumas de dos cuadrados
  $$(a^2 + b^2)\,(x^2 + y^2) = (ax - by)^2 + (ay + bx)^2$$
  se sigue de la fórmula $N (\alpha\beta) = N (\alpha)\,N (\beta)$
  para la norma $N (\alpha) = \alpha\,\overline{\alpha}$ sobre los enteros de
  Gauss $\ZZ [i]$.

  De la misma manera, podemos considerar el \textbf{álgebra de cuaterniones} con
  coeficientes en $\ZZ$:
  $$\mathbb{H} (\ZZ) = \{ a + bi + cj + dk \mid a,b,c,d \in \ZZ \},$$
  donde la multiplicación está definida por
  \begin{gather*}
    i^2 = j^2 = k^2 = -1,\\
    ij = k, ~ ji = -k, \\
    jk = i, ~ kj = -i, \\
    ki = j, ~ ik = -j.
  \end{gather*}
  De manera más concreta, tenemos una representación fiel del álgebra de
  cuaterniones $\mathbb{H} (\ZZ)$ por las matrices $M_2 (\ZZ [i])$ dada por
  \[ (a + bi + cj + dk) \mapsto \begin{pmatrix}
    a + bi & c + di \\
    -c + di & a - bi
  \end{pmatrix}. \]

  Podemos definir el conjugado de $\alpha = a + bi + cj + dk$ mediante
  $$\overline{\alpha} = a - bi - cj - dk.$$
  La norma se define como
  $$N (\alpha) = \alpha\,\overline{\alpha} = a^2 + b^2 + c^2 + d^2,$$
  o en términos de matrices,
  \[ \det \begin{pmatrix}
    a + bi & c + di \\
    -c + di & a - bi
  \end{pmatrix} =  a^2 + b^2 + c^2 + d^2. \]
  La norma es multiplicativa: $N (\alpha\beta) = N (\alpha) \, N (\beta)$.
\end{comentario}

\begin{lema}
  Para todo primo $p$ existen $m,n\in\ZZ$ tales que
  $$m^2 + n^2 + 1 \equiv 0 \pmod{p}.$$

  \begin{proof}
    Ejercicio.
  \end{proof}
\end{lema}

Fijemos entonces $m$ y $n$ como arriba y consideremos los siguientes vectores en
$V = \RR^4$:
\[ \omega_1 = (1,0,m,n), \quad
   \omega_2 = (0,1,n,-m), \quad
   \omega_3 = (0,0,p,0), \quad
   \omega_4 = (0,0,0,p). \]
Consideremos el producto escalar estándar sobre $\RR^4$, y para un vector
$v = \sum_i a_i \, e_i$ pongamos
$\|v\|^2 = \langle v,v\rangle = \sum_i a_i^2$. Calculando el determinante
correspondiente, es fácil verificar que los $\omega_i$ generan un retículo de
rango completo
$$\Lambda = \ZZ\,\omega_1 + \ZZ\,\omega_2 + \ZZ\,\omega_3 + \ZZ\,\omega_4 \subset \RR^4,$$
tal que
$$\covol \Lambda = p^2.$$

\begin{lema}
  Para todo $\omega\in \Lambda$ el número $\|\omega\|^2$ es un entero
  divisible por $p$.

  \begin{proof}
    Si
    $$\omega = a_1\,\omega_1 + a_2\,\omega_2 + a_3\,\omega_3 + a_4\,\omega_4 = (a_1, ~ a_2, ~ a_1\,m+a_2\,n + a_3\,p, ~ a_1\,n-a_2\,m + a_4\,p),$$
    entonces

    \begin{multline*}
      \|\omega\|^2 = a_1^2 + a_2^2 + (a_1\,m+a_2\,n + a_3\,p)^2 + (a_1\,n-a_2\,m + a_4\,p)^2 \\
      \equiv a_1^2 + a_2^2 + (a_1\,m+a_2\,n)^2 + (a_1\,n-a_2\,m)^2 \pmod{p}.
    \end{multline*}

    Luego,
    $$a_1^2 + a_2^2 + (a_1\,m+a_2\,n)^2 + (a_1\,n-a_2\,m)^2 = (a_1^2 + a_2^2)\,(m^2 + n^2 + 1)$$
    y $m^2 + n^2 + 1 \equiv 0 \pmod{p}$ por nuestra elección de $m$ y $n$.
  \end{proof}
\end{lema}

Sea $X$ la bola abierta en $\RR^4$ de radio $r = \sqrt{2p}$ centrada en el
origen:
$$X = \{ x \in \RR^4 \mid \|x\|^2 < 2p \}.$$
Recordemos que en general la bola $n$-dimensional de radio $r$ tiene volumen
$$\frac{\pi^{n/2}}{\Gamma \left(\frac{n}{2}+1\right)}\,r^n.$$
En este caso $n = 4$ y
$\Gamma \left(\frac{n}{2}+1\right) = \Gamma (3) = 2! = 2$. Tenemos
$$\vol X = \frac{\pi^2 r^4}{2} = 2 \pi^2 p^2 > 2^4\,\covol \Lambda = 16\,p^2$$
(de hecho, $2 \pi^2 = 19.73\ldots > 16$). Entonces, según el teorema de
Minkowski, existe un punto no nulo $\omega\in \Lambda$ tal que
$\omega \in X$. Ahora de
$$0 < \|\omega\|^2 < 2p, \quad p \mid \|\omega\|^2$$
podemos concluir que $\|\omega\|^2 = p$. Esto nos da una representación de $p$
como una suma de cuatro cuadrados. \qed

\vspace{1em}

Cabe mencionar que la representación de números enteros por sumas de cuadrados
es un problema clásico relacionado con mucha teoría de números profunda. Véase
por ejemplo el libro \cite{Grosswald-1985}.

%%%%%%%%%%%%%%%%%%%%%%%%%%%%%%%%%%%%%%%%%%%%%%%%%%%%%%%%%%%%%%%%%%%%%%%%%%%%%%%%

\section{Aplicación: teorema de aproximación de Dirichlet}
\marginpar{\small Lectura\\ adicional}

El siguiente famoso teorema de Dirichlet es el primer resultado en la
\textbf{aproximación diofántica}.

\begin{teorema}
  Dados números reales $\alpha$ y $N \ge 1$, existen $p,q \in \ZZ$ tales que
  $1 \le q \le N$ y
  $$\Bigl|\alpha - \frac{p}{q}\Bigr| < \frac{1}{qN} \le \frac{1}{q^2}.$$

  \begin{proof}
    Consideremos el conjunto convexo simétrico
    $$X = \Bigl\{ (x,y) \in \RR^2 \Bigm| |x| \le N + \frac{1}{2}, \, |\alpha x - y| < \frac{1}{N} \Bigr\}.$$

    \begin{center}
      \includegraphics{pic/dirichlet-approximation-minkowski.pdf}
    \end{center}

    Este es un paralelogramo de área $(2N + 1)\,\frac{2}{N} > 4$, y el teorema
    de Minkowski implica que existe un punto no nulo $(q,p) \in \ZZ^2$ tal que
    $(q,p) \in X$. Por la simetría, podemos asumir que $q \ge 1$. Tenemos
    entonces
    \[ 1 \le q \le N, \quad |q \alpha - p| < \frac{1}{N}. \qedhere \]
  \end{proof}
\end{teorema}

\begin{corolario}
  \label{cor:aproximaciones-racionales}
  Para $\alpha$ irracional existe un número infinito de fracciones $\frac{p}{q}$
  tales que
  $$\Bigl|\alpha - \frac{p}{q}\Bigr| < \frac{1}{q^2}.$$

  \begin{proof}
    Primero, el teorema de Dirichlet nos da una aproximación
    $$\Bigl|\alpha - \frac{p}{q}\Bigr| < \frac{1}{q N} \le \frac{1}{q^2}.$$
    Notamos que siempre podemos asumir que $\gcd (p,q) = 1$. Escojamos
    $$N' > \cfrac{1}{\Bigl|\alpha - \cfrac{p}{q}\Bigr|}$$
    y tomamos otra aproximación
    $$\Bigl|\alpha - \frac{p'}{q'}\Bigr| < \frac{1}{q'\,N'} = \cfrac{\Bigl|\alpha - \cfrac{p}{q}\Bigr|}{q'} \le \Bigl|\alpha - \frac{p}{q}\Bigr|.$$
    Aquí necesariamente $\frac{p}{q} \ne \frac{p'}{q'}$, y continuando de esta
    manera se obtienen diferentes $\frac{p_i}{q_i}$, donde
    $$0 < \Bigl|\alpha - \frac{p_n}{q_n}\Bigr| < \Bigl|\alpha - \frac{p_{n-1}}{q_{n-1}}\Bigr| < \cdots < \Bigl|\alpha - \frac{p_1}{q_1}\Bigr|$$
    y $\Bigl|\alpha - \frac{p_i}{q_i}\Bigr| < \frac{1}{q_i^2}$.
  \end{proof}
\end{corolario}

\begin{ejemplo}
  Para $\alpha = \sqrt{2}$ las fracciones
  $$\frac{p_n}{q_n} = \frac{1}{1}, ~ \frac{3}{2}, ~ \frac{7}{5}, ~ \frac{17}{12}, ~ \frac{41}{29}, ~ \frac{99}{70}, ~ \ldots$$
  cumplen la condición del corolario:
  \begin{center}\renewcommand{\arraystretch}{1.5}
    \begin{tabular}{rcccccc}
      \hline
      $\frac{p_n}{q_n}\colon$ & $\frac{1}{1}$ & $\frac{3}{2}$ & $\frac{7}{5}$ & $\frac{17}{12}$ & $\frac{41}{29}$ & $\frac{99}{70}$ \tabularnewline
      \hline
      $|\alpha - p_n/q_n|\colon$ & $0.414213$ & $0.085786$ & $0.014213$ & $0.002453$ & $0.000420$ & $0.000072$ \tabularnewline
      \hline
      $1/q_n^2\colon$ & $1.000000$ & $0.250000$ & $0.040000$ & $0.006944$ & $0.001189$ & $0.000204$ \tabularnewline
      \hline
      \end{tabular}
  \end{center}

\end{ejemplo}

Las aproximaciones de \ref{cor:aproximaciones-racionales} pueden ser obtenidas
mediante las \textbf{fracciones continuas} para $\alpha$. Vamos a revisarlas más
adelante porque estas tendrán una relación con el grupo de unidades
$\O_K^\times$ para $K = \QQ (\sqrt{d})$ campo cuadrático real.

\vspace{1em}

Para mayor información sobre la aproximación diofántica, véase por ejemplo
\cite{Schmidt-1980}.

%%%%%%%%%%%%%%%%%%%%%%%%%%%%%%%%%%%%%%%%%%%%%%%%%%%%%%%%%%%%%%%%%%%%%%%%%%%%%%%%

\pdfbookmark{Clase 20 (21/10/20)}{clase-20}
\section{Anillo de enteros como un retículo}
\marginpar{\small Clase 20 \\ 21/10/20}

Sea $K/\QQ$ un campo de números. Este tiene $n = [K:\QQ]$ encajes
$\tau\colon K\hookrightarrow \CC$, entre estos $r_1$ encajes reales
$\rho\colon K\hookrightarrow \RR$ y $2 r_2$ encajes complejos
$\sigma\colon K\hookrightarrow \CC$ tales que $\overline{\sigma}\ne\sigma$.
Consideremos el espacio complejo $n$-dimensional
$$K_\CC = \prod_\tau \CC$$
y el encaje correspondiente
\[ \Phi\colon K \hookrightarrow K_\CC, \quad
   \alpha \mapsto (\tau (\alpha))_\tau. \]

Vamos a dotar el espacio $K_\CC$ del producto hermitiano habitual
$$\langle z,z'\rangle = \sum_\tau z_\tau \, \overline{z'_\tau}.$$
En particular, notamos que $\langle z',z\rangle = \overline{\langle z,z'\rangle}$
y $\langle z,z\rangle > 0$ para $z \ne 0$.

El grupo $G_\RR = \Gal (\CC/\RR)$ actúa sobre $K_\CC$ mediante la conjugación
compleja y permutación de las coordenadas $\tau \mapsto \overline{\tau}$.
Esto nos da un automorfismo $\RR$-lineal de orden $2$
\[ F\colon K_\CC \to K_\CC, \quad
   (z_\tau)_\tau \mapsto (\overline{z}_{\overline{\tau}})_\tau. \]

Consideremos el subespacio real fijo por la acción de $G_\RR$:
$$K_\RR = (K_\CC)^{G_\RR} = \{ (z_\tau)_\tau \mid z_{\overline{\tau}} = \overline{z}_\tau \}.$$

Dado que $\langle F z, F z'\rangle = \overline{\langle z, z'\rangle}$, el producto
hermitiano sobre $K_\CC$ se restringe a un producto escalar
$$\langle\cdot,\cdot\rangle\colon K_\RR\times K_\RR\to \RR.$$
Efectivamente, para todo $x,y \in K_\RR$ se tiene
\begin{gather*}
  \langle x,y\rangle = \langle F x, F y\rangle = \overline{\langle x,y\rangle},\\
  \langle y,x\rangle = \overline{\langle x,y\rangle} = \langle x,y\rangle,\\
  \langle x,x\rangle > 0 \text{ si }x\ne 0.
\end{gather*}

Notamos que por la definición, para $\alpha \in K$ se tiene
$\overline{\tau} (\alpha) = \overline{\tau (\alpha)}$, y entonces el encaje
$\Phi\colon K\hookrightarrow K_\CC$ toma valores en $K_\RR$:
\[ \begin{tikzcd}
  K \ar[right hook->]{r}{\Phi}\ar[right hook->]{dr}{\Phi} & K_\RR\ar[right hook->]{d} \\
  & K_\CC
\end{tikzcd} \]

\begin{comentario}
  En términos de productos tensoriales,

  \begin{itemize}
  \item $K_\CC \cong K\otimes_\QQ \CC$,

  \item el encaje $\Phi\colon K \hookrightarrow K_\CC$ se identifica con
    $\alpha \mapsto \alpha\otimes 1$,

  \item la aplicación $F\colon K_\CC \to K_\CC$ corresponde a
    $\alpha \otimes z \mapsto \alpha \otimes \overline{z}$,

  \item $K_\RR \cong K\otimes_\QQ \RR$, y la inclusión
    $K_\RR \hookrightarrow K_\CC$ está inducida por $\RR \subset \CC$.
  \end{itemize}
\end{comentario}

\vspace{1em}

El siguiente resultado explica el significado geométrico del discriminante
$\Delta_K$.

\begin{proposicion}
  La imagen del anillo de enteros $\Lambda = \Phi (\O_K) \subset K_\RR$ es un
  retículo de rango completo tal que $\covol \Lambda = \sqrt{|\Delta_K|}$.

  \begin{proof}
    Si $\O_K = \ZZ\,\alpha_1 + \cdots + \ZZ\,\alpha_n$, entonces
    $\Lambda = \ZZ\,\Phi (\alpha_1) + \cdots + \ZZ\,\Phi (\alpha_n)$.
    Ahora si $\tau_i\colon K \hookrightarrow \CC$ son diferentes encajes,
    recordemos que
    $\Delta_K = \det (A)^2$, donde $A = (\tau_i \alpha_j)_{i,j}$.

    Por otra parte, el covolumen de $\Lambda$ se calcula mediante la matriz
    \[ (\langle \Phi (\alpha_i), \Phi (\alpha_j)\rangle)_{i,j} =
    \Bigl(\sum_k \tau_k \alpha_i, \overline{\tau_k \alpha_j}\Bigr)_{i,j} =
    A \, \overline{A}^t. \]

    Ahora
    \[ \covol \Lambda = \sqrt{|\det (A \overline{A}^t)|} = \sqrt{|\Delta_K|}. \qedhere \]
  \end{proof}
\end{proposicion}

\begin{corolario}
  Para todo ideal no nulo $I \subseteq \O_K$, la imagen correspondiente
  $\Lambda = \Phi (I)$ es un retículo de rango completo tal que
  $\covol \Lambda = \sqrt{|\Delta_K|}\cdot N_{K/\QQ} (I)$.

  \begin{proof}
    Si $I \ne 0$, entonces el índice
    $$N_{K/\QQ} (I) = \# (\O_K/I) = [\O_K : I]$$
    es finito, y luego $\Lambda = \Phi (I)$ es un subretículo en
    $\Lambda' = \Phi (\O_K)$ tal que $[\Lambda' : \Lambda] = N_{K/\QQ} (I)$.
    Dejo como un ejercicio verificar que
    \[ \covol \Lambda = \covol \Lambda' \cdot [\Lambda' : \Lambda]. \qedhere \]
  \end{proof}
\end{corolario}

El último cálculo explica el significado geométrico de la norma $N_{K/\QQ} (I)$.

\begin{ejemplo}
  \label{ejemplo:encaje-de-enteros-de-Eisenstein}
  Volvamos a los enteros de Eisenstein $\ZZ [\zeta_3] \subset \CC$.
  Hay dos encajes complejos
  $\sigma, \overline{\sigma}\colon \QQ (\zeta_3) \to \CC$ dados por
  $\sigma\colon \zeta_3 \mapsto \zeta_3$ y
  $\overline{\sigma}\colon \zeta_3 \mapsto \zeta_3^2$. En este caso particular
  $$K_\RR = \{ (z_\sigma, z_{\overline{\sigma}}) \in K_\CC \mid z_{\overline{\sigma}} = \overline{z_\sigma} \}.$$
  Esto nos da un isomorfismo de espacios vectoriales
  \[ \phi\colon K_\RR \xrightarrow{\cong} \RR^2, \quad
     (z_\sigma, z_{\overline{\sigma}}) \mapsto (x_\sigma, x_{\overline{\sigma}}) = (\Re z_\sigma, \Im z_\sigma). \]
  Poniendo
  $z_\sigma = x_\sigma + i y_\sigma$, $z'_\sigma = x'_\sigma + i y'_\sigma$,
  calculamos que
  \[ z_\sigma \, \overline{z'_\sigma} + z_{\overline{\sigma}} \, \overline{z'_{\overline{\sigma}}} =
     z_\sigma \, \overline{z'_\sigma} + \overline{z_\sigma} \, z'_\sigma =
     2\,(x_\sigma\,x'_\sigma + y_\sigma\,y'_\sigma) =
     2\,(x_\sigma\,x'_\sigma + x_{\overline{\sigma}}\,x'_{\overline{\sigma}}). \]
  Entonces, el producto escalar sobre $\RR^2$ que corresponde al producto escalar
  sobre $K_\RR$ es
  $$\langle x, x'\rangle = 2\,(x_\sigma\,x'_\sigma + x_{\overline{\sigma}}\,x'_{\overline{\sigma}}).$$
  Tenemos el encaje
  $$\Phi\colon \ZZ [\zeta_3] \hookrightarrow K_\RR \xrightarrow{\cong} \RR^2$$
  dado por
  \[ 1 \mapsto (1, 0), \quad
     \zeta_3 \mapsto (\Re \zeta_3, \Im \zeta_3). \]

  Ahora el volumen del retículo correspondiente será
  \[ \left|\det \begin{pmatrix}
    \langle \Phi (1), \Phi (1) \rangle & \langle \Phi (1), \Phi (\zeta_3) \rangle \\
    \langle \Phi (\zeta_3), \Phi (1) \rangle & \langle \Phi (\zeta_3), \Phi (\zeta_3) \rangle \\
  \end{pmatrix}\right|^{1/2} =
  \left|\det \begin{pmatrix}
    2 & 2 \Re \zeta_3 \\
    2 \Re \zeta_3 & 2\cdot |\zeta_3|^2
  \end{pmatrix}\right|^{1/2} = \sqrt{4 - 4\cdot \Re (\zeta_3)^2} = \sqrt{3}. \]
  En este caso $\Delta_K = -3$.
\end{ejemplo}

%%%%%%%%%%%%%%%%%%%%%%%%%%%%%%%%%%%%%%%%%%%%%%%%%%%%%%%%%%%%%%%%%%%%%%%%%%%%%%%%

\section{Cota de Minkowski}

\begin{comentario}
  Para ver el espacio $K_\RR$ de manera más explícita, sean
  $\rho_1, \ldots, \rho_{r_1}\colon K \hookrightarrow \RR$
  los encajes reales, y
  $\sigma_1, \, \overline{\sigma_1}, \ldots, \sigma_{r_2}, \, \overline{\sigma_{r_2}} \colon K \hookrightarrow \CC$
  los encajes complejos. Entonces,
  $$K_\RR = \{ (z_\tau) \in K_\CC \mid z_\rho \in \RR, \, z_{\overline{\sigma}} = \overline{z}_\sigma \}.$$
  Tenemos un isomorfismo de espacios $\RR$-vectoriales
  \[ \phi\colon K_\RR \xrightarrow{\cong} \RR^{r_1 + 2 r_2}, \quad
  (z_\tau)_\tau \mapsto (x_\tau)_\tau, \]
  donde
  \[ x_\rho = z_\rho, \quad
  x_\sigma = \Re (z_\sigma), \quad
  x_{\overline{\sigma}} = \Im (z_\sigma). \]

  Un cálculo similar al de \ref{ejemplo:encaje-de-enteros-de-Eisenstein}
  demuestra que el producto escalar correspondiente sobre $\RR^{r_1 + 2 r_2}$
  viene dado por
  $$\langle x,y\rangle = \sum_\tau n_\tau \, x_\tau \, y_\tau,$$
  donde
  \[ n_\tau = \begin{cases}
    1, & \text{si }\tau\text{ es real},\\
    2, & \text{si }\tau\text{ es complejo}.
  \end{cases} \]
  Este producto escalar define una medida sobre $\RR^{r_1 + 2r_2}$, respecto
  a cual
  $$\vol (X) = 2^{r_2}\cdot \vol_{Leb} (\phi (X)),$$
  donde $\vol_{Leb}$ denota el volumen respecto a la medida de Lebesgue
  habitual sobre $\RR^n$. A partir de ahora, cuando hablamos del volumen de un
  subconjunto de $K_\RR$, vamos a entender esta medida inducida por el producto
  escalar.
\end{comentario}

\begin{lema}
  Para $t > 0$ el conjunto convexo simétrico
  $$X_t = \{ (z_\tau) \in K_\RR \mid \sum_\tau |z_\tau| \le t \}$$
  tiene volumen
  $$\vol (X_t) = 2^{r_1}\,\pi^{r_2}\,\frac{t^n}{n!}.$$

  \begin{proof}
    Como vimos, tenemos $\vol (X_t) = 2^{r_2}\,\vol_{Leb} (\phi (X_t))$,
    y el conjunto $\phi (X_t)$ en $\RR^{r_1 + 2r_2}$ con coordenadas
    $(x_1, \ldots, x_{r_1}, \, y_1, z_1, \ldots, y_{r_2}, z_{r_2})$
    es el conjunto definido por la desigualdad
    $$|x_1| + \cdots + |x_{r_1}| + 2\sqrt{y_1^2 + z_1^2} + \cdots + 2\,\sqrt{y_{r_2} + z_{r_2}} \le t.$$
    Pasando a las coordenadas polares $y_i = u_i \, \cos \theta_i$,
    $z_i = u_i \, \sen \theta_i$, tenemos
    $$\vol_{Leb} (\phi (X_t)) = \int u_1 \cdots u_s\,dx_1 \cdots dx_{r_1}\,du_1 \cdots du_{r_2}\,d\theta_1 \cdots d \theta_{r_2},$$
    donde la integración es sobre el dominio
    \[ 0 \le \theta_i \le 2\pi, \quad u_i \ge 0, \quad
       |x_1| + \cdots + |x_{r_1}| + 2u_1 + \cdots + 2u_{r_2} \le t. \]
    Pasando a la integral sobre $x_i \ge 0$ y $2 u_i = w_i$,
    $$\vol_{Leb} (\phi (X_t)) = 2^{r_1}\,4^{-r_2}\,(2\pi)^{r_2}\,I_{r_1,r_2} (t),$$
    donde
    $$I_{r_1,r_2} (t) = \int w_1 \cdots w_{r_2} \, dx_1\cdots d x_{r_1} d w_1\cdots d w_{r_2},$$
    y la integral es sobre
    \[ x_i \ge 0, \quad w_i \ge 0, \quad
       x_1 + \cdots + x_{r_1} + w_1 + \cdots + w_{r_2} \le t. \]

    Tenemos
    $$I_{r_1,r_2} (t) = t^{r_1 + 2r_2} \, I_{r_1,r_2} (1) = t^n \, I_{r_1,r_2} (1).$$
    Reescribiendo el dominio como
    $$x_2 + \cdots + x_{r_1} + w_1 + \cdots + w_{r_2} \le t - x_1,$$
    tenemos por el teorema de Fubini
    $$I_{r_1, r_2} (1) = \int_0^1 I_{r_1-1,r_2} (1 - x_1)\,dx_1 = \int_0^1 (1 - x_1)^{n-1}\,dx_1 \cdot I_{r_1-1, r_2} (1) = \frac{1}{n}\,I_{r_1-1,r_2} (1),$$
    y entonces por inducción
    $$I_{r_1,r_2} = \frac{1}{n\,(n-1)\cdots (n - r_1 + 1)}\,I_{0,r_2} (1).$$
    De la misma manera
    $$I_{0,r_2} (1) = \int_0^1 w_1 \, (1-w_1)^{2r_2 - 2} \, dw_1 \, I_{0,r_2-1} (1),$$
    de donde por inducción
    $$I_{0,r_2} (1) = \frac{1}{(2r_2)!} \, I_{0,0} (1) = \frac{1}{(2r_2)!}.$$
    Entonces,
    $$I_{r_1, r_2} (1) = \frac{1}{n!},$$
    así que
    \[ \vol (X_t) = 2^{r_2} \cdot 2^{r_1}\,4^{-r_2}\,(2\pi)^{r_2}\,I_{r_1,r_2} (t) =
       2^{r_1}\,\pi^{r_2}\,\frac{t^n}{n!}. \qedhere \]
  \end{proof}
\end{lema}

\begin{teorema}[La cota de Minkowski]
  \label{thm:cota-de-minkowski}
  Dado un ideal no nulo $I \subseteq \O_K$, existe un elemento no nulo
  $\alpha \in I$, tal que
  $$|N_{K/\QQ} (\alpha)| \le M_K\cdot N_{K/\QQ} (I),$$
  donde
  $$M_K = \frac{n!}{n^n} \, \left(\frac{4}{\pi}\right)^{r_2} \, \sqrt{|\Delta_K|}$$
  es una constante que depende solamente de $K$, llamada la
  \textbf{cota de Minkowski}.

  \begin{proof}
    Consideremos el retículo $\Lambda = \Phi (I) \subset K_\RR$, y el conjunto
    convexo simétrico compacto $X_t \subset K_\RR$ del lema anterior, escogiendo
    $t$ tal que
    $$\vol (X_t) = 2^n \, \covol \Lambda;$$
    es decir,
    \[ 2^{r_1}\,\pi^{r_2}\,\frac{t^n}{n!} = 2^n\,\sqrt{|\Delta_K|} \cdot N_{K/\QQ} (I)
       \iff
       t^n = n!\,\left(\frac{4}{\pi}\right)^{r_2}\,\sqrt{|\Delta_K|}\cdot N_{K/\QQ} (I). \]
    En este caso $X_t \cap \Phi (I) \ne \{ 0 \}$; es decir, existe un elemento
    no nulo $\alpha \in I$ tal que $\Phi (\alpha) \in X_t$. (Basta tomar la igualdad
    $\vol (X_t) = 2^n \, \covol \Lambda$ porque $X_t$ es compacto.) Notamos que
    $$|N_{K/\QQ} (\alpha)| = \prod_\tau |\tau (\alpha)|,$$
    y tenemos la desigualdad entre la media aritmética y geométrica
    $$\frac{1}{n} \sum_\tau |\tau (\alpha)| \ge \left(\prod_\tau |\tau (\alpha)|\right)^{1/n}.$$

    De aquí
    \[ |N_{K/\QQ} (\alpha)| = \prod_\tau |\tau (\alpha)| \le \frac{1}{n^n} \Bigl(\sum_\tau |\tau (\alpha)|\Bigr)^n \le \frac{t^n}{n^n} = \frac{n!}{n^n}\,\left(\frac{4}{\pi}\right)^{r_2}\,\sqrt{|\Delta_K|}\cdot N_{K/\QQ} (I). \qedhere \]
  \end{proof}
\end{teorema}

En particular, si en \ref{thm:cota-de-minkowski} tomamos $I = \O_K$, entonces se
obtiene la desigualdad $1 \le |N_{K/\QQ} (\alpha)| \le M_K$, que puede ser
escrita como
\[ |\Delta_K| \ge \left(\frac{n^n}{n!}\right)^2 \, \left(\frac{\pi}{4}\right)^{2r_2}
              \ge \left(\frac{n^n}{n!}\right)^2 \, \left(\frac{\pi}{4}\right)^n. \]
Esta es una cota inferior para el discriminante en términos del grado de la
extensión $n = [K : \QQ]$. No es difícil verificar que la función de $n$ que
está a la derecha es creciente. (¡Ejercicio!) Para $n = 1$ a la derecha está
$\frac{\pi}{4}$, y para $n = 2$ tenemos $\frac{\pi^2}{4}$ que es mayor que $1$,
así que $|\Delta_K| > 1$ para $n > 1$.  Esto establece el siguiente resultado.

\begin{teorema}[Minkowski]
  Si $K/\QQ$ es una extensión no trivial, entonces $|\Delta_K| > 1$.
  En particular, en $K$ necesariamente se ramifican algunos primos.
\end{teorema}

%%%%%%%%%%%%%%%%%%%%%%%%%%%%%%%%%%%%%%%%%%%%%%%%%%%%%%%%%%%%%%%%%%%%%%%%%%%%%%%%

\section{Teorema de Hermite}

Ahora vamos a probar un teorema de Hermite que establece la finitud de campos de
números de discriminante acotado. Empecemos por un resultado auxiliar.

\begin{lema}
  Para todo campo de números $K/\QQ$ existe $\alpha \in \O_K$ tal que
  $K = \QQ (\alpha)$, y para cualquier encaje $\tau\colon K\hookrightarrow \CC$
  se tiene $|\tau (\alpha)| \le C$, donde la constante $C$ depende solamente
  del discriminante $\Delta_K$.

  \begin{proof}
    Consideremos dos casos diferentes.

    \begin{enumerate}
    \item[1)] Supongamos que $K$ tiene un encaje real
      $\rho\colon K \hookrightarrow \RR$. En este caso para $t > 1$ definamos
      el conjunto convexo simétrico
      $$X_t = \{ (x_\tau)_\tau \in K_\RR \mid |x_\rho| < t, ~ |x_\tau| < 1\text{ para }\tau \ne \rho \}.$$

    \item[2)] Si $K$ no tiene encajes reales, sean
      $\sigma, \overline{\sigma}\colon K \hookrightarrow \CC$ un par de encajes
      complejos conjugados. Definamos $X_t$ por las condiciones
      $$x_\sigma, x_{\overline{\sigma}} \in (-1, +1) + (-t, +t)\,i \subseteq \CC$$
      y
      $$|x_\tau| < 1\text{ para }\tau \ne \sigma,\overline{\sigma}.$$
    \end{enumerate}

    En ambos casos, podemos tomar $t$ suficientemente grande de tal manera que
    $\vol (X_t) > 2^n \, \sqrt{|\Delta_K|}$. El teorema de Minkowski entonces
    implica que existe un elemento no nulo $\alpha \in \O_K$ tal que
    $\Phi (\alpha) \in X_t$. Nos gustaría probar que $K = \QQ (\alpha)$.
    Todo encaje $\QQ (\alpha) \hookrightarrow \CC$ se extiende a
    $[K : \QQ (\alpha)]$ encajes $K \hookrightarrow \CC$. Entonces, sería
    suficiente ver que para cualesquiera dos encajes
    $\tau_1,\tau_2\colon K \hookrightarrow \CC$ se tiene
    $\tau_1 (\alpha) \ne \tau_2 (\alpha)$. Dejo al lector revisar las
    definiciones de $X_t$ de arriba y verificar que en cualquier caso
    $\tau_1 (\alpha) = \tau_2 (\alpha)$ para $\tau_1 \ne \tau_2$ implicaría que
    $|\tau (\alpha)| < 1$ para todo encaje $\tau\colon K\hookrightarrow \CC$,
    y luego
    $$|N_{K/\QQ} (\alpha)| = \prod_\tau |\tau (\alpha)| < 1.$$
    Sin embargo, $|N_{K/\QQ} (\alpha)| \in \ZZ_{\ge 1}$, dado que $\alpha$ es
    un entero algebraico no nulo.

    Entonces, podemos concluir que $K = \QQ (\alpha)$. Por la defnición de
    $X_t$, sabemos que los valores $|\tau (\alpha)|$ están acotados en términos
    de $t$, y como consecuencia en términos de $\Delta_K$.
  \end{proof}
\end{lema}

\begin{teorema}[Hermite]
  Para todo $C > 0$, salvo isomorfismo, hay un número finito de campos de
  números $K/\QQ$ con discriminante $|\Delta_K| < C$.

  \begin{proof}
    Primero, gracias a la cota de Minkowski, una cota sobre $|\Delta_K|$ implica
    una cota sobre el grado $[K : \QQ]$. Sería suficiente entonces ver que para
    todo grado fijo $n = [K : \QQ]$ existe un número finito de campos de números
    de discriminante fijo $\Delta_K = \Delta$.

    El lema anterior nos dice que $K = \QQ (\alpha)$, donde las raíces del
    polinomio mínimo $f = f^\alpha_\QQ \in \ZZ [x]$ están acotadas en términos
    de $\Delta$. Pero luego los coeficientes del polinomio mínimo pueden ser
    acotados en términos de $\Delta$. El grado $n = \deg (f)$ está fijo, lo que
    nos deja un número finito de posibilidades para $f$.
  \end{proof}
\end{teorema}

Cabe mencionar que el argumento de arriba usa la teoría de Minkowski y es más
reciente que los trabajos de Hermite. Para los detalles históricos (y qué
exactamente fue probado por Hermite), véase
\cite[Chapter~9]{Scharlau-Opolka-1985}.

\begin{ejemplo}
  Hay solamente dos campos cúbicos con $|\Delta_K| \le 100$. Estos están
  definidos por los polinomios
  \begin{center}
    \begin{tabular}{ll}
      $x^3 + x^2 - 2x - 1$ & ($\Delta_K = 49$) \\
      $x^3 - 3x - 1$ & ($\Delta_K = 81$)
    \end{tabular}
  \end{center}

  Por ejemplo, si aplicamos el lema de arriba a los campos cúbicos reales,
  entonces
  $$X_t = \{ (x_1,x_2,x_3) \in \RR^3 \mid |x_1| < t, \, |x_2|, |x_3| < 1 \}.$$
  Para que se cumpla $\vol (X_t) > 2^3\cdot \sqrt{|\Delta_K|}$, basta tomar
  $t > 80$. Tenemos entonces
  $$f = (x - \alpha_1)\,(x - \alpha_2)\,(x - \alpha_3) \in \ZZ[x],$$
  donde $\alpha_i$ son raíces reales con $|\alpha_i| < t$ y
  $|\alpha_2|, |\alpha_3| < 1$. Despejando la expresión para $f$, se obtiene una
  cota para los coeficientes. Esto reduce las consideraciones a un número finito
  de polinomios (aunque no es un modo muy eficaz de hacer los cálculos).
\end{ejemplo}

\begin{comentario}[Densidad de discriminantes]
  En general, si $N_n (C)$ es el número de campos de números $K/\QQ$
  (salvo isomorfismo) de grado $n$ y $|\Delta_K| < C$, el comportamiento
  asintótico de la «densidad» $N_n (C)/C$ con $C \to \infty$ es un objeto de
  recientes estudios.

  Por ejemplo, un teorema clásico de Davenport y Heilbronn (1971) dice que si
  $N_n (C)/C$ tiende a $\frac{1}{12\,\zeta (3)} = 0.069325\dots$ si se
  consideran los campos cúbicos reales ($r_1 = 3$, $r_2 = 0$, $\Delta_K > 0$)
  y a $\frac{1}{4\,\zeta (3)} = 0.207976\dots$ si se consideran los campos
  cúbicos complejos ($r_1 = r_2 = 1$, $\Delta_K < 0$).

  El artículo \cite{Belabas-1997} está dedicado a un algoritmo eficaz para
  enumerar los campos cúbicos de discriminante acotado; al final se encuentran
  unas tablas de campos de discriminantes pequeños y su número para
  $|\Delta_K| \le 10^{11}$.

  Para los resultados más recientes, véanse por ejemplo los artículos
  \cite{Bhargava-2005}, \cite{Bhargava-2010}, \cite{Belabas-Bhargava-Pomerance}.
\end{comentario}

%%%%%%%%%%%%%%%%%%%%%%%%%%%%%%%%%%%%%%%%%%%%%%%%%%%%%%%%%%%%%%%%%%%%%%%%%%%%%%%%

\pdfbookmark{Clase 21 (26/10/20)}{clase-21}
\section{Finitud del grupo de clases}
\marginpar{\small Clase 21 \\ 26/10/20}

Ocupando la cota de Minkowski \ref{thm:cota-de-minkowski}, no es difícil deducir
la finitud del grupo de clases
$$\Cl (K) = \Pic (\O_K) = \mathcal{I} (\O_K) / \mathcal{P} (\O_K).$$

\begin{lema}
  Para todo $C > 0$ hay un número finito de ideales $I \subseteq \O_K$ tales que
  $N_{K/\QQ} (I) \le C$.

  \begin{proof}
    Primero, si $I = \mathfrak{p}$ es un ideal primo, entonces
    $N_{K/\QQ} (\mathfrak{p}) = p^f$, donde $p$ es un primo tal que
    $\mathfrak{p}\cap\ZZ = p\ZZ$. Para todo primo racional $p$ hay un número
    finito de ideales primos $\mathfrak{p} \subset \O_K$ tales que
    $\mathfrak{p}\cap\ZZ = p\ZZ$ (es decir, $\mathfrak{p} \mid p$), y de estas
    consideraciones se ve que la afirmación es cierta para los ideales primos.

    En general, todo ideal no nulo $I \subset \O_K$ se factoriza de alguna
    manera en ideales primos:
    $I = \mathfrak{p}_1^{e_1}\cdots \mathfrak{p}_s^{e_s}$, y luego
    $$N_{K/\QQ} (I) = N_{K/\QQ} (\mathfrak{p}_1)^{e_1}\cdots N_{K/\QQ} (\mathfrak{p}_s)^{e_s}.$$

    De esta manera se ve que para encontrar los ideales de norma $\le C$,
    podemos considerar los primos racionales $p \le C$ y los ideales primos
    correspondientes $\mathfrak{p} \mid p$ con $N_{K/\QQ} (\mathfrak{p}) \le C$,
    y luego tomar diferentes productos de estos ideales $\mathfrak{p}$.
  \end{proof}
\end{lema}

\begin{lema}
  Para todo elemento $[I] \in \Cl (K)$ existe un ideal entero $J \subseteq \O_K$
  tal que $[I] = [J]$ y $N_{K/\QQ} (J) \le M_K$, donde $M_K$ es la cota de
  Minkowski, definida en \ref{thm:cota-de-minkowski}.

  \begin{proof}
    Consideremos un elemento $[I] \in \Cl (K)$ representado por un $\O_K$-ideal
    fraccionario no nulo $I \subseteq K$. En este caso para algún
    $\beta \in \O_K$ no nulo se tiene $\beta I^{-1} \subseteq \O_K$.
    Según el teorema \ref{thm:cota-de-minkowski}, existe entonces un elemento no
    nulo $\alpha \in \beta I^{-1}$ tal que
    $$|N_{K/\QQ} (\alpha)| \le M_K\cdot N_{K/\QQ} (\beta I^{-1}).$$
    Notamos que
    $\alpha \beta^{-1} I \subseteq (\beta I^{-1})\,(\beta^{-1} I) = \O_K$, así
    que el ideal $\alpha \beta^{-1} I$ es entero. Además,
    $[I] = [\alpha \beta^{-1} I]$ en el grupo de clases. La desigualdad de
    arriba nos dice que
    \[ N_{K/\QQ} (\alpha \beta^{-1} I) \le M_K. \qedhere \]
  \end{proof}
\end{lema}

\begin{teorema}
  El grupo $\Cl (K)$ es finito.

  \begin{proof}
    Hemos probado que cualquier elemento $[I] \in \Cl (K)$ puede ser
    representado por un ideal entero $J \subseteq \O_K$ tal que $[J] = [I]$ y
    $N_{K/\QQ} (J) \le M_K$. Aquí la constante $M_K$ depende solamente de $K$,
    así que hay un número finito de ideales $J$.
  \end{proof}
\end{teorema}

\begin{definicion}
  Para un campo de números $K/\QQ$ el número $h_K = \# \Cl (K)$ se llama el
  \textbf{número de clases} de $K$.
\end{definicion}

\begin{comentario}
  \label{com:clases-representadas-por-ideales-primos}
  Se puede probar que todo elemento de $\Cl (K)$ tiene forma $[\mathfrak{p}]$
  para algún \emph{ideal primo} $\mathfrak{p} \subset \O_K$.

  Este es un resultado sutil porque es cierto para $\O_K$ y no se cumple para
  los dominios de Dedekind en general, así que no es un asunto de álgebra
  conmutativa sino de la aritmética. Esto se sigue del teorema que afirma que
  los ideales primos que están en una clase fija en $\Cl (K)$ tienen densidad
  $\frac{1}{h_K}$. Como todos los teoremas densidad, esto se demuestra
  considerando ciertas funciones $L$. Véase por ejemplo
  \cite[\S VII.13]{Neukirch-ANT}.
\end{comentario}

%%%%%%%%%%%%%%%%%%%%%%%%%%%%%%%%%%%%%%%%%%%%%%%%%%%%%%%%%%%%%%%%%%%%%%%%%%%%%%%%

\section{Ejemplo: campos cuadráticos imaginarios}

\begin{ejemplo}
  Consideremos los campos cuadráticos imaginarios $K = \QQ (\sqrt{-d})$.
  En este caso $n = 2$ y $r_2 = 1$, así que la cota de Minkowski será
  $M_K = \frac{2}{\pi}\,\sqrt{|\Delta_K|}$, donde
  \[ \Delta_K = \begin{cases}
    -4d, & \text{si }d \equiv 1,2\pmod{4},\\
    -d, & \text{si }d \equiv 3\pmod{4}.
  \end{cases} \]

  \begin{center}\renewcommand{\arraystretch}{1.5}
    \begin{tabular}{rx{1cm}x{1cm}x{1cm}x{1cm}x{1cm}x{1cm}x{1cm}x{1cm}x{1cm}x{1cm}}
      \hline
      $d\colon$ & $1$ & $2$ & $3$ & $5$ & $6$ & $7$ & $10$ & $11$ & $13$ & $14$ \tabularnewline
      \hline
      $M_{\QQ (\sqrt{-d})}\colon$ & $1.27$ & $1.80$ & $1.10$ & $2.85$ & $3.12$ & $1.68$ & $4.03$ & $2.11$ & $4.59$ & $4.76$ \tabularnewline
      \hline
    \end{tabular}
  \end{center}

  \vspace{1em}

  \begin{itemize}
  \item Si $M_K < 2$, esto implica que $\Cl (K) = 0$. De esta manera sabemos que
    los campos
    $$\QQ (i), ~ \QQ (\sqrt{-2}), ~ \QQ (\sqrt{-3}), ~ \QQ (\sqrt{-7})$$
    tienen el grupo de clases trivial. Esto no es algo nuevo: ya sabemos que los
    anillos de enteros correspondientes
    \[ \ZZ [i], ~
       \ZZ [\sqrt{-2}], ~
       \ZZ \Bigl[\frac{1+\sqrt{-3}}{2}\Bigr], ~
       \ZZ \Bigl[\frac{1+\sqrt{-7}}{2}\Bigr] \]
    son dominios euclidianos.

  \item El anillo de enteros de $\QQ (\sqrt{-11})$ es
    $\ZZ \Bigl[\frac{1+\sqrt{-11}}{2}\Bigr]$, y es también un dominio
    euclidiano.

    La cota de Minkowski en este caso es $M_K \approx 2.11$. El primo $p = 2$ es
    inerte en $\QQ (\sqrt{-11})$, así que no hay ideales de norma $2$. Esto nos
    da otra prueba de que $\QQ (\sqrt{-11})$ tiene el grupo de clases trivial.

  \item Consideremos el campo $K = \QQ (\sqrt{-5})$. En este caso
    $M_K \approx 2.85$ nos dice que todo elemento en $\Cl (K)$ puede ser
    representado por un ideal entero de norma $1$ o $2$. El único ideal de norma
    $2$ es el ideal primo que está arriba de $p = 2$:
    $$2\O_K = \mathfrak{p}^2, \quad \mathfrak{p} = (2, 1 + \sqrt{-5}).$$

    Afirmamos que el ideal $\mathfrak{p}$ no es principal: en el caso contrario
    tendríamos $\mathfrak{p} = (\alpha)$ para algún $\alpha \in \O_K$,
    y luego $N_{K/\QQ} (\mathfrak{p}) = |N_{K/\QQ} (\alpha)|$. Sin embargo,
    $N_{K/\QQ} (\mathfrak{p}) = 2$, mientras que
    $N_{K/\QQ} (\alpha) = a^2 + 5b^2 \ne 2$.

    Entonces, $[\mathfrak{p}]$ es el único elemento no trivial del grupo de
    clases, y podemos concluir que $\Cl (K) \cong \ZZ/2\ZZ$.

  \item En el caso de $K = \QQ (\sqrt{-6})$ tenemos que examinar los ideales de
    norma $2$ y $3$. Estos son
    $$\mathfrak{p}_2 = (2, \sqrt{-6}), \quad \mathfrak{p}_3 = (3, \sqrt{-6}).$$
    Calculamos que
    \[ \mathfrak{p}_2^2 = 2\O_K, \quad
       \mathfrak{p}_3^2 = 3\O_K, \quad
       \mathfrak{p}_2\,\mathfrak{p}_3 = (\sqrt{-6}). \]
    Esto significa que en el grupo de clases los elementos $[\mathfrak{p}_2]$
    y $[\mathfrak{p}_3]$ tienen orden $2$, y además
    $[\mathfrak{p}_2]\cdot [\mathfrak{p}_3] = [\O_K]$ implica que    
    $[\mathfrak{p}_2] = [\mathfrak{p}_3]$. Un argumento similar al de arriba
    demuestra que ideal $\mathfrak{p}_2$ no es principal, así que
    $\Cl (K) \cong \ZZ/2\ZZ$.

  \item Para los campos $K = \QQ (\sqrt{-10}), \QQ (\sqrt{-13})$ de la misma
    manera se puede ver que $\Cl (K) \cong \ZZ/2\ZZ$. Dejo los detalles como
    un ejercicio.

  \item En fin, consideremos $K = \QQ (\sqrt{-14})$. La cota de Minkowski es
    $M_K \approx 4.76$, así que basta considerar los ideales primos arriba de
    $2$ y $3$. Estos son los siguientes:
    $$\mathfrak{p}_2 = (2, \sqrt{-14}), \quad 2\O_K = \mathfrak{p}_2^2,$$
    y
    \[ \mathfrak{p}_3 = (3, 1 + \sqrt{-14}), \quad
       \overline{\mathfrak{p}_3} = (3, 1 - \sqrt{-14}), \quad
       3\O_K = \mathfrak{p}_3\,\overline{\mathfrak{p}_3}. \]

    En el anillo $\ZZ [\sqrt{-14}]$ no hay elementos de norma $2$ y $3$, así que
    los ideales $\mathfrak{p}_2, \mathfrak{p}_3, \overline{\mathfrak{p}_3}$ no
    son principales.

    Calculamos que
    \[ \mathfrak{p}_3^2 = (9, 3 + \sqrt{-14}, -13 + 2\sqrt{-14}) =
       (1 - \sqrt{-14}/2) \, \mathfrak{p}_2, \]
    así que $[\mathfrak{p}_3]^2 = [\mathfrak{p}_2]$. Por otra parte,
    $[\mathfrak{p}_3]^4 = [\mathfrak{p}_2]^2 = [\O_K]$. Entonces,
    $[\mathfrak{p}_3]$ tiene orden $4$. Tenemos
    $[\overline{\mathfrak{p}_3}] = [\mathfrak{p}_3]^{-1} = [\mathfrak{p}_3]^3$.
    Podemos concluir que $\Cl (K) \cong \ZZ/4\ZZ$. \qedhere
  \end{itemize}
\end{ejemplo}

\begin{comentario}
  En los cálculos de arriba la parte más complicada es verificar si algún ideal
  $\mathfrak{p} \subset \O_K$ es principal, o en general si $[I] = [J]$ en el
  grupo de clases. Esto no se ve tan difícil para los campos cuadráticos
  imaginarios, pero también existe un algoritmo general para resolver este
  problema; véase por ejemplo \cite{Buchmann-Williams-1987}.
\end{comentario}

Una pequeña tabla de grupos de clases $\Cl (\QQ (\sqrt{-d}))$ se encuentra en el
apéndice \ref{ap:grupos-de-clases-cuadraticos-imaginarios}. De allí se nota que
estos grupos \emph{no suelen} ser triviales, y es fácil explicarlo.
Primero recordemos que
\[ \O_K = \begin{cases}
  \ZZ [\sqrt{-d}], & d \equiv 1,2 \pmod{4},\\
  \ZZ \Bigl[\frac{1+\sqrt{-d}}{2}\Bigr], & d \equiv 3 \pmod{4}.
\end{cases} \]
Ahora la norma sobre $\ZZ [\sqrt{-d}]$ tiene forma
$$N_{K/\QQ} (a + b\sqrt{-d}) = a^2 + d\,b^2,$$
y la norma sobre $\ZZ \Bigl[\frac{1+\sqrt{-d}}{2}\Bigr]$ tiene forma
$$N_{K/\QQ} \Bigl(a + b\,\frac{1+\sqrt{-d}}{2}\Bigr) = a^2 + ab + \frac{1+d}{4}\,b^2 = \frac{1}{4}\,((2a + b)^2 + db^2).$$
De aquí se ve que en $\O_K$ no hay elementos de norma $2$, con excepción de
$d = 1,2,7$. Entonces, si en $\O_K$ el primo $p = 2$ se ramifica o se escinde:
\[ 2\O_K = \mathfrak{p}^2 \quad\text{o}\quad
2\O_K = \mathfrak{p}\,\overline{\mathfrak{p}}, \]
el ideal $\mathfrak{p} \mid 2$ no tiene chances de ser principal:
$\mathfrak{p} = (\alpha)$ implica que
$N_{K/\QQ} (\alpha) = N_{K/\QQ} (\mathfrak{p}) = 2$.
Esto nos lleva al siguiente resultado.

\begin{proposicion}
  \label{prop:Cl(K)=0-d=3(8)}
  Supongamos $d \ne 1,2,7$. Si $d \equiv 1,2 \pmod{4}$ o $d \equiv 7 \pmod{8}$,
  entonces $\Cl (K) \ne 0$. Específicamente, en este caso un ideal
  $\mathfrak{p} \mid 2$ representa un elemento de orden $2$ en
  $\Cl (K)$.
\end{proposicion}

Otra observación curiosa: si $d$ no es primo, entonces $\Cl (K) \ne 0$.

\begin{proposicion}
  Si $d$ es un número compuesto y $p \mid d$, entonces el ideal
  $\mathfrak{p} \mid p$ representa un elemento de orden $2$ en $\Cl (K)$.

  \begin{proof}
    Tenemos $\mathfrak{p}^2 = p \O_K$. En el caso que nos interesa $d > 3$,
    y luego $\O_K^\times = \{ \pm 1 \}$. Si $\mathfrak{p}$ fuera principal,
    tendríamos $\sqrt{\pm p} \in K$, pero esto implicaría
    $K = \QQ (\sqrt{\pm p})$, lo cual no es cierto por nuestra hipótesis.
  \end{proof}
\end{proposicion}

Dejo como un ejercicio probar que si $d = p_1\cdots p_s$, entonces los ideales
correspondientes $\mathfrak{p}_1,\ldots,\mathfrak{p}_s \subset \O_K$ generan
un subgrupo en $\Cl (K)$ isomorfo a $(\ZZ/2\ZZ)^{s-1}$.

\begin{comentario}
  Gauss probó el siguiente resultado más fuerte: para $K = \QQ (\sqrt{d})$
  con $d < 0$ el subgrupo de $2$-torsión $\Cl (K)[2]$ es isomorfo a
  $(\ZZ/2\ZZ)^{\omega (\Delta_K) - 1}$, donde $\omega (\Delta_K)$ es el número
  de divisores primos del discriminante $\Delta_K$.

  Por ejemplo, si $p \equiv 1 \pmod{4}$, entonces
  $\Delta_{\QQ (\sqrt{-p})} = -4\cdot p$, y el subgrupo de $2$-torsión será
  $\ZZ/2\ZZ$.

  Otro ejemplo: para
  $K = \QQ (\sqrt{-3\cdot 5\cdot 7\cdot 11\cdot 13\cdot 17\cdot 19})$
  tenemos
  $\Delta_K = -4\cdot 3\cdot 5\cdot 7\cdot 11\cdot 13\cdot 17\cdot 19$,
  y el grupo de clases (calculado con la computadora) es
  $$\Cl (K) \cong \ZZ/24\ZZ \oplus (\ZZ/2\ZZ)^6.$$
  El rango de $2$-torsión es $7$, como nos dice el teorema de Gauss.

  Para los campos cuadráticos reales, Gauss también calculó la $2$-torsión en
  $\Cl (K)$, pero el resultado es un poco más difícil de formular, así que no
  entraré en los detalles. De la torsión $\Cl (K) [\ell]$ con $\ell \ne 2$
  o $[K : \QQ] > 2$, todavía no se sabe mucho, este es un tema de investigación
  contemporánea.
\end{comentario}

De lo que hemos visto hasta el momento se sigue que si nos interesan los campos
cuadráticos imaginarios con $\Cl (\QQ(\sqrt{-d})) = 1$, entonces $d = p$ es
necesariamente un primo tal que $p \equiv 3 \pmod{8}$. La única excepción es
$d = 1,2,7$.

\begin{ejemplo}
  Para $p < 200$ nos interesan entonces
  $$p = 19, \, 43, \, 59, \, 67, \, 83, \, 107, \, 131, \, 139, \, 163, \, 179.$$

  Una observación útil es la siguiente: si todos los primos racionales $q < M_K$
  son inertes en $K$, entonces $\Cl (K) = 0$. Hagamos una tabla con las cotas de
  Minkowski y el primer primo $q$ que \emph{no es} inerte en
  $K = \QQ (\sqrt{-p})$. Es fácil encontrarlo: este será el primer $q$ tal que
  $\legendre{-p}{q} = +1$.

  \begin{center}\renewcommand{\arraystretch}{1.5}
    \begin{tabular}{rccccccccccc}
      \hline
      $p\colon$ & $11$ & $19$ & $43$ & $59$ & $67$ & $83$ & $107$ & $131$ & $139$ & $163$ & $179$ \\
      \hline
      $\Delta_K\colon$ & $2.11$ & $2.77$ & $4.17$ & $4.89$ & $5.21$ & $5.80$ & $6.59$ & $7.29$ & $7.51$ & $8.13$ & $8.52$ \\
      \hline
      $q\colon$ & $3$ & $5$ & $11$ & $3$ & $17$ & $3$ & $3$ & $3$ & $5$ & $41$ & $3$ \\
      \hline
    \end{tabular}
  \end{center}

  De esta tabla se ve que $\Cl (\QQ (\sqrt{-p})) = 0$ para
  $p = 11, 19, 43, 67, 163$. Curiosamente, para estos primos todo $q < (p+1)/4$ es
  inerte en $\QQ (\sqrt{-p})$. Por otra parte, para
  $p = 59, 83, 107, 131, 139, 179$ el grupo de clases no será trivial.

  \begin{itemize}
  \item En $K = \QQ (\sqrt{-59})$ tenemos
    $$3\O_K = \mathfrak{p}_3\,\overline{\mathfrak{p}_3},$$
    donde $\mathfrak{p}_3$ y $\overline{\mathfrak{p}_3}$ no son ideales
    principales: en $\O_K$ no hay elementos de norma $3$. Además,
    \[ [\mathfrak{p}_3]^2 = [\overline{\mathfrak{p}_3}], \quad
    [\mathfrak{p}_3]^3 = [\mathfrak{p}_3\,\overline{\mathfrak{p}_3}] = [\O_K], \]
    así que $\Cl (K) \cong \ZZ/3\ZZ$.

  \item En $K = \QQ (\sqrt{-83})$ y $\QQ (\sqrt{-107})$ el primo $q = 3$ se
    escinde y $q = 5$ es inerte. De nuevo, en $\O_K$ no hay elementos de norma
    $3$, y los cálculos similares demuestran que $\Cl (K) \cong \ZZ/3\ZZ$.

  \item En $K = \QQ (\sqrt{-131})$ los primos $q = 3, 5, 7$ se escinden,
    y en $\O_K$ no hay elementos de norma $3, 5, 7$. Pongamos
    \[ 3\O_K = \mathfrak{p}_3 \, \overline{\mathfrak{p}_3}, \quad
    5\O_K = \mathfrak{p}_5 \, \overline{\mathfrak{p}_5}, \quad
    7\O_K = \mathfrak{p}_7 \, \overline{\mathfrak{p}_7}. \]
    Tenemos las siguientes relaciones en el grupo de clases:
    \[ [\mathfrak{p}_3]^2 = [\overline{\mathfrak{p}_5}] = [\overline{\mathfrak{p}_7}], \quad
       [\mathfrak{p}_3]^3 = [\mathfrak{p}_5] = [\mathfrak{p}_7], \quad
       [\mathfrak{p}_3]^4 = [\overline{\mathfrak{p}_3}], \quad
       [\mathfrak{p}_3]^5 = [\O_K]. \]
    Entonces, $\Cl (K) \cong \ZZ/5\ZZ$.

  \item En $K = \QQ (\sqrt{-139})$ los primos $q = 5, 7$ se escinden, y en
    $\O_K$ no hay elementos de norma $5$ y $7$. Tenemos las siguientes
    relaciones:
    \[ [\mathfrak{p}_5] = [\overline{\mathfrak{p}_7}], \quad
       [\mathfrak{p}_5]^2 = [\overline{\mathfrak{p}_5}] = [\overline{\mathfrak{p}_7}], \quad
       [\mathfrak{p}_5]^3 = [\O_K]. \]
    En este caso $\Cl (K) \cong \ZZ/3\ZZ$.
 
  \item En $K = \QQ (\sqrt{-179})$ los primos $q = 3,5$ se escinden, mientras
    que $q = 7$ es inerte. De nuevo, en $\O_K$ no hay elementos de norma
    $3$ y $5$. Las relaciones
    \[ [\mathfrak{p}_3]^2 = [\overline{\mathfrak{p}_5}], \quad
       [\mathfrak{p}_3]^3 = [\mathfrak{p}_5], \quad
       [\mathfrak{p}_3]^4 = [\overline{\mathfrak{p}_3}], \quad
       [\mathfrak{p}_3]^5 = [\O_K] \]
    nos permiten concluir que $\Cl (K) \cong \ZZ/5\ZZ$.
  \end{itemize}

  Invito que el lector verifique todos los detalles.
\end{ejemplo}

En el siguiente capítulo vamos a establecer la siguiente interpretación
del número de clases de $\QQ (\sqrt{-p})$.

\begin{teorema}[Dirichlet]
  Sea $p > 3$ un primo tal que $p \equiv 3 \pmod{4}$. Consideremos el campo
  cuadrático imaginario $K = \QQ (\sqrt{-p})$. Si $p \equiv 7 \pmod{8}$,
  entonces
  $$h_K = \sum_{1 \le a < p/2} \legendre{a}{p},$$
  y si $p \equiv 3 \pmod{8}$, entonces
  $$h_K = \frac{1}{3}\,\sum_{1 \le a < p/2} \legendre{a}{p}.$$
\end{teorema}

\begin{ejemplo}
  Si $p = 7$, entonces
  $$\legendre{1}{7} + \legendre{2}{7} + \legendre{3}{7} = 1 + 1 - 1 = 1$$
  (tenemos $3^2 \equiv 2 \pmod{7}$).

  Si $p = 11$, entonces
  $$\legendre{1}{11} + \legendre{2}{11} + \legendre{3}{11} + \legendre{4}{11} + \legendre{5}{11} = 1 - 1 + 1 + 1 + 1 = 3$$
  (note que $5^2 \equiv 3$ y $4^2 \equiv 5 \pmod{11}$).

  De estos cálculos se sigue que $\Cl (\QQ (\sqrt{-7})) = 0$ y
  $\Cl (\QQ (\sqrt{-11})) = 0$.
\end{ejemplo}

Estas fórmulas tienen que ver con el siguiente fenómeno: aunque en
$\FF_p^\times$ hay el mismo número de cuadrados y no-cuadrados, resulta que
para $p \equiv 3 \pmod{4}$ el intervalo $[1, (p-1)/2]$ contiene más residuos
cuadráticos que no-residuos (véase \cite{Moser-1951} para una prueba
elemental). Este «defecto» tiene que ver con el grupo de clases.

En la figura~\ref{fig:numeros-de-clases-imaginarios} se encuentran los números
de clases obtenidos mediante el último teorema.

\begin{figure}
  \begin{center}\renewcommand{\arraystretch}{1.5}
    \begin{tabular}{x{2cm}x{1cm}x{2cm}x{1cm}x{2cm}x{1cm}x{2cm}x{1cm}}
      \hline
      $K$ & $h_K$ & $K$ & $h_K$ & $K$ & $h_K$ & $K$ & $h_K$ \tabularnewline\hline
      $\QQ (\sqrt{-7})$ & $1$ & $\QQ (\sqrt{-283})$ & $3$ & $\QQ (\sqrt{-647})$ & $23$ & $\QQ (\sqrt{-1063})$ & $19$ \tabularnewline\hline
      $\QQ (\sqrt{-11})$ & $1$ & $\QQ (\sqrt{-307})$ & $3$ & $\QQ (\sqrt{-659})$ & $11$ & $\QQ (\sqrt{-1087})$ & $9$ \tabularnewline\hline
      $\QQ (\sqrt{-19})$ & $1$ & $\QQ (\sqrt{-311})$ & $19$ & $\QQ (\sqrt{-683})$ & $5$ & $\QQ (\sqrt{-1091})$ & $17$ \tabularnewline\hline
      $\QQ (\sqrt{-23})$ & $3$ & $\QQ (\sqrt{-331})$ & $3$ & $\QQ (\sqrt{-691})$ & $5$ & $\QQ (\sqrt{-1103})$ & $23$ \tabularnewline\hline
      $\QQ (\sqrt{-31})$ & $3$ & $\QQ (\sqrt{-347})$ & $5$ & $\QQ (\sqrt{-719})$ & $31$ & $\QQ (\sqrt{-1123})$ & $5$ \tabularnewline\hline
      $\QQ (\sqrt{-43})$ & $1$ & $\QQ (\sqrt{-359})$ & $19$ & $\QQ (\sqrt{-727})$ & $13$ & $\QQ (\sqrt{-1151})$ & $41$ \tabularnewline\hline
      $\QQ (\sqrt{-47})$ & $5$ & $\QQ (\sqrt{-367})$ & $9$ & $\QQ (\sqrt{-739})$ & $5$ & $\QQ (\sqrt{-1163})$ & $7$ \tabularnewline\hline
      $\QQ (\sqrt{-59})$ & $3$ & $\QQ (\sqrt{-379})$ & $3$ & $\QQ (\sqrt{-743})$ & $21$ & $\QQ (\sqrt{-1171})$ & $7$ \tabularnewline\hline
      $\QQ (\sqrt{-67})$ & $1$ & $\QQ (\sqrt{-383})$ & $17$ & $\QQ (\sqrt{-751})$ & $15$ & $\QQ (\sqrt{-1187})$ & $9$ \tabularnewline\hline
      $\QQ (\sqrt{-71})$ & $7$ & $\QQ (\sqrt{-419})$ & $9$ & $\QQ (\sqrt{-787})$ & $5$ & $\QQ (\sqrt{-1223})$ & $35$ \tabularnewline\hline
      $\QQ (\sqrt{-79})$ & $5$ & $\QQ (\sqrt{-431})$ & $21$ & $\QQ (\sqrt{-811})$ & $7$ & $\QQ (\sqrt{-1231})$ & $27$ \tabularnewline\hline
      $\QQ (\sqrt{-83})$ & $3$ & $\QQ (\sqrt{-439})$ & $15$ & $\QQ (\sqrt{-823})$ & $9$ & $\QQ (\sqrt{-1259})$ & $15$ \tabularnewline\hline
      $\QQ (\sqrt{-103})$ & $5$ & $\QQ (\sqrt{-443})$ & $5$ & $\QQ (\sqrt{-827})$ & $7$ & $\QQ (\sqrt{-1279})$ & $23$ \tabularnewline\hline
      $\QQ (\sqrt{-107})$ & $3$ & $\QQ (\sqrt{-463})$ & $7$ & $\QQ (\sqrt{-839})$ & $33$ & $\QQ (\sqrt{-1283})$ & $11$ \tabularnewline\hline
      $\QQ (\sqrt{-127})$ & $5$ & $\QQ (\sqrt{-467})$ & $7$ & $\QQ (\sqrt{-859})$ & $7$ & $\QQ (\sqrt{-1291})$ & $9$ \tabularnewline\hline
      $\QQ (\sqrt{-131})$ & $5$ & $\QQ (\sqrt{-479})$ & $25$ & $\QQ (\sqrt{-863})$ & $21$ & $\QQ (\sqrt{-1303})$ & $11$ \tabularnewline\hline
      $\QQ (\sqrt{-139})$ & $3$ & $\QQ (\sqrt{-487})$ & $7$ & $\QQ (\sqrt{-883})$ & $3$ & $\QQ (\sqrt{-1307})$ & $11$ \tabularnewline\hline
      $\QQ (\sqrt{-151})$ & $7$ & $\QQ (\sqrt{-491})$ & $9$ & $\QQ (\sqrt{-887})$ & $29$ & $\QQ (\sqrt{-1319})$ & $45$ \tabularnewline\hline
      $\QQ (\sqrt{-163})$ & $1$ & $\QQ (\sqrt{-499})$ & $3$ & $\QQ (\sqrt{-907})$ & $3$ & $\QQ (\sqrt{-1327})$ & $15$ \tabularnewline\hline
      $\QQ (\sqrt{-167})$ & $11$ & $\QQ (\sqrt{-503})$ & $21$ & $\QQ (\sqrt{-911})$ & $31$ & $\QQ (\sqrt{-1367})$ & $25$ \tabularnewline\hline
      $\QQ (\sqrt{-179})$ & $5$ & $\QQ (\sqrt{-523})$ & $5$ & $\QQ (\sqrt{-919})$ & $19$ & $\QQ (\sqrt{-1399})$ & $27$ \tabularnewline\hline
      $\QQ (\sqrt{-191})$ & $13$ & $\QQ (\sqrt{-547})$ & $3$ & $\QQ (\sqrt{-947})$ & $5$ & $\QQ (\sqrt{-1423})$ & $9$ \tabularnewline\hline
      $\QQ (\sqrt{-199})$ & $9$ & $\QQ (\sqrt{-563})$ & $9$ & $\QQ (\sqrt{-967})$ & $11$ & $\QQ (\sqrt{-1427})$ & $15$ \tabularnewline\hline
      $\QQ (\sqrt{-211})$ & $3$ & $\QQ (\sqrt{-571})$ & $5$ & $\QQ (\sqrt{-971})$ & $15$ & $\QQ (\sqrt{-1439})$ & $39$ \tabularnewline\hline
      $\QQ (\sqrt{-223})$ & $7$ & $\QQ (\sqrt{-587})$ & $7$ & $\QQ (\sqrt{-983})$ & $27$ & $\QQ (\sqrt{-1447})$ & $23$ \tabularnewline\hline
      $\QQ (\sqrt{-227})$ & $5$ & $\QQ (\sqrt{-599})$ & $25$ & $\QQ (\sqrt{-991})$ & $17$ & $\QQ (\sqrt{-1451})$ & $13$ \tabularnewline\hline
      $\QQ (\sqrt{-239})$ & $15$ & $\QQ (\sqrt{-607})$ & $13$ & $\QQ (\sqrt{-1019})$ & $13$ & $\QQ (\sqrt{-1459})$ & $11$ \tabularnewline\hline
      $\QQ (\sqrt{-251})$ & $7$ & $\QQ (\sqrt{-619})$ & $5$ & $\QQ (\sqrt{-1031})$ & $35$ & $\QQ (\sqrt{-1471})$ & $23$ \tabularnewline\hline
      $\QQ (\sqrt{-263})$ & $13$ & $\QQ (\sqrt{-631})$ & $13$ & $\QQ (\sqrt{-1039})$ & $23$ & $\QQ (\sqrt{-1483})$ & $7$ \tabularnewline\hline
      $\QQ (\sqrt{-271})$ & $11$ & $\QQ (\sqrt{-643})$ & $3$ & $\QQ (\sqrt{-1051})$ & $5$ & $\QQ (\sqrt{-1487})$ & $37$ \tabularnewline\hline
    \end{tabular}
  \end{center}

  \caption{Números de clases $h_{\QQ (\sqrt{-p})}$ para $p \equiv 3~(4)$}
  \label{fig:numeros-de-clases-imaginarios}
\end{figure}

Estos cálculos sugieren que \emph{los únicos} campos cuadráticos imaginarios
$K = \QQ (\sqrt{-d})$ con $\Cl (K) = 0$ corresponden a
\[ d = 1, \, 2, \, 3, \, 7, \, 11, \, 19, \, 43, \, 67, \, 163. \]

Esto fue conjeturado por Gauss. En 1936 Heilbronn y Linfoot probaron que a lo
sumo existe un número más, pero en este caso $d > 10^9$.
Heegner, Baker, y Stark probaron que este número no existe. La prueba de
Heegner fue publicada en 1952, pero contenía menores omisiones y no fue aceptada
hasta que Stark dio una prueba similar completa en 1967. Otra prueba totalmente
diferente fue publicada por Baker en 1966 (véase \cite[Chapter~5]{Baker-1990}).

Los $d$ de arriba se conocen como los \textbf{números de Heegner}.
Lamentablemente, Heegner falleció en 1965, antes de que su trabajo fuera
reconocido por la comunidad matemática\dots

\vspace{1em}

\begin{samepage}
Gauss también hizo otras conjeturas:
\begin{itemize}
\item $h_{\QQ (\sqrt{-d})} \to \infty$ con $d \to \infty$.

  Esto fue probado por Heilbronn en 1934.

\item Para cualquier $h$ fijo existe un número finito de campos cuadráticos
  imaginarios con $\# \Cl (K) = h$.

  Por ejemplo, Baker y Stark demostraron en 1971 que hay exactamente $18$ campos
  $K = \QQ (\sqrt{-d})$ con $h_K = 2$, y estos corresponden a
  $$d = 5, \, 6, \, 10, \, 13, \, 15, \, 22, \, 35, \, 37, \, 51, \, 58, \, 91, \, 115, \, 123, \, 187, \, 235, \, 267, \, 403, \, 427.$$

  Oesterlé demostró en 1985 que hay $16$ campos $K = \QQ (\sqrt{-d})$ con
  $h_K = 3$, y estos corresponden a
  $$d = 23, \, 31, \, 59, \, 83, \, 107, \, 139, \, 211, \, 283, \, 307, \, 331, \, 379, \, 499, \, 547, \, 643, \, 883, \, 907.$$

  De los trabajos de Goldfeld, Oesterlé, Gross y Zagier se sigue una cota
  explícita para $d$ en términos de $h_{\QQ (\sqrt{-d})}$. De esta manera el
  problema se reduce a un cálculo finito. Los campos correspondientes para
  pequeños valores de $h$ han sido enumerados gradualmente; por ejemplo,
  en \cite{Watkins-2004} se encuentra una tabla para $h \le 100$.
\end{itemize}
\end{samepage}

Para la historia detrás de los campos cuadráticos imaginarios con $\Cl (K) = 0$,
recomiendo el artículo \cite{Goldfeld-1985}.

%%%%%%%%%%%%%%%%%%%%%%%%%%%%%%%%%%%%%%%%%%%%%%%%%%%%%%%%%%%%%%%%%%%%%%%%%%%%%%%%

\section{Números de la suerte de Euler}

Euler descubrió\footnote{La publicación original es la carta de Euler a
  Bernoulli \cite{E461}. Allí el polinomio es $g (x) = x^2 - x + 41$,
  pero $g (x+1) = f (x)$.} que el polinomio $f (x) = x^2 + x + 41$ toma valores
primos para todo $x = 0, 1, \ldots, 39$.

\begin{center}\renewcommand{\arraystretch}{1.5}
  \begin{tabular}{rclc|crclc|crcl}
    $f (1)$  & $=$ & $43$  & & & $f (14)$ & $=$ & $251$ & & & $f (27)$ & $=$ & $797$ \\
    $f (2)$  & $=$ & $47$  & & & $f (15)$ & $=$ & $281$ & & & $f (28)$ & $=$ & $853$ \\
    $f (3)$  & $=$ & $53$  & & & $f (16)$ & $=$ & $313$ & & & $f (29)$ & $=$ & $911$ \\
    $f (4)$  & $=$ & $61$  & & & $f (17)$ & $=$ & $347$ & & & $f (30)$ & $=$ & $971$ \\
    $f (5)$  & $=$ & $71$  & & & $f (18)$ & $=$ & $383$ & & & $f (31)$ & $=$ & $1033$ \\
    $f (6)$  & $=$ & $83$  & & & $f (19)$ & $=$ & $421$ & & & $f (32)$ & $=$ & $1097$ \\
    $f (7)$  & $=$ & $97$  & & & $f (20)$ & $=$ & $461$ & & & $f (33)$ & $=$ & $1163$ \\
    $f (8)$  & $=$ & $113$ & & & $f (21)$ & $=$ & $503$ & & & $f (34)$ & $=$ & $1231$ \\
    $f (9)$  & $=$ & $131$ & & & $f (22)$ & $=$ & $547$ & & & $f (35)$ & $=$ & $1301$ \\
    $f (10)$ & $=$ & $151$ & & & $f (23)$ & $=$ & $593$ & & & $f (36)$ & $=$ & $1373$ \\
    $f (11)$ & $=$ & $173$ & & & $f (24)$ & $=$ & $641$ & & & $f (37)$ & $=$ & $1447$ \\
    $f (12)$ & $=$ & $197$ & & & $f (25)$ & $=$ & $691$ & & & $f (38)$ & $=$ & $1523$ \\
    $f (13)$ & $=$ & $223$ & & & $f (26)$ & $=$ & $743$ & & & $f (39)$ & $=$ & $1601$ \\
  \end{tabular}
\end{center}

Para $x = 40$ se tiene $f (40) = 41^2$, así que este valor ya no es primo.
De todos modos, es sorprendente que hasta este punto salgan números primos.
Este fenómeno tiene una explicación en la teoría algebraica de números.

\begin{figure}
  \begin{center}
    \includegraphics[width=15cm]{pic/ulam-spiral-euler.pdf}
  \end{center}

  \caption{Los primos de la forma $x^2 + x + 41$ en la espiral de Ulam (los puntos en azul corresponden a $x = 0,1,\ldots,39$)}
\end{figure}

\begin{teorema}[{Rabinowitsch\footnote{Seudónimo de George Yuri Rainich (1886--1968). El mismo Rabinowitsch del famoso «truco de Rabinowitsch» en la prueba del Nullstellensatz.}}, 1912]
  Sean $p$ un primo impar y $K = \QQ (\sqrt{-p})$. Si $\Cl (K) = 0$, entonces
  el polinomio\footnote{Recuerde que en este caso $p \equiv 3 \pmod{4}$ según
    \ref{prop:Cl(K)=0-d=3(8)}.}
  $$f (x) = x^2 + x + \frac{p+1}{4}$$
  toma valores primos para $0 \le x < \frac{p-3}{4}$.
\end{teorema}

Obviamente, esta observación es trivial si uno asume el teorema de
Baker--Heegner--Stark sobre los campos $K = \QQ (\sqrt{-d})$ con $\Cl (K) = 0$,
pero esto tampoco explicaría mucho\dots{} La prueba elemental que vamos a ver
viene del artículo \cite{Ayoub-Chowla}.

\begin{lema}
  \label{lema:Rabinowitsch-lema}
  Si $\Cl (K) = 0$ y $q$ es un primo tal que $q < \frac{p+1}{4}$, entonces
  $\legendre{q}{p} = -1$.

  \begin{proof}
    Como ya vimos (\ref{prop:Cl(K)=0-d=3(8)}), la hipótesis $\Cl (K) = 0$
    implica que $p \equiv 3 \pmod{4}$, y luego
    $\legendre{q}{p} = \legendre{-p}{q}$. Ahora si $\legendre{q}{p} = +1$,
    entonces $q$ se escinde en $K$: tenemos
    $$q \O_K = \mathfrak{q}\,\overline{\mathfrak{q}}.$$
    Por nuestra hipótesis, los ideales $\mathfrak{q}$ y
    $\overline{\mathfrak{q}}$ deben ser principales. Digamos que
    $\mathfrak{q}$ está generado por $a + b\,\frac{1+\sqrt{-p}}{2}$. Aquí
    necesariamente $b \ne 0$, y luego
    \[ q = N_{K/\QQ} (\mathfrak{q}) = N_{K/\QQ} \Bigl(a + b\,\frac{1+\sqrt{-p}}{2}\Bigr) = \frac{1}{4}\,((2a + b)^2 + pb^2) \ge \frac{p+1}{4}. \qedhere \]
  \end{proof}
\end{lema}

\begin{lema}
  Si $\Cl (K) = 0$, entonces $f (0) = \frac{p+1}{4}$ es primo.

  \begin{proof}
    Si $\Cl (K) = 0$, entonces $\O_K$ es un dominio de factorización única.
    Supongamos que existe un primo $q$ tal que
    $q < \frac{1+p}{4}$ y $q \mid \frac{1+p}{4}$. En este caso
    $\legendre{q}{p} = \legendre{-p}{q} = -1$ por el lema anterior, y luego $q$
    es inerte en $K$, y por lo tanto un elemento primo en $\O_K$. Ahora
    \[ q \mid \frac{p+1}{4} =
       \left(\frac{1 + \sqrt{-p}}{2}\right)\,\left(\frac{1 - \sqrt{-p}}{2}\right)
       \Longrightarrow
       q \mid \left(\frac{1 \pm \sqrt{-p}}{2}\right). \]
    Esto es imposible.
  \end{proof}
\end{lema}

\begin{proof}[Demostración del teorema de Rabinowitsch]
  Supongamos que para el polinomio $f (x)$ para algún $x < \frac{p-3}{4}$
  el valor $f (x)$ no es primo. Sea $q$ un primo impar tal que $q \mid f (x)$
  y $q^2 \le f (x)$. Tenemos entonces la desigualdad
  $$4q^2 \le (2x + 1)^2 + p < \left(\frac{p+1}{2}\right)^2.$$

  Por el lema \ref{lema:Rabinowitsch-lema}, tenemos $\legendre{q}{p} = -1$.
  Por otra parte,
  $$x^2 + x + \frac{p+1}{4} = aq$$
  para algún $a \in \ZZ$, y luego
  $$(2x + 1)^2 + p = 4aq,$$
  de donde $\legendre{-p}{q} = +1$. Pero bajo nuestra hipótesis
  $\legendre{-p}{q} = \legendre{q}{p}$. Contradicción.
\end{proof}

\begin{ejemplo}
  Si en lugar de $p = 163$ tomamos $p = 67$, se obtiene el polinomio
  $f (x) = x^2 + x + 17$ que toma valores primos para
  $x = 0,1,\ldots,15$.

  \begin{center}\renewcommand{\arraystretch}{1.5}
    \begin{tabular}{rclc|crclc|crcl}
      $f (1)$  & $=$ & $19$  & & & $f (6)$ & $=$ & $59$ & & & $f (11)$ & $=$ & $149$ \\
      $f (2)$  & $=$ & $23$  & & & $f (7)$ & $=$ & $73$ & & & $f (12)$ & $=$ & $173$ \\
      $f (3)$  & $=$ & $29$  & & & $f (8)$ & $=$ & $89$ & & & $f (13)$ & $=$ & $199$ \\
      $f (4)$  & $=$ & $37$  & & & $f (9)$ & $=$ & $107$ & & & $f (14)$ & $=$ & $227$ \\
      $f (5)$  & $=$ & $47$  & & & $f (10)$ & $=$ & $127$ & & & $f (15)$ & $=$ & $257$
    \end{tabular}
  \end{center}

  De la misma manera, para $p = 43$ se obtiene $f (x) = x^2 + x + 11$ que toma
  valores primos para $x = 0,1,\ldots,9$.
\end{ejemplo}

Los primos $q$ que dan lugar a $f (x) = x^2 + x + q$ que toma valores primos
para $x = 0,1,\ldots,q-2$ se conocen como los \textbf{números de la suerte de
  Euler}\footnote{\emph{Euler's lucky numbers}.}. Tampoco es difícil probar que
en este caso necesariamente se tiene $\Cl (\QQ (\sqrt{1 - 4q})) = 0$
(véase \cite[Chapter~5]{Ribenboim-Friends}). Entonces, el teorema de
Baker--Heegner--Stark implica que hay solo un número finito de números
de la suerte: estos se obtienen de $p = 7, 11, 19, 43, 67, 163$.

%%%%%%%%%%%%%%%%%%%%%%%%%%%%%%%%%%%%%%%%%%%%%%%%%%%%%%%%%%%%%%%%%%%%%%%%%%%%%%%%

\section{Ejemplo: campos cuadráticos reales}

Ahora veamos algunos campos cuadráticos reales. En este caso la cota de
Minkowski será $\frac{1}{2}\,\sqrt{|\Delta_K|}$.

\begin{ejemplo}
  Calculemos $M_K$ para los primeros $K = \QQ (\sqrt{d})$.

  \begin{center}\renewcommand{\arraystretch}{1.5}
    \begin{tabular}{rx{1cm}x{1cm}x{1cm}x{1cm}x{1cm}x{1cm}x{1cm}x{1cm}x{1cm}x{1cm}}
      \hline
      $d\colon$ & $2$ & $3$ & $5$ & $6$ & $7$ & $10$ & $11$ & $13$ & $14$ & $15$ \tabularnewline
      \hline
      $\Delta_{\QQ (\sqrt{d})}\colon$ & $8$ & $12$ & $5$ & $24$ & $28$ & $40$ & $44$ & $13$ & $56$ & $60$ \tabularnewline
      \hline
      $M_{\QQ (\sqrt{d})}\colon$ & $1.41$ & $1.73$ & $1.12$ & $2.45$ & $2.65$ & $3.16$ & $3.32$ & $1.80$ & $3.74$ & $3.87$ \tabularnewline
      \hline
    \end{tabular}
  \end{center}

  De aquí notamos que para $d = 2,3,5,13$ el grupo de clases es trivial. Podemos
  ver otros casos de la tabla uno por uno.

  \begin{itemize}
  \item Para $d = 6$, el primo $2$ se ramifica en $\O_K$: tenemos
    $2\O_K = \mathfrak{p}_2^2$, donde $\mathfrak{p}_2 = (2, \sqrt{6})$.
    Este ideal es principal: tenemos $\mathfrak{p}_2 = (2 + \sqrt{6})$.
    Para verlo, basta notar que
    $N_{K/\QQ} (2 + \sqrt{6}) = 2^2 - 6\cdot 1^2 = -2$, así que el elemento
    $2 + \sqrt{6}$ genera un ideal primo que está sobre $p = 2$.

  \item Para $d = 7$ pasa lo mismo: se puede ver que el ideal
    $\mathfrak{p}_2 = (2, 1 + \sqrt{7})$ es principal, generado por
    $3 + \sqrt{7}$.

  \item Para $d = 10$ tenemos
    $$2\O_K = \mathfrak{p}_2^2, \quad \mathfrak{p}_2 = (2, \sqrt{10})$$
    y
    $$3\O_K = \mathfrak{p}_3\,\overline{\mathfrak{p}_3}, \quad \mathfrak{p}_3 = (3, 1+\sqrt{10}).$$
    Estos ideales no son principales. La norma es
    $N_{K/\QQ} (a + b\sqrt{10}) = a^2 - 10\,b^2$, y esta no puede ser igual
    a $2$ o $3$. Para verlo, basta por ejemplo reducir la norma módulo $5$.

    Además, podemos calcular que
    $$(1 - \sqrt{10}/2)\,\mathfrak{p}_2 = \mathfrak{p}_3,$$ así que
    $[\mathfrak{p}_3] = [\mathfrak{p}_2]$ en el grupo de clases. De la misma manera
    $[\overline{\mathfrak{p}_3}] = [\mathfrak{p}_2]$.
    Entonces, $\Cl (K) \cong \ZZ/2\ZZ$.

  \item Dejo al lector ver que para $d = 11$ y $14$ el grupo de clases será
    trivial.

  \item Para $d = 15$ tenemos
    $$2\O_K = \mathfrak{p}_2^2, \quad \mathfrak{p}_2 = (2, 1 + \sqrt{15})$$
    y
    $$3\O_K = \mathfrak{p}_3^2, \quad \mathfrak{p}_3 = (3, \sqrt{15}).$$
    De nuevo, reduciendo la norma
    $N_{K/\QQ} (a + b\sqrt{15}) = a^2 + 15\,b^2$ módulo $5$, notamos que
    no hay elementos de norma $2$ y $3$, así que los ideales $\mathfrak{p}_2$ y
    $\mathfrak{p}_3$ no son principales. Por otra parte, podemos verificar que
    $[\mathfrak{p}_2] = [\mathfrak{p}_3]$, y luego
    $\Cl (K) \cong \ZZ/2\ZZ$. \qedhere
  \end{itemize}
\end{ejemplo}

Una tabla de grupos de clases para los campos cuadráticos reales se encuentra
en el apéndice \ref{ap:grupos-de-clases-cuadraticos-reales}. Aquí la situación
es mucho más misteriosa que con los campos imaginarios. Una conjetura de Gauss,
todavía abierta, afirma que $\Cl (\QQ (\sqrt{d})) = 0$ para un número infinito
de $d > 1$. Cohen y Lenstra predicen que $\Cl (\QQ (\sqrt{p})) = 0$ para
aproximadamente $75.445\%$ de los primos.  La densidad exacta (conjetural) es
\[ \frac{1}{2\,C_\infty\,\eta_\infty (2)}, \quad \text{donde }
   \eta_\infty (p) = \prod_{n \ge 1} \frac{1}{1 - p^{-n}}, \,
    C_\infty = \prod_{n \ge 2} \zeta (n). \]
Es algo que se puede verificar con la computadora, solo hay que tomar
\emph{bastantes} primos:
\begin{center}
  \begin{tabular}{ll}
    primos & $\Cl (K) = 0$ \\
    \hline
    $100$    & $91$ \\
    $1000$   & $845$ \\
    $10000$  & $7970$ \\
    $100000$ & $77962$ \\
    $1000000$ & $769230$
  \end{tabular}
\end{center}
Cohen y Lenstra tienen otras conjeturas de este tipo sobre la estructura de
grupos de clase; para más detalles, consulte el artículo
\cite{Cohen-Lenstra-1984}.

\vspace{1em}

En general, para un campo de números específico $K$, hay un algoritmo para
calcular $\Cl (K)$. Sin embargo, si nos interesa el comportamiento de grupos de
clases para una familia de campos de números, no se sabe mucho. Por ejemplo, no
se conoce \emph{ninguna} familia infinita de campos de números con
$\Cl (K) = 0$. Como mencionamos, esta es una gran conjetura para los campos
cuadráticos reales.

%%%%%%%%%%%%%%%%%%%%%%%%%%%%%%%%%%%%%%%%%%%%%%%%%%%%%%%%%%%%%%%%%%%%%%%%%%%%%%%%

\section{Perspectiva: campos ciclotómicos}

Para los campos ciclotómicos $K = \QQ (\zeta_n)$ nuestros cálculos con la cota
de Minkowski ya no serán muy prácticos porque los discriminantes $\Delta_K$
serán bastante grandes. He aquí un pequeño ejemplo que sí es posible hacer
a mano.

\begin{ejemplo}
  Consideremos el campo ciclotómico $K = \QQ (\zeta_7)$. En este caso
  $n = 6$, $r_2 = 3$, y $\Delta_K = -7^5$, así que la cota de Minkowski
  viene dada por
  $$M_K = \frac{6!}{6^6} \, \left(\frac{4}{\pi}\right)^3 \, 7^{5/2} \approx 4.12.$$
  En este caso el orden de $p = 2$ mód $7$ es igual a $f = 3$, así que
  $$2\O_K = \mathfrak{p}_2 \, \mathfrak{p}_2'.$$
  Por otra parte, el orden de $p = 3$ mód $7$ es igual a $f = 6$, lo que
  significa que $p = 3$ es inerte en $K$.

  Factorizando el polinomio ciclotómico, se obtiene
  $$\Phi_7 (x) \equiv (x^3 + x + 1)\,(x^3 + x^2 + 1) \pmod{2}.$$
  Entonces,
  \[ \mathfrak{p}_2 = (2, 1 + \zeta_7 + \zeta_7^3), \quad
     \mathfrak{p}_2' = (2, 1 + \zeta_7^2 + \zeta_7^3). \]
  Afirmamos que estos ideales son principales. En efecto,
  $$(1 + \zeta_7 + \zeta_7^3)\,(1 + \zeta_7^2 + \zeta_7^3) = 2\zeta_7^3,$$
  así que
  \[ \mathfrak{p}_2 = (1 + \zeta_7 + \zeta_7^3), \quad
     \mathfrak{p}_2' = (1 + \zeta_7^2 + \zeta_7^3). \]
  Esto demuestra que $\Cl (K) = 0$.
\end{ejemplo}

De hecho, Kummer descubrió que el primer campo ciclotómico con el grupo de
clases no trivial es $K = \QQ (\zeta_{23})$, donde $\Cl (K) \cong \ZZ/3\ZZ$.
Como generador, se puede tomar uno de los ideales primos sobre $p = 2$,
o también uno de los $22$ ideales no principales sobre $p = 47$ que encontramos
en \ref{ejemplo:Q-zeta-23-y-47}.

Las cotas de Minkowski para los campos ciclotómicos son las siguientes.

\begin{center}\renewcommand{\arraystretch}{1.5}
  \begin{tabular}{rccccccccccc}
    \hline
    $K\colon$ & $\QQ (\zeta_3)$ & $\QQ (\zeta_4)$ & $\QQ (\zeta_5)$ & $\QQ (\zeta_7)$ & $\QQ (\zeta_8)$ & $\QQ (\zeta_9)$ & $\QQ (\zeta_{11})$ & $\QQ (\zeta_{12})$ & $\QQ (\zeta_{13})$ & $\cdots$ & $\QQ (\zeta_{23})$ \tabularnewline
    \hline
    $M_K\colon$ & $1.10$ & $1.27$ & $1.70$ & $4.13$ & $2.43$ & $4.47$ & $58.96$ & $1.82$ & $306.42$ & $\cdots$ & $9324406.48$ \tabularnewline
    \hline
    & & & & & & & & & & & ¡oops! \tabularnewline
  \end{tabular}
\end{center}

En el apéndice \ref{ap:grupos-de-clases-ciclotomicos} se encuentra una pequeña
tabla de grupos de clases para los campos ciclotómicos. Se sabe que el grupo
$\Cl (\QQ (\zeta_n))$ es trivial precisamente para
$$n = 1, \, 3, \, 4, \, 5, \, 7, \, 8, \, 9, \, 11, \, 12, \, 13, \, 15, \, 16, \, 17, \, 19, \, 20, \, 21, \, 24, \, 25, \, 27, \, 28, \, 32, \, 33, \, 35, \, 36, \, 40, \, 44, \, 45, \, 48, \, 60, \, 84.$$
Aquí para evitar las redundancias, se consideran $n \not\equiv 2 \pmod{4}$
(sino, $\QQ (\zeta_n) = \QQ (\zeta_{n/2})$ para $n \equiv 2 \pmod{4}$).
Para la prueba véase \cite[Chapter 11]{Washington-GTM83}.

\vspace{1em}

La \textbf{teoría de Iwasawa} es una rama de la teoría de números que se origina
en el estudio de grupos de clases de campos ciclotómicos. Un texto introductorio
en español sobre el tema es \cite{Futterer-Villanueva}.

%%%%%%%%%%%%%%%%%%%%%%%%%%%%%%%%%%%%%%%%%%%%%%%%%%%%%%%%%%%%%%%%%%%%%%%%%%%%%%%%

\section{Campos con número de clases 2}
\marginpar{\small Lectura\\ adicional}

Ya sabemos que el anillo de enteros $\O_K$ es un dominio de factorización única
si y solamente si se tiene ${h_K = 1}$. Recordemos que la factorización única
significa que para todo elemento no nulo $\alpha \in \O_K$, si hay dos
factorizaciones
\[ \tag{*} \alpha = \pi_1 \cdots \pi_r = \pi_1' \cdots \pi_s', \]
donde los $\pi_i$ y $\pi_j$ son irreducibles, entonces $r = s$, y salvo una
permutación, los factores son asociados: $\pi_i \sim \pi_i'$ (lo que equivale a
$\pi_i \O_K = \pi_i' \O_K$). Supongamos ahora que $\O_K$ no tiene factorización
única, pero pidamos una propiedad más débil: dada una factorización en
elementos irreducibles (*), se tiene $r = s$. La caracterización de esta
propiedad fue obtenida por Leonard Carlitz en un breve artículo
\cite{Carlitz-1960}.

\begin{teorema}
  La unicidad de longitud de factorización se cumple si y solamente si
  $h_K = 1,2$.
\end{teorema}

Ya conocemos el caso de $h_K = 1$, así que vamos a suponer que
$h_K > 1$. Primero asumamos que $h_K = 2$. Empecemos por la siguiente
observación.

\begin{lema}
  Sea $K/\QQ$ un campo de números con $h_K = 2$. Supongamos que $\pi \in \O_K$
  es un elemento irreducible que no es primo. Entonces,
  $\pi\O_K = \mathfrak{q}\,\mathfrak{q}'$, donde
  $\mathfrak{q}$ y $\mathfrak{q}'$ son ideales primos no principales
  (no necesariamente distintos).

  \begin{proof}
    Consideramos una factorización en ideales primos
    $$\pi\O_K = \mathfrak{p}_1 \cdots \mathfrak{p}_r\,\mathfrak{q}_1 \cdots \mathfrak{q}_s,$$
    donde los ideales $\mathfrak{p}_i = \pi_i \O_K$ son principales y los
    $\mathfrak{q}_j$ no son principales. Por nuestra hipótesis de que $h_K = 2$,
    todos $\mathfrak{q}_j$ representan el mismo elemento no trivial
    $[\mathfrak{q}]$ en el grupo de clases. La factorización de arriba nos dice
    que $[\mathfrak{q}]^s = [\O_K]$, y luego $s$ es necesariamente par, así
    que la factorización tiene forma
    $$\pi\O_K = \mathfrak{p}_1 \cdots \mathfrak{p}_r\,\mathfrak{q}_1 \mathfrak{q}_1'\, \cdots \mathfrak{q}_t \mathfrak{q}_t'$$
    (donde $t = s/2$). Pero ahora
    $[\mathfrak{q}_j \mathfrak{q}_j'] = [\mathfrak{q}]^2 = [\O_K]$,
    así que los ideales $\mathfrak{q}_j \mathfrak{q}_j' = \rho_j \O_K$ son
    principales. Entonces, se obtiene una factorización
    $$\pi \sim \pi_1 \cdots \pi_r\,\rho_1 \cdots \rho_t.$$
    Aquí $\pi$ es irreducible, y el ideal $\pi \O_K$ no es principal por nuestra
    hipótesis, así que se tiene $r = 0$, $t = 1$, y $\pi \sim \rho$. Esto nos da
    la factorización deseada $\pi \O_K = \mathfrak{q}\,\mathfrak{q}'$.
  \end{proof}
\end{lema}

Ahora ocupando el lema, supongamos que para $\alpha \ne 0$ hay dos
factorizaciones en elementos irreducibles
$$\alpha = \pi_1 \cdots \pi_r \rho_1 \cdots \rho_s = \pi_1' \cdots \pi_{r'}' \rho_1' \cdots \rho_{s'},$$
donde los $\pi_i$ son primos y $\rho_j$ no lo son. El lema nos da entonces una
factorización en ideales primos
$$\alpha \O_K = \mathfrak{p}_1 \cdots \mathfrak{p}_r\,\mathfrak{q}_1 \cdots \mathfrak{q}_{2s} = \mathfrak{p}_1' \cdots \mathfrak{p}_{r'}'\,\mathfrak{q}_1' \cdots \mathfrak{q}_{2s'}',$$
donde los $\mathfrak{p}_i$ son ideales principales y $\mathfrak{q}_j$ no lo son.
Ahora por la unicidad de factorizaciones en ideales primos, esto implica que
$r' = r$ y $s' = s$.

\vspace{1em}

Esta es una de las implicaciones del teorema de Carlitz, y la otra es más
difícil: para esto necesitamos usar que toda clase en $\Cl (K)$ puede ser
representada por un ideal primo (véase el comentario
\ref{com:clases-representadas-por-ideales-primos}).

Nos gustaría probar que si para las factorizaciones irreducibles en $\O_K$ se
tiene la unicidad de longitud, entonces $h_K \le 2$. Para esto asumamos que
$h_K > 2$. Esto nos deja dos posibilidades: o en el grupo de clases existe un
elemento de orden $> 2$, o hay por lo menos dos distintos elementos de orden $2$.

Primero, supongamos que algún elemento tiene orden $n > 2$. En este caso por el
resultado mencionado, existen ideales primos $\mathfrak{p}_1$ y
$\mathfrak{p}_2$, donde $[\mathfrak{p}_i]$ tiene orden $n$
y $[\mathfrak{p}_2] = [\mathfrak{p}_1]^{-1}$, así que
$$\mathfrak{p}_1^n = \pi_1 \O_K, \quad \mathfrak{p}_2^n = \pi_2 \O_K, \quad \mathfrak{p}_1\,\mathfrak{p}_2 = \rho \O_K.$$
Aquí los elementos $\pi_1, \pi_2, \rho$ son necesariamente irreducibles. Por
ejemplo, si $\pi_1 = \alpha\,\beta$ con $\alpha,\beta \notin \O_K^\times$,
entonces
$\pi_1 \O_K = \alpha \O_K \, \beta \O_K = \mathfrak{p}_1^\ell \mathfrak{p}_1^m$,
donde $\mathfrak{p}_1^\ell = \alpha \O_K$ y $\mathfrak{p}_1^m = \beta \O_K$,
$\ell,m > 0$ y $\ell + m = n$. Sin embargo, esto contradice el hecho de que
$n$ es el orden de $\mathfrak{p}_1$ en el grupo de clases. Se obtiene entonces
$\rho^n \sim \pi\,\pi'$, y estas son dos factorizaciones en elementos
irreducibles de diferente longitud.

Supongamos ahora que en el grupo de clases hay dos elementos de orden $2$.
En este caso tenemos tres ideales primos
$\mathfrak{p}_1,\mathfrak{p}_2,\mathfrak{p}_3$ donde
$[\mathfrak{p}_i]$ son diferentes elementos de orden $2$ y
$[\mathfrak{p}_3] = [\mathfrak{p}_1\mathfrak{p}_2]$, así que
\[ \mathfrak{p}_1^2 = \pi_1 \O_K, \quad
   \mathfrak{p}_2^2 = \pi_2 \O_K, \quad
   \mathfrak{p}_3^2 = \pi_3 \O_K, \quad
   \mathfrak{p}_1\mathfrak{p}_2\mathfrak{p}_3 = \pi \O_K, \]
donde de nuevo se puede verificar que $\pi_1, \pi_2, \pi_3$ son elementos
irreducibles. Pero ahora $\pi^2 \sim \pi_1\pi_2\pi_3$.

Esto concluye la prueba del teorema de Carlitz. \qed

\begin{ejemplo}
  El grupo de clases de $K = \QQ (\sqrt{-14})$ es isomorfo a $\ZZ/4\ZZ$,
  y podemos tomar como el generador la clase de
  $\mathfrak{p}_3 = (3, 1 + \sqrt{-14})$. En este caso
  $[\mathfrak{p}_3]^{-1} = [\overline{\mathfrak{p}_3}]$, y tenemos
  \[ \mathfrak{p}_3^4 = (5 + 2\sqrt{-14}), \quad
     \overline{\mathfrak{p}_3}^4 = (5 - 2\sqrt{-14}), \quad
     \mathfrak{p}_3 \mathfrak{p}_3 = 3 \O_K. \]
  Analizando las normas, no es difícil ver directamente que los elementos $3$ y
  $5 \pm 2\sqrt{-14}$ son irreducibles en $\O_K = \ZZ [\sqrt{-14}]$. Ahora
  $$3^4 = (5 + 2\sqrt{-14})\,(5 - 2\sqrt{-14})$$
  son dos factorizaciones irreducibles de diferente longitud.
\end{ejemplo}

\begin{ejemplo}
  El grupo de clases de $K = \QQ (\sqrt{-21})$ es isomorfo a
  $\ZZ/2\ZZ \oplus \ZZ/2\ZZ$. En este caso
  \[ 2\O_K = \mathfrak{p}_2^2, \quad
     3\O_K = \mathfrak{p}_3^2, \quad
     5\O_K = \mathfrak{p}_5 \, \overline{\mathfrak{p}_5}, \]
  y $[\mathfrak{p}_2], [\mathfrak{p}_3], [\mathfrak{p}_5]$ son diferentes clases
  no triviales: analizando $N_{K/\QQ} (a + b\sqrt{-21}) = a^2 + 21 b^2$,
  se ve que no hay elementos de norma $2,3,5$.

  Se tiene $[\mathfrak{p}_1\,\mathfrak{p}_2] = [\mathfrak{p}_5]$.
  Específicamente,
  $\mathfrak{p}_2 = (2, 1 + \sqrt{-21})$,
  $\mathfrak{p}_3 = (3, \sqrt{-21})$,
  $\mathfrak{p}_5 = (5, 2 + \sqrt{-21})$, y calculamos que
  \[ \mathfrak{p}_2\,\mathfrak{p}_3\,\mathfrak{p}_5 = (3 - \sqrt{-21}), \quad
     \mathfrak{p}_5^2 = (2 + \sqrt{-21}). \]
  Los elementos $2$, $3$, $5$, $2 + \sqrt{-21}$, $3 - \sqrt{-21}$
  son irreducibles: esto se sigue del hecho de que en $\O_K$ no hay elementos de
  norma $2, 3, 5, 6, 10, 15$. Tenemos entonces factorizaciones irreducibles de
  diferente longitud
  \[ (3 - \sqrt{-21})^2 = 2\cdot 3\cdot (-2 - \sqrt{-21}). \qedhere \]
\end{ejemplo}

%%%%%%%%%%%%%%%%%%%%%%%%%%%%%%%%%%%%%%%%%%%%%%%%%%%%%%%%%%%%%%%%%%%%%%%%%%%%%%%%

\iffalse
\section{Ecuación de Pell}

Antes de probar el teorema de unidades de Dirichlet, vamos a ver su caso
particular relacionado con los campos cuadráticos reales.

\begin{teorema}
  \label{thm:pell-minkowski}
  Sea $d > 1$ un entero libre de cuadrados. La ecuación
  $$x^2 - dy^2 = 1$$
  tiene una solución entera distinta de $(\pm 1, 0)$.

  \begin{proof}
    Consideremos el campo de números $K = \QQ (\sqrt{d})$. Nuestra ecuación
    puede ser interpretada como
    $$N_{K/\QQ} (x + y\sqrt{d}) = 1.$$
    Vamos a encajar el anillo de números $\ZZ [\sqrt{d}]$ como un retículo
    $\Lambda$ en $\RR^2$ mediante
    $$a + b\sqrt{d} \mapsto (a + b\sqrt{d}, a - b\sqrt{d}).$$
    Respecto a este encaje, las soluciones que nos interesan son los puntos
    del retículo $\Lambda$ en la hipérbola $xy = 1$. Sea $X$ un conjunto
    convexo simétrico suficientemente grande para que
    $X \cap \Lambda \ne \{ 0 \}$. Según el teorema de Minkowski, hay que
    tomar $X$ tal que $\vol X > 4 \covol \Lambda$.

    Ahora para todo $\lambda > 0$ podemos considerar el conjunto
    $X_\lambda = (\lambda, \lambda^{-1})\,X$. Este tiene el mismo volumen que
    $X$, así que $X_\lambda \cap \Lambda \ne \{ 0 \}$
    (véase la figura~\ref{fig:pell-minkowski})

    De esta manera para todo $\lambda > 0$ se obtiene un elemento
    $\alpha_\lambda \in \ZZ [\sqrt{d}]$. Los puntos que corresponden a
    $X \cap \Lambda$ tienen norma acotada, y la transformación
    $(\lambda, \lambda^{-1})$ preserva la cota sobre la norma. Entonces,
    tenemos $|N_{K/\QQ} (\alpha_\lambda)| \le C$ para todo $\alpha_\lambda$.
    Hay un número finito de ideales de norma acotada, así que existen
    $\lambda \ne \lambda'$ tales que $(\alpha_\lambda) = (\alpha_{\lambda'})$
    y $\alpha_\lambda \ne \pm\alpha_{\lambda'}$. En este caso
    $u = \alpha_\lambda/\alpha_{\lambda'}$ es una unidad en el anillo
    $\ZZ [\sqrt{d}]$ distinta de $\pm 1$. Tenemos $N_{K/\QQ} (u) = \pm 1$.
    Si $N_{K/\QQ} (u) = -1$, entonces $N_{K/\QQ} (u^2) = +1$. De esta manera
    se obtiene una solución no trivial de la ecuación de Pell.
  \end{proof}
\end{teorema}

\begin{figure}
  \begin{center}
    \includegraphics{pic/pell-minkowski.pdf}
  \end{center}

  \caption{Argumento de \ref{thm:pell-minkowski}}
  \label{fig:pell-minkowski}
\end{figure}

\begin{corolario}
  El grupo de unidades $\ZZ [\sqrt{d}]^\times$ es infinito.

  \begin{proof}
    Hemos encontrado una unidad $u \in \ZZ [\sqrt{d}]^\times$ distinta de
    $\pm 1$, y luego $u^n$ para $n \in \ZZ$ son distintas unidades.
  \end{proof}
\end{corolario}

%%%%%%%%%%%%%%%%%%%%%%%%%%%%%%%%%%%%%%%%%%%%%%%%%%%%%%%%%%%%%%%%%%%%%%%%%%%%%%%%

\section{Teorema de unidades de Dirichlet}

**TODO**
\fi

%%%%%%%%%%%%%%%%%%%%%%%%%%%%%%%%%%%%%%%%%%%%%%%%%%%%%%%%%%%%%%%%%%%%%%%%%%%%%%%%

\pagebreak

\phantomsection

\addcontentsline{toc}{section}{Ejercicios}
\section*{Ejercicios}

\begin{ejercicio}
  Demuestre que si $G$ es un grupo topológico Hausdorff, entonces todo subgrupo
  discreto $H \subset G$ es cerrado.
\end{ejercicio}

\begin{ejercicio}
  Demuestre directamente que $\ZZ [\sqrt{2}]$ no es un subgrupo discreto de
  $\RR$.
\end{ejercicio}

\begin{ejercicio}
  Demuestre que si $X$ es un conjunto convexo simétrico compacto tal que
  $\vol X = 2^n\cdot \covol \Lambda$, entonces $X \cap \Lambda \ne \{ 0 \}$.
\end{ejercicio}

\begin{ejercicio}
  Demuestre que para todo primo $p$ existen $m,n \in \ZZ$ tales que
  $m^2 + n^2 + 1 \equiv 0 \pmod{p}$.
\end{ejercicio}

\begin{ejercicio}
  Demuestre que
  $$a_n = \left(\frac{n^n}{n!}\right)^2\,\left(\frac{\pi}{4}\right)^n$$
  crece con $n$.
\end{ejercicio}

\begin{ejercicio}
  Supongamos que $d = p_1\cdots p_s$, donde $s > 1$ y los $p_i$ son diferentes
  primos y consideremos el campo cuadrático imaginario $K = \QQ (\sqrt{-d})$.
  Demuestre que los ideales correspondientes
  $\mathfrak{p}_1,\ldots,\mathfrak{p}_s \subset \O_K$ generan un subgrupo en
  $\Cl (K)$ isomorfo a $(\ZZ/2\ZZ)^{s-1}$.
\end{ejercicio}

\begin{ejercicio}
  Demuestre que para $K = \QQ (\sqrt{21})$ se tiene $\Cl (K) = 0$.
\end{ejercicio}

\begin{ejercicio}
  Sea $K/\QQ$ un campo de números. Demuestre que para cualquier ideal
  $I \subset \O_K$ existe una extensión finita $L/K$ tal que el ideal
  correspondiente $I\,\O_L$ es principal.

  \noindent (Use que $[I]$ tiene orden finito en $\Cl (K)$.)
\end{ejercicio}

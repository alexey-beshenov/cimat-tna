\chapter{Teoría de Minkowski}

\pdfbookmark{Clase 19 (19/10/20)}{clase-19}
\marginpar{\small Clase 19 \\ 19/10/20}

El objetivo de este capítulo es probar los siguientes teoremas.

\begin{enumerate}
\item[1)] Para una extensión no trivial $K/\QQ$ se tiene $|\Delta_K| > 1$.

\item[2)] \textbf{Teorema de Hermite}: para $C$ fijo, existe un número finito de
  campos de números $K/\QQ$, salvo isomorfismo, con discriminante
  $|\Delta_K| < C$.

\item[3)] Dado un campo de números $K/\QQ$, el grupo de clases
  $\Cl (K) = \Pic (\O_K)$ es finito.

  Este resultado es bastante sutil y no se sigue del álgebra conmutativa:
  si en lugar de $\O_K$ se toma otro dominio de Dedekind $R$, el grupo
  $\Pic (R)$ ya no tiene por qué ser finito.

\item[4)] \textbf{Teorema de unidades de Dirichlet}: el grupo $\O_K^\times$ es
  finitamente generado de rango $r_1 + r_2 - 1$, donde $r_1$ es el número de
  encajes reales $K\hookrightarrow \RR$ y $2r_2$ es el número de encajes
  complejos $K\hookrightarrow \CC$. En otras palabras, existen unidades
  $\epsilon_1, \ldots, \epsilon_{r_1 + r_2 - 1} \in \O_K^\times$,
  llamadas \textbf{unidades fundamentales}, tales que
  $$\O_K^\times \cong \mu_K \times \langle\epsilon_1\rangle \times \cdots \times \langle\epsilon_{r_1+r_2-1}\rangle.$$
  Aquí $\mu_K$ es el subgrupo de torsión que consiste en las raíces de unidad
  en $K$, mientras que $\epsilon_i$ son generadores de diferentes componentes
  cíclicas infinitas.
\end{enumerate}

La herramienta principal en las pruebas será el teorema de Minkowski sobre
puntos de retículos en conjuntos convexos simétricos. El término clásico para
los resultados de Minkowski es «geometría de números»
(\emph{Geometrie der Zahlen}), pero hoy en día el punto de vista geométrico a
la teoría de números está contenido más bien en la teoría de esquemas (véase por
ejemplo \cite{Eisenbud-Harris} y \cite{Gortz-Wedhorn}). Por esto un nombre más
adecuado para este capítulo sería «teoría de Minkowski».

%%%%%%%%%%%%%%%%%%%%%%%%%%%%%%%%%%%%%%%%%%%%%%%%%%%%%%%%%%%%%%%%%%%%%%%%%%%%%%%%

\section{Retículos y el teorema de Minkowski}

\begin{definicion}
  Sea $V$ un espacio vectorial real. Un \textbf{retículo} en $V$ es un subgrupo
  aditivo $\Lambda \subset V$ de la forma
  $$\Lambda = \ZZ\,\omega_1 + \cdots + \ZZ\,\omega_n,$$
  donde $\omega_1,\ldots,\omega_n \in V$ son vectores linealmente
  independientes. En este caso se dice que $\omega_1,\ldots,\omega_n$ es una
  \textbf{base} de $\Lambda$. Si $n = \dim_\RR V$, se dice que $\Lambda$
  \textbf{tiene rango completo} en $V$. El conjunto
  $$\Pi = \Bigl\{ \sum_i \lambda_i\,\omega_i \Bigm| 0 \le \lambda_i < 1 \Bigr\}$$
  se llama un \textbf{dominio fundamental} de $\Lambda$.
\end{definicion}

\begin{comentario}
  Si $\Lambda$ tiene rango completo, entonces $V$ puede ser recubierto por el
  dominio fundamental $\Pi$ trasladado por los elementos de $\Lambda$:
  $$V = \bigsqcup_{\omega \in \Lambda} \Pi + \omega.$$
  Esta unión es disjunta. El dominio fundamental se identifica con el cociente
  $V/\Lambda$.
\end{comentario}

\begin{comentario}
  Vamos a considerar $V$ como un espacio vectorial topológico, dotado de la
  topología real estándar. Para las aplicaciones que nos interesan, se puede
  pensar que $V = \RR^n$.
\end{comentario}

\begin{ejemplo}
  Consideremos los enteros de Eisenstein
  $\ZZ [\zeta_3] \subset \CC$. Identificando $\CC$ con $\RR^2$ de la manera
  habitual, podemos ver $\ZZ [\zeta_3]$ como un retículo generado por los
  vectores $\omega_1 = (1,0)$ y
  $\omega_2 = \left(-\frac{1}{2}, \, \frac{\sqrt{3}}{2}\right)$.

  \begin{center}
    \includegraphics{pic/eisenstein-integers-lattice.pdf}
  \end{center}
\end{ejemplo}

\begin{ejemplo}
  El subgrupo $\Lambda = \ZZ\cdot 1 + \ZZ\cdot \sqrt{2}$ no es un retículo en
  $\RR$: los vectores $1$ y $\sqrt{2}$ no son linealmente independientes.
\end{ejemplo}

\begin{lema}
  Un retículo $\Lambda \subset V$ tiene rango completo si y solamente si existe
  un conjunto acotado $X \subseteq V$ tal que
  \[ \tag{*} V = \bigcup_{\omega \in \Lambda} X + \omega. \]

  \begin{proof}
    Si $\Lambda = \ZZ\,\omega_1 + \cdots + \ZZ\,\omega_n$ tiene rango completo,
    entonces podemos tomar como $X$ el dominio fundamental $\Pi$.

    Viceversa, si existe un conjunto acotado $X$ tal que se cumple
    (*), denotemos por $V_0$ el subespacio vectorial generado por los elementos
    de $\Lambda$. Nos gustaría ver que $V_0 = V$. Para todo $v \in V$ y
    $k \in \NN$ podemos escribir $k v = x_k + \omega_k$ para algunos $x_k \in X$
    y $\omega_k \in \Lambda$. Puesto que $X$ es acotado,
    $$\lim_{k\to\infty} \frac{1}{k}\,x_k = 0.$$
    Ahora tenemos
    $$v = \lim_{k\to\infty} \frac{1}{k}\,\omega_k \in V_0,$$
    usando que $V_0$ es un subespacio cerrado de $V$.
  \end{proof}
\end{lema}

Aunque nuestra definición de retículos menciona explícitamente una $\ZZ$-base de
$\Lambda$, hay otra caracterización más canónica.

\begin{lema}
  Un subgrupo aditivo $\Lambda \subset V$ es un retículo si y solamente
  si $\Lambda$ es discreto.
\end{lema}

Recordemos que $\Lambda \subset V$ es un subespacio \textbf{discreto} si para
todo $\omega \in \Lambda$ existe un entorno abierto $U \ni \omega$ en $V$ tal
que $\Lambda \cap U = \{ \omega \}$.

\begin{ejemplo}
  Para $\Lambda = \ZZ\cdot 1 + \ZZ\cdot \sqrt{2}$, usando la irracionalidad de
  $\sqrt{2}$, se puede ver que para cualquier $\epsilon > 0$ existe un número
  infinito de $a + b\sqrt{2} \in \ZZ [\sqrt{2}]$ tales que
  $|a + b\sqrt{2}| < \epsilon$.
\end{ejemplo}

\begin{proof}
  Primero, si $\Lambda = \ZZ\,\omega_1 + \cdots + \ZZ\,\omega_n$ es un retículo en
  $V$, entonces para todo punto $\omega = \sum_i a_i\,\omega_i \in \Lambda$
  podemos tomar el entorno abierto
  $$U = \Bigl\{ \sum_i \lambda_i\,\omega_i \Bigm| |a_i - \lambda_i| < 1 \Bigr\},$$
  y se cumple $\Lambda \cap U = \{ \omega \}$.

  Viceversa, supongamos que $\Lambda \subset V$ es un subgrupo discreto.
  En general, si $G$ es un grupo topológico de Hausdorff, entonces cualquier
  subgrupo discreto $H \subset G$ es cerrado. En nuestro caso particular,
  $\Lambda$ será cerrado. Sea $V_0$ el subespacio de $V$ generado por los
  elementos de $\Lambda$. Podemos entonces escoger una base de $V_0$ que
  consiste en elementos $\omega_1,\ldots,\omega_n \in \Lambda$. Esta base nos da
  un subretículo
  $$\Lambda_0 = \ZZ\,\omega_1 + \cdots + \ZZ\,\omega_n \subseteq \Lambda.$$
  Dado que $\Lambda_0$ tiene rango completo en $V_0$, tenemos
  $$V_0 = \bigcup_{\omega \in \Lambda_0} \Pi_0 + \omega,$$
  donde $\Pi_0$ es el dominio fundamental que corresponde a la base
  $\omega_1,\ldots,\omega_n$. Vamos a probar que el cociente $\Lambda/\Lambda_0$
  es finito. Sean $\omega_i \in \Lambda$ representantes de diferentes elementos
  en $\Lambda/\Lambda_0$. Escribamos
  $$\omega_i = x_i + \omega_{0i},$$
  donde $x_i \in \Pi_0$ y $\omega_{0i} \in \Lambda_0$. Aquí
  $x_i = \omega_i - \omega_{0i} \in \Lambda \cap \Pi_0$. El espacio
  $\Lambda \cap \Pi_0$ es discreto y acotado, así que es finito. Se sigue que
  el cociente $\Lambda/\Lambda_0$ es finito. Ahora
  $$\Lambda_0 \subseteq \Lambda \subseteq \frac{1}{[\Lambda : \Lambda_0]}\,\Lambda_0,$$
  y $\Lambda$ es también un grupo abeliano de rango $n$, así que admite una
  $\ZZ$-base finita $\omega_1', \ldots, \omega_n'$. Estos vectores son
  linealmente independientes sobre $\RR$ porque generan el espacio $V_0$ de
  dimensión $n$.
\end{proof}

Ahora supongamos que $V$ tiene estructura de espacio euclidiano; es decir, viene
con una forma bilineal definida positiva
$$\langle\cdot,\cdot\rangle\colon V\times V \to \RR.$$

\begin{definicion}
  Para un retículo $\Lambda = \ZZ\,\omega_1 + \cdots + \ZZ\,\omega_n$ el
  \textbf{covolumen} viene dado por
  $$\covol (\Lambda) = \vol (\Pi) = |\det (\langle\omega_i,\omega_j\rangle)|^{1/2}.$$
\end{definicion}

Notamos que el covolumen no depende de una base particular: para otra base
$\omega_1', \ldots, \omega_n'$ la matriz de cambio de base tiene determinante
$\pm 1$ (recuerde también nuestra discusión del discriminante en el
capítulo~\ref{ch:algebra-z-lineal}).

\begin{definicion}
  Sea $X \subseteq V$ un subconjunto.

  \begin{itemize}
  \item Se dice que $X$ es \textbf{simétrico} (respecto al origen) si para todo
    $x \in X$ se tiene $-x \in X$.

  \item Se dice que $X$ es \textbf{convexo} si para cualesquiera $x,y\in X$ la
    recta entre $x$ e $y$ también está en $X$:
    $$\{ \lambda\,y + (1-\lambda)\,x \mid 0 \le \lambda \le 1 \} \subseteq X.$$
  \end{itemize}
\end{definicion}

Todo conjunto convexo simétrico no vacío $X \subseteq V$ necesariamente contiene
el punto $0$. Ahora dado un retículo $\Lambda \subset V$, si $X$ es
suficientemente grande, entonces $X$ contiene otro punto de $\Lambda$ a parte de
$0$. Este es el contenido del siguiente teorema.

\begin{teorema}[{Minkowski\footnote{Hermann Minkowski (1864--1909)}}]
  Sean $\Lambda \subset V$ un retículo de rango completo y $X \subseteq V$
  un conjunto convexo simétrico tal que
  $$\vol X > 2^n\,\covol \Lambda.$$
  Entonces, $X$ contiene un punto no nulo de $\Lambda$.
\end{teorema}

En algún sentido, este resultado es una versión continua del principio
del palomar.

\begin{center}
  \includegraphics{pic/minkowski.pdf}
\end{center}

\begin{comentario}
  Para entender el significado del múltiplo $2^n$ en la cota del teorema,
  podemos considerar el hipercubo abierto con $2^n$ vértices en
  $(\pm 1, \pm 1, \ldots, \pm 1)$. Consideremos el retículo
  $\Lambda = \ZZ^n \subset \RR^n$. El volumen del cubo es
  $2^n = \covol \Lambda$, pero el cubo no contiene ningún punto de $\Lambda$
  salvo $0$.

  \begin{center}
    \includegraphics{pic/minkowski-2n.pdf}
  \end{center}

  Dejo como un ejercicio probar que si $X$ es compacto, entonces también
  funcionaría $\vol (X) = 2^n \, \covol \Lambda$.
\end{comentario}

Para demostrar el teorema, necesitamos el siguiente resultado auxiliar.

\begin{lema}[{Blichfeldt\footnote{Hans Frederick Blichfeldt (1873--1945), matemático danés.}}]
  Dado un conjunto medible $X\subset V$ tal que $\vol X > \covol \Lambda$,
  existen dos diferentes puntos $x,x'\in X$ tales que $x-x'\in \Lambda$.

  \begin{proof}
    Puesto que
    $$V = \bigsqcup_{\omega\in \Lambda} \Pi + \omega,$$
    tenemos
    $$X = \bigsqcup_{\omega\in \Lambda} X \cap (\Pi + \omega),$$
    así que
    $$\vol X = \sum_{\omega\in \Lambda} \vol (X \cap (\Pi + \omega)) = \sum_{\omega\in \Lambda} \vol ((X-\omega) \cap \Pi).$$
    Aquí los conjuntos $(X-\omega) \cap \Pi$ están en el dominio fundamental
    $\Pi$, y por nuestra hipótesis $\vol X > \vol \Pi$, así que podemos
    deducir que existen $\omega, \omega'\in \Lambda$ tales que
    $$(X-\omega) \cap (X-\omega') \ne \emptyset.$$
    Tomando $y \in (X-\omega) \cap (X-\omega')$, se obtiene
    \[ x = y + \omega, \quad x' = y + \omega' \in X, \quad
       x - x' = \omega-\omega' \in \Lambda. \qedhere \]
  \end{proof}
\end{lema}

Notamos que en realidad el lema no usa la covexidad de $X$ y se aplica a
cualquier conjunto medible, pero en particular, los conjuntos convexos son
medibles.

\vspace{1em}

Ahora estamos listos para demostrar el teorema de Minkowski, y aquí será
importante la hipótesis de que $X$ es convexo y simétrico. Consideremos
el conjunto
$$\frac{1}{2}\,X = \Bigl\{ \frac{1}{2}\,x \Bigm| x \in X \Bigr\}.$$
Tenemos
$$\vol \left(\frac{1}{2}\,X\right) = \frac{1}{2^n} \vol X > \covol \Lambda,$$
así que por el lema de Blichfeldt existen dos puntos distintos
$x,x' \in \frac{1}{2}\,X$ tales que $x-x' \in \Lambda$. Para terminar la
demostración, sería suficiente ver que este punto pertenece a $X$. Por la
hipótesis que $X$ es simétrico, $-x'\in \frac{1}{2}\,X$, y luego
$$x = \frac{1}{2}\,y, \quad -x' = \frac{1}{2}\,y' \quad\text{para algunos }y,y'\in X.$$
El punto
$$x - x' = \frac{1}{2}\,y + \frac{1}{2}\,y'$$
pertenece a $X$, siendo una combinación convexa de dos puntos $y,y'\in X$. \qed

\vspace{1em}

Notamos que el argumento no es constructivo y no nos da una manera eficaz de
obtener un punto no nulo en $X \cap \Lambda$.

%%%%%%%%%%%%%%%%%%%%%%%%%%%%%%%%%%%%%%%%%%%%%%%%%%%%%%%%%%%%%%%%%%%%%%%%%%%%%%%%

\section{Aplicación: teorema de cuatro cuadrados}

Prácticamente todo este capítulo será dedicado a varias aplicaciones del teorema
de Minkowski, pero primero sería instructivo ver algún ejemplo más sencillo de
su uso. En esta sección vamos a probar el siguiente famoso resultado, conocido
como el \textbf{teorema de cuatro cuadrados}.

\begin{teorema}[Lagrange]
  Para todo entero $n \ge 0$ existen $a,b,c,d \in \ZZ$ tales que
  $n = a^2 + b^2 + c^2 + d^2$.
\end{teorema}

Primero, gracias a la identidad
\begin{multline*}
  (a^2 + b^2 + c^2 + d^2)\cdot (x^2 + y^2 + z^2 + w^2) \\
  = (a x - b y - c z - d w)^2 +
    (a y + b x + c w - d z)^2 +
    (a z - b w + c x + d y)^2 +
    (a w + b z - c y + d x)^2,
\end{multline*}
descubierta por Euler, notamos que es suficiente probar el teorema para el caso
cuando $n = p$ es un número primo.

\begin{comentario}
  He aquí una breve explicación. La identidad para las sumas de dos cuadrados
  $$(a^2 + b^2)\,(x^2 + y^2) = (ax - by)^2 + (ay + bx)^2$$
  se sigue de la fórmula $N (\alpha\beta) = N (\alpha)\,N (\beta)$
  para la norma $N (\alpha) = \alpha\,\overline{\alpha}$ sobre los enteros de
  Gauss $\ZZ [i]$.

  De la misma manera, podemos considerar el \textbf{álgebra de cuaterniones} con
  coeficientes en $\ZZ$:
  $$\mathbb{H} (\ZZ) = \{ a + bi + cj + dk \mid a,b,c,d \in \ZZ \},$$
  donde la multiplicación está definida por
  \begin{gather*}
    i^2 = j^2 = k^2 = -1,\\
    ij = k, ~ ji = -k, \\
    jk = i, ~ kj = -i, \\
    ki = j, ~ ik = -j.
  \end{gather*}
  De manera más concreta, tenemos una representación fiel del álgebra de
  cuaterniones $\mathbb{H} (\ZZ)$ por las matrices $M_2 (\ZZ [i])$ dada por
  \[ (a + bi + cj + dk) \mapsto \begin{pmatrix}
    a + bi & c + di \\
    -c + di & a - bi
  \end{pmatrix}. \]

  Podemos definir el conjugado de $\alpha = a + bi + cj + dk$ mediante
  $$\overline{\alpha} = a - bi - cj - dk.$$
  La norma se define como
  $$N (\alpha) = \alpha\,\overline{\alpha} = a^2 + b^2 + c^2 + d^2,$$
  o en términos de matrices,
  \[ \det \begin{pmatrix}
    a + bi & c + di \\
    -c + di & a - bi
  \end{pmatrix} =  a^2 + b^2 + c^2 + d^2. \]
  La norma es multiplicativa: $N (\alpha\beta) = N (\alpha) \, N (\beta)$.
\end{comentario}

\begin{lema}
  Para todo primo $p$ existen $m,n\in\ZZ$ tales que
  $$m^2 + n^2 + 1 \equiv 0 \pmod{p}.$$

  \begin{proof}
    Ejercicio.
  \end{proof}
\end{lema}

Fijemos entonces $m$ y $n$ como arriba y consideremos los siguientes vectores en
$V = \RR^4$:
\[ \omega_1 = (1,0,m,n), \quad
   \omega_2 = (0,1,n,-m), \quad
   \omega_3 = (0,0,p,0), \quad
   \omega_4 = (0,0,0,p). \]
Consideremos el producto escalar estándar sobre $\RR^4$, y para un vector
$v = \sum_i a_i \, e_i$ pongamos
$\|v\|^2 = \langle v,v\rangle = \sum_i a_i^2$. Calculando el determinante
correspondiente, es fácil verificar que los $\omega_i$ generan un retículo de
rango completo
$$\Lambda = \ZZ\,\omega_1 + \ZZ\,\omega_2 + \ZZ\,\omega_3 + \ZZ\,\omega_4 \subset \RR^4,$$
tal que
$$\covol \Lambda = p^2.$$

\begin{lema}
  Para todo $\omega\in \Lambda$ el número $\|\omega\|^2$ es un entero
  divisible por $p$.

  \begin{proof}
    Si
    $$\omega = a_1\,\omega_1 + a_2\,\omega_2 + a_3\,\omega_3 + a_4\,\omega_4 = (a_1, ~ a_2, ~ a_1\,m+a_2\,n + a_3\,p, ~ a_1\,n-a_2\,m + a_4\,p),$$
    entonces

    \begin{multline*}
      \|\omega\|^2 = a_1^2 + a_2^2 + (a_1\,m+a_2\,n + a_3\,p)^2 + (a_1\,n-a_2\,m + a_4\,p)^2 \\
      \equiv a_1^2 + a_2^2 + (a_1\,m+a_2\,n)^2 + (a_1\,n-a_2\,m)^2 \pmod{p}.
    \end{multline*}

    Luego,
    $$a_1^2 + a_2^2 + (a_1\,m+a_2\,n)^2 + (a_1\,n-a_2\,m)^2 = (a_1^2 + a_2^2)\,(m^2 + n^2 + 1)$$
    y $m^2 + n^2 + 1 \equiv 0 \pmod{p}$ por nuestra elección de $m$ y $n$.
  \end{proof}
\end{lema}

Sea $X$ la bola abierta en $\RR^4$ de radio $r = \sqrt{2p}$ centrada en el
origen:
$$X = \{ x \in \RR^4 \mid \|x\|^2 < 2p \}.$$
Recordemos que en general la bola $n$-dimensional de radio $r$ tiene volumen
$$\frac{\pi^{n/2}}{\Gamma \left(\frac{n}{2}+1\right)}\,r^n.$$
En este caso $n = 4$ y
$\Gamma \left(\frac{n}{2}+1\right) = \Gamma (3) = 2! = 2$. Tenemos
$$\vol X = \frac{\pi^2 r^4}{2} = 2 \pi^2 p^2 > 2^4\,\covol \Lambda = 16\,p^2$$
(de hecho, $2 \pi^2 = 19.73\ldots > 16$). Entonces, según el teorema de
Minkowski, existe un punto no nulo $\omega\in \Lambda$ tal que
$\omega \in X$. Ahora de
$$0 < \|\omega\|^2 < 2p, \quad p \mid \|\omega\|^2$$
podemos concluir que $\|\omega\|^2 = p$. Esto nos da una representación de $p$
como una suma de cuatro cuadrados. \qed

\vspace{1em}

Cabe mencionar que la representación de números enteros por sumas de cuadrados
es un problema clásico relacionado con mucha teoría de números profunda. Véase
por ejemplo el libro \cite{Grosswald-1985}.

%%%%%%%%%%%%%%%%%%%%%%%%%%%%%%%%%%%%%%%%%%%%%%%%%%%%%%%%%%%%%%%%%%%%%%%%%%%%%%%%

\section{Aplicación: teorema de aproximación de Dirichlet}
\marginpar{\small Lectura\\ adicional}

El siguiente famoso teorema de Dirichlet es el primer resultado en la
\textbf{aproximación diofántica}.

\begin{teorema}
  Dados números reales $\alpha$ y $N \ge 1$, existen $p,q \in \ZZ$ tales que
  $1 \le q \le N$ y
  $$\Bigl|\alpha - \frac{p}{q}\Bigr| < \frac{1}{qN} \le \frac{1}{q^2}.$$

  \begin{proof}
    Consideremos el conjunto convexo simétrico
    $$X = \Bigl\{ (x,y) \in \RR^2 \Bigm| |x| \le N + \frac{1}{2}, \, |\alpha x - y| < \frac{1}{N} \Bigr\}.$$

    \begin{center}
      \includegraphics{pic/dirichlet-approximation-minkowski.pdf}
    \end{center}

    Este es un paralelogramo de área $(2N + 1)\,\frac{2}{N} > 4$, y el teorema
    de Minkowski implica que existe un punto no nulo $(q,p) \in \ZZ^2$ tal que
    $(q,p) \in X$. Por la simetría, podemos asumir que $q \ge 1$. Tenemos
    entonces
    \[ 1 \le q \le N, \quad |q \alpha - p| < \frac{1}{N}. \qedhere \]
  \end{proof}
\end{teorema}

\begin{corolario}
  \label{cor:aproximaciones-racionales}
  Para $\alpha$ irracional existe un número infinito de fracciones $\frac{p}{q}$
  tales que
  $$\Bigl|\alpha - \frac{p}{q}\Bigr| < \frac{1}{q^2}.$$

  \begin{proof}
    Primero, el teorema de Dirichlet nos da una aproximación
    $$\Bigl|\alpha - \frac{p}{q}\Bigr| < \frac{1}{q N} \le \frac{1}{q^2}.$$
    Notamos que siempre podemos asumir que $\gcd (p,q) = 1$. Escojamos
    $$N' > \cfrac{1}{\Bigl|\alpha - \cfrac{p}{q}\Bigr|}$$
    y tomamos otra aproximación
    $$\Bigl|\alpha - \frac{p'}{q'}\Bigr| < \frac{1}{q'\,N'} = \cfrac{\Bigl|\alpha - \cfrac{p}{q}\Bigr|}{q'} \le \Bigl|\alpha - \frac{p}{q}\Bigr|.$$
    Aquí necesariamente $\frac{p}{q} \ne \frac{p'}{q'}$, y continuando de esta
    manera se obtienen diferentes $\frac{p_i}{q_i}$, donde
    $$0 < \Bigl|\alpha - \frac{p_n}{q_n}\Bigr| < \Bigl|\alpha - \frac{p_{n-1}}{q_{n-1}}\Bigr| < \cdots < \Bigl|\alpha - \frac{p_1}{q_1}\Bigr|$$
    y $\Bigl|\alpha - \frac{p_i}{q_i}\Bigr| < \frac{1}{q_i^2}$.
  \end{proof}
\end{corolario}

\begin{ejemplo}
  Para $\alpha = \sqrt{2}$ las fracciones
  $$\frac{p_n}{q_n} = \frac{1}{1}, ~ \frac{3}{2}, ~ \frac{7}{5}, ~ \frac{17}{12}, ~ \frac{41}{29}, ~ \frac{99}{70}, ~ \ldots$$
  cumplen la condición del corolario:
  \begin{center}\renewcommand{\arraystretch}{1.5}
    \begin{tabular}{rcccccc}
      \hline
      $\frac{p_n}{q_n}\colon$ & $\frac{1}{1}$ & $\frac{3}{2}$ & $\frac{7}{5}$ & $\frac{17}{12}$ & $\frac{41}{29}$ & $\frac{99}{70}$ \tabularnewline
      \hline
      $|\alpha - p_n/q_n|\colon$ & $0.414213$ & $0.085786$ & $0.014213$ & $0.002453$ & $0.000420$ & $0.000072$ \tabularnewline
      \hline
      $1/q_n^2\colon$ & $1.000000$ & $0.250000$ & $0.040000$ & $0.006944$ & $0.001189$ & $0.000204$ \tabularnewline
      \hline
      \end{tabular}
  \end{center}

\end{ejemplo}

Las aproximaciones de \ref{cor:aproximaciones-racionales} pueden ser obtenidas
mediante las \textbf{fracciones continuas} para $\alpha$. Vamos a revisarlas más
adelante porque estas tendrán una relación con el grupo de unidades
$\O_K^\times$ para $K = \QQ (\sqrt{d})$ campo cuadrático real.

\vspace{1em}

Para mayor información sobre la aproximación diofántica, véase por ejemplo
\cite{Schmidt-1980}.

%%%%%%%%%%%%%%%%%%%%%%%%%%%%%%%%%%%%%%%%%%%%%%%%%%%%%%%%%%%%%%%%%%%%%%%%%%%%%%%%

\pdfbookmark{Clase 20 (21/10/20)}{clase-20}
\section{Anillo de enteros como un retículo}
\marginpar{\small Clase 20 \\ 21/10/20}

Sea $K/\QQ$ un campo de números. Este tiene $n = [K:\QQ]$ encajes
$\tau\colon K\hookrightarrow \CC$, entre estos $r_1$ encajes reales
$\rho\colon K\hookrightarrow \RR$ y $2 r_2$ encajes complejos
$\sigma\colon K\hookrightarrow \CC$ tales que $\overline{\sigma}\ne\sigma$.
Consideremos el espacio complejo $n$-dimensional
$$K_\CC = \prod_\tau \CC$$
y el encaje correspondiente
\[ \Phi\colon K \hookrightarrow K_\CC, \quad
   \alpha \mapsto (\tau (\alpha))_\tau. \]

Vamos a dotar el espacio $K_\CC$ del producto hermitiano habitual
$$\langle z,z'\rangle = \sum_\tau z_\tau \, \overline{z'_\tau}.$$
En particular, notamos que $\langle z',z\rangle = \overline{\langle z,z'\rangle}$
y $\langle z,z\rangle > 0$ para $z \ne 0$.

El grupo $G_\RR = \Gal (\CC/\RR)$ actúa sobre $K_\CC$ mediante la conjugación
compleja y permutación de las coordenadas $\tau \mapsto \overline{\tau}$.
Esto nos da un automorfismo $\RR$-lineal de orden $2$
\[ F\colon K_\CC \to K_\CC, \quad
   (z_\tau)_\tau \mapsto (\overline{z}_{\overline{\tau}})_\tau. \]

Consideremos el subespacio real fijo por la acción de $G_\RR$:
$$K_\RR = (K_\CC)^{G_\RR} = \{ (z_\tau)_\tau \mid z_{\overline{\tau}} = \overline{z}_\tau \}.$$

Dado que $\langle F z, F z'\rangle = \overline{\langle z, z'\rangle}$, el producto
hermitiano sobre $K_\CC$ se restringe a un producto escalar
$$\langle\cdot,\cdot\rangle\colon K_\RR\times K_\RR\to \RR.$$
Efectivamente, para todo $x,y \in K_\RR$ se tiene
\begin{gather*}
  \langle x,y\rangle = \langle F x, F y\rangle = \overline{\langle x,y\rangle},\\
  \langle y,x\rangle = \overline{\langle x,y\rangle} = \langle x,y\rangle,\\
  \langle x,x\rangle > 0 \text{ si }x\ne 0.
\end{gather*}

Notamos que por la definición, para $\alpha \in K$ se tiene
$\overline{\tau} (\alpha) = \overline{\tau (\alpha)}$, y entonces el encaje
$\Phi\colon K\hookrightarrow K_\CC$ toma valores en $K_\RR$:
\[ \begin{tikzcd}
  K \ar[right hook->]{r}{\Phi}\ar[right hook->]{dr}{\Phi} & K_\RR\ar[right hook->]{d} \\
  & K_\CC
\end{tikzcd} \]

\begin{comentario}
  En términos de productos tensoriales,

  \begin{itemize}
  \item $K_\CC \cong K\otimes_\QQ \CC$,

  \item el encaje $\Phi\colon K \hookrightarrow K_\CC$ se identifica con
    $\alpha \mapsto \alpha\otimes 1$,

  \item la aplicación $F\colon K_\CC \to K_\CC$ corresponde a
    $\alpha \otimes z \mapsto \alpha \otimes \overline{z}$,

  \item $K_\RR \cong K\otimes_\QQ \RR$, y la inclusión
    $K_\RR \hookrightarrow K_\CC$ está inducida por $\RR \subset \CC$.
  \end{itemize}
\end{comentario}

\vspace{1em}

El siguiente resultado explica el significado geométrico del discriminante
$\Delta_K$.

\begin{proposicion}
  La imagen del anillo de enteros $\Lambda = \Phi (\O_K) \subset K_\RR$ es un
  retículo de rango completo tal que $\covol \Lambda = \sqrt{|\Delta_K|}$.

  \begin{proof}
    Si $\O_K = \ZZ\,\alpha_1 + \cdots + \ZZ\,\alpha_n$, entonces
    $\Lambda = \ZZ\,\Phi (\alpha_1) + \cdots + \ZZ\,\Phi (\alpha_n)$.
    Ahora si $\tau_i\colon K \hookrightarrow \CC$ son diferentes encajes,
    recordemos que
    $\Delta_K = \det (A)^2$, donde $A = (\tau_i \alpha_j)_{i,j}$.

    Por otra parte, el covolumen de $\Lambda$ se calcula mediante la matriz
    \[ (\langle \Phi (\alpha_i), \Phi (\alpha_j)\rangle)_{i,j} =
    \Bigl(\sum_k \tau_k \alpha_i, \overline{\tau_k \alpha_j}\Bigr)_{i,j} =
    A \, \overline{A}^t. \]

    Ahora
    \[ \covol \Lambda = \sqrt{|\det (A \overline{A}^t)|} = \sqrt{|\Delta_K|}. \qedhere \]
  \end{proof}
\end{proposicion}

\begin{corolario}
  Para todo ideal no nulo $I \subseteq \O_K$, la imagen correspondiente
  $\Lambda = \Phi (I)$ es un retículo de rango completo tal que
  $\covol \Lambda = \sqrt{|\Delta_K|}\cdot N_{K/\QQ} (I)$.

  \begin{proof}
    Si $I \ne 0$, entonces el índice
    $$N_{K/\QQ} (I) = \# (\O_K/I) = [\O_K : I]$$
    es finito, y luego $\Lambda = \Phi (I)$ es un subretículo en
    $\Lambda' = \Phi (\O_K)$ tal que $[\Lambda' : \Lambda] = N_{K/\QQ} (I)$.
    Dejo como un ejercicio verificar que
    \[ \covol \Lambda = \covol \Lambda' \cdot [\Lambda' : \Lambda]. \qedhere \]
  \end{proof}
\end{corolario}

El último cálculo explica el significado geométrico de la norma $N_{K/\QQ} (I)$.

\begin{ejemplo}
  \label{ejemplo:encaje-de-enteros-de-Eisenstein}
  Volvamos a los enteros de Eisenstein $\ZZ [\zeta_3] \subset \CC$.
  Hay dos encajes complejos
  $\sigma, \overline{\sigma}\colon \QQ (\zeta_3) \to \CC$ dados por
  $\sigma\colon \zeta_3 \mapsto \zeta_3$ y
  $\overline{\sigma}\colon \zeta_3 \mapsto \zeta_3^2$. En este caso particular
  $$K_\RR = \{ (z_\sigma, z_{\overline{\sigma}}) \in K_\CC \mid z_{\overline{\sigma}} = \overline{z_\sigma} \}.$$
  Esto nos da un isomorfismo de espacios vectoriales
  \[ \phi\colon K_\RR \xrightarrow{\cong} \RR^2, \quad
     (z_\sigma, z_{\overline{\sigma}}) \mapsto (x_\sigma, x_{\overline{\sigma}}) = (\Re z_\sigma, \Im z_\sigma). \]
  Poniendo
  $z_\sigma = x_\sigma + i y_\sigma$, $z'_\sigma = x'_\sigma + i y'_\sigma$,
  calculamos que
  \[ z_\sigma \, \overline{z'_\sigma} + z_{\overline{\sigma}} \, \overline{z'_{\overline{\sigma}}} =
     z_\sigma \, \overline{z'_\sigma} + \overline{z_\sigma} \, z'_\sigma =
     2\,(x_\sigma\,x'_\sigma + y_\sigma\,y'_\sigma) =
     2\,(x_\sigma\,x'_\sigma + x_{\overline{\sigma}}\,x'_{\overline{\sigma}}). \]
  Entonces, el producto escalar sobre $\RR^2$ que corresponde al producto escalar
  sobre $K_\RR$ es
  $$\langle x, x'\rangle = 2\,(x_\sigma\,x'_\sigma + x_{\overline{\sigma}}\,x'_{\overline{\sigma}}).$$
  Tenemos el encaje
  $$\Phi\colon \ZZ [\zeta_3] \hookrightarrow K_\RR \xrightarrow{\cong} \RR^2$$
  dado por
  \[ 1 \mapsto (1, 0), \quad
     \zeta_3 \mapsto (\Re \zeta_3, \Im \zeta_3). \]

  Ahora el volumen del retículo correspondiente será
  \[ \left|\det \begin{pmatrix}
    \langle \Phi (1), \Phi (1) \rangle & \langle \Phi (1), \Phi (\zeta_3) \rangle \\
    \langle \Phi (\zeta_3), \Phi (1) \rangle & \langle \Phi (\zeta_3), \Phi (\zeta_3) \rangle \\
  \end{pmatrix}\right|^{1/2} =
  \left|\det \begin{pmatrix}
    2 & 2 \Re \zeta_3 \\
    2 \Re \zeta_3 & 2\cdot |\zeta_3|^2
  \end{pmatrix}\right|^{1/2} = \sqrt{4 - 4\cdot \Re (\zeta_3)^2} = \sqrt{3}. \]
  En este caso $\Delta_K = -3$.
\end{ejemplo}

%%%%%%%%%%%%%%%%%%%%%%%%%%%%%%%%%%%%%%%%%%%%%%%%%%%%%%%%%%%%%%%%%%%%%%%%%%%%%%%%

\section{Cota de Minkowski}

\begin{comentario}
  Para ver el espacio $K_\RR$ de manera más explícita, sean
  $\rho_1, \ldots, \rho_{r_1}\colon K \hookrightarrow \RR$
  los encajes reales, y
  $\sigma_1, \, \overline{\sigma_1}, \ldots, \sigma_{r_2}, \, \overline{\sigma_{r_2}} \colon K \hookrightarrow \CC$
  los encajes complejos. Entonces,
  $$K_\RR = \{ (z_\tau) \in K_\CC \mid z_\rho \in \RR, \, z_{\overline{\sigma}} = \overline{z}_\sigma \}.$$
  Tenemos un isomorfismo de espacios $\RR$-vectoriales
  \[ \phi\colon K_\RR \xrightarrow{\cong} \RR^{r_1 + 2 r_2}, \quad
  (z_\tau)_\tau \mapsto (x_\tau)_\tau, \]
  donde
  \[ x_\rho = z_\rho, \quad
  x_\sigma = \Re (z_\sigma), \quad
  x_{\overline{\sigma}} = \Im (z_\sigma). \]

  Un cálculo similar al de \ref{ejemplo:encaje-de-enteros-de-Eisenstein}
  demuestra que el producto escalar correspondiente sobre $\RR^{r_1 + 2 r_2}$
  viene dado por
  $$\langle x,y\rangle = \sum_\tau n_\tau \, x_\tau \, y_\tau,$$
  donde
  \[ n_\tau = \begin{cases}
    1, & \text{si }\tau\text{ es real},\\
    2, & \text{si }\tau\text{ es complejo}.
  \end{cases} \]
  Este producto escalar define una medida sobre $\RR^{r_1 + 2r_2}$, respecto
  a cual
  $$\vol (X) = 2^{r_2}\cdot \vol_{Leb} (\phi (X)),$$
  donde $\vol_{Leb}$ denota el volumen respecto a la medida de Lebesgue
  habitual sobre $\RR^n$. A partir de ahora, cuando hablamos del volumen de un
  subconjunto de $K_\RR$, vamos a entender esta medida inducida por el producto
  escalar.
\end{comentario}

\begin{lema}
  Para $t > 0$ el conjunto convexo simétrico
  $$X_t = \{ (z_\tau) \in K_\RR \mid \sum_\tau |z_\tau| \le t \}$$
  tiene volumen
  $$\vol (X_t) = 2^{r_1}\,\pi^{r_2}\,\frac{t^n}{n!}.$$

  \begin{proof}
    Como vimos, tenemos $\vol (X_t) = 2^{r_2}\,\vol_{Leb} (\phi (X_t))$,
    y el conjunto $\phi (X_t)$ en $\RR^{r_1 + 2r_2}$ con coordenadas
    $(x_1, \ldots, x_{r_1}, \, y_1, z_1, \ldots, y_{r_2}, z_{r_2})$
    es el conjunto definido por la desigualdad
    $$|x_1| + \cdots + |x_{r_1}| + 2\sqrt{y_1^2 + z_1^2} + \cdots + 2\,\sqrt{y_{r_2} + z_{r_2}} \le t.$$
    Pasando a las coordenadas polares $y_i = u_i \, \cos \theta_i$,
    $z_i = u_i \, \sen \theta_i$, tenemos
    $$\vol_{Leb} (\phi (X_t)) = \int u_1 \cdots u_s\,dx_1 \cdots dx_{r_1}\,du_1 \cdots du_{r_2}\,d\theta_1 \cdots d \theta_{r_2},$$
    donde la integración es sobre el dominio
    \[ 0 \le \theta_i \le 2\pi, \quad u_i \ge 0, \quad
       |x_1| + \cdots + |x_{r_1}| + 2u_1 + \cdots + 2u_{r_2} \le t. \]
    Pasando a la integral sobre $x_i \ge 0$ y $2 u_i = w_i$,
    $$\vol_{Leb} (\phi (X_t)) = 2^{r_1}\,4^{-r_2}\,(2\pi)^{r_2}\,I_{r_1,r_2} (t),$$
    donde
    $$I_{r_1,r_2} (t) = \int w_1 \cdots w_{r_2} \, dx_1\cdots d x_{r_1} d w_1\cdots d w_{r_2},$$
    y la integral es sobre
    \[ x_i \ge 0, \quad w_i \ge 0, \quad
       x_1 + \cdots + x_{r_1} + w_1 + \cdots + w_{r_2} \le t. \]

    Tenemos
    $$I_{r_1,r_2} (t) = t^{r_1 + 2r_2} \, I_{r_1,r_2} (1) = t^n \, I_{r_1,r_2} (1).$$
    Reescribiendo el dominio como
    $$x_2 + \cdots + x_{r_1} + w_1 + \cdots + w_{r_2} \le t - x_1,$$
    tenemos por el teorema de Fubini
    $$I_{r_1, r_2} (1) = \int_0^1 I_{r_1-1,r_2} (1 - x_1)\,dx_1 = \int_0^1 (1 - x_1)^{n-1}\,dx_1 \cdot I_{r_1-1, r_2} (1) = \frac{1}{n}\,I_{r_1-1,r_2} (1),$$
    y entonces por inducción
    $$I_{r_1,r_2} = \frac{1}{n\,(n-1)\cdots (n - r_1 + 1)}\,I_{0,r_2} (1).$$
    De la misma manera
    $$I_{0,r_2} (1) = \int_0^1 w_1 \, (1-w_1)^{2r_2 - 2} \, dw_1 \, I_{0,r_2-1} (1),$$
    de donde por inducción
    $$I_{0,r_2} (1) = \frac{1}{(2r_2)!} \, I_{0,0} (1) = \frac{1}{(2r_2)!}.$$
    Entonces,
    $$I_{r_1, r_2} (1) = \frac{1}{n!},$$
    así que
    \[ \vol (X_t) = 2^{r_2} \cdot 2^{r_1}\,4^{-r_2}\,(2\pi)^{r_2}\,I_{r_1,r_2} (t) =
       2^{r_1}\,\pi^{r_2}\,\frac{t^n}{n!}. \qedhere \]
  \end{proof}
\end{lema}

\begin{teorema}[La cota de Minkowski]
  \label{thm:cota-de-minkowski}
  Dado un ideal no nulo $I \subseteq \O_K$, existe un elemento no nulo
  $\alpha \in I$, tal que
  $$|N_{K/\QQ} (\alpha)| \le M_K\cdot N_{K/\QQ} (I),$$
  donde
  $$M_K = \frac{n!}{n^n} \, \left(\frac{4}{\pi}\right)^{r_2} \, \sqrt{|\Delta_K|}$$
  es una constante que depende solamente de $K$, llamada la
  \textbf{cota de Minkowski}.

  \begin{proof}
    Consideremos el retículo $\Lambda = \Phi (I) \subset K_\RR$, y el conjunto
    convexo simétrico compacto $X_t \subset K_\RR$ del lema anterior, escogiendo
    $t$ tal que
    $$\vol (X_t) = 2^n \, \covol \Lambda;$$
    es decir,
    \[ 2^{r_1}\,\pi^{r_2}\,\frac{t^n}{n!} = 2^n\,\sqrt{|\Delta_K|} \cdot N_{K/\QQ} (I)
       \iff
       t^n = n!\,\left(\frac{4}{\pi}\right)^{r_2}\,\sqrt{|\Delta_K|}\cdot N_{K/\QQ} (I). \]
    En este caso $X_t \cap \Phi (I) \ne \{ 0 \}$; es decir, existe un elemento
    no nulo $\alpha \in I$ tal que $\Phi (\alpha) \in X_t$. (Basta tomar la igualdad
    $\vol (X_t) = 2^n \, \covol \Lambda$ porque $X_t$ es compacto.) Notamos que
    $$|N_{K/\QQ} (\alpha)| = \prod_\tau |\tau (\alpha)|,$$
    y tenemos la desigualdad entre la media aritmética y geométrica
    $$\frac{1}{n} \sum_\tau |\tau (\alpha)| \ge \left(\prod_\tau |\tau (\alpha)|\right)^{1/n}.$$

    De aquí
    \[ |N_{K/\QQ} (\alpha)| = \prod_\tau |\tau (\alpha)| \le \frac{1}{n^n} \Bigl(\sum_\tau |\tau (\alpha)|\Bigr)^n \le \frac{t^n}{n^n} = \frac{n!}{n^n}\,\left(\frac{4}{\pi}\right)^{r_2}\,\sqrt{|\Delta_K|}\cdot N_{K/\QQ} (I). \qedhere \]
  \end{proof}
\end{teorema}

En particular, si en \ref{thm:cota-de-minkowski} tomamos $I = \O_K$, entonces se
obtiene la desigualdad $1 \le |N_{K/\QQ} (\alpha)| \le M_K$, que puede ser
escrita como
\[ |\Delta_K| \ge \left(\frac{n^n}{n!}\right)^2 \, \left(\frac{\pi}{4}\right)^{2r_2}
              \ge \left(\frac{n^n}{n!}\right)^2 \, \left(\frac{\pi}{4}\right)^n. \]
Esta es una cota inferior para el discriminante en términos del grado de la
extensión $n = [K : \QQ]$. No es difícil verificar que la función de $n$ que
está a la derecha es creciente. (¡Ejercicio!) Para $n = 1$ a la derecha está
$\frac{\pi}{4}$, y para $n = 2$ tenemos $\frac{\pi^2}{4}$ que es mayor que $1$,
así que $|\Delta_K| > 1$ para $n > 1$.  Esto establece el siguiente resultado.

\begin{teorema}[Minkowski]
  Si $K/\QQ$ es una extensión no trivial, entonces $|\Delta_K| > 1$.
  En particular, en $K$ necesariamente se ramifican algunos primos.
\end{teorema}

%%%%%%%%%%%%%%%%%%%%%%%%%%%%%%%%%%%%%%%%%%%%%%%%%%%%%%%%%%%%%%%%%%%%%%%%%%%%%%%%

\section{Teorema de Hermite}

Ahora vamos a probar un teorema de Hermite que establece la finitud de campos de
números de discriminante acotado. Empecemos por un resultado auxiliar.

\begin{lema}
  Para todo campo de números $K/\QQ$ existe $\alpha \in \O_K$ tal que
  $K = \QQ (\alpha)$, y para cualquier encaje $\tau\colon K\hookrightarrow \CC$
  se tiene $|\tau (\alpha)| \le C$, donde la constante $C$ depende solamente
  del discriminante $\Delta_K$.

  \begin{proof}
    Consideremos dos casos diferentes.

    \begin{enumerate}
    \item[1)] Supongamos que $K$ tiene un encaje real
      $\rho\colon K \hookrightarrow \RR$. En este caso para $t > 1$ definamos
      el conjunto convexo simétrico
      $$X_t = \{ (x_\tau)_\tau \in K_\RR \mid |x_\rho| < t, ~ |x_\tau| < 1\text{ para }\tau \ne \rho \}.$$

    \item[2)] Si $K$ no tiene encajes reales, sean
      $\sigma, \overline{\sigma}\colon K \hookrightarrow \CC$ un par de encajes
      complejos conjugados. Definamos $X_t$ por las condiciones
      $$x_\sigma, x_{\overline{\sigma}} \in (-1, +1) + (-t, +t)\,i \subseteq \CC$$
      y
      $$|x_\tau| < 1\text{ para }\tau \ne \sigma,\overline{\sigma}.$$
    \end{enumerate}

    En ambos casos, podemos tomar $t$ suficientemente grande de tal manera que
    $\vol (X_t) > 2^n \, \sqrt{|\Delta_K|}$. El teorema de Minkowski entonces
    implica que existe un elemento no nulo $\alpha \in \O_K$ tal que
    $\Phi (\alpha) \in X_t$. Nos gustaría probar que $K = \QQ (\alpha)$.
    Todo encaje $\QQ (\alpha) \hookrightarrow \CC$ se extiende a
    $[K : \QQ (\alpha)]$ encajes $K \hookrightarrow \CC$. Entonces, sería
    suficiente ver que para cualesquiera dos encajes
    $\tau_1,\tau_2\colon K \hookrightarrow \CC$ se tiene
    $\tau_1 (\alpha) \ne \tau_2 (\alpha)$. Dejo al lector revisar las
    definiciones de $X_t$ de arriba y verificar que en cualquier caso
    $\tau_1 (\alpha) = \tau_2 (\alpha)$ para $\tau_1 \ne \tau_2$ implicaría que
    $|\tau (\alpha)| < 1$ para todo encaje $\tau\colon K\hookrightarrow \CC$,
    y luego
    $$|N_{K/\QQ} (\alpha)| = \prod_\tau |\tau (\alpha)| < 1.$$
    Sin embargo, $|N_{K/\QQ} (\alpha)| \in \ZZ_{\ge 1}$, dado que $\alpha$ es
    un entero algebraico no nulo.

    Entonces, podemos concluir que $K = \QQ (\alpha)$. Por la defnición de
    $X_t$, sabemos que los valores $|\tau (\alpha)|$ están acotados en términos
    de $t$, y como consecuencia en términos de $\Delta_K$.
  \end{proof}
\end{lema}

\begin{teorema}[Hermite]
  Para todo $C > 0$, salvo isomorfismo, hay un número finito de campos de
  números $K/\QQ$ con discriminante $|\Delta_K| < C$.

  \begin{proof}
    Primero, gracias a la cota de Minkowski, una cota sobre $|\Delta_K|$ implica
    una cota sobre el grado $[K : \QQ]$. Sería suficiente entonces ver que para
    todo grado fijo $n = [K : \QQ]$ existe un número finito de campos de números
    de discriminante fijo $\Delta_K = \Delta$.

    El lema anterior nos dice que $K = \QQ (\alpha)$, donde las raíces del
    polinomio mínimo $f = f^\alpha_\QQ \in \ZZ [x]$ están acotadas en términos
    de $\Delta$. Pero luego los coeficientes del polinomio mínimo pueden ser
    acotados en términos de $\Delta$. El grado $n = \deg (f)$ está fijo, lo que
    nos deja un número finito de posibilidades para $f$.
  \end{proof}
\end{teorema}

Cabe mencionar que el argumento de arriba usa la teoría de Minkowski y es más
reciente que los trabajos de Hermite. Para los detalles históricos (y qué
exactamente fue probado por Hermite), véase
\cite[Chapter~9]{Scharlau-Opolka-1985}.

\begin{ejemplo}
  Hay solamente dos campos cúbicos con $|\Delta_K| \le 100$. Estos están
  definidos por los polinomios
  \begin{center}
    \begin{tabular}{ll}
      $x^3 + x^2 - 2x - 1$ & ($\Delta_K = 49$) \\
      $x^3 - 3x - 1$ & ($\Delta_K = 81$)
    \end{tabular}
  \end{center}

  Por ejemplo, si aplicamos el lema de arriba a los campos cúbicos reales,
  entonces
  $$X_t = \{ (x_1,x_2,x_3) \in \RR^3 \mid |x_1| < t, \, |x_2|, |x_3| < 1 \}.$$
  Para que se cumpla $\vol (X_t) > 2^3\cdot \sqrt{|\Delta_K|}$, basta tomar
  $t > 80$. Tenemos entonces
  $$f = (x - \alpha_1)\,(x - \alpha_2)\,(x - \alpha_3) \in \ZZ[x],$$
  donde $\alpha_i$ son raíces reales con $|\alpha_i| < t$ y
  $|\alpha_2|, |\alpha_3| < 1$. Despejando la expresión para $f$, se obtiene una
  cota para los coeficientes. Esto reduce las consideraciones a un número finito
  de polinomios (aunque no es un modo muy eficaz de hacer los cálculos).
\end{ejemplo}

\begin{comentario}[Densidad de discriminantes]
  En general, si $N_n (C)$ es el número de campos de números $K/\QQ$
  (salvo isomorfismo) de grado $n$ y $|\Delta_K| < C$, el comportamiento
  asintótico de la «densidad» $N_n (C)/C$ con $C \to \infty$ es un objeto de
  recientes estudios.

  Por ejemplo, un teorema clásico de Davenport y Heilbronn (1971) dice que si
  $N_n (C)/C$ tiende a $\frac{1}{12\,\zeta (3)} = 0.069325\dots$ si se
  consideran los campos cúbicos reales ($r_1 = 3$, $r_2 = 0$, $\Delta_K > 0$)
  y a $\frac{1}{4\,\zeta (3)} = 0.207976\dots$ si se consideran los campos
  cúbicos complejos ($r_1 = r_2 = 1$, $\Delta_K < 0$).

  El artículo \cite{Belabas-1997} está dedicado a un algoritmo eficaz para
  enumerar los campos cúbicos de discriminante acotado; al final se encuentran
  unas tablas de campos de discriminantes pequeños y su número para
  $|\Delta_K| \le 10^{11}$.

  Para los resultados más recientes, véanse por ejemplo los artículos
  \cite{Bhargava-2005}, \cite{Bhargava-2010}, \cite{Belabas-Bhargava-Pomerance}.
\end{comentario}

%%%%%%%%%%%%%%%%%%%%%%%%%%%%%%%%%%%%%%%%%%%%%%%%%%%%%%%%%%%%%%%%%%%%%%%%%%%%%%%%

\iffalse
\section{Finitud del grupo de clases}

Ocupando la cota de Minkowski \ref{thm:cota-de-minkowski}, no es difícil deducir
la finitud del grupo de clases.
$$\Cl (K) = \Pic (\O_K) = \mathcal{I} (\O_K) / \mathcal{P} (\O_K).$$

\begin{lema}
  Para todo $C > 0$ hay un número finito de ideales $I \subseteq \O_K$ tales que
  $N_{K/\QQ} (I) \le C$.

  \begin{proof}
    Primero, si $I = \mathfrak{p}$ es un ideal primo, entonces
    $N_{K/\QQ} (\mathfrak{p}) = p^f$, donde $p$ es un primo tal que
    $\mathfrak{p}\cap\ZZ = p\ZZ$. Para todo primo racional $p$ hay un número
    finito de ideales primos $\mathfrak{p} \subset \O_K$ tales que
    $\mathfrak{p}\cap\ZZ = p\ZZ$ (es decir, $\mathfrak{p} \mid p$), y de estas
    consideraciones se ve que la afirmación es cierta para ideales primos.

    En general, todo ideal no nulo $I \subset \O_K$ se factoriza de alguna
    manera en ideales primos:
    $I = \mathfrak{p}_1^{e_1}\cdots \mathfrak{p}_s^{e_s}$, y luego
    $$N_{K/\QQ} (I) = N_{K/\QQ} (\mathfrak{p}_1)^{e_1}\cdots N_{K/\QQ} (\mathfrak{p}_s)^{e_s}.$$

    De esta manera se ve que para encontrar los ideales de norma $\le C$,
    podemos considerar los primos racionales $p \le C$ y los ideales primos
    correspondientes $\mathfrak{p} \mid p$ con $N_{K/\QQ} (\mathfrak{p}) \le C$,
    y luego considerar diferentes productos de estos ideales $\mathfrak{p}$.
  \end{proof}
\end{lema}

\begin{lema}
  Para todo elemento $[I] \in \Cl (K)$ existe un ideal entero $J \subseteq \O_K$
  tal que $[I] = [J]$ y $N_{K/\QQ} (J) \le M_K$, donde $M_K$ es la cota de
  Minkowski.

  \begin{proof}
    Consideremos un elemento $[I] \in \Cl (K)$ representado por un $\O_K$-ideal
    fraccionario no nulo $I \subseteq K$. En este caso existe el ideal
    fraccionario inverso $I^{-1}$, y para algún $\beta \in \O_K$ se tiene
    $$\beta I^{-1} \subseteq \O_K.$$
    Según el teorema \ref{thm:cota-de-minkowski}, existe entonces un elemento no
    nulo $\alpha \in \beta I^{-1}$ tal que
    $$|N_{K/\QQ} (\alpha)| \le M_K\cdot N_{K/\QQ} (\beta I^{-1}).$$
    Notamos que
    $\alpha \beta^{-1} I \subseteq (\beta I^{-1})\,(\beta^{-1} I) = \O_K$, así
    que el ideal $\alpha \beta^{-1} I$ es entero. Además,
    $[I] = [\alpha \beta^{-1} I]$ en el grupo de clases. La desigualdad de
    arriba nos dice que
    \[ N_{K/\QQ} (\alpha \beta^{-1} I) \le M_K. \qedhere \]
  \end{proof}
\end{lema}

\begin{teorema}
  El grupo $\Cl (K)$ es finito.

  \begin{proof}
    Hemos probado que cualquier elemento $[I] \in \Cl (K)$ puede ser
    representado por un ideal entero $J \subseteq \O_K$ tal que $[J] = [I]$ y
    $N_{K/\QQ} (J) \le M_K$. Aquí la constante $M_K$ depende solamente de $K$,
    así que hay un número finito de ideales $J$.
  \end{proof}
\end{teorema}

%%%%%%%%%%%%%%%%%%%%%%%%%%%%%%%%%%%%%%%%%%%%%%%%%%%%%%%%%%%%%%%%%%%%%%%%%%%%%%%%

\section{Ejemplos de cálculos del grupo de clases}

\begin{ejemplo}
  Consideremos los campos cuadráticos imaginarios $K = \QQ (\sqrt{d})$, donde
  $d < 0$. En este caso $n = 2$ y $r_2 = 1$, así que la cota de Minkowski será
  $M_K = \frac{2}{\pi}\,\sqrt{|\Delta_K|}$.

  \begin{center}\renewcommand{\arraystretch}{1.5}
    \begin{tabular}{rx{1cm}x{1cm}x{1cm}x{1cm}x{1cm}x{1cm}x{1cm}x{1cm}x{1cm}x{1cm}}
      \hline
      $d\colon$ & $-1$ & $-2$ & $-3$ & $-5$ & $-6$ & $-7$ & $-10$ & $-11$ & $-13$ & $-14$ \tabularnewline
      \hline
      $\Delta_{\QQ (\sqrt{d})}\colon$ & $-4$ & $-8$ & $-3$ & $-20$ & $-24$ & $-7$ & $-40$ & $-11$ & $-52$ & $-56$ \tabularnewline
      \hline
      $M_{\QQ (\sqrt{d})}\colon$ & $1.27$ & $1.80$ & $1.10$ & $2.85$ & $3.12$ & $1.68$ & $4.03$ & $2.11$ & $4.59$ & $4.76$ \tabularnewline
      \hline
    \end{tabular}
  \end{center}

  \vspace{1em}

  \begin{itemize}
  \item Si $M_K < 2$, esto implica que $\Cl (K) = 0$. De esta manera sabemos que
    los campos
    $$\QQ (i), ~ \QQ (\sqrt{-2}), ~ \QQ (\sqrt{-3}), ~ \QQ (\sqrt{-7})$$
    tienen el grupo de clases trivial. Esto no es algo nuevo: ya sabemos que los
    anillos de enteros correspondientes
    \[ \ZZ [i], ~
       \ZZ [\sqrt{-2}], ~
       \ZZ \Bigl[\frac{1+\sqrt{-3}}{2}\Bigr], ~
       \ZZ \Bigl[\frac{1+\sqrt{-7}}{2}\Bigr] \]
    son dominios euclidianos.

  \item El anillo de enteros de $\QQ (\sqrt{-11})$ es
    $\ZZ \Bigl[\frac{1+\sqrt{-11}}{2}\Bigr]$ y es también un dominio euclidiano.

    La cota de Minkowski en este caso es $M_K = 2.11$. El primo $p = 2$ es
    inerte en $\QQ (\sqrt{-11})$, así que no hay ideales de norma $2$. Esto nos
    da otra prueba de que $\QQ (\sqrt{-11})$ tiene el grupo de clases trivial.

  \item Consideremos el campo $K = \QQ (\sqrt{-5})$. En este caso $M_K = 2.85$
    nos dice que todo elemento en $\Cl (K)$ puede ser representado por un ideal
    entero de norma $1$ o $2$. El único ideal de norma $2$ es el ideal primo que
    está arriba de $p = 2$:
    $$2\O_K = \mathfrak{p}^2, \quad \mathfrak{p} = (2, 1 + \sqrt{-5}).$$

    Afirmamos que el ideal $\mathfrak{p}$ no es principal: en el caso contrario
    tendríamos $\mathfrak{p} = (\alpha)$ para algún $\alpha \in \O_K$,
    y luego $N_{K/\QQ} (\mathfrak{p}) = |N_{K/\QQ} (\alpha)|$. Sin embargo,
    $N_{K/\QQ} (\mathfrak{p}) = 2$, mientras que
    $N_{K/\QQ} (\alpha) = a^2 + 5b^2 \ne 2$.

    Entonces, $[\mathfrak{p}]$ es el único elemento no trivial del grupo de
    clases, y podemos concluir que $\Cl (K) \cong \ZZ/2\ZZ$.

  \item En el caso de $K = \QQ (\sqrt{-6})$ tenemos que examinar los ideales de
    norma $2$ y $3$. Estos son
    $$\mathfrak{p}_2 = (2, \sqrt{-6}), \quad \mathfrak{p}_3 = (3, \sqrt{-6}).$$
    Calculamos que
    \[ \mathfrak{p}_2^2 = 2\O_K, \quad
       \mathfrak{p}_3^2 = 3\O_K, \quad
       \mathfrak{p}_2\,\mathfrak{p}_3 = (\sqrt{-6}). \]
    Esto implica que en el grupo de clases los elementos $[\mathfrak{p}_2]$
    y $[\mathfrak{p}_3]$ tienen orden $2$, y además
    $[\mathfrak{p}_2]\cdot [\mathfrak{p}_3] = [\O_K]$ implica que    
    $[\mathfrak{p}_2] = [\mathfrak{p}_3]$. Un argumento similar al de arriba
    demuestra que ideal $\mathfrak{p}_2$ no es principal, así que
    $\Cl (K) \cong \ZZ/2\ZZ$.

  \item Para los campos $K = \QQ (\sqrt{-10}), \QQ (\sqrt{-13})$ de la misma
    manera se puede ver que $\Cl (K) \cong \ZZ/2\ZZ$. Dejo los detalles como
    un ejercicio.

  \item En fin, consideremos $K = \QQ (\sqrt{-14})$. La cota de Minkowski es
    $M_K = 4.76\ldots$, así que basta considerar los ideales primos arriba de $2$
    y $3$. Estos son los siguientes:
    $$\mathfrak{p}_2 = (2, \sqrt{-14}), \quad 2\O_K = \mathfrak{p}_2^2,$$
    y
    \[ \mathfrak{p}_3 = (3, 1 + \sqrt{-14}), \quad
       \overline{\mathfrak{p}_3} = (3, 1 - \sqrt{-14}), \quad
       3\O_K = \mathfrak{p}_3\,\overline{\mathfrak{p}_3}. \]

    En el anillo $\ZZ [\sqrt{-14}]$ no hay elementos de norma $2$ y $3$, así que
    los ideales $\mathfrak{p}_2, \mathfrak{p}_3, \overline{\mathfrak{p}_3}$ no
    son principales.

    Calculamos que
    \[ \mathfrak{p}_3^2 = (9, 3 + \sqrt{-14}, -13 + 2\sqrt{-14}) =
       (1 - \sqrt{-14}/2) \, \mathfrak{p}_2, \]
    así que $[\mathfrak{p}_3]^2 = [\mathfrak{p}_2]$. Por otra parte,
    $[\mathfrak{p}_3]^4 = [\mathfrak{p}_2]^2 = [\O_K]$. Entonces,
    $[\mathfrak{p}_3]$ tiene orden $4$. Tenemos
    $[\overline{\mathfrak{p}_3}] = [\mathfrak{p}_3]^{-1} = [\mathfrak{p}_3]^3$.
    Podemos concluir que $\Cl (K) \cong \ZZ/4\ZZ$. \qedhere
  \end{itemize}
\end{ejemplo}

En los cálculos de arriba la parte más complicada es verificar si algún ideal
$\mathfrak{p} \subset \O_K$ es principal. Existe un algoritmo general para
resolver este problema; véase por ejemplo \cite{Buchmann-Williams-1987}.

\begin{ejemplo}
  Consideremos los siguientes campos cuadráticos imaginarios:

  \begin{center}\renewcommand{\arraystretch}{1.5}
    \begin{tabular}{rx{2cm}x{2cm}x{2cm}x{2cm}}
      \hline
      $d\colon$ & $-19$ & $-43$ & $-67$ & $-163$ \tabularnewline
      \hline
      $\Delta_{\QQ (\sqrt{d})}\colon$ & $-19$ & $-43$ & $-67$ & $-163$ \tabularnewline
      \hline
      $M_{\QQ (\sqrt{d})}\colon$ & $2.77$ & $4.17$ & $5.21$ & $8.12$ \tabularnewline
      \hline
    \end{tabular}
  \end{center}

  \begin{itemize}
  \item En $K = \QQ (\sqrt{-19})$ el primo $p = 2$ es inerte, así que en $\O_K$
    no hay ideales no triviales de norma $\le M_K$. Podemos concluir que
    $\Cl (K) = 0$.

  \item En $K = \QQ (\sqrt{-43})$ los primos $p = 2, 3$ son inertes, así que los
    ideales de norma $\le M_K$ serán $I = \O_K, 2\O_K$. Son principales, así que
    $\Cl (K) = 0$.

  \item En $K = \QQ (\sqrt{-67})$ los primos $p = 2,3,5$ son inertes, y de la
    misma manera esto implica que $\Cl (K) = 0$.

  \item En $K = \QQ (\sqrt{-163})$ los ideales $p = 2,3,5,7$ son inertes,
    de donde de nuevo se deduce que $\Cl (K) = 0$. \qedhere
  \end{itemize}
\end{ejemplo}

\begin{comentario}
  Los cálculos de arriba nos dicen que
  \[ \Cl (\QQ (\sqrt{d})) = 0
     \quad\text{para}\quad
     d = -1, -2, -3, -7, -11, -19, -43, -67, -163.\]

  De hecho, estos son los únicos campos cuadráticos imaginarios con el grupo de
  clases trivial. Esto fue conjeturado por Gauss y probado por Kurt Heegner y
  Harold Stark. La prueba de Heegner fue publicada en 1952, pero contenía
  menores omisiones y no fue aceptada hasta que Stark dio una prueba completa en
  1967. Otra prueba diferente fue publicada por Alan Baker en 1966 y reducía el
  problema a un cálculo finito.

  Heegner falleció en 1965, antes de que su trabajo fue reconocido por
  la comunidad matemática\dots
\end{comentario}

\begin{ejemplo}
  Consideremos el campo ciclotómico $K = \QQ (\zeta_7)$. En este caso
  $n = 6$, $r_2 = 3$, y $\Delta_K = -7^5$, así que la cota de Minkowski
  viene dada por
  $$M_K = \frac{6!}{6^6} \, \left(\frac{4}{\pi}\right)^3 \, 7^{5/2} = 4.12\ldots$$
  En este caso el orden de $p = 2$ mód $7$ es igual a $f = 3$, así que
  $$2\O_K = \mathfrak{p}_2 \, \mathfrak{p}_2'.$$
  Por otra parte, el orden de $p = 3$ mód $7$ es igual a $f = 6$, lo que
  significa que $p = 3$ es inerte en $K$.

  Factorizando el polinomio ciclotómico mód $2$ se obtiene
  $$\Phi_7 \equiv (x^3 + x + 1)\,(x^3 + x^2 + 1) \pmod{2}.$$
  Entonces,
  \[ \mathfrak{p}_2 = (2, 1 + \zeta_7 + \zeta_7^3), \quad
     \mathfrak{p}_2' = (2, 1 + \zeta_7^2 + \zeta_7^3). \]
  Afirmamos que estos ideales son principales. En efecto,
  $$(1 + \zeta_7 + \zeta_7^3)\,(1 + \zeta_7^2 + \zeta_7^3) = 2\zeta_7^3,$$
  así que
  \[ \mathfrak{p}_2 = (1 + \zeta_7 + \zeta_7^3), \quad
     \mathfrak{p}_2' = (1 + \zeta_7^2 + \zeta_7^3). \]
  Esto demuestra que $\Cl (K) = 0$.
\end{ejemplo}

\begin{comentario}
  En general, los cálculos con la cota de Minkowski no son muy prácticos para
  hacerlos a mano. Por ejemplo, el primer campo ciclotómico con el grupo de
  clases no trivial es $K = \QQ (\zeta_{23})$, donde
  $\Cl (K) \cong \ZZ/3\ZZ$. Las cotas de Minkowski para los campos ciclotómicos
  son las siguientes.

  \begin{center}\renewcommand{\arraystretch}{1.5}
    \begin{tabular}{rccccccccccc}
      \hline
      $K\colon$ & $\QQ (\zeta_3)$ & $\QQ (\zeta_4)$ & $\QQ (\zeta_5)$ & $\QQ (\zeta_7)$ & $\QQ (\zeta_8)$ & $\QQ (\zeta_9)$ & $\QQ (\zeta_{11})$ & $\QQ (\zeta_{12})$ & $\QQ (\zeta_{13})$ & $\cdots$ & $\QQ (\zeta_{23})$ \tabularnewline
      \hline
      $M_K\colon$ & $1.10$ & $1.27$ & $1.70$ & $4.13$ & $2.43$ & $4.47$ & $58.96$ & $1.82$ & $306.42$ & $\cdots$ & $9324406.48$ \tabularnewline
      \hline
    \end{tabular}
  \end{center}
\end{comentario}

%%%%%%%%%%%%%%%%%%%%%%%%%%%%%%%%%%%%%%%%%%%%%%%%%%%%%%%%%%%%%%%%%%%%%%%%%%%%%%%%

\section{Ecuación de Pell}

Antes de probar el teorema de unidades de Dirichlet, vamos a ver su caso
particular relacionado con los campos cuadráticos reales.

\begin{teorema}
  \label{thm:pell-minkowski}
  Sea $d > 1$ un entero libre de cuadrados. La ecuación
  $$x^2 - dy^2 = 1$$
  tiene una solución entera distinta de $(\pm 1, 0)$.

  \begin{proof}
    Consideremos el campo de números $K = \QQ (\sqrt{d})$. Nuestra ecuación
    puede ser interpretada como
    $$N_{K/\QQ} (x + y\sqrt{d}) = 1.$$
    Vamos a encajar el anillo de números $\ZZ [\sqrt{d}]$ como un retículo
    $\Lambda$ en $\RR^2$ mediante
    $$a + b\sqrt{d} \mapsto (a + b\sqrt{d}, a - b\sqrt{d}).$$
    Respecto a este encaje, las soluciones que nos interesan son los puntos
    del retículo $\Lambda$ en la hipérbola $xy = 1$. Sea $X$ un conjunto
    convexo simétrico suficientemente grande para que
    $X \cap \Lambda \ne \{ 0 \}$. Según el teorema de Minkowski, hay que
    tomar $X$ tal que $\vol X > 4 \covol \Lambda$.

    Ahora para todo $\lambda > 0$ podemos considerar el conjunto
    $X_\lambda = (\lambda, \lambda^{-1})\,X$. Este tiene el mismo volumen que
    $X$, así que $X_\lambda \cap \Lambda \ne \{ 0 \}$
    (véase la figura~\ref{fig:pell-minkowski})

    De esta manera para todo $\lambda > 0$ se obtiene un elemento
    $\alpha_\lambda \in \ZZ [\sqrt{d}]$. Los puntos que corresponden a
    $X \cap \Lambda$ tienen norma acotada, y la transformación
    $(\lambda, \lambda^{-1})$ preserva la cota sobre la norma. Entonces,
    tenemos $|N_{K/\QQ} (\alpha_\lambda)| \le C$ para todo $\alpha_\lambda$.
    Hay un número finito de ideales de norma acotada, así que existen
    $\lambda \ne \lambda'$ tales que $(\alpha_\lambda) = (\alpha_{\lambda'})$
    y $\alpha_\lambda \ne \pm\alpha_{\lambda'}$. En este caso
    $u = \alpha_\lambda/\alpha_{\lambda'}$ es una unidad en el anillo
    $\ZZ [\sqrt{d}]$ distinta de $\pm 1$. Tenemos $N_{K/\QQ} (u) = \pm 1$.
    Si $N_{K/\QQ} (u) = -1$, entonces $N_{K/\QQ} (u^2) = +1$. De esta manera
    se obtiene una solución no trivial de la ecuación de Pell.
  \end{proof}
\end{teorema}

\begin{figure}
  \begin{center}
    \includegraphics{pic/pell-minkowski.pdf}
  \end{center}

  \caption{Argumento de \ref{thm:pell-minkowski}}
  \label{fig:pell-minkowski}
\end{figure}

\begin{corolario}
  El grupo de unidades $\ZZ [\sqrt{d}]^\times$ es infinito.

  \begin{proof}
    Hemos encontrado una unidad $u \in \ZZ [\sqrt{d}]^\times$ distinta de
    $\pm 1$, y luego $u^n$ para $n \in \ZZ$ son distintas unidades.
  \end{proof}
\end{corolario}

%%%%%%%%%%%%%%%%%%%%%%%%%%%%%%%%%%%%%%%%%%%%%%%%%%%%%%%%%%%%%%%%%%%%%%%%%%%%%%%%

\section{Teorema de unidades de Dirichlet}

**TODO**

%%%%%%%%%%%%%%%%%%%%%%%%%%%%%%%%%%%%%%%%%%%%%%%%%%%%%%%%%%%%%%%%%%%%%%%%%%%%%%%%
\fi

\pagebreak

\phantomsection

\addcontentsline{toc}{section}{Ejercicios}
\section*{Ejercicios}

\begin{ejercicio}
  Demuestre que si $G$ es un grupo topológico Hausdorff, entonces todo subgrupo
  discreto $H \subset G$ es cerrado.
\end{ejercicio}

\begin{ejercicio}
  Demuestre directamente que $\ZZ [\sqrt{2}]$ no es un subgrupo discreto de
  $\RR$.
\end{ejercicio}

\begin{ejercicio}
  Demuestre que si $X$ es un conjunto convexo simétrico compacto tal que
  $\vol X = 2^n\cdot \covol \Lambda$, entonces $X \cap \Lambda \ne \{ 0 \}$.
\end{ejercicio}

\begin{ejercicio}
  Demuestre que para todo primo $p$ existen $m,n \in \ZZ$ tales que
  $m^2 + n^2 + 1 \equiv 0 \pmod{p}$.
\end{ejercicio}

\begin{ejercicio}
  Demuestre que
  $$a_n = \left(\frac{n^n}{n!}\right)^2\,\left(\frac{\pi}{4}\right)^n$$
  crece con $n$.
\end{ejercicio}

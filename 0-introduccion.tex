\addcontentsline{toc}{chapter}{Introducción}
\chapter*{Introducción}

\begin{framed}
  Esta es una versión preliminar de mis apuntes del curso.

  Para la última versión, visite la página
  \url{http://cadadr.org/cimat-tna/apuntes.html}

  Preguntas, comentarios y correcciones:
  \texttt{alexey.beshenov@cimat.mx}
\end{framed}

\epigraph{I attended a course in algebraic number theory from Artin which was
  extremely elegant, although perhaps too advanced for me. However, it wasn't
  until a few years later that I learned what an algebraic number was. The
  course was so streamlined that algebraic numbers were never actually
  mentioned.}
  {John Milnor, citado por Steven Krantz}

La teoría de números algebraicos estudia\dots{} los
\textbf{números algebraicos}, es decir, los números complejos $\alpha \in \CC$
que satisfacen una relación algebraica no trivial
$$a_n \alpha^n + a_{n-1} \alpha^{n-1} + \cdots + a_1 \alpha + a_0 = 0,$$
donde $a_i \in \QQ$ y $a_n \ne 0$. Estos números viven en
los \textbf{campos de números} que son extensiones finitas $K/\QQ$.
A saber, los campos de números son de la forma $K = \QQ
(\alpha_1,\ldots,\alpha_s)$, donde los $\alpha_i$ son números algebraicos.

Un ejemplo sencillo de campo de números es
$$\QQ (\sqrt{-5}) = \{ a + b\sqrt{-5} \mid a,b \in \QQ \},$$
la extensión cuadrática de los números racionales que se obtiene añadiendo la
raíz cuadrada $\sqrt{-5}$.

La teoría de números surge al considerar subanillos en los campos de números
$R \subset K$, que sería lógico denominar los \textbf{anillos de números}.
(No es un término muy común, pero lo adoptaremos en nuestro curso, siguiendo
a \cite{Stevenhagen-NR}.) Por ejemplo
$$\ZZ [\sqrt{-5}] = \{ a + b\sqrt{-5} \mid a,b \in \ZZ \}$$
es un anillo de números dentro del campo de números $\QQ (\sqrt{-5})$.

Los anillos de números son objetos unidimensionales. Específicamente, a
cualquier anillo conmutativo $R$ se puede asociar su \emph{dimensión de Krull}
$\dim R$, y para cualquier anillo de números se cumple $\dim R = 1$. En este
sentido la teoría de anillos de números se parece mucho a la teoría de curvas
algebraicas.

Los anillos de números son generalizaciones bastante sencillas del anillo de los
números enteros $\ZZ$, pero en los anillos de números, entre otras cosas,
ya no necesariamente se cumple el \emph{teorema fundamental de la aritmética}
(que afirma que todo número se expresa esencialmente de manera única como un
producto de números primos). Por ejemplo, en el anillo $\ZZ [\sqrt{-5}]$
$$2\cdot 3 = (1 + \sqrt{-5})\,(1 - \sqrt{-5})$$
son dos factorizaciones distintas del número $6$. La idea de Richard Dedekind
consistía en remplazar las factorizaciones en números primos por factorizaciones
de \emph{ideales} en \emph{ideales primos} del anillo. En el ejemplo de arriba,
\[ (2) = \mathfrak{p}^2, \quad
   (3) = \mathfrak{q}_1 \mathfrak{q}_2, \quad
   (1 + \sqrt{-5}) = \mathfrak{p} \mathfrak{q}_1, \quad
   (1 - \sqrt{-5}) = \mathfrak{p} \mathfrak{q}_2, \]

donde
\[ \mathfrak{p} = (2, 1 + \sqrt{-5}); \quad
   \mathfrak{q}_1 = (3, 1 + \sqrt{-5}); \quad
   \mathfrak{q}_2 = (3, 2 + \sqrt{-5}) \]
son ideales primos en $\ZZ [\sqrt{-5}]$. Los anillos de números donde los
ideales se descomponen de manera única en ideales primos se conocen como
los \textbf{anillos de Dedekind}. Todas estas nociones serán introducidas y
consideradas en detalles en el curso.

El objetivo principal será definir algunos invariantes fundamentales de los
campos de números: el \textbf{anillo de enteros} $\O_K \subset K$,
\textbf{grupo de clases} $\Cl (K)$, y \textbf{grupo de unidades}
$\O_K^\times$, demostrar sus propiedades básicas y aprender a
calcularlos.

Todos los invariantes que serán considerados en el curso se pueden calcular
algorítmicamente. En particular, veremos ejemplos de cálculos en el programa
PARI/GP (\url{https://pari.math.u-bordeaux.fr/}) y la base de datos LMFDB
(\url{https://lmfdb.org/}). Todo el material teórico será acompañado de
problemas con pruebas y cálculos particulares.

\section{Para qué sirve este curso}

Este curso podría ser interesante para los que estudian álgebra conmutativa,
ya que serán consideradas algunas nociones fundamentales de esta área (ideales
primos, anillos de valuación discreta, anillos de Dedekind, el grupo de Picard
de un anillo conmutativo, el grupo de unidades, etc.), basándose en ejemplos muy
concretos y calculables. En cierto sentido, el álgebra conmutativa
históricamente se originó en la teoría de números algebraicos. (El mismo término
«anillo» fue introducido por Hilbert en un contexto de anillos de números,
e «ideal» es la abreviación del «número ideal».)

Además, la similitud entre los anillos de números y curvas algebraicas que
mencioné arriba, haría este material útil para los que están aprendiendo
superficies de Riemann, singularidades de curvas, etc. y los interesados
en la geometría algebraica moderna (la teoría de esquemas etc.).

Por último, y no menos importante, este curso es fundamental para los
estudiantes con intención de aprender la teoría de números.

\section{Conocimientos preliminares}

Tendré que suponer que los oyentes conozcan las nociones como anillo
(conmutativo), ideal (primo, maximal), anillo cociente, módulo sobre un anillo
(módulo libre, rango), y campo (incluso la teoría de campos finitos).
Tampoco estaría mal conocer la teoría de Galois básica, pero el lector puede
consultar el apéndice \ref{ap:teoria-de-Galois} para un breve resumen.

De todas maneras, cuando sea necesario en el transcurso, trataremos las nociones
poco conocidas. Uno de mis objetivos es presentar diferentes herramientas
algebraicas, así como ejemplos muy concretos.

\section{Referencias}

Mi fuente principal de inspiración son los apuntes de
Peter Stevenhagen \cite{Stevenhagen-NR} de un curso que se imparte en
la universidad de Leiden (Países Bajos). Además, podrían ser útiles diferentes
libros de texto sobre el tema; he aquí algunas fuentes que puedo recomendar.

Algunos apuntes en línea, a parte de \cite{Stevenhagen-NR}, son los siguientes:
\begin{itemize}
\item el curso de Andrew Sutherland en MIT:
  \url{https://dspace.mit.edu/handle/1721.1/124987}

\item J.S. Milne: 
  \url{https://www.jmilne.org/math/CourseNotes/ant.html}

\item Paul Garrett:
  \url{http://www-users.math.umn.edu/~garrett/m/number_theory/}

\item varios apuntes de Keith Conrad:
  \url{https://kconrad.math.uconn.edu/blurbs/}

\item un curso de Robert B. Ash:
  \url{https://faculty.math.illinois.edu/~r-ash/ANT.html}
\end{itemize}

Algunos libros introductorios son
\cite[Chapters 12, 13, 17]{Ireland-Rosen}, 
\cite{Alaca-Williams},
\cite{Kato-NT-2},
\cite{Frohlich-Taylor},
\cite{Marcus-NF},
\cite{Samuel-TAN},
\cite[Chapters 4, 5]{Borevich-Shafarevich},
\cite{Cox-2013}.

Para experimentos en PARI/GP, véase el libro
\cite{Rodriguez-Villegas-2007}.

Lectura avanzada:
\cite{Neukirch-ANT},
\cite{Lang-ANT},
\cite{Cassels-Frohlich}.

En fin, en este curso veremos varios cálculos específicos, pero no hablaremos
de algoritmos serios. Las fuentes recomendadas sobre la teoría algorítmica
son
\cite{Pohst-Zassenhaus},
\cite{Lenstra-1992},
\cite{Cohen-GTM138}.

\section{Agradecimientos}

Agradezco a CIMAT por la oportunidad de dar este curso, y en particular
al Dr.~Xavier Gómez Mont y Dr.~Pedro Luis del Ángel.

Pavel Solomatin y Dmitry Shvetsov han echo varias observaciones útiles acerca de
una versión preliminar de mis notas, y hemos tenido muchas conversaciones sobre
la teoría de números y pedagogía.
